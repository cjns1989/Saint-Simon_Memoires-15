\PassOptionsToPackage{unicode=true}{hyperref} % options for packages loaded elsewhere
\PassOptionsToPackage{hyphens}{url}
%
\documentclass[oneside,14pt,french,]{extbook} % cjns1989 - 27112019 - added the oneside option: so that the text jumps left & right when reading on a tablet/ereader
\usepackage{lmodern}
\usepackage{amssymb,amsmath}
\usepackage{ifxetex,ifluatex}
\usepackage{fixltx2e} % provides \textsubscript
\ifnum 0\ifxetex 1\fi\ifluatex 1\fi=0 % if pdftex
  \usepackage[T1]{fontenc}
  \usepackage[utf8]{inputenc}
  \usepackage{textcomp} % provides euro and other symbols
\else % if luatex or xelatex
  \usepackage{unicode-math}
  \defaultfontfeatures{Ligatures=TeX,Scale=MatchLowercase}
%   \setmainfont[]{EBGaramond-Regular}
    \setmainfont[Numbers={OldStyle,Proportional}]{EBGaramond-Regular}      % cjns1989 - 20191129 - old style numbers 
\fi
% use upquote if available, for straight quotes in verbatim environments
\IfFileExists{upquote.sty}{\usepackage{upquote}}{}
% use microtype if available
\IfFileExists{microtype.sty}{%
\usepackage[]{microtype}
\UseMicrotypeSet[protrusion]{basicmath} % disable protrusion for tt fonts
}{}
\usepackage{hyperref}
\hypersetup{
            pdftitle={SAINT-SIMON},
            pdfauthor={Mémoires XV},
            pdfborder={0 0 0},
            breaklinks=true}
\urlstyle{same}  % don't use monospace font for urls
\usepackage[papersize={4.80 in, 6.40  in},left=.5 in,right=.5 in]{geometry}
\setlength{\emergencystretch}{3em}  % prevent overfull lines
\providecommand{\tightlist}{%
  \setlength{\itemsep}{0pt}\setlength{\parskip}{0pt}}
\setcounter{secnumdepth}{0}

% set default figure placement to htbp
\makeatletter
\def\fps@figure{htbp}
\makeatother

\usepackage{ragged2e}
\usepackage{epigraph}
\renewcommand{\textflush}{flushepinormal}

\usepackage{indentfirst}
\usepackage{relsize}

\usepackage{fancyhdr}
\pagestyle{fancy}
\fancyhf{}
\fancyhead[R]{\thepage}
\renewcommand{\headrulewidth}{0pt}
\usepackage{quoting}
\usepackage{ragged2e}

\newlength\mylen
\settowidth\mylen{...................}

\usepackage{stackengine}
\usepackage{graphicx}
\def\asterism{\par\vspace{1em}{\centering\scalebox{.9}{%
  \stackon[-0.6pt]{\bfseries*~*}{\bfseries*}}\par}\vspace{.8em}\par}

\usepackage{titlesec}
\titleformat{\chapter}[display]
  {\normalfont\bfseries\filcenter}{}{0pt}{\Large}
\titleformat{\section}[display]
  {\normalfont\bfseries\filcenter}{}{0pt}{\Large}
\titleformat{\subsection}[display]
  {\normalfont\bfseries\filcenter}{}{0pt}{\Large}

\setcounter{secnumdepth}{1}
\ifnum 0\ifxetex 1\fi\ifluatex 1\fi=0 % if pdftex
  \usepackage[shorthands=off,main=french]{babel}
\else
  % load polyglossia as late as possible as it *could* call bidi if RTL lang (e.g. Hebrew or Arabic)
%   \usepackage{polyglossia}
%   \setmainlanguage[]{french}
%   \usepackage[french]{babel} % cjns1989 - 1.43 version of polyglossia on this system does not allow disabling the autospacing feature
\fi

\title{SAINT-SIMON}
\author{Mémoires XV}
\date{}

\begin{document}
\maketitle

\hypertarget{chapitre-premier.}{%
\chapter{CHAPITRE PREMIER.}\label{chapitre-premier.}}

1717

~

{\textsc{Attention générale sur le voyage du czar à Paris.}} {\textsc{-
Le roi de Prusse tenté et détourné d'y venir.}} {\textsc{- Vues et
conduite de ce prince.}} {\textsc{- Liaison entre le roi de Prusse et le
czar.}} {\textsc{- Inquiétude du roi d'Angleterre sur le czar.}}
{\textsc{- Il est forcé de réformer dix mille hommes.}} {\textsc{-
Servitude de la Hollande pour l'Angleterre.}} {\textsc{- Union et traité
entre le czar et le roi de Prusse.}} {\textsc{- Mesures du czar avec la
France et avec le roi de Pologne.}} {\textsc{- Mesures sur le séjour des
troupes moscovites dans le Mecklembourg.}} {\textsc{- Le pape veut lier
le czar avec l'empereur contre le Turc.}} {\textsc{- Manèges d'Albéroni
en France pour son chapeau.}} {\textsc{- Véritables raisons du
changement de conduite d'Albéroni à l'égard du pape.}} {\textsc{- Le
pape écrit au czar\,; il le veut liguer avec l'empereur et obtenir le
libre exercice de la religion catholique dans ses États.}} {\textsc{- Le
czar l'amuse et se moque de lui.}} {\textsc{- Il en parle très sensément
au maréchal de Tessé.}} {\textsc{- Molinez, inquisiteur général
d'Espagne, revenant de Rome en Espagne, arrêté à Milan.}} {\textsc{-
Embarras et caractère du pape.}} {\textsc{- Promotion d'Albéroni est
l'unique affaire.}} {\textsc{- Il se moque de Molinez, s'assure du
régent sur sa promotion.}} {\textsc{- Ses vanteries.}} {\textsc{- La
cour d'Espagne à l'Escurial malgré la reine.}} {\textsc{- Aldovrandi y
arrive.}} {\textsc{- Manèges d'Albéroni.}} {\textsc{- L'Angleterre
reprend la négociation de la paix entre l'empereur et l'Espagne.}}
{\textsc{- Divisions domestiques en Angleterre.}} {\textsc{- Son
inquiétude sur le czar.}} {\textsc{- Troupes russiennes sortent du
Mecklembourg.}} {\textsc{- Le Danemark, inquiet sur le nord, éprouve le
mécontentement de la Russie.}} {\textsc{- Le czar veut traiter avec la
France.}} {\textsc{- Obstacles du traité.}} {\textsc{- Le czar en mesure
avec l'empereur à cause du czarowitz.}} {\textsc{- Plaintes et avis du
roi de Prusse.}} {\textsc{- Offices du régent sur le nord.}} {\textsc{-
Scélératesse du nonce Bentivoglio.}} {\textsc{- Le Prétendant à Rome\,;
y sert Albéroni.}} {\textsc{- Soupçons de nouveaux délais de sa
promotion.}} {\textsc{- Hauteurs et manèges du pape.}} {\textsc{- Départ
de Cadix de la flotte d'Espagne.}} {\textsc{- Scélératesse d'Albéroni.}}
{\textsc{- Giudice à Rome.}} {\textsc{- Misère de sa conduite, de sa
position, de sa réputation.}} {\textsc{- Friponnerie d'Ottobon.}}
{\textsc{- Chiaous à Marseille.}} {\textsc{- Vie solitaire et pénitente
de Ragotzi.}}

~

Le voyage du czar en France, au commencement de mai, devint l'attention
de toute l'Europe, en particulier de l'Angleterre. Le roi de Prusse y
serait venu en même temps si on ne lui en eût fait craindre du ridicule,
et que l'empereur n'en prit un violent ombrage. Ces deux princes étaient
également mécontents du roi d'Angleterre, ils ne comptaient pas d'avoir
rien à espérer de l'empereur. Leur vue était de conclure une paix
avantageuse avec la Suède.

Le roi de Prusse sollicitait le régent d'ordonner positivement au comte
de La Marck, ambassadeur de France auprès du roi de Suède, d'engager
promptement une négociation pour la paix entre eux et d'en poursuivre
vivement la conclusion. Il insistait à profiter de la guerre du Turc,
dont l'empereur ne serait pas plutôt débarrassé qu'il voudrait agir en
maître des affaires de l'empire et de celles du nord, où il prendrait
des liaisons préjudiciables à la France. Il avertissait le régent de se
défier de Georges tout occupé de ménager l'empereur à cause de ses États
d'Allemagne, et de ceux qu'il avait usurpés sur la Suède, et à qui il
voulait faire torcher deux cent cinquante mille livres sterling, que le
parlement allait lever pour le payement des arrérages dus aux alliés de
l'Angleterre et des subsides de la dernière guerre. Irrité d'être
frustré de sa part sur cette somme, il désirait prendre avec la France
des engagements plus forts que ceux qu'il avait déjà avec elle par un
traité secret. Il avait paru éluder la proposition que le régent lui
avait faite d'entrer dans la triple alliance, alarmé aussi du bruit
répandu que le roi d'Angleterre y faisait admettre le Danemark. Il
n'était pas aisé de compter sur le roi de Prusse, léger, inconstant,
plein de variations subites, et qui prodiguait à l'empereur tout ce
qu'il espérait lui pouvoir concilier sa protection.

Il fit savoir au czar, à Paris, en mai, qu'ils ne devaient compter ni
l'un ni l'autre sur l'empereur pour la conservation de leurs conquêtes
sur la Suède\,; qu'il était de leur intérêt commun de ne pas attendre
que l'empereur fût débarrassé de la guerre du Turc pour traiter avec la
Suède, et qu'ils ne le pourraient faire avantageusement que par le moyen
de la France. C'était lui dire de s'attacher à cette couronne. Cet avis
était fondé sur ce qu'il lui était revenu que les ministres de Vienne
avaient dit à celui de Russie que, sensible à la confiance du czar,
l'empereur prendrait volontiers des mesures plus étroites avec lui
pourvu qu'il ne s'agît point des affaires du nord, dont jusqu'alors il
ne s'était point mêlé, et qu'il ne pouvait dans ces affaires exercer que
son office de juge supérieur. Que d'ailleurs si le czar voulait prendre
avec lui quelques mesures sur la guerre du Turc, il en serait fort aise.

Quelque temps après le roi de Prusse apprit que l'empereur, irrité plus
que jamais du séjour des troupes moscovites dans le Mecklembourg, malgré
les promesses de bouche et par écrit de les en retirer, avait dit qu'il
les en ferait sortir à main armée, et demandé à ceux qui lui
représentaient les suites d'un pareil engagement s'ils craignaient les
Moscovites, qu'il n'avait, lui, aucun sujet d'appréhender. Le roi de
Prusse fit communiquer ces avis aux czar, et ses soupçons des desseins
secrets du roi d'Angleterre de joindre à ses troupes celles du Danemark
et des princes de la basse Allemagne pour chasser les Moscovites du
Mecklembourg, sous le nom et l'autorité de l'empereur. Le czar répondit
à la confiance du roi de Prusse, et l'assura qu'il pensait sérieusement
à un traité avec la France\,; qu'il lui communiquerait tout ce qu'il y
ferait, et lui promit de ne rien conclure sans sa participation.

Georges connaissait très bien le caractère de son gendre, capable
d'entrer en beaucoup de choses contre lui. Mais, se reposant sur sa
perpétuelle instabilité, il tournait toute son inquiétude sur le voyage
du czar à Paris, persuadé que c'était dans le dessein d'y prendre des
liaisons étroites, dont le séjour des troupes moscovites dans le
Mecklembourg augmentait l'alarme. Il n'avait plus de prétexte de
conserver ses troupes. Le roi de Suède désavouait ses ministres. Nul
vaisseau ni préparatif dans le port de Gottembourg. Ainsi, Georges se
trouva forcé de déclarer au parlement qu'il réformait dix mille hommes.
La France ne donnait plus d'alarmes à l'Angleterre, surtout depuis la
triple alliance, et la Hollande persévérait dans son ancienne habitude
de lui être soumise. Elle ne voulut admettre le roi de Prusse dans la
triple alliance, dont il l'avait fait sonder, qu'autant que le roi
d'Angleterre le désirerait\,; et ce prince, voulant découvrir si le czar
y était reçu, le Pensionnaire répondit au ministre de Prusse que
l'alliance n'était qu'entre puissances voisines, pour maintenir l'amitié
et la sûreté commune, et ne regardait en aucune manière le czar\,;
qu'elle deviendrait trop universelle si elle s'étendait à des princes
éloignés, et que, par même raison, il serait étrange que le Danemark y
voulût entrer. La clarté de cette réponse enraya le ministre de Prusse
sur l'admission du czar, de peur de nuire à son maître.

Leurs ministres à Paris semblaient marcher fort de concert. Kniphausen,
qui avait la confiance du roi de Prusse, était venu de Hollande à Paris
relever Vireck. Schaffirof, vice-chancelier du czar, avait aussi la
sienne et l'accompagnait dans ses voyages. Ils convinrent que l'intérêt
commun de leurs maîtres était de bien examiner laquelle de l'alliance
avec l'empereur ou avec la France serait plus avantageuse\,; qu'avant de
s'engager avec la dernière, il fallait voir clairement si elle voulait
et pouvait faire sincèrement quelque chose de solide pour eux, sinon la
laisser et conclure un traité avec l'empereur, à condition qu'il
promettrait de n'user d'aucune voie de fait pour les forcer directement
ni indirectement à restituer les conquêtes qu'ils avaient faites, si,
comme ils ne l'espéraient pas, ils ne pouvaient l'engager à les leur
garantir. En attendant, ne rien faire qui pût le rebuter, entretenir
même de la confiance avec lui, dans la crainte des mesures que le roi de
Suède y pourrait prendre. Rien ne paraissait mieux cimenté que leur
union, et ils se promirent de s'avertir mutuellement de tout ce qu'ils
apprendraient.

Un ministre de l'empereur fit entendre, en ce temps-là, à celui de
Russie que, si la confiance s'établissait entre leurs maîtres,
l'empereur était disposé à étendre les traités\,; mais qu'il ne croyait
pas en devoir faire part au roi de Prusse que le traité ne fût bien
digéré, et même les préliminaires convenus. Quelque temps après,
Schaffirof remit à Kniphausen le projet d'un traité à faire entre leurs
maîtres. L'objet principal était d'empêcher que le roi de Danemark, qui
possédait alors la Poméranie antérieure, ne la remît entre les mains du
roi de Suède par une paix particulière, ou à quelque autre puissance
suspecte aux princes ligués. Ce projet avait sept articles.

\begin{enumerate}
\def\labelenumi{\arabic{enumi}.}
\item
  Renouveler les traités signés à l'occasion de la guerre du nord,
  particulièrement les conventions nouvellement faites entre leurs
  maîtres dans la conférence d'Havelsberg\,;
\item
  Donner l'attention nécessaire pour empêcher que le roi de Suède ou
  quelque prince d'intelligence avec lui reprît Stralsund et Wismar\,;
\item
  Promettre d'observer le traité fait avec le roi de Danemark, tant que
  ce prince l'observerait lui-même, et qu'il conserverait ce qu'il
  possédait dans la Poméranie antérieure en deçà de la rivière de
  Penne\,;
\item
  Engagement réciproque de secours mutuels pour s'opposer au roi de
  Danemark, s'il prétendait disposer, sans concert avec eux, des pays
  dont il est en possession\,;
\item
  Promesse du czar, pour satisfaire à cet engagement, de faire marcher
  les troupes qu'il avait dans le Mecklembourg, ou d'autres des plus
  prochains endroits, si elles en étaient sorties\,; qu'il les joindrait
  à celles du roi de Prusse\,; qu'elles agiraient conjointement pour
  chasser les Suédois ou autres puissances suspectes de l'île de Rugen
  et des autres lieux de la Poméranie antérieure, avec promesse du czar
  d'y employer encore des forces maritimes\,;
\item
  Le czar consentait aux démarches que le roi de Prusse jugerait à
  propos de faire, pour obtenir du roi de Danemark la cession de la
  Poméranie antérieure. Le czar promettait d'y contribuer de tout son
  pouvoir, et la chose faite, de se porter pour garant de cette cession
  pendant la guerre jusqu'à la paix, suivant ce qui avait été pratiqué à
  l'égard de Stettin\,;
\item
  Ils convenaient qu'après que Wismar serait rasé, il serait donné au
  duc de Mecklembourg une indemnité des pertes par lui souffertes du roi
  de Danemark, suivant la promesse du roi de Prusse à ce prince. Le czar
  et le roi de Prusse s'engageaient à procurer cette cession, lors de la
  paix avec la Suède, et à solliciter pour cet effet le consentement de
  l'empereur et de l'empire, et des alliés du nord, de ne pas permettre
  qu'il fût disposé autrement de Wismar, et, si on l'entreprenait, de
  s'y opposer avec le nombre de troupes qui serait jugé nécessaire.
\end{enumerate}

Ce dernier article fit tant de difficulté que Schaffirof céda. Il pria
seulement Kniphausen d'envoyer le projet au roi de Prusse, de faire ce
qu'il pourrait pour en obtenir son agrément, et l'assura que ce
changement n'empêcherait pas la signature du traité', pourvu qu'on y
voulût insérer qu'à l'égard de Wismar on s'en tiendrait à la déclaration
donnée à Stettin.

Le czar en même temps cherchait à traiter avec la France. L'article des
subsides qu'il demandait en faisait la difficulté principale. La
conduite de cette négociation sous ses yeux ne pouvait se cacher à ses
alliés alarmés des engagements qu'il pourrait prendre à leur préjudice.
Le roi de Pologne, qui avait un ministre à Paris, y en envoya un second,
pour y veiller encore mieux, pendant le séjour du czar. Schaffirof les
assura tous deux que le czar ne ferait jamais d'accommodement avec la
Suède, sans la participation du roi de Pologne\,; que les Français ne
lui avaient encore fait aucune proposition là-dessus, et n'en feraient
apparemment pas, avant d'avoir reçu des nouvelles du comte de La Marck,
leur ambassadeur auprès du roi de Suède, et qu'il ne s'était parlé
encore que d'un simple traité d'amitié. Il leur confia sous le secret
que, si la France proposait un traité d'alliance pareil à celui qu'elle
avait fait avec l'Angleterre et la Hollande, le czar y pourrait
consentir, mais à bonnes enseignes, et à condition qu'elle abandonnerait
la Suède. Il leur dit aussi qu'il n'avait tenu qu'à l'empereur de se
lier avec le czar\,; mais que, comme il avait répondu avec mépris,
quoique depuis il eût changé de ton, le czar pourrait aussi s'entendre
avec la France, s'il y trouvait son compte. Il ajouta que le czar avait
déjà la parole du roi de Prusse, qu'il souhaitait de trouver le roi de
Pologne dans les mêmes dispositions. Schaffirof les pria d'en écrire à
leur maître, et leur demanda le secret, et les assura que, si le traité
se concluait, il y serait laissé une porte ouverte au roi de Pologne
pour y entrer. Les envoyés de Pologne jugeaient le succès de la
négociation fort incertain à cause des garanties que le czar et la
Prusse ne manqueraient pas de demander\,; mais comme ils pouvaient se
tromper, leur but était de suspendre la négociation, s'il leur était
possible, jusqu'à ce qu'ils eussent des nouvelles de leur maître.

On prétend que Los, un des envoyés de Pologne, conseilla au roi son
maître d'engager, s'il pouvait, la France à lui faire des propositions,
parce que, si elles ne lui convenaient pas, elles lui serviraient à lui
faire un mérite auprès de l'empereur. Ce même Los suivait le czar
partout où il allait, en espion plutôt qu'en ministre.

L'empereur souffrait avec impatience le séjour des troupes russiennes
dans le Mecklembourg. L'envoyé de Prusse en informa le czar, en
adoucissant les termes forts des Impériaux. Les ministres du czar
avouèrent que, suivant les promesses du czar, elles en devaient sortir à
la fin d'avril\,; que cette prolongation portait plus de préjudice que
d'avantage à leur maître, et promirent de presser le czar là-dessus\,;
mais ils assurèrent que ce retardement n'était causé que par quelques
ombrages qu'il avait conçus des intérêts et de la conduite du roi
d'Angleterre à son égard. Une des raisons qui retenaient encore le czar
était sa propre sûreté. Il voulait avoir des troupes en Allemagne
pendant qu'il était hors de ses États, et à portée de se mettre à leur
tête quand il sortirait de l'empire.

Ses ministres étaient persuadés qu'il n'y avait rien à craindre de la
maison de Brunswick ni de l'empereur, malgré ses menaces, quoiqu'ils
sussent qu'il se proposait actuellement d'unir les forces des rois
d'Angleterre et de Danemark pour chasser les Moscovites du Mecklembourg.
Ils s'en plaignirent à un émissaire que le roi de Danemark avait envoyé
observer le czar à Paris, nommé Westphal. Ils lui reprochèrent que son
maître avait faussement publié que le czar prenait les intérêts du duc
de Holstein, et que c'était là-dessus que les Danois prenaient des
engagements contraires aux Moscovites, le menacèrent d'une rupture
ouverte si le Danemark faisait le moindre acte d'hostilité sous quelque
prétexte que ce fût. Ils nièrent aussi qu'il y eût aucune proposition de
mariage entre le duc de Holstein et la fille aînée du czar, comme le
bruit s'en était répandu, et qui s'accomplit depuis.

Ces plaintes étaient fondées. Il s'agissait alors à Vienne de former une
armée pour forcer les Moscovites à se retirer. L'empereur comptait sur
les troupes de Brunswick et de Danemark. Le roi d'Angleterre lui
promettait vingt-cinq mille hommes incessamment pour exécuter ses
ordres. Sur cette assurance, le projet était fait à Vienne d'intimer au
czar un terme fort court pour faire sortir ses troupes des terres de
l'empire\,; s'il refusait, le déclarer ennemi de l'empire et de tenir
une diète pour cela. Le roi d'Angleterre, comme directeur du cercle de
la basse Saxe, devait agir ensuite au nom de l'empereur et de l'empire
avec une armée composée des troupes de Danemark, Hanovre, Wolfenbuttel,
Gotha et Munster, et camper le 15 juin aux environs de Lauenbourg. Le
payement de ces troupes devait être pris sur les vingt-cinq mille livres
sterling accordées au roi d'Angleterre par son parlement.

Tandis que ces mesures se prenaient, dont le pape était très mal
informé, il pensait à faire une ligue entre l'empereur et le czar pour
la défense de la chrétienté, et il donna ordre à son nonce Bentivoglio,
à Paris, de travailler secrètement et prudemment à la former. Il avait
trouvé plusieurs exemples de ses prédécesseurs, de saint Pie V entre
autres, et d'Innocent XI, qui avaient écrit des brefs aux grands ducs de
Moscovie. Il résolut de les imiter, et il avertit Bentivoglio qu'il lui
en enverrait un incessamment à remettre à ce prince.

Albéroni, qui s'était plaint avec tant d'éclat, sous le nom du roi
d'Espagne, de la promotion de Borromée, comme vendu aux Allemands, et
comme une marque du pouvoir prédominant de l'empereur à Rome, prit un
ton tout différent en France, dans la crainte que cette couronne ne se
mît en prétention d'un chapeau, en équivalent. Il y devint l'avocat du
pape, soutint que le chapeau de Borromée n'était qu'une affaire de
famille indispensable depuis le mariage d'un neveu du pape avec la riche
nièce de ce prélat. Avec ces raisons, Cellamare eut ordre de représenter
au régent que sa prétention ne ferait que retarder inutilement celle
d'Albéroni, et il eut permission pour l'empêcher d'entrer en des
engagements avec la France. A. la vérité, il ne s'expliquait pas sur
quoi ni jusqu'où, apparemment pour avoir plus de liberté d'en désavouer
Cellamare. Il voyait une grande facilité à se servir de la flotte
promise au pape, pour ses vues particulières sur l'Italie, pendant la
guerre du Turc, qui liait les bras à l'empereur. Il comptait que la
France le laisserait faire, et l'Angleterre et la Hollande aussi, par
leur intérêt d'empêcher que Livourne tombât entre les mains de
l'empereur. Mais avant de tromper le pape sur l'usage de la flotte, dont
l'espérance du secours lui devait valoir le chapeau, il fallait le tenir
bien réellement, à quoi tout délai était empêchement dirimant pour le
chapeau et pour l'entreprise qu'il méditait par cette flotte. Telles
furent les véritables raisons du subit changement de conduite d'Albéroni
qui, après tant d'éclats et de menaces, chercha à se faire un mérite
auprès du pape de ce changement même, comme obtenu enfin par lui de
Leurs Majestés Catholiques, et de faire partir l'escadre, et de mander
Aldovrandi à la cour pour y terminer les différends entre les deux
cours, ce qui le porta à faire écrire le roi d'Espagne au pape avec des
engagements réitérés, sous la garantie du duc de Parme, pour emporter sa
promotion à ce coup, et être libre après de l'emploi de sa flotte, sans
avoir plus rien à ménager ni à craindre pour son chapeau.

Il avait envie de pénétrer le motif du voyage du czar ai Paris, ainsi
que toutes les autres puissances. Le comte de Konigseck, ambassadeur de
l'empereur, y était plus attentif qu'aucun des ministres étrangers. Il
pria Vireck, nouvellement rappelé à Berlin, de suivre le czar à
Fontainebleau, où Kniphausen, qui le relevait, alla aussi. Ils y virent
Ragotzi entrer en conférence avec le czar, et Ragotzi ne cacha point à
Kniphausen que les Turcs le pressaient de se rendre auprès d'eux, et que
son dessein était d'y aller.

Le prince Kurakin, étant à Rome pour la raison qui a été expliquée en
son lieu, avait fait espérer au pape que le czar accorderait le libre
exercice de la religion catholique dans ses États. Le pape crut que
Bentivoglio pourrait l'obtenir en parlant au czar ou à ses ministres,
mais il voulut que ce fût comme sans dessein qu'il en embarquât la
négociation, en parlant de cela à Kurakin, à propos de l'estime qu'il
s'était acquise à Rome. Les papes, en écrivant aux grands-ducs de
Moscovie, ne leur avaient jamais donné de Majesté. Celui-ci ne crut pas
devoir être arrêté par des bagatelles. Il énonça toutes les qualités que
le czar prenait, dans le bref qu'il lui écrivit, et qu'il adressa à
Bentivoglio pour le lui remettre, au cas qu'il reçût aussi la patente du
libre exercice de la religion catholique, à condition toutefois que ce
ne fût pas avec celle de la permission d'introduire le schisme grec dans
aucun pays catholique, ce qui aurait rendu l'affaire impossible.

Craignant aussi que le peu de temps qu'il restait au czar à demeurer à
Paris fût trop court pour la consommer, il voulut que Bentivoglio lui
fit agréer qu'il envoyât un ministre auprès de lui, avec ou sans
caractère. Mais il ne crut pas devoir traiter avec ce prince dans Paris,
sous les yeux du régent, sans l'informer de ce dont il s'agissait. Il
ordonna donc à son nonce de lui en rendre compte\,; mais de ne lui point
parler des ordres secrets qu'il lui avait envoyés de tâcher de lier le
czar avec l'empereur, pour faire la guerre aux Turcs. Le nonce s'adressa
donc au prince Kurakin, qui lui donna de bonnes paroles, et à qui il dit
qu'il avait un bref pour le czar, où toutes ses qualités étaient
énoncées. Il eut une audience de ce prince, mais sans parler d'affaires.

Kurakin lui avait dit que celle-là devait passer par Schaffirof, comme
vice-chancelier, parce qu'il s'agissait d'une expédition de
chancellerie. Kurakin lui dit aussi que les catholiques jouissaient
actuellement de cette liberté en Moscovie, où il y avait même déjà des
maisons de jésuites et de capucins établies à Moscou. Le nonce revit
Kurakin et Schaffirof\,; ce dernier lui dit les mêmes choses, et ajouta
que le czar voulait établir un couvent de capucins à Pétersbourg, qu'il
n'y serait de retour de plus de trois mois, qu'alors l'affaire se
pourrait finir à la satisfaction du pape, et que le ministre que le pape
enverrait prendrait alors caractère, pourvu que ce fût un homme de
distinction.

Sur la ligue, Bentivoglio avait cru toucher les Russes par la facilité
de reprendre Azoff pendant la guerre de Hongrie, mais Kurakin lui fit
voir par de bonnes raisons combien cette place leur était indifférente.
Il dit pourtant au nonce dans une autre conversation que, dès que le
czar serait délivré de la guerre de Suède, il se lierait non seulement
avec l'empereur, mais avec les Vénitiens, enfin avec le pape, parce
qu'il voulait être bien avec lui.

En effet, le czar avait dit au maréchal de Tessé qu'il ne s'éloignerait
pas de reconnaître le pape pour premier patriarche orthodoxe, mais aussi
qu'il ne s'accommoderait pas de certains assujettissements que la cour
de Rome prétendait imposer aux princes, au préjudice de leur
souveraineté\,; qu'il voulait bien croire le pape infaillible, mais à la
tête du concile général. C'est que la vérité et la raison sont de tous
pays, et ce monarque, presque encore barbare, nous faisait une
excellente leçon.

La guerre subsistait toujours entre l'empereur et le roi d'Espagne\,;
mais l'éloignement des États suspendait naturellement les actes
d'hostilité. Ils étaient de plus interdits en Italie par le traité de
neutralité d'Utrecht. Molinez, grand inquisiteur d'Espagne, voulant s'y
rendre de Rome, prit néanmoins des passeports du pape pour plus de
sûreté, et Paulucci prit encore assurance de Schrottembach, cardinal,
chargé des affaires de l'empereur, en absence d'ambassadeur. Avec ces
précautions, Molinez partit de Rome à la fin de mai, et ne laissa pas
d'être arrêté à Milan, par ordre du prince de Loewenstein, gouverneur
général du Milanais, qui était frère de M\textsuperscript{me} de
Dangeau. Sur cette nouvelle, le cardinal Acquaviva alla trouver le
cardinal Albane, qui, en l'absence du cardinal Paulucci, faisait la
charge de secrétaire d'État que son oncle lui destinait, à qui il fit
ses plaintes, insistant sur le mépris des passeports du pape. Albane
répondit que Schrottembach improuvait cette violence, et que le pape
ferait ce qu'il voudrait. Sur cette assurance, Acquaviva alla au pape, à
qui il proposa d'en faire son affaire particulière, et d'en obtenir
réparation, ou de la laisser démêler au roi d'Espagne. Si le pape s'en
chargeait, il fallait réclamer Molinez comme ecclésiastique et comme
officier intime, principal et immédiat du saint-siège\,; ne s'amuser
point à dépêcher inutilement des courriers à Vienne, mais parler haut,
et marquer dans Rome combien il était blessé de la mauvaise foi des
Allemands\,; le déclarer lui-même aux ministres de l'empereur, ou leur
refuser toute audience, jusqu'à ce qu'il eût reçu toute satisfaction, et
que Molinez fût en liberté. Si, au contraire, Sa Sainteté voulait
laisser au roi d'Espagne le soin de se venger de la mauvaise foi des
Allemands, Acquaviva protestait que ce monarque, regardant cet incident
comme une infraction manifeste à la neutralité d'Italie, emploierait les
vaisseaux qu'il avait actuellement en mer à tirer raison de la violation
des traités.

Il semblait que le pontificat de Clément XI fût destiné aux événements
capables de l'embarrasser. Ils s'accumulaient\,; chaque jour en
produisait un nouveau dont il ne pouvait se démêler. Il était plus
susceptible qu'aucun de ses prédécesseurs, de frayeur, d'agitation et de
trouble, et plus incapable que personne du monde de se décider et de
sortir d'embarras. Il mécontentait ordinairement tous ceux dont il
n'avait point affaire\,; il traitait avec hauteur ceux dont il croyait
n'avoir rien à craindre\,; il se comportait avec tant de bassesse et de
timidité à l'égard de ceux dont il appréhendait la puissance, qu'ils ne
lui savaient aucun gré de ce qu'ils en arrachaient par force et par
terreur. Il croyait exceller à écrire en latin et à composer des
homélies et des brefs. Il y perdait beaucoup de temps. Il était sans
cesse tiraillé dans son intérieur domestique. Son incertitude, ses
variations, sa faiblesse avait ôté toute confiance en ses paroles. Des
cardinaux hardis, comme Fabroni et d'autres, hasardaient sous son nom
quelquefois ce qu'il leur plaisait, et ne le lui disaient que quand les
choses étaient faites. Il était désolé, mais il n'osait les défaire. Les
larmes, dont il avait une source et une facilité abondante, étaient sa
ressource dans tous ses embarras\,; mais elles ne l'en tiraient pas. Au
fond, un très bon homme et honnête homme, doux, droit et pieux, s'il fût
resté particulier sans affaires.

Effrayé au dernier point de la dernière partie du discours d'Acquaviva,
il s'écria qu'il fallait bien se garder de prendre une voie si
dangereuse\,; qu'il allait dépêcher de vives plaintes à Vienne\,; qu'il
ne perdrait point de vue cette affaire, qu'il avait si bien regardée
comme la sienne, avant qu'Acquaviva lui en eût parlé, qu'il lui montrât
la réponse qu'il faisait à l'archevêque de Milan qui lui avait écrit
qu'il avait inutilement demandé au gouverneur général du Milanais de
remettre Molinez à sa garde (car il faut remarquer que l'immunité
ecclésiastique se mêle de tout et entre dans tout). Mais au fond, la
détention de Molinez occupait peu ceux qui devaient y être les plus
sensibles. La promotion d'Albéroni était l'affaire unique que le pape
voulait éluder, malgré tant de paroles positives, et malgré le départ
tant désiré de l'escadre espagnole. Il craignait de déplaire à
l'empereur, de révolter Rome et le sacré collège\,; il cherchait des
délais, malgré la dernière lettre du roi d'Espagne et la garantie du duc
de Parme. Il voulait que les différends avec l'Espagne fussent
accommodés à son gré auparavant.

Albéroni ne se découragea point, et comme le pape se défendait sur
l'équivalent du chapeau d'Albéroni, que les couronnes pourraient lui
demander, si un motif public comme l'accommodement à son gré n'en était
une raison à leur fermer la bouche, Albéroni commença par obtenir une
lettre du régent au cardinal de La Trémoille, par laquelle il lui
mandait de suspendre toute demande capable de traverser sa promotion, et
il se proposa de terminer au gré du pape les différends entre les deux
cours, dès qu'Aldovrandi serait arrivé, qu'il attendait avec impatience.

Dans cette situation personnelle, il n'avait garde de déranger le bon
état de son affaire, en laissant donner par le roi d'Espagne des marques
de ressentiment de l'arrêt de la personne de Molinez\,; il n'avait nulle
estime pour lui, et l'appelait ordinairement \emph{solemnissima bestia}.
Il disait qu'il méritait bien cette aventure, qu'il demeurerait
longtemps au château de Milan s'il en était cru, et qu'il ne valait pas
la peine de déranger les projets de l'escadre pour la délivrance de cet
oracle des Espagnols. En même temps il se vantait de ce qu'il avait fait
et prétendait faire pour le service du roi d'Espagne. Il disait qu'il
avait armé trente vaisseaux en moins de huit mois, envoyé six cent mille
écus à la Havane, pour employer en tabac qui serait vendu en Europe au
profit du roi\,; employé cent cinquante mille écus en achats de
provisions pour la marine, cent quatre-vingt mille écus en bronze pour
l'artillerie, dont les places étaient dépourvues, et cent vingt mille
pistoles pour la citadelle de Barcelone. Enfin, ajoutait-il, l'Espagne
n'en avait pas tant fait en trois siècles, et ne l'eût pu faire encore
s'il eût laissé répandre et distribuer l'argent comme par le passé. À
l'avenir il voulait établir une marine, régler les finances de manière
que les troupes fussent bien payées, {[}et{]} un fonds sûr pour le
payement des maisons royales, en sorte que les rois ne vivraient plus
dans la misère de leurs prédécesseurs. Il voulait encore des troupes
étrangères, et persistait à demander au roi d'Angleterre la permission
de lever dans ses États des Anglais ou des Irlandais. L'Angleterre, de
son côté, et la Hollande aussi, le pressaient d'un règlement sur le
commerce de Cadix. Patiño était chargé d'assembler là-dessus chez lui
les marchands de toutes les nations, et son occupation de l'escadre
servait d'excuses aux délais.

Le roi d'Espagne eut des évanouissements qui firent craindre pour les
suites. On en accusa l'air de Ségovie où il était depuis quelque temps.
Il voulut aller à l'Escurial. On n'a point su pourquoi la reine s'y
opposa fortement\,; mais le roi lui parla avec tant de hauteur,
qu'étourdie d'un langage si inusité pour elle, elle n'osa hasarder une
résistance, pour conserver son pouvoir despotique dans les choses
importantes. Ainsi on fut à l'Escurial.

Aldovrandi y arriva le 10 juin, et y fit la jalousie des ministres
étrangers par les distinctions qu'il y reçut, et qui montrèrent
qu'Albéroni ne connaissait d'autre affaire que celle de sa promotion, et
qu'il était inutile de lui parler d'aucune autre. Lui et Aubenton, en
bons serviteurs du pape, se mirent à disposer avec le nonce les affaires
à une heureuse fin. Ils lui conseillèrent d'attendre qu'elles fussent
comme conclues avant de voir Leurs Majestés Catholiques, et il se
conforma à leurs désirs. Il louait sans cesse Albéroni sur l'escadre, et
ce dernier se plaignait du pape avec un modeste mépris. En même temps il
rassura Cellamare sur la continuation de son amitié, quoi que pût dire
et faire contre lui à Rome son oncle le cardinal del Giudice, qui allait
y arriver.

On laissait dormir depuis quelque temps la négociation de la paix entre
l'empereur et l'Espagne, lorsque Widword, envoyé d'Angleterre en
Hollande, alla trouver Beretti, lui dire par ordre de Sunderland,
nouveau secrétaire d'État, que le roi d'Angleterre avait dépêché un
courrier à l'empereur pour l'obliger enfin à déclarer s'il voulait
traiter la paix avec le roi d'Espagne\,; que ces instances se faisaient
de concert avec la France\,; que lorsqu'il en serait temps, les états
généraux seraient invités de prendre part à la négociation comme
médiateurs et comme arbitres. Beretti, qui n'avait point d'ordre, et qui
n'avait pas d'opinion du succès de cette démarche, n'oublia rien pour
donner de la crainte à cet envoyé, des négociations secrètes du roi de
Sicile avec l'empereur, de la mauvaise foi des Autrichiens, de
l'ambition et de la puissance de leur maître.

L'Angleterre, en effet, n'était guère en état de se mêler beaucoup du
dehors par les embarras du dedans. Le prince de Galles cabalait
ouvertement contre le roi son père, et faisait porter contre Cadogan des
accusations au parlement. Tout y était en mouvement sur celles du comte
d'Oxford, prêtes à être jugées. Les ennemis de la cour, qui faisaient le
plus grand nombre, étaient affligés de son union avec le régent, qui
obtint enfin du czar, si pressé d'ailleurs, la sortie des troupes du
pays de Mecklembourg, et des assurances de témoignages d'amitié pour le
roi d'Angleterre qui, non plus que ses ministres, n'y compta guère, mais
qui le ménageait pour tâcher d'effacer les sujets qu'il lui avait donnés
de mécontentement et de plaintes.

Ils en étaient d'autant plus inquiets que le czar avait été voir la
reine douairière d'Angleterre, et avait paru touché de son état et de
celui du roi Jacques son fils. Les suites que cette compassion pouvait
avoir alarmèrent Stairs. Il prit une audience du czar, à qui il dit
merveilles de l'estime et des intentions du roi d'Angleterre à son
égard. Il vit après Schaffirof avec les mêmes protestations, et lui
parla des troupes du Mecklembourg. Schaffirof se contenta de lui
répondre qu'il en rendrait compte au czar, sans lui montrer que la
résolution de la sortie de ces troupes était prise et l'ordre envoyé. Il
conseilla à son maître de se faire un mérite auprès du roi d'Angleterre
d'une affaire faite. Le czar le crut, et Schaffirof écrivit en
conséquence à Stairs. Schaffirof avertit aussi l'envoyé de Prusse de
l'ordre envoyé à ces troupes. Ainsi ils eurent l'adresse de faire valoir
au régent et au roi d'Angleterre l'exécution d'une résolution, que la
crainte de se voir tomber une puissante armée sur les bras ne leur avait
plus permis de différer.

En même temps le roi de Danemark s'inquiétait de ce qu'on ne parlait
point d'attaquer la Suède\,; il craignait d'en être attaqué lui-même en
Norvège. Il demandait au czar une diversion qui l'en mît à l'abri. Le
czar, peu content de ce prince, éluda ses demandes. Il répondit qu'il
n'était pas en état de rien entreprendre contre la Suède sans le secours
de vaisseaux que l'Angleterre et le Danemark lui avaient promis\,; que
d'ailleurs le roi d'Angleterre était seul, et sans lui assez puissant
pour garantir les États du roi de Danemark d'une invasion des Suédois,
et lui procurer `une paix avantageuse. Les Danois, qui entendirent bien
la signification de cette réponse, étaient, ainsi que les envoyés de
Pologne, extrêmement inquiets de ce que le czar traitait avec le régent.
Ils se relayaient autour de ce monarque, et se communiquaient tout ce
qu'ils pouvaient apprendre. Il partit enfin de Paris sans qu'ils fussent
éclaircis de rien. Mais Schaffirof, qui y demeura quelques jours après
lui, confia sous le dernier secret à un des agents du roi de Pologne
tout ce qui s'était passé dans la négociation avec la France, et que le
traité aurait été conclu si l'envoyé de Prusse n'en eût pas arrêté la
signature. Il ajouta que le principal but du czar, en prenant avec la
France des engagements apparents, qui dans le fond ne l'obligeaient à
rien, avait été de brouiller la France avec la Suède\,; qu'une
convention vague d'assistance générale était si aisée à éluder qu'il
était persuadé qu'elle ne pouvait blesser l'empereur, qui en sentirait
aisément le peu de solidité\,; que sur ce fondement ils en presseraient
la conclusion\,; et s'ouvrant tout à fait, il avoua qu'il la désirait
par l'espérance des présents aux ministres qui font la signature, et se
plaignit amèrement du mauvais procédé de la cour de Berlin qui l'avait
retardée, et qu'il dit être connue de tout le monde pour être légère, et
sans principes ni suite dans ses résolutions.

Schaffirof ne disait pas tout. La Suède, bien moins que l'Angleterre,
avait été la pierre d'achoppement. La Suède était trop abattue pour
faire ombrage à la Russie. D'ailleurs le czar, qui avait beaucoup de
grand, n'avait pu refuser son estime au roi de Suède. Content de l'avoir
réduit dans l'état où il se trouvait, il ne voulait pas l'accabler, mais
il cherchait, au contraire, à s'en faire un ami. Il ne voulait pas moins
conserver ses conquêtes. Ce but s'accordait parfaitement avec sa haine
pour le roi d'Angleterre, et avec son mécontentement du Danemark. Il
cherchait donc les moyens de les obliger à restituer ce qu'ils avaient
pris ou usurpé sur la Suède, à s'en faire un mérite auprès d'elle, en
conservant ce qu'il lui avait pris. Mais il trouva l'Angleterre si
absolue dans le cabinet du régent, qu'il perdit bientôt toute espérance
de faire restituer par aucun moyen Brême et Verden enlevés à la Suède en
pleine paix par les Hanovriens, dans les temps les plus calamiteux de la
Suède.

Le czar avait un autre embarras avec l'empereur, qui l'obligeait à le
ménager. Le czarowitz, dont la tragique histoire est entre les mains de
tout le monde, s'était sauvé de Russie pendant l'absence du czar, et
s'était réfugié à Vienne. L'empereur l'avait promptement fait passer à
Naples, où il n'avait pu être si bien caché que le czar n'en fût
informé. Il demandait à l'empereur de le lui remettre entre les mains.
Quoique l'empereur n'eût pas lieu de s'intéresser beaucoup au sort d'un
prince qui, ayant épousé la soeur de l'impératrice sa femme, l'avait
tuée, grosse, d'un coup de pied dans le ventre, sans autre cause que sa
férocité, l'empereur ne laissait pas de faire beaucoup de difficultés de
rendre un prince qui s'était jeté entre ses bras, comme dans son unique
asile, à un père aussi irrité qu'était le czar, qui adorait la czarine,
belle-mère de ce prince, et qui en avait un fils qu'il préférait à cet
aîné fugitif pour lui succéder. Le roi de Prusse, de son côté, se
plaignait, dans la défiance qu'il avait de ses alliés, que la France ne
pressait pas assez la paix entre la Suède et lui, et menaçait que, si
elle n'était faite avant la fin de la guerre de Hongrie, la ligue du
nord se jetterait entre les bras de l'empereur, dont elle achèterait
l'appui tout ce qu'il le lui voudrait vendre. Ces plaintes étaient
injustes. Le régent n'oubliait rien pour calmer les troubles du nord. Il
avait disposé le roi d'Angleterre à relâcher le comte de Gyllembourg,
dès que le roi de Suède eut désavoué ses ministres, et déclaré qu'ils
avaient agi sans sa participation. La détention du baron de Goertz, en
Hollande, apportait un obstacle à la conclusion de cette affaire. Le roi
d'Angleterre le regardait comme un ennemi dangereux, et tâchait de
prolonger sa prison. Elle faisait tort au commerce des Hollandais dans
le nord, et ils se lassaient d'être les geôliers du roi d'Angleterre.
Ses ministres en Hollande ne se sentant pas assez forts pour persuader
la république contre ses intérêts, voulaient s'appuyer auprès d'elle de
l'appui du régent, des amis duquel ils sentaient tout le poids auprès
d'elle. Cette étroite intelligence entre le roi d'Angleterre et le
régent était un des moyens dont le nonce Bentivoglio se servait le plus
pour décrier à Rome le régent, qui sacrifiait, disait-il, la religion
pour s'appuyer des protestants\,; car tout était bon à ce furieux pour
mettre le feu du schisme, de l'interdit, de la guerre civile, s'il eût
pu, en France, dans la folle persuasion que cela seul le ferait
subitement cardinal. Il gémissait amèrement sur le jugement rendu entre
les princes du sang et les bâtards. Leur privation de l'habilité de
succéder à la couronne était l'ouvrage des jansénistes, et le plus
funeste coup porté à la religion. Il désirait ardemment et il espérait
des conjonctures funestes au gouvernement, qui donneraient lieu à leur
rétablissement. Pourrait-on imaginer que des propos si diamétralement
contraires à l'Évangile sortissent de la bouche d'un archevêque,
représentant le pape, écrivant à Rome\,? Mais sa vie publique répondait
à ses discours, et les désordres effrénés de la sienne étaient
l'approbation signalée des ombres qui se remarquent dans la vie du feu
roi.

Le Prétendant était alors à Rome, où le pape avait pour lui tous les
égards et les distinctions qu'il devait, mais qui, à vingt mille écus
près qu'il lui donna, n'allaient qu'à des honneurs et à des compliments
pour lui et pour la reine sa mère. Il n'espérait d'assistance que de
l'Espagne. Il voulut donc flatter Albéroni, et dans une audience qu'il
eut du pape, il le pressa sur sa promotion. Le pape lui répondit
seulement qu'il attendait un projet d'édit du roi d'Espagne
qu'Aldovrandi devait lui envoyer\,; mais après l'audience il lui en fit
faire un reproche tendre par son neveu don Alexandre, et {[}le fit{]}
avertir en même temps de se garder de ceux qui ne lui donnaient de ces
sortes de conseils que pour le trahir. Le pape, à l'occasion du premier
consistoire, en parla au cardinal Gualtieri, qui fit si bien comprendre
la nécessité où se trouvait ce malheureux prince que le pape se repentit
de ce qu'il lui avait fait dire, chose qui lui arrivait souvent après
ses démarches.

Acquaviva, à qui le Prétendant avait fort recommandé Castel-Blanco, qui
lui avait rendu de grands services, lui avait dit ce qui s'était passé
entre le pape et lui sur Albéroni. Il réfléchit sur cet édit attendu
d'Espagne, dont jusque-là le pape n'avait pas dit un mot. Il en inféra
qu'il y voulait trouver occasion de délais, pour laisser vaquer
plusieurs chapeaux, et en contenter à la fois l'Espagne et les autres
couronnes qui auraient à se plaindre d'un chapeau seul donné à Albéroni,
et ce soupçon était très conforme au caractère du pape. Sa Sainteté
faisait presser le roi d'Espagne de finir au plus tôt les affaires de la
nonciature de Madrid. Si elles étaient terminées avant la promotion, il
se proposait de dire au consistoire qu'il y avait plus de gloire pour
lui de faire cardinal celui qui avait tant contribué au bien du
saint-siège, que pour le sujet même qu'il élevait à la pourpre. C'était
par là qu'il se préparait à se défendre contre les plaintes, et {[}à{]}
imposer silence aux prétentions des couronnes sur des chapeaux en
équivalent de celui-là. Acquaviva ne se fiait ni à ces propos ni aux
promesses du prélat Alamanni, qui répondait de la promotion, même avant
que le tribunal de la nonciature fût rouvert à Madrid, si le roi
d'Espagne persistait à la demander.

Le pape avait écrit au roi d'Espagne et au duc de Parme comme des
excuses sur la promotion de Borromée, et de nouvelles promesses de celle
d'Albéroni, dont il voulait leur persuader que le délai ne roulait point
sur là défiance de l'exécution des paroles du roi d'Espagne, et fit
encore {[}écrire{]} par le cardinal Paulucci au P. Daubenton, son plus
fidèle agent, pour presser le roi d'Espagne de finir tous les points à
la satisfaction du pape avant la promotion. Cette lettre était pleine de
tout ce qu'on y put mettre de raisons d'une part, et de témoignages
d'estime, d'affection et de confiance, de l'autre, pour le jésuite.

Ces lettres étant demeurées sans effet jusqu'à l'arrivée d'Aldovrandi à
l'Escurial, le pape redoubla de promesses que, sitôt que les différends
seraient terminés à sa satisfaction, il ferait la promotion sans
attendre de vacances. Il se plaignait qu'elle serait faite depuis deux
mois si le roi d'Espagne ne les avait perdus en plaintes inutiles sur
celle de Borromée, et à tenir Aldovrandi à Perpignan\,; enfin qu'il
était nécessaire qu'il pût annoncer au consistoire que la nonciature
était rouverte, le nonce en possession de toutes ses anciennes
prérogatives, que les nouveautés contraires à l'ancienne juridiction
ecclésiastique étaient abolies, la flotte à la voile pour le secours de
l'Italie et de la chrétienté, et qu'Albéroni avait été le ministre
auprès du roi d'Espagne de toutes ces grandes choses. Le pape, qui
sentait tout le parti qu'il pouvait tirer de l'excès de l'ambition
d'Albéroni, et de l'excès aussi de son pouvoir sur l'esprit du roi et de
la reine d'Espagne, manda à Aldovrandi que, s'il ne pouvait obtenir
l'ouverture de sa nonciature avant que la promotion d'Albéroni fût faite
et déclarée, il le trouvait bon, mais à cette condition que le décret
que le roi d'Espagne devait publier, suivant la minute jointe à ses
instructions, fût signé avant la promotion sans aucune variation, et
qu'il en fût remis un exemplaire authentique entre les mains
d'Aldovrandi pour le lui envoyer. Il voulait, de plus, recevoir par le
duc de Parme des assurances précises de l'ouverture du tribunal de la
nonciature après immédiatement la nouvelle de la promotion, et d'une
pleine et entière satisfaction suivant les instructions qu'il avait
données à son nonce, qu'il avait chargé, de plus, d'obtenir
l'éloignement de quelques personnes notées à la cour de Rome\,: salaire
trop accoutumé de la fidélité et de la capacité de ceux qui ont le mieux
servi les rois contre les entreprises de cette dangereuse et implacable
cour.

Malgré tant de dispositions apparentes, on soupçonnait encore le pape de
vouloir se préparer des délais, dans la crainte où il était du
ressentiment de l'empereur. La flotte d'Espagne, si désirée du pape,
partit enfin de Cadix, composée de douze vaisseaux de guerre, un pour
hôpital, un pour les magasins, et deux brûlots. Albéroni flattait
toujours le pape qu'elle prenait le plus court chemin du Levant, sans
toucher aux côtes d'Italie, pour abréger de cent lieues. Albéroni, à ce
qu'on a cru depuis, avait averti le duc de Parme de la véritable
destination de la flotte. Il l'avertit aussi d'éviter tout commerce avec
les correspondants du Prétendant, dont la maison était toujours remplie
de fripons et de traîtres, et duquel il blâmait le voyage de Rome comme
une curiosité dévote qui ne serait pas applaudie en Angleterre. En même
temps Albéroni, voulant tout mettre à profit pour plaire au pape dans
cette crise de sa promotion, le pressait de se faire obéir en France par
quelque coup d'éclat sur la constitution.

Giudice, arrivé à Rome, y fut d'abord sèchement visité par Acquaviva\,;
on le soupçonnait de se vouloir donner à l'empereur. Il était accusé
d'en avoir fort avancé le traité, en 1714, avec le comte de Lamberg,
ambassadeur de l'empereur, et de l'avoir brusquement rompu, lorsque la
princesse des Ursins fut chassée et qu'il fut rappelé en Espagne.
Lamberg même ne le nommait plus depuis que le double traître. Il avait
vu, en passant à Turin, le roi de Sicile, qui ne s'était ouvert en rien
sur quoi que ce soit avec lui, et ne lui avait parlé que de choses
passées. Ses différends avec Rome étaient pour lors en assez grand
mouvement, et le pape lui avait fait une réponse extrêmement captieuse,
et pleine des plus grands ménagements pour l'empereur. Giudice donc ne
put rapporter aucune considération de son passage à Turin. Étant à
Gênes, il avait voulu visiter la princesse des Ursins, qui l'avait
crûment refusé, sous prétexte de son respect pour le roi d'Espagne, qui
ne lui permettait pas de voir personne qui fût dans sa disgrâce. La
Trémoille fut moins réservé que sa sueur, qu'il n'aimait guère, ni elle
lui. Il était depuis longtemps ami de Giudice, il le vit souvent, et
avec une confiance fort déplacée avec un homme moins franc et plus rusé
que lui, sur un mauvais pied à Rome, et d'une réputation peu entière.

La cour de Rome est pleine de gens, et du plus haut rang, qui font
métier d'apprendre tout ce qu'ils peuvent, et d'en profiter. On
prétendit que le cardinal Ottobon ne s'oublia pas, dans ce qu'il sut
démêler de ces deux cardinaux, pour gagner la confiance du roi d'Espagne
et se réconcilier l'empereur. Il s'empressait pour la promotion
d'Albéroni pendant qu'il faisait tous ses efforts pour effacer les
soupçons de la cour de Vienne, et retirer par ce moyen une partie des
revenus de ses bénéfices situés dans l'État de Milan, que les Allemands
avaient confisqués.

Un chiaous, dépêché par le Grand Seigneur, arriva en France et m'y
ramènera en même temps. La Porte voulait savoir des nouvelles du
gouvernement de France depuis la mort du roi, dans le dessein de vivre
toujours bien avec elle. Elle voulait aussi exciter des mouvements en
Transylvanie, et proposer des partis avantageux à Ragotzi pour y
retourner.

La vie qu'il menait, surtout depuis la mort du roi, ne répondait guère à
une pareille proposition. Il s'était aussitôt après tout à fait retiré
dans une maison qu'il avait prise dès auparavant, et où il allait
quelquefois, aux Camaldules de Grosbois. Il y avait peu de domestiques,
n'y voyait presque personne, vivait très frugalement dans une grande
pénitence, au pain et à l'eau une ou deux fois la semaine, et assidu à
tous les offices du jour et de la nuit. Presque plus à Paris, où il ne
voyait que Dangeau, le maréchal de Tessé et deux ou trois autres amis\,;
M. le comte de Toulouse, avec qui, deux ou trois fois l'année, il allait
faire quelques, chasses à Fontainebleau\,; le roi et le régent,
uniquement par devoir et de fort loin à loin\,; d'ailleurs beaucoup de
bonnes oeuvres, mais toujours fort informé de ce qui se passait en
Transylvanie, en Hongrie et dans les pays voisins\,; avec cela,
sincèrement retiré, pieux et pénitent, et charmé de sa vie solitaire,
sans ennui et sans recherche d'aucun amusement ni d'aucune dissipation,
et jouissant toujours de tout ce qu'on a vu en son temps que le feu roi
lui avait donné.

\hypertarget{chapitre-ii.}{%
\chapter{CHAPITRE II.}\label{chapitre-ii.}}

1717

~

{\textsc{Le général et l'intendant de nos îles paquetés et renvoyés en
France par les habitants de la Martinique.}} {\textsc{- Mort de la
duchesse de La Trémoille\,; du fils unique du maréchal de Montesquiou\,;
de Busenval\,; d'Harlay, conseiller d'État.}} {\textsc{- Caractère et
singularités de ce dernier.}} {\textsc{- Mort de Dongois, greffier en
chef du parlement.}} {\textsc{- Mort et deuil d'un fils du prince de
Conti.}} {\textsc{- Affaire de Courson, intendant de Bordeaux et
conseiller d'État, et de la ville, etc., de Périgueux.}} {\textsc{-
Courson, cause de la chute de des Forts, son beau-frère\,; et seul
coupable, se soutient.}} {\textsc{- Le maréchal de Tallard entre au
conseil de régence.}} {\textsc{- Question de préséance entre le maréchal
d'Estrées et lui, jugée en sa faveur.}} {\textsc{- Son aventure au même
conseil.}} {\textsc{- Duc d'Albert gouverneur d'Auvergne.}} {\textsc{-
Maréchal de Tessé quitte le conseil de marine.}} {\textsc{- Grâces
accordées aux conseillers du grand conseil.}} {\textsc{- Le roi
Stanislas près d'être enlevé aux Deux-Ponts\,; quelque temps après reçu
en asile à Weissembourg en basse Alsace.}} {\textsc{- Naissance du
prince de Conti et d'un fils du roi de Portugal.}} {\textsc{- Fête
donnée par son ambassadeur.}} {\textsc{- La Forêt\,; quel\,; perd un
procès de suite importante.}} {\textsc{- Le régent assiste, à la royale,
à la procession de Notre-Dame, le 15 août.}} {\textsc{- Le parlement
refuse d'enregistrer la création de deux charges dans les bâtiments.}}
{\textsc{- Fête de Saint-Louis.}} {\textsc{- Rare leçon du maréchal de
Villeroy.}}

~

Il arriva à la Martinique une chose si singulière et si bien concertée
qu'elle peut être dite sans exemple. Varennes y avait succédé à
Phélypeaux, qui avait été ambassadeur à Turin, et comme lui était
capitaine général de nos îles. Ricouart y était intendant. Ils vivaient
à la Martinique dans une grande union, et y faisaient très bien leurs
affaires. Les habitants en étaient fort maltraités. Ils se plaignirent à
diverses reprises et toujours inutilement. Poussés à bout enfin de leur
tyrannie et de leurs pillages et hors d'espérance d'en avoir justice,
ils résolurent de se la faire eux-mêmes. Rien de si sagement concerté,
de plus secrètement conduit parmi cette multitude, ni de plus doucement
ni de plus plaisamment exécuté. Ils les surprirent un matin chacun chez
eux au même moment, les paquetèrent, scellèrent tous leurs papiers et
leurs effets, n'en détournèrent aucun, ne firent mal à pas un de leurs
domestiques, les jetèrent dans un vaisseau qui était là de hasard prêt à
partir pour la France, et tout de suite le firent mettre à la voile. Ils
chargèrent en même temps le capitaine d'un paquet pour la cour dans
lequel ils protestèrent de leur fidélité et de leur obéissance,
demandèrent pardon de ce qu'ils faisaient, firent souvenir de tant de
plaintes inutiles qu'ils avaient faites, et s'excusèrent sur la
nécessité inévitable où les mettait l'impossibilité absolue de souffrir
davantage la cruauté de leurs vexations. On aurait peine, je crois, à
représenter l'étonnement de ces deux maîtres des îles de se voir
emballés de la sorte, et partis en un clin d'oeil, leur rage en chemin,
leur honte à leur arrivée.

La conduite des insulaires ne put être approuvée dans la surprise
qu'elle causa, ni blâmée par ce qui parut du motif extrême de leur
entreprise, dont le secret et la modération se firent admirer. Leur
conduite, en attendant un autre capitaine général et un autre intendant,
fut si soumise et si tranquille, qu'on ne put s'empêcher de la louer.
Varennes et Ricouart n'osèrent plus se montrer après les premières fois,
et demeurèrent pour toujours sans emplois. On murmura fort avec raison
qu'ils en fussent quittes à si bon marché. En renvoyant leurs
successeurs à la Martinique, pour qui ce fut une bonne leçon, on
n'envoya point de réprimande aux habitants par la honte tacite de ne les
avoir pas écoutés et de les avoir réduits par là à la nécessité de se
délivrer eux-mêmes.

Le maréchal de Montesquiou perdit son fils unique, et la marquise de
Gesvres mourut, dont on a vu en son temps l'étrange procès avec son
mari. Le vieux et très ennuyeux Busenval mourut aussi fort pauvre,
lieutenant général, ayant été premier sous-lieutenant des gens d'armes
de la garde. La duchesse de La Trémoille mourut aussi fort jeune et fort
jolie, mais peu heureuse, ne laissant qu'un fils unique. Elle était fort
riche et de grande naissance, Mottier de La Fayette, et héritière de son
père mort lieutenant général, et de sa mère, fille de Marillac, doyen du
conseil, qui avait perdu ses deux fils sans enfants, en sorte que
M\textsuperscript{me} de La Fayette était demeurée seule héritière.

En même temps mourut un homme avec l'acclamation publique d'en être
délivré, quoiqu'il ne fût pas en place ni en passe de faire ni bien ni
mal, étant conseiller d'État sans nulle commission extraordinaire. Ce
fut Harlay, fils unique du feu premier président, digne d'être le fléau
de son père, comme son père d'être le sien, et comme ils se le firent
sentir toute leur vie, sans toutefois s'être jamais séparés
d'habitation. On a vu en son lieu quel était le père. Le fils, avec bien
moins d'esprit et une ambition démesurée nourrie par la plus folle
vanité, avait un esprit méchant, guindé, pédant, précieux, qui voulait
primer partout, qui courait également après les sentences qui toutefois
ne coulaient pas de source, et les bons mots de son père, qu'il
rappelait tristement. C'était le plus étrange composé de l'austère
écorce de l'ancienne magistrature et du petit maître de ces temps-ci,
avec tous les dégoûts de l'un et tous les ridicules de l'autre. Son ton
de voix, sa démarche, son attitude, tout était d'un mauvais comédien
forcé\,; gros joueur par air, chasseur par faste, magnifique en singe de
grand seigneur. Il se ruina autant qu'il le put avec un extérieur
austère, un fond triste et sombre, une humeur insupportable, et pourtant
aussi parfaitement débauché et aussi ouvertement qu'un jeune
académiste\footnote{Ce mot était employé, aux XVIIe et XVIIIe siècles,
  pour désigner les jeunes gens qui suivaient des écoles, appelées
  académies, où l'on enseignait l'équitation. M\textsuperscript{me} de
  Motteville, à l'année 1645, parlant de l'entrée des ambassadeurs de
  Pologne à Paris, dit\,: «\,Après eux venaient nos
  \emph{académistes}.\,» Saint-Évremond a employé le mot
  \emph{académistes} dans le sens d'académiciens, dans une pièce dirigée
  contre l'Académie française.}.

On ferait un livre et fort divertissant du domestique entre le père et
le fils. Jamais ils ne se parlaient de rien\,; mais les billets
mouchaient à tous moments d'une chambre à l'autre, d'un caustique amer
et réciproque presque toujours facétieux. Le père se levait pour son
fils, même étant seuls, ôtait gravement son chapeau, ordonnait qu'on
apportât un siège à M. du Harlay, et ne se couvrait et ne s'asseyait que
quand le siège était en place. C'était après des compliments et dans le
reste un poids et une mesure de paroles. À table de même, enfin une
comédie continuelle. Au fond, ils se détestaient parfaitement l'un
l'autre, et tous deux avaient parfaitement raison.

Le ver rongeur du fils était de n'être de rien, et cette rage le rendait
ennemi de presque tout ce qui avait part au gouvernement, et frondeur de
tout ce qui s'y faisait. Sa faiblesse et sa vanité étaient là-dessus si
pitoyables, que, sachant très bien que M. le duc d'Orléans ne lui avait
jamais parlé, ni fait parler de rien, ni envoyé chez lui, et qu'il n'y
avait ni affaire ni occasion qui lui pût attirer de message de ce prince
ni de visite de personne des conseils, il défendait souvent et bien haut
à ses gens devant ceux qui le venaient voir, de laisser entrer personne,
quelque considérables qu'ils fussent, même de la part de M. le duc
d'Orléans, parce qu'il voulait être en repos, et qu'encore était-il
permis quelquefois d'être avec ses amis et de reprendre haleine. Ses
valets s'en moquaient, et ses prétendus amis en riaient, et au partir de
là en allaient rire avec les leurs.

Sa femme, demoiselle de Bretagne, riche héritière et d'une grande vertu,
en eut grand besoin, et fut avec lui une des plus malheureuses femmes du
monde. Ils n'eurent qu'une fille unique qui épousa le dernier fils de M.
de Luxembourg, dont le premier président était l'âme damnée, et ce fils
est devenu maréchal de France.

Harlay mourut comme il avait vécut. Il avait une bonne et nombreuse
bibliothèque, avec quantité de manuscrits sur différentes matières. Il
les donna à Chauvelin, depuis garde des sceaux, qui en sut faire un
échelon à sa fortune, et parce qu'il n'était rien moins que dévot, il
lui donna aussi tout ce qu'il avait de livres de dévotion, et tout le
reste de sa bibliothèque aux jésuites. Il n'avait au plus que soixante
ans, et se plut à ces legs ridicules. Je me suis peut-être trop étendu
sur un particulier qui n'a jamais figuré. J'ai succombé à la tentation
de déployer un si singulier caractère.

Dongois, greffier en chef du parlement, qui s'était bien réjoui en sa
vie de la rareté de ces deux hommes, mourut en même temps à
quatre-vingt-trois ans, et fut universellement regretté. C'était un très
honnête homme, très droit, extrêmement instruit et capable, qui faisait
très supérieurement sa charge\,; fort obligeant, très considéré du
parlement qui avait souvent recours à ses lumières en beaucoup
d'occasions, et qui avait au dehors et parmi les seigneurs et à la cour
beaucoup d'amis.

M. le prince de Conti perdit un fils enfant, qui était appelé comte de
La Marche, dont le roi prit le deuil pour huit jours.

Courson, fils de Bâville, intendant ou plutôt roi de Languedoc, ne
ressemblait en rien à son père. On a vu en son lieu qu'il pensa plus
d'une fois être assommé à coups de pierres en divers lieux de son
intendance de Rouen, dont il fallut l'ôter tant il s'y était rendu
odieux, mais le crédit de son père le sauva et le fit envoyer intendant
à Bordeaux. C'était dehors et dedans un gros boeuf, fort brutal, fort
insolent et dont les mains n'étaient pas nettes, ni à son exemple celles
de ses secrétaires qui faisaient toute l'intendance, dont il était très
incapable, et de plus très paresseux.

Il fit, entre autres tyrannies, des taxes sèches\footnote{Qui se
  payaient argent comptant.} très violentes dans Périgueux, par ses
ordonnances en forme, sans aucun édit ni arrêt du conseil\,; et voyant
qu'on ne se pressait pas d'y satisfaire, les augmenta, multiplia les,
frais, et à la fin mit dans des cachots des échevins et d'autres
honnêtes et riches bourgeois. Il en fit tant qu'ils députèrent pour
porter leurs plaintes, et allèrent de porte en porte chez tous ceux du
conseil de régence, après avoir été plus de deux mois à se morfondre
dans les antichambres du duc de Noailles.

Le comte de Toulouse, qui était homme fort juste, et qui les avait
entendus, blessé de ce qu'ils ne pouvaient obtenir de réponse, m'en
parla. J'en étais aussi indigné que lui. Je lui répondis que s'il
voulait m'aider nous aurions raison de cette affaire. J'en parlai à M.
le duc d'Orléans, qui n'en savait rien que superficiellement. Je lui
remontrai la nécessité de voir clair en des plaintes de cette nature\,;
l'injustice de ruiner ces députés de Périgueux sur le pavé de Paris pour
les lasser et ne les point entendre, et la cruauté de laisser languir
d'honnêtes bourgeois dans des cachots sans savoir pourquoi, et de quelle
autorité ils y étaient. Il en convint et nie promit d'en parler au duc
de Noailles. Au premier conseil d'après pour finances, j'avertis le
comte de Toulouse, et tous deux {[}nous{]} demandâmes au duc de Noailles
quand il rapporterait l'affaire de ces gens de Périgueux.

Il ne s'attendait à rien moins, et voulut nous éconduire. Je lui dis
qu'il y avait assez longtemps que les uns étaient dans les cachots et
les autres sur le pavé de Paris\,; que c'était une honte que cela, et ne
se pouvait souffrir davantage. Le comte de Toulouse reprit fort
sèchement sur le même ton. M. le duc d'Orléans arriva et on se mit en
place.

Comme le duc de Noailles ouvrait son sac, je dis fort haut à M. le duc
d'Orléans que M. le comte de Toulouse et moi venions de demander à M. de
Noailles quand il rapporterait au conseil l'affaire de Périgueux\,; que
ces gens-là, innocents ou coupables, n'avaient qu'un cri pour être ouïs
et jugés\,; et qu'il me paraissait de l'honneur du conseil de ne les pas
faire languir davantage. En finissant je regardai le comte de Toulouse,
qui dit aussi quelque chose de court mais d'assez fort. M. le duc
d'Orléans répondit qu'il ne demandait pas mieux. Le duc de Noailles se
mit à barbouiller sur l'accablement d'affaires, qu'il n'avait pas eu le
temps, etc. Je l'interrompis et lui dis qu'il fallait le prendre, et
l'avoir pris il y avait longtemps, parce qu'il n'y avait {[}rien{]} de
si pressé que de ne pas ruiner des gens sur le pavé de Paris, et en
laisser pourrir d'autres dans des cachots sans savoir pourquoi. M. le
duc d'Orléans reprit un mot en même sens, et ordonna au duc de Noailles
de se mettre en état de rapporter l'affaire à la huitaine.

D'excuses en excuses il différa encore trois semaines. À la fin je dis à
M. le duc d'Orléans que c'était se moquer de lui ouvertement, et faire
un déni de justice le plus public et le plus criant. Le conseil d'après
il se trouva que M. le duc d'Orléans lui avait dit qu'il rie voulait
plus attendre. M. le comte de Toulouse et moi continuâmes à lui demander
si à la fin il apportait l'affaire de Périgueux. Nous ne doutâmes plus
alors qu'elle serait aussitôt rapportée, niais les ruses n'étaient pas à
bout.

C'était un mardi après dîner, où souvent M. le duc d'Orléans abrégeait
le conseil pour aller à l'Opéra. Dans cette confiance le duc de Noailles
tint tout le conseil en différentes affaires. J'étais entre le comte de
Toulouse et lui. À chaque fin d'affaire je lui demandais\,: «\,Et
l'affaire de Périgueux\,? --- Tout à l'heure,\,» répondait-il, et en
commençait une autre. À la fin je m'aperçus du projet\,; je le dis tout
bas au comte de Toulouse qui s'en doutait déjà, et nous convînmes tous
deux de n'en être pas la dupe. Quand il eut épuisé son sac il était cinq
heures. En remettant ses pièces il le referma et dit à M. le duc
d'Orléans, qu'il avait encore l'affaire de Périgueux qu'il lui avait
ordonné d'apporter, mais qui serait longue et de détail\,; qu'il voulait
sans doute aller à l'Opéra\,; que ce serait pour la première fois\,; et
tout de suite, sans attendre de réponse, il se lève, pousse son tabouret
et tourne pour s'en aller. Je le pris par le bras\,: «\,Doucement, lui
dis-je, il faut savoir ce qu'il plaît à Son Altesse Royale. Monsieur,
dis-je à M, le duc d'Orléans, toujours tenant ferme la manche du duc de
Noailles, vous souciez-vous beaucoup aujourd'hui de l'Opéra\,? --- Mais
non, me répondit-il, on peut voir l'affaire de Périgueux. --- Mais sans
l'étrangler, repris je. --- Oui, dit M. le duc d'Orléans qui, regardant
M. le Duc qui souriait\,: Vous ne vous souciez pas d'y aller, lui
dit-il. --- Non, monsieur\,; voyons l'affaire, répondit M. le Duc. ---
Oh\,! remettez-vous donc là, monsieur, dis-je au duc de Noailles d'un
ton très ferme en le tirant très fort, reprenez votre siège et rouvrez
votre sac.\,» Sans dire une parole il tira son tabouret à grand bruit,
et s'assit dessus à le rompre. La rage lui sortait par les yeux. Le
comte de Toulouse riait et avait dit son mot aussi sur l'Opéra, et toute
la compagnie nous regardait, souriant presque tous, mais assez étonnée.

Le duc de Noailles étala ses papiers et se mit à rapporter. À mesure
qu'il s'agissait de quelque pièce, je la feuilletais, et par-ci, par là
je le reprenais. Il n'osait se fâcher dans ses réponses, mais il
écumait. Il fit un éloge de Bâville, de la considération qu'il méritait,
excusa Courson, et bavarda là-dessus tant qu'il put pour exténuer tout
et en faire perdre les principaux points de vue. Voyant que cela ne
finissait point pour lasser et se rendre maître de l'arrêt, je
l'interrompis et lui dis que le père et le fils étaient deux, qu'il ne
s'agissait ici que des faits du fils, de savoir si un intendant était
autorisé ou non, par son emploi, de taxer les gens à volonté, et de
mettre des impôts dans les villes et dans les campagnes de son
département, sans édit qui les ordonne, sans même d'arrêt du conseil, et
uniquement sur ses propres ordonnances particulières, et de tenir des
gens domiciliés quatre ou cinq mois dans des cachots, sans forme ni
figure de procès, parce qu'ils ne payaient point ces taxes sèches à
volonté, et encore accablés de frais. Puis me tournant à lui pour le
bien regarder\,: «\,C'est sur cela, monsieur, ajoutai-je, qu'il faut
opiner net et précis, puisque votre rapport est fait, et non pas nous
amuser ici au panégyrique de M. de Bâville, qui n'est point dans le
procès.\,» Le duc de Noailles, hors de soi, d'autant plus qu'il voyait
le régent sourire, et M. le Duc qui me regardait et riait un peu plus
ouvertement, se mit à opiner ou plutôt à balbutier. Il n'osa pourtant ne
pas conclure à l'élargissement des prisonniers. «\,Et les frais, dis-je,
et l'ordonnance de ces taxes, qu'en faites-vous\,? --- Mais en
élargissant, dit-il, l'ordonnance tombe.\,» Je ne voulus pas pousser
plus loin pour lors. On opina à l'élargissement, à casser l'ordonnance,
quelques uns au remboursement des frais aux dépens de l'intendant, et à
lui faire défense de récidiver.

Quand ce fut à mon tour, j'opinai de même, mais j'ajoutai que ce n'était
pas assez pour dédommager des gens aussi injustement et aussi
maltraités\,; que j'étais d'avis d'une somme à leur être adjugée, telle
qu'il plairait au conseil de la régler\,; et qu'à l'égard d'un intendant
qui abusait de l'autorité de sa place au point d'usurper celle du roi
pour imposer des taxes inconnues, de son chef, telles qu'il lui plaît,
sur qui il lui plaît, par ses seules ordonnances, qui jette dans les
cachots qui bon lui semble de son autorité privée, et qui met ainsi une
province au pillage, j'étais d'avis que Son Altesse Royale fût suppliée
d'en faire une telle justice qu'elle demeurât en exemple à tous les
intendants.

Le chancelier, adorateur de la robe et du duc de Noailles, se jeta dans
l'éloquence pour adoucir. Le comte de Toulouse et M. le Duc furent de
mon avis. Ceux qui avaient opiné devant moi firent la plupart des signes
que j'avais raison, mais ne reprirent point la parole. M. le duc
d'Orléans prononça l'élargissement et la cassation de l'ordonnance de
Courson et de tout ce qui s'en était suivi\,; qu'à l'égard du reste, il
se chargeait de faire dédommager ces gens-là, de bien laver la tête à
Courson, qui méritait pis, mais dont le père méritait d'être ménagé.
Comme on voulut se lever, je dis qu'il serait bon d'écrire l'arrêt tout
de suite, et M. le duc d'Orléans l'approuva. Noailles se jeta sur du
papier et de l'encre comme un oiseau de proie et se mit à écrire, moi à
me baisser et à lire à mesure ce qu'il écrivait. Il s'arrêta sur la
cassation de l'ordonnance et la prohibition de pareille récidive sans y
être autorisé par édit ou par arrêt du conseil. Je lui dictai la
clause\,; il regarda la compagnie, comme demandant des yeux. «\, Oui,
lui dis-je, il a passé comme cela\,; il n'y a qu'à le demander
encore.\,» M. le duc d'Orléans dit qu'oui. Noailles écrivit. Je pris le
papier et le relus\,; il l'avait écrit. Il le reprit en furie, le jeta
avec les autres pêle-mêle dans son sac, jeta son tabouret à dix pas de
là en se tournant, et s'en alla brossant comme un sanglier, sans
regarder ni saluer personne, et nous à rire. M. le Duc vint à moi, et
plusieurs autres qui, avec M. le comte de Toulouse, s'en divertirent.
Effectivement M. de Noailles se posséda si peu, qu'en se tournant pour
s'en aller, il frappa la table en jurant et disant qu'il n'y avait plus
moyen d'y tenir.

Je sus par des familiers de l'hôtel de Noailles, qui le dirent à de mes
amis, qu'en arrivant chez lui il s'était mis au lit sans vouloir voir
personne, que la fièvre lui prit, qu'il avait été d'une humeur
épouvantable le lendemain, et qu'il lui était échappé qu'il ne pouvait
plus soutenir les algarades et les scènes que je lui faisais essuyer. On
peut juger que cela ne m'en corrigea pas.

L'histoire en fut apparemment révélée par quelqu'un aux députés de
Périgueux (car dès le soir elle se débita par la ville) qui me vinrent
faire de grands remerciements. Noailles eut si peur de moi qu'il ne leur
fit attendre leur expédition que deux jours.

Peu de mois après, Courson fut révoqué aux feux de joie de sa province.
Cela ne le corrigea ni ne l'empêcha point d'obtenir dans les suites une
des deux places de conseiller au conseil royal des finances, car il
était déjà conseiller d'État lors de cette affaire de Périgueux. Des
Forts, mari de sa sueur, était devenu contrôleur général. Il se fia à
lui des actions de la compagnie des Indes et de leur mouvement sur la
place. Courson et sa soeur, à l'insu de des Forts, dont la netteté des
mains ne fut jamais soupçonnée, y firent si bien leurs affaires que le
désarroi de la place éclata. Chauvelin, lors à l'apogée de sa fortune,
ennemi déclaré de des Forts, le fit chasser d'autant plus aisément que
le cardinal Fleury était excédé de M\textsuperscript{me} des Forts et de
ses manèges, et le criminel Courson fut conservé à l'indignation
publique, qui ne s'y méprit pas, parce que Chauvelin voulut tout faire
retomber plus à plomb sur des Forts. J'ajoute cette suite, qui excède le
temps de ces Mémoires, pour achever tout de suite ce qui regarde
Courson.

Le maréchal de Tallard, dont on a vu le caractère, t. IV, p.~98, avait
été mis dans le conseil de régence par le testament du feu roi. Enragé
de n'être de rien, on a vu aussi qu'il se retira à la Planchette, petite
maison près de Paris, criant, dans ses accès de désespoir, qu'il voulait
porter le testament du feu roi écrit sur son dos. Il mourait de rage et
d'ennui dans sa solitude, et n'y put durer longtemps. Son attachement
aux Rohan, quoique servile, n'empêchait pas qu'il n'en fût compté. Il
n'en était pas de même du sien, de tous temps, pour le maréchal de
Villeroy qui, le rencontrant même à la tête des armées, conserva
toujours ses grands airs avec lui, et ne cessa en aucun temps de le
traiter comme son protégé. L'autre, impatient du joug, se rebecquait
quelquefois\,; mais comme l'ambition et la faveur furent toujours ses
idoles, il se rendit plus que jamais le très humble esclave du maréchal
de Villeroy, depuis le grand vol que M\textsuperscript{me} de Maintenon
lui fit prendre après son rappel, qu'elle moyenna à la mort de
M\textsuperscript{me} la duchesse de Bourgogne, lors Dauphine, et qu'il
conservait encore auprès de M. le duc d'Orléans, qui le craignait et qui
le ménageait, jusqu'à aller sans cesse au-devant de tout ce qui lui
pouvait plaire, aussi misérablement qu'inutilement.

Villeroy prit son temps de l'issue de l'affaire des bâtards et de cette
prétendue noblesse, dont on avait su faire peur au régent, pour lui
représenter la triste situation de Tallard et profiter du malaise qui
troublait encore ce prince. Le moment fut favorable\,; il crut
s'acquérir Villeroy et les Rohan en traitant bien Tallard. Il imagina
que, tenant tous aux bâtards, et par conséquent à cette prétendue
noblesse, le bon traitement fait à Tallard plairait au public et lui
ramènerait bien des gens. Les affaires, importantes avaient déjà pris le
chemin unique de son cabinet, et n'étaient presque plus portées au
conseil de régence que toutes délibérées, et seulement pour la forme.
Ainsi, le régent crut paraître faire beaucoup et donner peu en effet, en
y faisant entrer Tallard, qui de honte, de dépit et d'embarras, ne se
présentait que des moments fort rares au Palais-Royal. La parole fut
donc donnée au maréchal de Villeroy, avec permission de le dire à
Tallard sous le secret, qui, dès le lendemain, se présenta devant M. le
duc d'Orléans. Il avait voulu se réserver de lui déclarer et de fixer le
jour de son entrée au conseil de régence. Un peu après qu'il fut là en
présence, parmi les courtisans, le régent lui dit qu'il le mettait dans
le conseil de régence, et d'y venir prendre place le surlendemain.

Dès que je le sus, je sentis la difficulté qui se devait présenter sur
la préséance entre lui et le maréchal d'Estrées qui y venait rapporter
les affaires de marine, et qui d'ailleurs y entrait avec les autres
chefs et présidents des conseils quand on les y appelait pour des
affaires importantes. J'aimais bien mieux Estrées que Tallard, et pour
l'estime nulle sorte de comparaison à en faire en rien. Le public même
n'en faisait aucune, et tout était de ce côté-là à l'avantage du
maréchal d'Estrées, mais j'aimais mieux que lui l'ordre et la règle, et
sans intérêt (car je n'y en pouvais avoir aucun entre eux), l'intégrité
des dignités de l'État. Tous deux étaient maréchaux de France, et dans
cet office de la couronne Estrées était l'ancien de beaucoup\,; mais il
n'était point duc et Tallard l'était vérifié au parlement\,; il est vrai
qu'Estrées était grand d'Espagne, beaucoup plus anciennement que Tallard
n'était duc, et que, comme aux cérémonies de la cour les grands
d'Espagne, comme je l'ai expliqué ailleurs, coupaient les ducs, suivant
l'ancienneté des uns à l'égard des autres, Estrées précédait Tallard aux
cérémonies de l'ordre et en toutes celles de la cour. Mais, dès la
première fois que le conseil de régence s'était assemblé, il avait été
réglé, comme je l'ai rapporté en son lieu, que le maréchal de Villars
précéderait le maréchal d'Harcourt, celui-ci duc vérifié beaucoup plus
ancien que l'autre, mais Villars plus ancien pair qu'Harcourt, parce que
les séances du conseil de régence se devaient régler sur celles qui
s'observent au parlement, et aux états généraux et aux autres cérémonies
d'État où la pairie l'emporte. Il en résultait qu'entre deux hommes qui
n'étaient pas pairs, mais dont l'office de la couronne qu'ils avaient
tous deux se trouvait effacé par une autre dignité, c'était cette
dignité qui devait régler leur rang. Ils en avaient chacun une égale,
mais différente\,: l'une était étrangère, l'autre de l'État. Cette
dignité étrangère roulait à la vérité par ancienneté avec la première de
l'État dans les cérémonies de la cour\,; mais comme telle, elle ne
pouvait être admise dans une séance qui se réglait pour le rang par la
pairie, parce qu'il s'y agissait de matières d'État où elle ne pouvait
avoir aucune part\,; au lieu que la dignité de duc vérifié en étant une
réelle et effective de l'État, avait, comme telle, plein caractère pour
être admise aux affaires de l'État, et ne l'y pouvait être que dans le
rang qui lui appartenait, d'où il résultait qu'encore que le maréchal
d'Estrées eût dans les cérémonies de la cour la préséance sur le
maréchal de Tallard, celui-ci la devait avoir sur l'autre dans les
cérémonies de l'État, et singulièrement au conseil de régence établi
pour suppléer en tout à l'âge du roi pour le gouvernement de l'État.

Je ne pus avertir Tallard qu'aux Tuileries, un peu avant le conseil. Sa
joie extrême allait jusqu'à l'indécence, et ne lui en avait pas laissé
la réflexion\,; il en dit un mot au maréchal d'Estrées qui devait
rapporter ses affaires de marine, et tous deux en parlèrent à M. le duc
d'Orléans, quand il arriva un moment après, qui leur dit que le conseil
les jugerait sur-le-champ. On se mit en place\,; les deux maréchaux se
tinrent debout derrière la place où j'étais. Estrées parla le premier\,;
Tallard, étourdi du bateau, s'embarrassa. Je sentis qu'il se tirerait
mal d'affaire, je l'interrompis, et dis à M. le duc d'Orléans que, s'il
avait agréable de prier MM. les deux maréchaux de sortir pour un moment,
je m'offrais d'expliquer la question en deux mots, et qu'on y opinerait
plus librement en leur absence qu'en leur présence. Au lieu de me
répondre, il s'adressa aux deux maréchaux, et leur dit qu'en effet il
serait mieux qu'ils voulussent bien sortir, et qu'il les ferait rappeler
sitôt que le jugement serait décidé. Ils firent la révérence sans rien
dire, et sortirent.

J'expliquai aussitôt après la question en la manière que je viens de la
rapporter, quoique avec un peu plus d'étendue, mais de fort peu. Je
conclus en faveur de Tallard, et tous les avis furent conformes au mien.
La Vrillière écrivit sur-le-champ la décision sur le registre du
conseil\,; puis alla, par ordre du régent, appeler les deux maréchaux, à
qui La Vrillière ne dit rien de leur jugement. Ils se tinrent debout au
même lieu où ils s'étaient mis d'abord\,; nous nous rassîmes en même
temps que M. le duc d'Orléans, qui à l'instant prononça l'arrêt que le
maréchal d'Estrées prit de fort bonne grâce et très honnêtement, et
Tallard fort modestement. Le régent leur dit de prendre place, se leva,
et nous tous, et nous rassîmes aussitôt. Tallard, par son rang, échut
vis-à-vis de moi, quelques places au-dessous.

L'excès de la joie, le sérieux du spectacle, l'inquiétude d'une dispute
imprévue, firent sur lui une étrange impression. Vers le milieu du
conseil, je le vis pâlir, rougir, frétiller doucement sur son siège, ses
yeux qui s'égaraient, un homme en un mot fort embarrassé de sa personne.
Quoique sans aucun commerce avec lui que celui qu'on a avec tout le
monde, la pitié m'en prit\,; je dis à M. le duc d'Orléans que je croyais
que M. de Tallard se trouvait mal. Aussitôt il lui dit de sortir, et de
revenir quand il voudrait. Il ne se fit pas prier, et s'en alla très
vite. Il rentra un quart d'heure après. En sortant du conseil, il me dit
que je lui avais sauvé la vie\,; qu'il avait indiscrètement pris de la
rhubarbe le matin, qu'il venait de mettre comble la chaise percée du
maréchal de Villeroy, qu'il ne savait ce qu'il serait devenu sans moi,
ni ce qui lui serait arrivé, parce qu'il n'aurait jamais osé demander la
permission de sortir. Je ris de bon coeur de son aventure, mais je ne
pris pas le change de sa rhubarbe\,; il était trop transporté de joie
pour avoir oublié le conseil, et trop avisé pour avoir pris ce jour-là
de quoi se purger.

Le duc d'Albret obtint le gouvernement d'Auvergne, sur la démission de
M. de Bouillon, qui avait dessus cent mille écus de brevet de retenue\,:
un pareil fut donné au duc d'Albret.

Le maréchal de Tessé entrait au conseil de marine comme général des
galères. On a vu à propos du voyage du czar, auprès duquel il fut mis,
la vie qu'il menait depuis la mort du feu roi. Il était fort dégoûté de
n'être de rien\,; je ne sais si l'entrée de Tallard au conseil de
régence acheva de le dépiter\,; mais peu de jours après il pria le
régent de lui permettre, retiré comme il était, ou plutôt comme il se
croyait, de se retirer aussi du conseil de marine. Mais il se garda bien
d'en rendre les appointements. Ce vide ne fit aucune sensation.

La facilité de M, le duc d'Orléans se laissa aller à l'adoration du
chancelier pour la robe, et aux sollicitations du duc de Noailles pour
la capter, d'accorder aux gens du monde les plus inutiles, qui sont les
conseillers du grand conseil, deux grandes et fort étranges grâces\,:
l'une qu'ils feraient désormais souche de noblesse\,; l'autre, exemption
de lods et ventes\footnote{Droit que prélevait le seigneur pour la vente
  des terres comprises dans sa censive (domaine soumis à la redevance
  appelée cens.)} des terres et maisons relevant du roi.

Le roi Stanislas pensa être enlevé aux Deux-Ponts par un parti qui avait
fait cette entreprise. Elle fut découverte au moment qu'elle allait
réussir. On prit trois de ces gens-là que le roi de Pologne avait mis en
campagne. Comme les affaires du nord n'étaient pas finies, il ne
craignait point de violer le territoire de la souveraineté, personnelle
surtout, du roi de Suède. Quelque temps après, le régent, touché de
l'état fugitif de ce malheureux roi, qui n'était en sûreté nulle part,
lui donna asile à Weissembourg en basse Alsace.

M\textsuperscript{me} la princesse de Conti accoucha de M. le prince de
Conti d'aujourd'hui, tandis que M. son mari était à l'Ile-Adam.
L'ambassadeur de Portugal donna une superbe fête pour la naissance d'un
fils du roi de Portugal. Il y eut un grand bal en masque, où
M\textsuperscript{me} la duchesse de Berry, M. le duc d'Orléans et
beaucoup de gens allèrent masqués.

La Forêt, gentilhomme français, huguenot, et depuis longtemps attaché au
service du roi d'Angleterre avant qu'il vînt à la couronne, était
parvenu aux premières charges de la cour de Hanovre, et à être fort
avant dans les bonnes grâces de son maître. Il se trouva dans un cas
singulier sur la jouissance de ses biens en France, qui, avec le secours
du crédit du roi d'Angleterre auprès de M. le duc d'Orléans, qu'il y
employa tout entier, lui en fit espérer la restitution, dont il intenta
la demande. L'affaire, très soigneusement examinée par la considération
du roi d'Angleterre, ne se trouva point dans le cas que La Forêt
prétendait, et très dangereuse de plus à lui être adjugée, par la porte
que cet arrêt eût ouverte aux autres réfugiés pour les mêmes
prétentions. Ainsi La Forêt perdit son procès tout d'une voix au conseil
du dedans, puis en celui de régence.

Le 15 août fut dans Paris l'étrange spectacle du triomphe du parlement
sur la royauté, et de l'ignominie des deux augustes qualités réunies
ensemble, de petit-fils de France et de régent du royaume, dont M. le
duc d'Orléans, entraîné par le duc de Noailles, Effiat, les Besons,
Canillac et autres serfs du parlement, se cacha merveilleusement de moi.
On a vu sur l'année passée qu'il voulut aller à la procession du voeu du
roi son grand-père, qui a montré plus d'une fois au parlement, de
paroles et d'effet, qu'il savait le contenir dans les bornes du devoir
et du respect, et qui l'aurait étrangement humilié, s'il eût pu imaginer
ce qui se passerait entre son petit-fils et cette compagnie
soixante-quatorze ans après sa mort, à l'occasion de la procession qu'il
avait pieusement instituée. La faute de l'année précédente aurait dû
corriger\,; et puisque M. le duc d'Orléans avait eu la faiblesse de ne
pas faire rentrer le parlement dans ses bornes, au moins n'en fallait-il
pas volontairement subir l'usurpation monstrueuse sans aucune sorte de
nécessité. Une procession n'était ni de son goût, ni de la vie qu'il
menait, ni par cela même de l'édification publique. Ni le feu roi, ni
aucune personne royale n'y avait jamais assisté, et {[}ils{]} s'étaient
toujours contentés de celle de leur chapelle. Il n'avait donc qu'à rire
avec mépris de la folle chimère du parlement, s'il n'avait pas la force
de mieux faire, et ne plus penser d'aller à cette procession.

Le parlement venait de refuser très sèchement d'enregistrer la création
de deux charges dans les bâtiments, qui auraient été vendues quatre cent
mille livres les deux, au profit du roi, sous prétexte, dirent
Messieurs, que leurs gages augmenteraient les dépenses de l'État. Le
même esprit de misère qui venait de mettre Tallard dans le conseil de
régence fit aller M. le duc d'Orléans à la procession\,; et comme les
\emph{mezzo-termine} étaient de son goût, le premier président, de
concert avec le duc et la duchesse du Maine, lui en suggérèrent un qui
portait tellement son excommunication sur le front, qu'il est incroyable
qu'un prince d'autant d'esprit que M. le duc d'Orléans y put donner, et
que, de tous ceux qui l'excitèrent à cette procession, pas un ne s'en
aperçut ou ne lui fut assez attaché pour l'en avertir\,; car le
singulier est que je ne le sus que le matin même du 15, que la
procession était pour l'après-dînée\,; et qu'il n'y avait plus qu'à
hausser les épaules. Ce \emph{mezzo-termine}, si bien imaginé pour
accommoder toutes choses, fut une procuration du roi à M. le duc
d'Orléans pour tenir sa place à la procession, où en cette qualité il
irait des Tuileries à Notre-Dame, et en reviendrait comme le roi, et
avec le même accompagnement de carrosses, pages, valets de pied, gardes
du roi, Cent Suisses, etc., ayant à Notre-Dame, et pendant la
procession, le premier gentilhomme de la chambre en année, et le duc de
Villeroy, capitaine des gardes du corps en quartier, avec le bâton,
derrière lui, et le capitaine des Cent Suisses devant lui, et les
aumôniers du roi de quartier en rochet, manteau long et bonnet carré,
pour le servir comme le roi. Avec cette royale mascarade, le parlement
eut la complaisance de le vouloir bien souffrir à sa droite, et se
réserva le plaisir de s'en bien moquer. On laisse à penser quel effet
opéra une telle comparse, fondée sur aucune sorte d'apparence d'usage,
de coutume, encore moins de nécessité, faite par un prince qui se
donnait publiquement, par ses discours et par sa conduite, pour se
moquer de bien pis que d'une procession, et qui, par les renonciations,
la paix d'Utrecht et l'âge où le roi {[}était{]}, était encore pour
longtemps l'héritier présomptif de la couronne. Quoi donc de plus simple
et de plus naturel à répandre et à persuader que M. le duc d'Orléans,
dans la soif et dans l'espérance de régner, avait saisi une occasion de
se donner la satisfaction de se montrer en roi en une cérémonie
publique, en avant-goût de ce qui lui pouvait arriver, et pour
accoutumer Paris à lui en voir toute la pompe et la majesté en plein,
comme il en exerçait le pouvoir\,?

Avec les horreurs semées lors de la perte des princes, père, mère, frère
et oncle du roi, sans cesse rafraîchies par leurs pernicieux auteurs, ou
peut imaginer ce qui fut répandu dans Paris, dans les provinces, dans
les pays étrangers et dans l'esprit du roi, par la facilité et
l'autorité de l'accès auprès de lui de ceux qui voulaient accréditer ces
exécrables soupçons et en grossir les idées. Aussi firent-elles un grand
bruit, et la fête n'avait été proposée ni imaginée pour autre chose.
Après la chose faite, M. le duc d'Orléans n'osa jamais m'en parler, et
l'indignation me retint autant de lui en rien dire aussi, que
l'inutilité de le faire après coup. L'autre effet fut d'affermir le
monde dans la folle idée de la supériorité, tout au moins de l'égalité
du parlement avec le régent, qui se semait depuis longtemps avec art, et
qui de cette époque prit faveur générale, et d'enfler le parlement au
point qu'on verra bientôt, rallié avec tous les ennemis du régent et
d'une multitude de fous qui ne doutaient pas de figurer et de faire
fortune dans les troubles.

La fête de Saint-Louis donna dix jours après le contraste plénier de
celle-ci. La musique de l'Opéra a coutume, ce jour-là, de divertir
gratuitement le public d'un beau concert dans le jardin des Tuileries.
La présence du roi dans ce palais y attirait encore plus de monde, dans
l'espérance de le voir paraître quelquefois sur les terrasses qui sont
de plain-pied aux appartements. Il parut très sensiblement cette année
un redoublement de zèle, par l'affluence innombrable qui accourut non
seulement dans le jardin, mais de l'autre côté, dans les cours, dans la
place, et qui ne laissa pas une place vide, je ne dis pas aux fenêtres,
mais sur les toits des maisons en vue des Tuileries. Le maréchal de
Villeroy persuadait à grand'peine le roi de se montrer, tantôt à la vue
du jardin, tantôt à celle des cours, et dès qu'il paraissait, c'étaient
des cris de\,: Vive le roi\,! cent fois redoublés. Le maréchal de
Villeroy faisait remarquer au roi cette multitude prodigieuse, et
sentencieusement lui disait\,: «\,Voyez, mon maître, voyez tout ce
peuple, cette affluence, ce nombre de peuple immense, tout cela est à
vous, vous en êtes le maître\,;» et sans cesse lui répétait cette leçon
pour la lui bien inculquer. Il avait peur apparemment qu'il n'ignorât
son pouvoir. L'admirable Dauphin son père en avait reçu de bien
différentes, dont il avait bien su profiter. Il était bien fortement
persuadé qu'en même temps que la puissance est donnée aux rois pour
commander et pour gouverner, les peuples ne sont pas aux rois, mais les
rois aux peuples, pour leur rendre justice, les faire vivre selon les
lois, et les rendre heureux par l'équité, la sagesse, la douceur et la
modération de leur gouvernement. C'est ce que je lui ai souvent ouï dire
avec effusion de coeur et persuasion intime, dans le désir et la
résolution bien ferme de se conduire en conséquence, non seulement étant
en particulier avec lui, et y travaillant pour l'avenir dans ces
principes, mais je le lui ai ouï dire et répéter plusieurs fois tout
haut en public, en plein salon de Marly, à l'admiration et aux délices
de tous ceux qui l'entendaient.

\hypertarget{chapitre-iii.}{%
\chapter{CHAPITRE III.}\label{chapitre-iii.}}

1717

~

{\textsc{Comité pour les finances.}} {\textsc{- Ma conduite à cet
égard.}} {\textsc{- Je propose en particulier au chancelier la réforme
de quelques troupes distinguées avec les raisons et la manière de la
faire.}} {\textsc{- Il l'approuve\,; mais elle demeure entre nous deux
par la faiblesse du régent.}} {\textsc{- Fin et résolution du comité des
finances mises en édit.}} {\textsc{- Démêlé ajusté entre le premier
président avec les enquêtes pour le choix et le nombre des commissaires
du parlement, quand il en faut nommer.}} {\textsc{- Le parlement veut
qu'on lui rendre compte des finances avant d'opiner sur l'enregistrement
de l'édit, et l'obtient.}} {\textsc{- Il l'enregistre enfin avec
peine.}} {\textsc{- Misère du régent\,; peur et valetage du duc de
Noailles.}} {\textsc{- Évêques prétendent inutilement des carreaux à
l'anniversaire du feu roi.}} {\textsc{- Entreprise de nouveau condamnée
entre les princesses du sang, femmes et filles, au mariage de Chalmazel
avec une soeur du maréchal d'Harcourt.}} {\textsc{-
M\textsuperscript{me} la duchesse d'Orléans achète Bagnolet.}}
{\textsc{- Maison donnée à Paris aux chanceliers\,; et Champ donné à la
princesse de Conti pour La Vallière, aux dépens du financier
Bourvalais.}} {\textsc{- Ragotzi s'en va en Turquie\,; ce qu'il devient
jusqu'à sa mort.}} {\textsc{- Victoire du prince Eugène sur les Turcs.}}
{\textsc{- Prise de Belgrade.}} {\textsc{- Mort de Villette et
d'Estrade.}} {\textsc{- Le fils du dernier obtient sa mairie de
Bordeaux.}} {\textsc{- M\textsuperscript{me} de Mouchy et Rion, dame
d'atours et premier écuyer en second de M\textsuperscript{me} la
duchesse de Berry.}} {\textsc{- Changement parmi ses dames.}} {\textsc{-
Diverses grâces de M. le duc d'Orléans.}} {\textsc{- Retour de Hongrie
des François.}} {\textsc{- Mort du duc de Ventadour\,; extinction de son
duché-pairie.}} {\textsc{- Mort de Moncault.}} {\textsc{- J'achète pour
mes enfants deux régiments de cavalerie.}} {\textsc{- Abbé Dubois
repasse en Angleterre.}} {\textsc{- Peterborough arrêté dans l'État
ecclésiastique.}}

~

Le comité qui s'assemblait plusieurs fois la semaine pour les finances
allait son train. Le duc de Noailles y montra, comme il voulut, l'état
présent des finances, en exposa les embarras, y présenta des expédients,
lut des mémoires. J'étais là, comme on l'a vu, malgré moi, et cette
langue de finance dont on {[}a{]} su faire une science, et, si ce mot se
peut hasarder, un grimoire, pour que l'intelligence en soit cachée à
ceux qui n'y sont pas initiés, et qui, magistrats et traitants,
banquiers, etc., ont grand intérêt que les autres en demeurent dans
l'ignorance\,; cette langue, dis-je, m'était tout à fait étrangère.
Néanmoins ma maxime constante ayant toujours été que l'humeur doit être
toujours bannie des affaires autant que l'acception des choses et des
personnes et toute prévention, j'écoutais de toutes mes oreilles, malgré
mon dégoût de la matière, et ce que je n'entendais pas, je n'étais pas
honteux de le dire et de me le faire expliquer. C'était le fruit de
l'aveu de mon ignorance en finances, que j'avais fait si haut et si
clair en plein conseil de régence, lorsque je m'excusai d'être de ce
comité, et que le régent finit par me le commander.

Il arriva assez souvent qu'y ayant diversité d'avis, quelquefois même
assez vive, je me trouvai de celui du duc de Noailles, et que je
disputai même assez fortement pour le soutenir. Le chancelier ravi m'en
faisait compliment après\,; et M. le duc d'Orléans, à qui l'un et
l'autre le dirent, et qui avait remarqué la même chose quelquefois au
conseil de régence, les assura qu'il n'en était point surpris, et ne
laissa pas de m'en marquer sa satisfaction. Je lui dis, et au
chancelier, que l'avis du duc de Noailles, bon ou mauvais, et sa
personne, étaient pour moi deux choses absolument distinctes et
séparées\,; que je cherchais partout le bon et le vrai, et que je m'y
attachais partout où je le croyais voir, comme je me roidissais aussi
contre ce que j'y croyais opposé\,; qu'il pouvait bien être qu'en ce
dernier cas, je parlais plus ferme et plus dur quand je trouvais l'avis
du duc de Noailles à combattre, que si j'avais eu à attaquer celui d'un
autre\,; mais aussi {[}que{]} j'étais de son avis sans répugnance quand
je le trouvais bon, et que je m'élevais pour le soutenir fortement en
faveur du bon et du vrai quand je le voyais disputer, sans que, pour
tout cela, je changeasse de sentiment pour sa personne.

Comme ce travail se prolongeait, les assemblées se multiplièrent\,; et
une après-dînée, à la fin d'une, il fut convenu que nous nous
rassemblerions le lendemain matin et encore l'après-dînée, et que, pour
n'avoir pas la peine de tant aller et venir, le chancelier donnerait à
dîner à tout le comité. Le lendemain matin, au sortir de la séance, le
chancelier, qui, dès la veille, m'avait prié, outre le général, en
particulier à dîner, s'approcha de moi en me disant, comme encore d'un
air d'invitation, qu'on allait dîner. Je le priai de me dire précisément
à quelle heure il comptait rentrer en séance, afin que je m'y trouvasse
ponctuellement. À sa surprise et son redoublement de prières de rester,
je lui avouai franchement que je ne pouvais me résoudre à dîner avec le
duc de Noailles\,; que tant qu'il voudrait sans lui je réparerais ce que
je perdais ce jour-là. Il me parut affligé au dernier point, me pressa,
me conjura, me représenta le bruit que cela allait faire. Je lui dis
qu'il n'y aurait rien de nouveau, et que personne n'ignorait à quel
point nous étions ensemble. Ce colloque, qui se faisait avec émotion sur
le chemin de la porte, fut remarqué. Je vis par hasard le duc de
Noailles, qui du fond de la chambre nous regardait, et parlant aux uns
et aux autres. Le duc de La Force vint en tiers, un instant après le
maréchal de Villeroy, puis l'archevêque de Bordeaux, qui se joignirent
au chancelier, et qui tous ensemble, comme par force, me retinrent. Je
consentis donc enfin, mais avec une répugnance extrême, et à condition
encore que le duc de Noailles se placerait au plus loin de moi, sans
quoi je leur déclarai que je sortirais de table. Ils s'en chargèrent, et
cela fut exécuté. Le dîner fut grand et bon, et tout m'y montra qu'on
était aise que j'y fusse demeuré. Le duc de Noailles y parut, tout
désinvolte qu'il est, fort empêtré. Il voulut pourtant un peu
bavarder\,; mais on voyait qu'il avait peine à dire. Vers le milieu du
repas, il se trouva mal ou en fit le semblant, et passa dans une autre
chambre. Un moment après, la chancelière l'alla voir et revint se mettre
à table. Personne autre n'en sortit ni ne marqua de soins que le
chancelier, qui y envoya une fois ou deux. On dit que c'était des
vapeurs, et finalement il acheva de dîner dans cette chambre plus à son
aise qu'il n'eût apparemment fait à table. Je n'en sourcillai jamais. Il
se retrouva avec la compagnie à prendre du café, et peu après nous nous
remîmes en séance, où il rapporta comme si de rien n'eût été. Je fus
fort remercié de la compagnie, et particulièrement du chancelier et de
la chancelière d'être demeuré à dîner, et je ne cachai à personne que
ç'avait été un vrai sacrifice de ma part, dont l'absence du duc de
Noailles m'avait fort soulagé dans la dernière moitié du repas. Ce dîner
avec lui, ce qui s'était répandu que j'étais souvent de son avis, et
grossi, dont lui-même était bien homme à s'être paré, fit courir quelque
bruit que nous étions raccommodés, qui fut bientôt détruit par la
continuité de la façon dont j'en usais avec lui. Ce fut la seule fois
qu'il y eut comité matin et soir. Ils redoublèrent d'après-dînée et de
longueur. Je crus que le chancelier n'avait pas voulu, et sagement, nous
exposer, le duc de Noailles et moi, à l'inconvénient d'un second dîner.

Le travail achevé, et tous les avis à peu près réunis sur chaque point,
j'allai voir le chancelier en particulier. Je lui dis que je venais lui
communiquer une pensée que je n'avais pas voulu hasarder dans le comité,
raisonner avec lui, et, s'il trouvait que ce que je pensais fût bon, le
proposer lui et moi à M. le duc d'Orléans, sinon l'oublier l'un et
l'autre. Je lui dis que, peiné de voir toute la difficulté qui se
trouvait à égaler, du moins en pleine paix, la recette du roi à sa
dépense, je pensais qu'il serait à propos de réformer la gendarmerie, et
même les gens d'armes et les chevau-légers de la garde, avec les deux
compagnies des mousquetaires, en augmentant de deux brigades chacune des
quatre compagnies des gardes du corps.

Mes raisons étaient celles-ci\,: il n'y a point d'escadron de ces
troupes, l'un dans l'autre, qui en simples maîtres et en officiers, tout
compris, ne coûte quatre escadrons de cavalerie ordinaire. Quelque
valeureuses qu'on ait éprouvé ces troupes, on ne peut espérer qu'elles
puissent battre leur quadruple, ni même qu'elles puissent se soutenir
contre ce nombre. Ainsi, quant aux actions, rien à perdre de ce
côté-là\,; au contraire à y gagner, si en temps de guerre on juge à
propos de faire la même dépense pour avoir le quadruple d'escadrons
ordinaires en leur place\,; et en attendant une épargne de plusieurs
millions dont la supputation est évidente. Le courant du service dans
les armées y gagnerait en toute façon. C'est une dispute continuelle sur
les prétentions de la gendarmerie, qui vont toujours croissant et qui la
rend odieuse à la cavalerie, jusqu'à causer toutes les campagnes des
embarras et des accidents. Les maîtres ne sont point officiers, et ne
veulent point passer pour cavaliers. Ils se prétendent égaux aux gens
d'armes et aux chevau-légers de la garde, lesquels sont maison du roi.
De là des disputes pour marcher et pour obéir, pour des préférences de
fourrages, pour des distinctions de quartiers, pour des difficultés avec
les officiers généraux et avec ceux du détail, et pour toutes sortes de
détachements\,; et comme tout cela est soutenu par un esprit de corps
(on n'oserait dire de petite république, par ce nombreux essaim
d'officiers, triplés et quadruplés en charges par compagnie, dont chacun
se pique à qui soutiendra plus haut ce qu'ils appellent l'honneur du
corps), personne ne se veut brouiller jusqu'aux querelles avec tant de
têtes échauffées, et le général lui-même a plus court de céder, mais
d'éviter de les avoir dans son armée, où ils ne font presque aucun
service par ces difficultés, et les renvoie le plus tôt qu'il est
possible, eux-mêmes étant dans la prétention d'arriver les derniers à
l'armée et d'en partir les premiers, en sorte qu'il est rare qu'ils
fassent une campagne entière, dont les armées mêmes se sentent fort
soulagées. Voilà ce qui est particulier à la gendarmerie.

À l'égard de ce qui lui est commun avec les gens d'armes et les
chevau-légers de la garde et les mousquetaires, le voici\,: deux grands
inconvénients pour la guerre, par le grand nombre des officiers de tous
ces corps, qui font une foule d'équipages qui sont fort à charge pour
les subsistances, et qui augmentent très considérablement l'embarras des
marches et des mouvements d'une armée. Mais ce nombre d'officiers en
produit un autre plus fâcheux\,: c'est qu'ils ne sont en effet que des
capitaines, des lieutenants, des cornettes de cavalerie, et ce qui est
la même chose sous le nom d'enseigne qu'on a donné pour avoir quatre
officiers, qui quelquefois sont doublés, comme ils le sont toujours dans
les gens d'armes et chevau-légers de la garde et dans les deux
compagnies de mousquetaires. Or, n'étant que tels, ils en sont bornés au
même service quand ils sont en détachement, et comme ils vieillissent
dans ces charges, ils y deviennent anciens officiers généraux sans
savoir plus et souvent moins qu'un lieutenant de cavalerie\,; d'où il
est aisé de juger de ce qui en peut arriver quand ils se trouvent
chargés de quelque chose. Le feu roi, de la création duquel sont les
mousquetaires gris et noirs et la gendarmerie, et qui se plaisait aux
détails et aux revues des troupes et à leur magnificence, mit les
officiers de ces troupes sur le pied peu à peu de devenir officiers
généraux à leur rang, et les fit presque tous colonels par leurs
charges, et fort tôt après les avoir achetées ceux dont les charges ne
les font pas. Cela fait donc dans les armées un amas très nombreux de
colonels, brigadiers, officiers généraux, qui n'ont ni n'ont jamais eu
de troupes, qui n'ont jamais été en détachement que comme simples
cornettes, lieutenants ou capitaines de cavalerie, et qui, nonobstant
leurs grades, continuent, tant qu'ils ont ces charges, d'être détachés
sur le même pied. Il est vrai que sur le gros de l'armée ils marchent à
leur tour suivant leur grade d'armée\,; mais, au nombre qu'ils sont de
chaque grade, marcher ainsi se borne à deux ou trois fois par campagne,
qui n'est pas le moyen d'apprendre, quand précédemment surtout on n'a
rien appris ni eu occasion d'apprendre. Cette double façon d'être
détaché produit une cacophonie ridicule en ce que le lieutenant, détaché
avec sa troupe distinguée, et qui dans le total du détachement ne sert
que comme un lieutenant de cavalerie à la tête de quinze ou vingt
maîtres, est souvent brigadier\footnote{Général de brigade.} et même
maréchal de camp, aux ordres, non seulement de son cadet de même grade
ou même inférieur qui commande le tout, mais à ceux des colonels et des
lieutenants-colonels détachés avec lui à leur tour de marcher, et qui,
sous le chef, commandent à tout le détachement. Voilà en peu de mots
pour la guerre\,; venons aux autres inconvénients.

Celui de la gendarmerie est unique\,: c'est ce qu'il en coûte de plus au
roi que pour ses troupes ordinaires, en place de fourrages pour les
officiers, et en traitements de quartiers d'hiver pour le total de la
gendarmerie, ainsi qu'en routes et en étapes, ce qui gît encore en un
calcul bien aisé. Pour ce qui est des gens d'armes, chevau-légers et,
mousquetaires, c'est une autre manière de compter avec eux qui va encore
plus loin. Ces troupes, en si petit nombre pour la guerre, quand même
(ce qui ne peut être) les quatre compagnies iraient tout entières, parce
qu'il en demeure toujours pour le guet et par force congés, ne sont, ou
d'aucun usage ailleurs, ou d'un usage inutile. Jamais leur guet n'est
auprès du roi dans pas un lieu de ses demeures\,; ce guet l'accompagne
seulement de Versailles à Fontainebleau ou à Compiègne, ou en de vrais
voyages. Dans ces voyages même ils ne sont jamais dans les lieux où le
roi couche, excepté que, en des cas assez rares, un petit détachement de
mousquetaires des deux compagnies {[}s'y trouve{]}, pour fournir aux
sentinelles extérieures et suppléer au régiment des gardes ou autre
garde d'infanterie par les chemins, les gardes du corps environnant
toujours le carrosse du roi aux deux côtés et derrière, et quelques-uns
devant\,; qu'en avant de tout et en arrière de tout, il y a un
détachement de gens d'armes et de chevau-légers, et quatre mousquetaires
à la tête de l'attelage du roi, qui tous se relayent de distance en
distance. De service de cour, aucun autre qu'un officier principal de
chacun de ces corps en quartier, qui prend l'ordre du roi au sortir de
son souper, quand le capitaine ne s'y trouve pas, et un maître de chaque
corps, botté, en uniforme, qui prend l'ordre du roi tous les jours sur
son passage, pour aller à la messe\,; et à ces deux ordres du matin et
du soir jamais rien à faire, parce que, s'il y avait quelque ordre à
donner pour la guerre, pour une revue, pour un voyage, etc., cela se
passait toujours du roi au capitaine, ou si la chose pressait, et qu'il
n'y frit pas, à l'officier de quartier. Par ce court détail je ne voyais
point d'utilité pour la guerre ni pour le service, encore moins pour
celui de, la cour, ni {[}pour{]} sa décoration, à entretenir des troupes
si chères, et qui, à la valeur près, n'étaient bonnes que pour la
magnificence et la décoration des revues, auxquelles le feu roi ne
s'était que trop plu.

Question après de la manière de s'en soulager. Rien de plus aisé pour la
gendarmerie\,: la réformer, laisser crier les intéressés, continuer une
pension aux maréchaux des logis, et rembourser toutes les charges. Pour
y parvenir, s'imaginer après la réforme qu'elle n'est point faite, faire
en tout genre de dépense pour la gendarmerie les mêmes fonds que si elle
subsistait, rembourser de cette somme tous les ans un nombre de charges
en entier, et continuer les appointements de toutes jusqu'au jour de
leur remboursement, le rendre libre de toute dette qui n'aurait point
dessus des hypothèques spéciales, promettre (et tenir parole) à ceux qui
seraient mestres de camp et brigadiers de la préférence pour des
régiments\,; moyennant quoi, en trois ans ou quatre au plus, on serait
soulagé de toute cette dépense.

Pour ce qui est des gens d'armes, des chevau-légers, je sentis bien la
difficulté de la faiblesse de M. le duc d'Orléans pour le prince de
Rohan et le duc de Chaulnes, qui les commandaient. Je proposais la même
forme que je viens d'expliquer pour la gendarmerie, et je dis au
chancelier que c'était son affaire pour ôter ce nombre d'exempts de
taille et d'autres impositions, et cette quantité de lettres d'État, la
plupart très indirectes, qui, pour de l'argent que les plaideurs
donnaient à des gens d'armes ou à des chevau-légers, se mêlaient sans
intérêt, dans leurs affaires sous quelque couleur forcée, et arrêtaient
de leur chef les procédures et les jugements tant qu'il leur plaisait.
Pour les mousquetaires, la difficulté des capitaines n'était pas la
même, mais la manière de réformer et de rembourser {[}était{]} pareille.
Les huit brigades d'augmentation dans les gardes du corps n'étaient pas
une dépense en comparaison de l'épargne qu'on eût faite. Ceux-là au
moins auraient servi utilement à la guerre et à la cour.

Je trouvais leur guet trop faible, outre qu'on pouvait remettre cette
augmentation à l'ouverture d'une guerre ou au mariage du roi. Les deux
hôtels des mousquetaires les auraient logés dans Paris, chacun à leur
tour, où on aurait eu des troupes plus nombreuses et plus sages que
cette jeunesse à qui il fallait des gouverneurs. De plus, il pouvait y
avoir des temps difficiles où la faiblesse du guet est un grand
inconvénient, et où de l'augmenter en est un autre qui marque de la
crainte et enhardit ceux qui se proposent d'en donner, et dans d'autres
temps où il vient un dauphin, une dauphine et des fils de France qui
n'ont pas encore leur maison, le guet, au nombre qu'il est, et qui ne
peut être plus fort par rapport à la force des compagnies, ne peut
suffire au service, et n'y suffisait même pas par cette raison du temps
du feu roi, qu'il était plus nombreux, parce que les compagnies étaient
plus nombreuses. Il en arriverait une augmentation d'escadrons de gardes
du corps pour la guerre, qui répareraient en grande partie et bien moins
chèrement ceux des gens d'armes, chevau-légers, mousquetaires et
gendarmerie, dont le service serait sans embarras et se ferait bien
mieux, étant d'un même corps.

Enfin on éviterait, en réformant les mousquetaires, d'autres
inconvénients qui n'y sont compensés d'aucun avantage. On en a voulu
faire une école militaire, et y faire passer sans exception toute la
jeunesse qui demande de l'emploi. Or, cette école n'apprend rien pour la
guerre ni pour la discipline des troupes\,; on n'y apprend que
l'exercice et à escadronner, à obéir, et force pédanteries, dont on se
moque tout bas en attendant qu'on en sorte et qu'on puisse en rire tout
haut. Ainsi cette jeunesse passe le temps d'une année au moins, et
souvent davantage, à se débaucher dans Paris et à y dépenser très
inutilement\,; et quand elle entre dans les troupes, elle y est neuve à
tout, comme si elle sortait de sa province, et c'est alors qu'elle
commence à apprendre utilement et qu'elle oublie tout ce qu'elle a
appris d'inutile. Les détachements qui vont à la guerre ne l'instruisent
pas davantage. Ils y servent en simples maîtres, ou, s'il y a des
attaques à un siège, en simples grenadiers. Or la jeunesse noble,
beaucoup moins l'illustre, qui est à la vérité destinée à la guerre et à
tous ses hasards, ne l'est pas à ce genre de service\,; et c'est en
abuser d'une façon barbare que de la prodiguer en troupes au service de
simples maîtres et de simples grenadiers.

Avant l'invention de cette étrange mode, la jeunesse ne perdait point
ainsi son temps, et n'était point prodiguée à tas à des attaques
d'ouvrages. Chacun d'elle avait un parent ou un ami de son père, avec
qui il se mettait cadet, et qui en prenait soin pour tout. Ils
devenaient bientôt officiers, et toujours sous les mêmes yeux. Cela
faisait des enfants du corps, et de ces corps une famille\,; et le soin
et la dépendance du jeune homme le préservaient d'une infinité
d'inconvénients, lui apprenaient à vivre, à s'instruire, à se conduire,
et en avançant ainsi, à devenir bons officiers, et capables d'en élever
d'autres comme eux-mêmes l'avaient été. Il est vrai que la beauté des
revues et des camps de plaisir et de magnificence ne serait plus la
même. Mais le feu roi n'était plus, et c'était un gain, à bien de
différents égards, que d'en perdre l'usage et de se bien garder de le
renouveler.

Le chancelier goûta infiniment toutes ces raisons. Mais quand nous
discutâmes ensuite, non le moyen de les persuader au régent, parce que
leur évidence était palpable, mais d'exécuter cette réforme, nous
convînmes aisément que nous ne viendrions jamais à bout de lui en
inspirer la résolution, ou que, s'il la prenait, contre notre espérance,
jamais les cris et les brigues des intéressés ne la lui laisseraient
exécuter. Cette prodigieuse faiblesse, qui perdit constamment une
régence qui aurait pu être si belle, si utile au royaume, si glorieuse
au régent, et dont les suites auraient été en tout d'un aussi grand
avantage, fut l'obstacle continuel à tout bien, et la cause perpétuelle
de la douleur de tous ceux qui désiraient sincèrement le bien de l'État
et la gloire du régent. Nous comprîmes enfin, le chancelier et moi,
qu'en proposant au régent une réforme si utile, elle ne se ferait
jamais, et que tout le fruit que nous retirerions de notre zèle serait
la haine de tant d'intéressés. Cette considération nous ferma donc là
bouche, et la chose en demeura entre nous deux.

Le long et ennuyeux travail du comité étant fini, il s'assembla
plusieurs fois chez M. le duc d'Orléans, où les dernières résolutions
furent prises fort unanimement. Les principales furent de ne point
toucher aux rentes de l'hôtel de ville\,; d'ôter le dixième, tant pour
tenir la parole si solennellement donnée en l'imposant de le supprimer à
la paix, que parce que, dans le fait, on n'en pouvait presque plus rien
tirer. Le fonds de un million deux cent mille livres destiné par an aux
bâtiments fut réduit à la moitié\,; {[}il y eut{]} plusieurs
retranchements de pensions fort inutilement données, et des diminutions
sur d'autres. Les menus plaisirs du roi de dix mille livres par mois, et
sa garde-robe à trente-six mille livres, furent réduits, les menus
plaisirs à moitié, la garde-robe à vingt-quatre mille livres. À l'âge du
roi tout cela s'en allait en pillage. Il y eut encore d'autres choses
retranchées et de la diminution sur les intérêts des sommés empruntées
au denier vingt.

Les chefs et présidents des conseils furent mandés à un conseil
extraordinaire du jeudi après-dîner, 19 août, où lé duc de Noailles
rendit compte de ce qui avait été concerté. Il fut réglé que l'édit en
serait dressé en conformité, pour être envoyé enregistrer au parlement.
Le lendemain le comité s'assembla encore chez M. le duc d'Orléans pour
voir le projet d'édit et le perfectionner.

Le premier président avait un démêlé avec les enquêtes et les requêtes
sur le nombre et le choix des députés quand il s'agirait d'en nommer aux
occasions qui le demanderaient. La grand'chambre semblait partiale pour
le premier président, parce que, maître du choix dans cette chambre, il
voulait exclure les autres, qui cependant ne sont pas moins qu'elle des
chambres du parlement. Après bien du bruit, ils convinrent que la
grand'chambre aurait seule sept députés, et les cinq chambres des
enquêtes et les deux des requêtes chacune un, ce qui en fait sept
autres\,; ainsi à elles sept la moitié des députés, et la grand'chambre
seule une autre moitié. Cette affaire ne se passa pas bien pour le
premier président, qui demeura assez mal avec la compagnie, laquelle
depuis longtemps le regardait comme un double fripon, dont le métier
était de tirer tant qu'il pouvait d'argent de M. le duc d'Orléans.

L'édit porté au parlement lui parut une trop belle occasion pour n'en
pas profiter. Messieurs opinèrent qu'il leur fallait faire voir un
détail des revenus et des dépenses du roi avant qu'ils décidassent s'ils
enregistreraient l'édit. Le premier président alla en rendre compte au
régent, et le lendemain après dîner, il reçut une députation du
parlement, à laquelle il dit qu'il ne souffrirait point qu'il fût donné
la moindre atteinte à l'autorité royale, tandis qu'il en serait le
dépositaire. Les quatorze commissaires députés s'assemblèrent. Les gens
du roi furent ensuite au Palais-Royal. Le parlement s'assembla ensuite,
et enregistra la suppression du dixième, de beaucoup de francs salés, et
d'autres articles. Sur ceux qui restaient, M. le duc d'Orléans eut la
faiblesse, poussé par la frayeur qui avait saisi le duc de Noailles, et
son désir de faire sa cour au parlement, de les faire discuter par ce
duc en sa présence, le dimanche matin 5 septembre, aux quatorze députés
du parlement, et il y fit aussi entrer le sieur Law pour leur expliquer
les avantages qui en reviendraient à la compagnie du Mississipi
\footnote{Les lettres patentes qui établissaient la compagnie d'Occident
  ou du Mississipi sont de la fin d'août 1717\,; elles furent
  enregistrées au parlement de Paris le 6 septembre de la même année.}.
De tout cela pas un mot au conseil de régence, et, s'il se pouvait,
beaucoup moins à moi en particulier\,; aussi n'en dis-je pas une parole
à M. le duc d'Orléans, suivant ma coutume, quand il s'agissait du
parlement.

Il s'assembla le lendemain matin, et après-dîner, pour entendre le
rapport des commissaires, et comme il ne fut pas encore pour achever
l'enregistrement, et qu'il était le 6 septembre, il fut prorogé par le
roi jusqu'au 14. Il demanda jour et heure au régent pour venir faire des
remontrances au roi. Ils y vinrent le jeudi 9\,; le régent les présenta,
et le roi leur dit que le chancelier leur expliquerait sa volonté. La
députation fut nombreuse. Enfin, le lendemain matin vendredi 10, l'édit
entier fut enregistré avec une déclaration du roi qui en expliquait
quelques endroits. Aussitôt après, le parlement eut liberté d'entrer en
vacance, et les conseils en eurent aussi une de trois semaines. Ainsi,
le parlement, qui se prétend le tuteur des rois mineurs et des majeurs
aussi quand il peut, voulut montrer ici que ce n'est pas en vain, et en
fit une fonction solennelle.

La faiblesse du maître et du ministre à qui il eut affaire ne servit à
rien à tous deux. Le parlement s'enorgueillit jusqu'à l'ivresse,
l'autorité du régent déchut\,; il ne tarda pas à s'apercevoir de l'un et
de l'autre. Pour le duc de Noailles, qui mourait toujours de peur de la
robe à qui il était accoutumé de faire une cour servile, il ne s'en fit
que mépriser, et il ne fut pas longtemps à l'éprouver. À l'égard de Law,
qui pensait mieux là-dessus, il ne put qu'obéir. Le régent, en tenant
bon et se moquant d'une prétention aussi dangereuse qu'inepte, aurait
hautement forcé le parlement à enregistrer son édit, ayant le public
derrière lui pour la suppression du dixième et d'autres points qui
l'intéressaient si fortement. Ce prince ne sut pas profiter de cet
avantage, dont il eût pu tirer un si utile parti, et il encouragea, au
contraire, et ouvrit la voie à ceux qui par divers intérêts se
réunissaient entre eux, pour brouiller, réduire son autorité, et le
mettre au point de dépendre de leurs volontés, qui n'étaient pas, à
beaucoup près, de lui laisser le gouvernement des affaires, et qui
bientôt lui en donnèrent beaucoup.

L'anniversaire qui se fait tous les ans à Saint-Denis pour le roi
dernier mort produisit une prétention toute nouvelle. La reconnaissance
n'est plus à la mode depuis longtemps. Il y eut très peu de gens de la
cour\,; M. du Maine et son second fils, quelque peu d'évêques et le
cardinal de Polignac. Ces évêques s'avisèrent de vouloir avoir des
carreaux\,: le rare est qu'il n'y eut que le cardinal de Polignac qui
s'y opposa, et qui l'empêcha, sur quoi les évêques osèrent s'en aller et
se plaindre au régent. Jamais ils n'en avaient eu ni prétendu, et j'ai
dit ailleurs que la règle des honneurs c'est que chacun est en présence
du corps ou de sa représentation comme il était en présence de cette
même personne vivante\,; or, les évêques n'ont jamais eu ni imaginé
d'avoir des carreaux en aucun lieu où est le roi. Ces messieurs se
pouvaient contenter de leurs conquêtes sur les évêques pairs en ces
cérémonies, à qui ils ne voulurent pas souffrir leurs carreaux, étant
avec eux en corps de clergé, et qui l'emportèrent sur la faiblesse des
prélats pairs. C'était bien là une preuve que les autres évêques n'en
avaient jamais eu ni prétendu. Ils pouvaient encore se souvenir qu'il
n'y avait pas un grand nombre d'années qu'ils y étaient sur la même
ligne avec les cardinaux, derrière qui, même s'il n'y en avait qu'un
seul, ils avaient toujours été placés auparavant.

Le mariage de Chalmazel, aujourd'hui premier maître d'hôtel de la reine,
et qui est homme de condition, avec une soeur d'Harcourt, fit renaître
une autre prétention, quoique solennellement et contradictoirement jugée
et condamnée par le feu roi, entre les femmes et les filles des princes
du sang, comme on l'a vu en son lieu, et comme le jugement en avait sans
cesse été exécuté depuis. M\textsuperscript{me} la duchesse d'Orléans
fit signer à M\textsuperscript{lle}s ses filles ce contrat de mariage
avec elle, et immédiatement après elle\,; en sorte que les femmes des
princes du sang ne trouvèrent plus d'espace lorsqu'on leur présenta ce
contrat où elles pussent signer au-dessus de ces princesses filles.
M\textsuperscript{me} la duchesse d'Orléans au désespoir du jugement du
feu roi, comme on l'a vu en son temps, n'avait pu se résoudre de
démordre de sa prétention qu'elle conserva toujours \emph{in petto},
dont le but était de faire de ses enfants un ordre nouveau,
d'arrière-petits-fils de France, dont le rang serait supérieur à celui
des princes du sang, et de s'élever par là imperceptiblement elle-même à
celui des fils et filles de France. La régence de M. le duc d'Orléans
lui parut un temps favorable à réussir en cette entreprise.

Elle s'y trompa. Les princes du sang et les princesses leurs femmes
firent grand bruit. Elles portèrent leurs plaintes à M. le duc
d'Orléans, le règlement du feu roi à la main\,; M. le duc d'Orléans leur
fit des excuses, et leur promit que ce dont elles se plaignaient
n'arriverait plus. Il ne s'était jamais mis cette prétention dans la
tête\,; il avait laissé faire M\textsuperscript{me} la duchesse
d'Orléans du temps du feu roi, pour ne se donner pas la peine de la
contrarier dans une fantaisie qu'elle avait fort à coeur\,; il ne se
soucia en aucune façon de la condamnation que le feu roi en fit, et ne
pensa jamais à en revenir. D'ailleurs il était fatigué des riottes qui
se perpétuaient sur des riens entre M\textsuperscript{me} la duchesse de
Berry et M\textsuperscript{me} la duchesse d'Orléans, et bien plus
encore de ne pouvoir apaiser la dernière sur ce qui avait été jugé entre
les princes du sang et ses frères sur l'habilité de succéder à la
couronne. Ainsi M\textsuperscript{me} la duchesse d'Orléans eut tout le
dégoût de son entreprise, que M. le duc d'Orléans ne s'embarrassa pas de
lui donner.

Dans sa mauvaise humeur, dégoûtée de son appartement de Montmartre, d'où
elle ne voyait que des toits, des minuties des religieuses pour des
clefs et des passages, de l'éloignement des jardins qu'elle y avait fait
ajuster avec beaucoup de goût et de dépense, elle acheta la maison de
Bagnolet, et peu à peu plusieurs voisines, dont elle fit un lieu immense
et délicieux. Madame passait presque toute l'année à Saint-Cloud\,;
c'était aussi la seule maison de campagne à portée qu'eût M. le duc
d'Orléans. Elle en voulut une qui ne fût qu'à elle et que pour elle, et
dont elle fût à portée de jouir à tout moment.

Le duc de Noailles fit une galanterie aux dépens du roi à son ami le
chancelier. Il y avait à Versailles et à Fontainebleau une maison pour
la demeure du chancelier, qu'on appelait \emph{la Chancellerie\,;} mais
il n'y en avait jamais eu à Paris, où jusqu'alors les chanceliers
avaient toujours logé à leurs dépens chez eux. Bourvalais, un des plus
riches traitants et des plus maltraités par la chambre de justice, fut
dépouillé d'une superbe maison qu'il avait bâtie dans la place de
Vendôme, et d'une maison de campagne à Champ, qu'il avait rendue
charmante, et que, d'une maison de bouteille, il avait fait chef-lieu
d'une grande et belle terre à force d'acquisitions.
M\textsuperscript{me} la princesse de Conti eut Champ pour une pièce de
pain qu'elle donna à La Vallière, et la maison de Paris devint la
chancellerie, qui, outre le don du roi, lui coûta fort cher par tout ce
que d'Antin y fit pour faire sa cour au chancelier qui jusqu'alors était
demeuré très mal logé dans son ancienne maison de la rue Pavée, qu'il
louait auprès de celle de son père.

Le chiaoux, principalement venu pour débaucher le prince Ragotzi, y
réussit. Jamais on ne vit mieux qu'en lui la petitesse des personnages à
qui le hasard a fait faire grand bruit dans le monde quand ils sont
rapprochés. Ragotzi était un homme sans talents et sans esprit que des
plus communs, grand homme de bien et d'honneur, d'une pénitence
également austère et sincère qui, différente de celle des camaldules
chez qui il était retiré, n'était guère moins dure, qui y gardait une
solitude véritable et suivie, qui n'en sortait que par des bienséances
nécessaires, et qui, sans rien de contraint ni de déplacé, vivait,
lorsqu'il était parmi le monde, comme un homme qui en est, et qui
toutefois se souvient bien qu'il n'y est que par emprunt. De grandes
aumônes étaient jointes à sa pénitence, une grande règle dans son
domestique et dans sa maison, et cependant avec toutes les décences d'un
fort grand seigneur. Il est inconcevable comment un homme qui, après
tant de tempêtes, goûte un tel port, se rejette de nouveau à la merci
des vagues, et trouve des gens de bien qui, consultés par lui de bonne
foi, lui conseillent de s'y rembarquer\,; et mille fois plus
inconcevable encore comment il s'est pu conserver dans son même genre de
vie jusqu'à la mort, pendant plusieurs années, et chez les Turcs, et
parmi un faste et des dissipations qu'il ne put éviter. Il sut avant son
départ la défaite des Turcs dont on parlera tout à l'heure, et ne laissa
de poursuivre sa pointe. Arrivé à Constantinople et à Andrinople, il y
fut reçu et traité avec une grande distinction, mais sans avoir pu y
être d'aucun usage, à cause du changement des conjonctures. Il y demeura
peu, et s'en alla habiter un beau château sur la mer Noire, à quinze ou
vingt lieues de Constantinople, magnifiquement meublé pour lui par le
Grand Seigneur, où la chasse et la prière partagèrent presque tout son
temps au milieu d'une nombreuse suite. Les convenances entre l'empereur
et la Porte le tirèrent après quelques années d'un voisinage qui
inquiétait la cour de Vienne. Il fut envoyé dans une des plus agréables
îles de l'Archipel, où il vécut comme il faisait sur les bords de la mer
Noire, avec la même splendeur, avec la même piété, et y est mort au bout
de quelques années, laissant deux fils fort au-dessous du rien. Il
écrivait rarement au comte de Toulouse, aux maréchaux de Villeroy et de
Tessé, à M\textsuperscript{me} de Dangeau, et à quelques autres amis
d'ici, en homme qui aurait mieux aimé y être demeuré, mais toutefois
content de son sort, et tout abandonné à la Providence.

On apprit que le prince Eugène, ayant formé le siège de Belgrade, s'y
était trouvé assiégé lui-même par une puissante armée de Turcs,
commandée par le grand vizir, qui le serait de si près entre elle et la
place, qu'ils étaient à vue, et quelle ôtait à celle de l'empereur tous
moyens de mouvements et de subsistance, et qui en deux jours se
retrancha parfaitement et très régulièrement. Dans cette extrémité
subite, le prince Eugène ne vit de ressource que dans le hasard d'une
bataille. Il profita de la sécurité des Turcs, qui n'imaginèrent jamais
qu'avec Belgrade derrière lui, et nulle retraite, il osât les attaquer
dans leurs retranchements. Un grand et long brouillard couvrit ses
promptes dispositions. Il commença son attaque un peu avant qu'il fût
dissipé, au moment que les Turcs s'y attendaient le moins, et il eut le
bonheur de remporter une victoire complète le 16 août, en quatre heures
de temps. M. le comte de Charolais et le prince de Dombes s'y
distinguèrent. Estrades eut une jambe emportée auprès de lui, dont il
mourut peu après\,; et Villette, qui s'était battu à Paris avec Jonsac,
y fut tué. Les Turcs y perdirent infiniment de monde, tous leurs canons
et tous leurs bagages. Ils se retirèrent avec assez de confusion.
Belgrade capitula aussitôt. Le prince Eugène perdit aussi
considérablement, et plusieurs officiers distingués.

Il profita le reste de la campagne d'une victoire qui l'en laissa
maître, et dans laquelle il eut divers succès dont le plus grand pour
l'empereur fut de reculer sa frontière aussi loin, et de faire avec les
Turcs une paix prompte et avantageuse.

La mairie de Bordeaux de vingt mille livres de rente qu'avait d'Estrades
après son père, et le maréchal son grand-père, fut donnée à son fils qui
s'était trouvé à la bataille.

J'ai expliqué en son temps quelle était M\textsuperscript{me} de Mouchy,
favorite confidente de M\textsuperscript{me} la duchesse de Berry, et
quel était Rion, son favori d'une autre sorte. Elle voulut doubler en
leur faveur les charges de dame d'atours et de premier écuyer,
qu'avaient M\textsuperscript{me} de Pons et le chevalier d'Hautefort,
qui en furent fort affligés. Il y avait longtemps que
M\textsuperscript{me}s de Beauvau et de Clermont s'ennuyaient des
préférences et des façons de M\textsuperscript{me} de Mouchy, et
qu'elles ne restaient dans la maison que par amitié et par considération
pour M\textsuperscript{me} de Saint-Simon. M\textsuperscript{me} de
Mouchy n'y avait point de place\,; elles ne purent soutenir de la voir
tout à coup dame d'atours, elles vinrent trouver M\textsuperscript{me}
de Saint-Simon, et lui dire que cela était plus fort qu'elles. Elles
allèrent parler à M. le duc d'Orléans, avec lequel elles ne se
contraignirent pas sur M\textsuperscript{me} de Mouchy, et quittèrent
leurs places avec grand éclat, dont M\textsuperscript{me} la duchesse de
Berry fut vivement piquée. Il en vaqua en même temps une troisième par
la mort de la jeune M\textsuperscript{me} d'Aydie, soeur de Rion.
M\textsuperscript{me}s de Laval et de Brassac furent choisies pour ces
places dont leur peu de bien avait besoin. C'était aussi des femmes de
mérite et de nom qui, en laissant regretter les autres, pouvaient aussi
les remplacer. La première était sueur du chevalier d'Hautefort, l'autre
fille du maréchal de Tourville.

M. le duc d'Orléans donna trois mille livres de pension à un gentilhomme
nommé Marsillac, dont les mains étaient fort estropiées de blessures. Il
y aura lieu de parler de lui dans la suite, et de voir de plus en plus
que ce prince n était pas toujours heureux à placer ses bienfaits. Il
plaça mieux l'archevêché de Besançon qu'il donna à l'abbé de Mornay, qui
faisait très dignement et capablement l'ambassade de Portugal depuis que
le feu roi l'y avait envoyé. C'était le frère de MM. de
Grammont-Franc-Comtois, et lieutenants généraux\,; il l'avait après son
oncle, et qui était mort\,; et M. le duc d'Orléans après quelques
longueurs avait obtenu pour le roi le même indult pour la Franche-Comté
que le feu roi avait eu. Il donna à l'abbé de Tressan, évêque de Vannes,
son premier aumônier, l'évêché de Nantes, vacant par la mort d'un
Beauvau qui l'avait possédé fort longtemps, et je lui proposai l'abbé de
Caumartin pour Vannes, à qui il le donna, et qui est mort depuis évêque
de Blois. C'est le même dont j'ai parlé à propos de M. de Noyon et de sa
réception à l'Académie française. Il accorda l'abbaye de Montmartre à
M\textsuperscript{me} la duchesse d'Orléans pour M\textsuperscript{me}
de Montpipeau de la maison de Rochechouart, et l'agrément de la charge
de secrétaire du cabinet du président Duret, à Verneuil, qui a eu depuis
la plume et une charge d'introducteur des ambassadeurs. Son père avait
été lieutenant des gardes de Monsieur\,; son nom est Chassepoux, sieur
de Croquefromage\,; celui de sa femme est Bigre. Je n'ai pu retenir le
ridicule de ces noms.

Le prince de Dombes, et ce qui était allé en Hongrie de François en
revinrent, excepté M. le comte de Charolais.

Le duc de Ventadour mourut retiré, depuis quelques années, aux
Incurables, séparé de sa femme depuis un grand nombre d'années, ne
laissant qu'une très riche héritière mariée au prince de Rohan, qui
s'était chargé de tous ses biens et de ses dettes moyennant quarante
mille livres de rente qu'il lui payait par quartier. C'était un homme
fort laid et fort contrefait qui, avec beaucoup d'esprit et de valeur,
avait toujours mené la vie la plus obscure et la plus débauchée. Par sa
mort son duché-pairie fut éteint.

Moncault, soldat de fortune, et qui la devait au maréchal de Duras et à
son esprit, mourut en même temps. Il était lieutenant général et
gouverneur de la citadelle de Besançon. Il avait su s'enrichir et marier
son fils à une fille d'Armenonville.

Dès l'hiver dernier on me pressa de présenter mes enfants au roi et au
régent, et il est vrai qu'ils étaient en âge où cela ne pouvait plus se
différer. Néanmoins j'y résistai, parce que je voulus leur apprendre ce
qu'ils devaient à la mémoire de Louis VIII, qui nous doit être si
précieuse et si sacrée, et que les prémices de leurs hommages lui
fussent rendues. Je les menai donc à son anniversaire à Saint-Denis, où
je ne manquais jamais à l'exemple de mon père, et ce devoir si principal
pour nous rempli, je les présentai. Je trouvai en ce temps-ci deux
régiments à vendre, tous deux de cavalerie, et gris. Le régent m'en
accorda l'agrément, et je les achetai pour eux du duc de Saint-Aignan,
ambassadeur en Espagne, et de Villepreux qui se retirait par vieillesse.

L'abbé Dubois partit dans le même temps pour retourner à Londres, et on
apprit que le comte de Peterborough avait été arrêté voyageant en Italie
par ordre du légat de Bologne. C'était un homme fort remuant, qui toute
sa vie s'était mêlé de beaucoup d'affaires en Angleterre et de beaucoup
d'autres au dehors, tant de guerre que de paix et de différentes
intrigues, et à qui les plus grands et les plus fréquents voyages ne
coûtaient rien. Il avait la Jarretière, tantôt bien, tantôt mal avec le
gouvernement d'Angleterre, mais craint et ménagé.

\hypertarget{chapitre-iv.}{%
\chapter{CHAPITRE IV.}\label{chapitre-iv.}}

1717

~

{\textsc{Mépris d'Albéroni pour la détention de Molinez.}} {\textsc{-
Ses réflexions sur la situation de l'Europe.}} {\textsc{- Son dégoût de
Beretti.}} {\textsc{- Conduite et pensée de cet ambassadeur.}}
{\textsc{- Inquiétude et avis de Beretti.}} {\textsc{- Différents
sentiments sur l'empereur en Angleterre.}} {\textsc{- Manège intérieur
de cette cour.}} {\textsc{- Même diversité de sentiments sur l'union
établie entre le régent et le roi d'Angleterre.}} {\textsc{-
Empressement et offres des ministres d'Angleterre au régent pour l'unir
avec l'empereur et y faire entrer l'Espagne.}} {\textsc{- Saint-Saphorin
employé par le roi d'Angleterre à Vienne\,; quel.}} {\textsc{- Son avis
sur les traités à faire.}} {\textsc{- Roi de Prusse suspect à Vienne et
à Londres.}} {\textsc{- Son caractère et sa conduite.}} {\textsc{-
Ministres hanovriens dévoués à l'empereur, qui veut tenir le roi
d'Angleterre en dépendance.}} {\textsc{- Complaisance de ce dernier à
lui payer un reste de subsides, qui excite du bruit en Angleterre et
dans le nord.}} {\textsc{- Hauteur de l'empereur sur Peterborough.}}
{\textsc{- Secret profond de l'entreprise sur la Sardaigne.}} {\textsc{-
Conseils du duc de Parme au roi d'Espagne.}} {\textsc{- Colère du pape
sur l'accommodement signé en Espagne.}} {\textsc{- Contretemps du
Prétendant.}} {\textsc{- Adresse hardie d'Acquaviva.}} {\textsc{-
Congrégation consultée favorable à Albéroni, contraire à Aldovrandi, qui
excuse Albéroni sur la destination de la flotte espagnole.}} {\textsc{-
L'entreprise de l'Espagne, au-dessus de ses forces sans alliés, donne
lieu à beaucoup de divers raisonnements.}} {\textsc{- Albéroni se moque
d'Aldovrandi et de Mocenigo.}} {\textsc{- L'entreprise généralement
blâmée, colorée de l'enlèvement de Molinez.}} {\textsc{- Vanteries et
fausseté impudente d'Albéroni.}} {\textsc{- Inquiétude pour la Sicile.}}
{\textsc{- Le secret confié au seul duc de Parme.}} {\textsc{- Ses avis
et ses conseils.}} {\textsc{- Albéroni fait cardinal dans le consistoire
du 12 juillet.}} {\textsc{- Cris sur sa promotion.}} {\textsc{- Giudice
s'y distingue.}} {\textsc{- Malaise du roi d'Angleterre dans sa cour et
dans sa famille.}} {\textsc{- Comte d'Oxford absous en parlement.}}
{\textsc{- Éclat entre le roi d'Angleterre et le prince de Galles.}}
{\textsc{- Inquiétude sur l'entreprise d'Espagne moindre en Hollande
qu'à Londres.}} {\textsc{- Applaudissements et avis de Beretti.}}
{\textsc{- Son intérêt personnel.}} {\textsc{- Les Impériaux somment le
roi d'Angleterre de secours avec peu de succès.}} {\textsc{- Caractère
du comte de Peterborough.}} {\textsc{- Secret profond de la destination
de l'entreprise de l'Espagne.}} {\textsc{- Double hardiesse
d'Albéroni.}} {\textsc{- Plaintes et menaces de Gallas, qui font
trembler le pape.}} {\textsc{- Frayeur de toute l'Italie.}} {\textsc{-
Hauteur et sécurité d'Albéroni.}} {\textsc{- Aldovrandi veut persuader
que l'entreprise se fait malgré Albéroni.}} {\textsc{- Mouvements
partout contre cette entreprise, et opinions diverses.}}

~

L'accommodement des différends entre les cours de Rome et de Madrid
avait été conclu entre Aldovrandi et Albéroni, et signé par eux. Il
avait été porté au duc de Parme par un courrier dépêché de l'Escurial le
17 juin, et les deux plénipotentiaires attendaient avec impatience
l'approbation du pape sur un ouvrage dont l'élévation de l'un et la
fortune de l'autre dépendaient également. Dans cette attente Albéroni
s'inquiétait peu de la prison de Molinez. Il l'accusait d'imprudence
d'avoir passé par Milan, et il disait qu'il n'y aurait pas grand mal
quand il n'arriverait jamais en Espagne. Quelque occupé qu'il fût de se
voir enfin revêtu incessamment de la pourpre, il ne laissait pas que de
tenir les yeux ouverts sur la situation de l'Europe. Il n'était point
alarmé de la trouver pleine de semences de troubles\,; il mettait le
point de sagesse à savoir en profiter quand ils arriveraient.

L'affaire des bâtards et celle de la constitution étaient sur la France
la matière de ses réflexions. Son dessein, depuis longtemps, était de
fortifier le roi d'Espagne pour les événements à venir par des alliances
avec l'Angleterre et la Hollande. Il s'était ralenti sur la première,
jugeant que les Anglais ayant un intérêt capital d'assurer leur commerce
avec l'Espagne, ils feraient les premières avances, et qu'il serait
dangereux de leur marquer trop d'empressement. Il se persuadait que la
Hollande désirait sincèrement de faire une ligue avec l'Espagne, dont la
seule crainte de l'empereur retardait l'accomplissement.

Beretti, son homme de confiance, lui était devenu insupportable. Il se
repentait de l'avoir choisi pour l'ambassade de Hollande. Il manda au
duc de Parme que depuis qu'il était dans cet emploi il s'était fait
connaître pour un homme vain, ardent, d'une vivacité dangereuse,
difficile à corriger, injuste en ses demandes, importun pour les
obtenir. Il ne voulut pas même laisser Beretti dans l'ignorance de tout
ce qu'il pensait de lui\,; car après lui avoir reproché souvent la
prolixité de ses lettres et l'inutilité de ses raisonnements, il lui
déclara franchement que le roi d'Espagne se passerait très bien
d'entretenir à grands frais un ambassadeur en Hollande, et qu'il
suffirait à son service d'avoir un bon espion à la Haye.

Mais plus il recevait de ces reproches, plus il vantait ses services
d'avoir ouvert les yeux aux principaux de la république sur le danger
des desseins et de la grandeur de l'empereur, dont il prétendait avoir
fait échouer les négociations, et il était vrai qu'il avait obtenu
là-dessus les assurances les plus positives des membres des États les
plus accrédités. Il était en même temps persuadé que les Anglais étaient
portés à favoriser l'alliance de l'empereur avec les Provinces-Unies. Il
prétendait que Stanhope, qui avait été longtemps à la suite de
l'empereur, conservait pour lui un attachement personnel, que Cadogan
était dans les mêmes sentiments, et bien plus encore Bernsdorff et
Bothmar, ministres hanovriens du roi d'Angleterre.

Beretti, peu rassuré par les protestations de Chateauneuf que la France
ne concourrait jamais à l'alliance des États généraux avec l'empereur,
s'alarmait d'avoir ouï dire que cet ambassadeur et l'abbé Dubois
seraient chargés de traiter l'accommodement en Hollande entre l'empereur
et l'Espagne. Il croyait cette négociation très prochaine sur ce que
Widword, envoyé d'Angleterre à la Haye, lui avait dit que Sunderland lui
mandait que Stairs avait communiqué un plan du traité au régent, que ce
prince l'avait approuvé, et qu'il était prêt à contribuer efficacement
au succès de ce projet. Ainsi Beretti pressait infiniment pour qu'on lui
envoyât de Madrid des instructions de la manière dont il aurait à se
conduire si cette négociation s'ouvrait à la Haye. Il craignait, ou en
faisait le semblant, que le roi d'Espagne ne fût trahi de tous côtés,
peut-être davantage que cette négociation ne sortît de ses mains pour
passer en celles des ministres de France.

L'empereur avait donné ses pouvoirs au marquis de Prié et au baron
d'Heems, pour terminer ce qui restait de différends avec les États
généraux sur le traité de la Barrière, et pour traiter une alliance avec
eux et avec l'Angleterre. Ces deux affaires paraissaient encore
éloignées, surtout celle de l'alliance. Beretti en fit tant de plaintes
et de bruit, que le Pensionnaire s'en plaignit à Widword. Son inquiétude
était extrême de ne rien recevoir de Madrid. Enfin, pour forcer,
Albéroni à s'expliquer, il lui manda qu'il était souvent pressé par
Widword de lui rendre enfin réponse des intentions de l'Espagne sur la
négociation de paix qu'il s'agissait d'entamer avec l'empereur, et
s'étendait sur sa réponse en termes généraux et en de grands
raisonnements qu'il avait faits à ce ministre, dont il se vantait
d'avoir la confiance et de ceux de Londres aussi, même de quelques-uns
qu'il ne connaissait pas, pour se faire croire le plus propre à conduire
cette négociation, qu'il mourait de peur de se voir enlever. Il assura
qu'il savait du même Widword que les Impériaux convenaient d'assurer aux
enfants de la reine d'Espagne la succession de Toscane\,; qu'ils
voulaient réserver le point de Mantoue à discuter lors du traité\,;
qu'on n'en pouvait demander davantage sans prétendre tout mettre en
préliminaires\,; que Widword lui avait dit que le roi d'Angleterre avait
grande impatience de voir si les intentions de l'empereur étaient
sincères ou artificieuses sur cette paix\,; que le régent n'en avait pas
une moindre, et que, si l'empereur usait de mauvaise foi, la France,
l'Angleterre et la Hollande prendraient ensemble les mesures nécessaires
pour le contraindre par la force à concourir au repos de l'Europe, parce
qu'il était de leurs intérêts de borner ses vastes desseins et sa trop
grande puissance en Italie et en Allemagne.

Georges avait autant lieu de craindre cette puissance démesurée, soit
comme prince de l'empire, soit comme roi d'Angleterre. Il ménageait avec
soin les bonnes grâces de l'empereur, auquel ses ministres allemands
étaient dévoués, et lui représentaient sans cesse le besoin qu'il avait
du chef de l'empire pour conserver les États qu'il avait enlevés à la
Suède, dont il n'avait d'autre titre que de les avoir achetés du
Danemark après qu'il s'en était emparé. Les Anglais pensaient
différemment. Ils auraient mieux aimé que leur roi fût moins puissant au
dehors de leurs îles, et il n'y avait pas lieu de se flatter qu'ils
voulussent l'aider à soutenir la querelle de Brème et de Verden aux
dépens de leur commerce avec la Suède.

Pour tâcher de rompre cet obstacle, Georges, étant à Hanovre la dernière
fois, s'était laissé persuader par ses ministres allemands de donner la
place de secrétaire d'État au comte de Sunderland, à condition qu'il le
servirait dans cette affaire. Mais ce comte, petit-fils de celui qui, en
la même qualité, avait si cruellement abusé de la confiance de Jacques
II, qu'il trahissait pour le prince d'Orange, ne fut pas plutôt de
retour en Angleterre, qu'il soutint qu'il était de l'intérêt de la
nation de presser la restitution de ces deux duchés, pour obtenir plus
promptement par là le rétablissement du commerce avec la Suède.

Quoique la cessation des hostilités entre cette couronne et celle
d'Angleterre fût également désirée des Anglais et des Hollandais,
Georges continuait à se rendre difficile à renvoyer Gyllembourg en
Suède, et à consentir à la délivrance du baron de Goertz de sa prison en
Hollande, dont les vaisseaux, arrêtés en Suède, animaient les villes de
commerce qui en souffraient considérablement, contre les délais de
Georges et la lâche complaisance des chefs de la république pour lui.

Widword n'espérait plus d'empêcher l'élargissement de ce ministre
suédois que par les offices du régent, dont le poids en Hollande et en
Angleterre faisait faire de grandes réflexions aux ministres d'Espagne
sur les mesures que le roi d'Angleterre et le régent prenaient ensemble
et sur leur intérêt de s'unir pour les événements à venir. Les Anglais
même en étaient peinés. Ils disaient librement que l'Angleterre n'avait
jamais été si malheureuse que dans les temps où elle s'était trouvée
unie avec la France. Les ministres d'Angleterre pensaient tout
autrement. Ils paraissaient travailler de bonne foi à rendre l'alliance
plus étroite, en y faisant entrer l'empereur. Ils pressaient le régent
d'y concourir pour ses propres intérêts, et l'assuraient que la cour de
Vienne était disposée à suivre le plan que Stanhope y avait donné pour
assurer la tranquillité de l'Europe. Ils souhaitaient que le roi
d'Espagne y voulût entrer. S'ils le refusaient, ils assuraient le régent
que l'empereur et le roi d'Angleterre prendraient avec Son Altesse
Royale les mesures nécessaires pour lui garantir ses droits sur la
couronne en cas d'ouverture de la succession. Ils offraient même
d'insérer dans le traité la clause de laisser le roi d'Espagne jouir
tranquillement des États qu'il possédait, et la faculté d'accéder à
l'alliance après qu'elle aurait été conclue, croyant que ce monarque, la
voyant faite, se désabuserait des espérances qu'il conservait
apparemment sur la couronne de France.

Un nommé Saint-Saphorin, Suisse du canton de Berne, fort décrié depuis
longtemps par plusieurs actions contre l'honneur et la probité, et par
ses manèges encore et ses déclamations contre la France, était celui
dont le roi d'Angleterre se servait à Vienne, et croyait se pouvoir
confier à lui. Il s'applaudissait d'avoir su conduire les choses au
point où elles en étaient. Il conseillait de ne pas songer au roi de
Prusse, quoique la France le désirât, mais d'attendre que tout fût réglé
et d'accord, parce qu'on aurait alors ce prince à bon marché. Il mandait
que la seule proposition d'y faire intervenir le roi de Prusse
alarmerait les Impériaux au point de renverser les bonnes dispositions
où les offices du roi d'Angleterre avaient mis l'empereur pour le
régent\,; que ses ministres avaient déjà dit que, s'ils s'apercevaient
que le régent voulût comme les forcer par les alliances qu'il
contracterait dans l'empire, ils rejetteraient toute proposition et
prendraient tout autre parti plutôt que de subir la loi qu'on leur
voudrait imposer, parce que enfin l'empereur ne s'était rendu aux
instances du roi d'Angleterre que par considération pour lui, et non par
la nécessité de ses affaires\,; qu'il était même persuadé que, demeurant
libre de tout engagement et attendant tranquillement les occasions
favorables de faire valoir ses prétentions, il trouverait des avantages
plus grands qu'en se pressant de traiter\,; qu'il fallait donc suivre le
sentiment de ces ministres de Vienne, achever premièrement l'alliance
avec la France et convenir après, de concert, du choix des princes qu'il
serait à propos d'y faire entrer. Alors l'empereur ne s'opposerait pas à
mettre le roi de Prusse dans ce nombre, s'il se gouvernait bien, mais
qu'il fallait compter que l'empereur romprait toute négociation, si
l'Angleterre et la Hollande insistaient à comprendre quelque autre
puissance dans l'alliance avant qu'elle fût signée. Les intentions du
roi de Prusse étaient également suspectes à Vienne et à Londres, parce
que son caractère était également connu dans les deux cours.

Ce prince, uniquement occupé de son intérêt, embrassait tous les moyens
propres à y parvenir. Souvent il se trompait dans le choix\,; mais la
route qu'il croyait la plus sûre était d'exciter les troubles dans
l'Europe. Il se flattait d'être assez habile pour en profiter, et dans
cette confiance, il entreprenait légèrement et se désistait encore plus
légèrement lorsqu'il craignait le péril ou l'engagement qu'il avait
pris. La crainte était ce qui agissait le plus sur lui. Il n'était pas
difficile, surtout à l'empereur, d'user de ce moyen pour le contenir. Il
tremblait à la moindre menace de Vienne, et la moindre apparence de
faveur de cette cour aurait pu rompre les traités les plus solennels
qu'il aurait faits. Ce prince, lié avec la France, ne cessait de
protester à Vienne qu'il était dévoué à la maison d'Autriche. Absolument
détourné, comme on l'a vu, par ses ministres de venir en France pendant
que le czar y était, il avait fait dire à l'empereur que la crainte de
lui déplaire avait rompu son voyage. Ainsi on conseillait au régent
d'abandonner la pensée de faire entrer le roi de Prusse dans le traité
comme un projet inutile, en ce que l'accession de ce prince ne
fortifierait pas l'union qu'il s'agissait de former avec l'empereur, et
dangereux en ce que les instances que Son Altesse Royale continuerait en
faveur du roi de Prusse seraient à Vienne un sujet d'ombrage et de
jalousie qu'il serait difficile de dissiper. C'est ce que disaient les
ministres les plus confidents du roi d'Angleterre, les Allemands
surtout, qui avaient beaucoup de complaisance pour l'empereur, lequel
n'y répondait pas avec la même franchise.

Il était bien aise que le roi d'Angleterre, comme prince de l'empire,
eût besoin de lui, pour conserver les États usurpés sur la Suède, et il
le voulait tenir toujours dans sa dépendance. Saint-Saphorin crut même
s'apercevoir que cette cour était fâchée que les offices du régent
eussent contribué à la sortie des troupes moscovites du Mecklembourg,
parce qu'elle aurait cru profiter de leur plus long séjour pour disposer
encore plus aisément du roi d'Angleterre.

Ce prince avait demandé à l'empereur de faire sortir des Pays-Bas les
partisans du Prétendant. L'empereur le lui avait promis. Cependant il
restreignit ses ordres aux principaux chefs, et il en écrivit même si
faiblement au marquis de Prié, que les ministres d'Angleterre ne lui en
surent nul gré, et qu'ils crurent que plus la France abandonnait ce
malheureux prince, plus l'empereur lui était favorable. Cela ne
refroidit pas néanmoins les ménagements du roi d'Angleterre pour
l'empereur. Ses ministres, surtout les Allemands, engagèrent la nation
Anglaise à lui payer les restes des subsides dus de la guerre
précédente. Le projet était de lui faire donner sous ce prétexte cent
mille livres sterling. L'empereur prétendait que la dette se montait
bien plus haut. Les Anglais qui n'étaient pas dans le ministère
soutenaient au contraire que la nation n'en devait rien, et ils
traitaient de fort étranges les demandes que faisait l'empereur d'être
payé d'un reste de subsides d'une guerre dont il avait seul profité, et
que l'Angleterre avait faite uniquement pour l'intérêt de la maison
d'Autriche. Les rois de Danemark et de Prusse se plaignaient de la
complaisance que les Anglais avaient pour l'empereur, pendant qu'ils ne
recevaient aucun payement des subsides qu'ils devaient toucher pour la
guerre du nord qu'ils soutenaient actuellement de concert avec le roi
d'Angleterre.

Cette complaisance n'empêchait pas que la cour de Vienne ne se plaignît,
à la moindre occasion, de tout ce qui pouvait lui déplaire de la part
des Anglais. Elle prétendait que le comte de Peterborough avait donné
des conseils inconsidérés aux princes d'Italie. L'empereur en fit porter
ses plaintes à Londres, avec des menaces de le faire arrêter s'il
traversait en Italie des pays occupés par ses troupes. Peterborough
reçut une réprimande et avis d'éviter d'entrer dans les États de
l'empereur. Ce prince informa ses ministres en France des propositions
qu'il recevait de l'Angleterre pour conserver, disait-il, la paix
universelle dans l'Europe, et former une amitié plus étroite avec le
régent. Mais l'avis qu'il en donna, vers le mois de juillet, au comte de
Koenigseck, son ambassadeur à Paris, n'était que général. Il lui
apprenait seulement que la cour d'Angleterre attendait de nouveaux avis
de Paris\,; qu'elle ne voulait rien proposer que sur un fondement
solide\,; qu'elle avait cependant laissé entendre que, si là cour de
Madrid était trop difficile, l'ouvrage s'achèverait avec le régent à
l'exclusion de l'Espagne. L'empereur ordonnait de plus à Koenigseck des
assurances agréables d'entretenir avec Stairs une intelligence étroite.

Koenigseck se persuadait assez que le régent n'avait nulle part à
l'entreprise de Sardaigne, et qu'il verrait avec peine une occasion de
renouveler la guerre. Cependant il ne pouvait croire qu'il n'en eût pas
été informé avant l'exécution. Il était vrai pourtant que le régent n'en
avait eu nulle connaissance. On ne croyait pas qu'aucun prince d'Italie,
non pas même le duc de Parme, eût eu part au secret si bien gardé par
Albéroni. Au moins l'ignorait-il au commencement de juillet, qu'il
conseillait au roi d'Espagne de tenir parole au pape sur l'envoi et la
destination de sa flotte. Il l'exhortait en même temps à donner quelques
marques de ressentiment de la détention de Molinez, qui était une telle
infraction au droit des gens, qu'elle ne pouvait être passée sous
silence, mais d'y employer des paroles, non les armes\,; de s'adresser
aux garants de la neutralité de l'Italie, et d'exciter les autres
princes de l'Europe à prendre des mesures contre les desseins de
l'empereur, qu'il montrait assez, d'usurper le souverain domaine de
toute l'Italie.

Ce prince s'étendait à remontrer le danger de laisser l'Italie en proie
à l'empereur, qui rendrait même le roi d'Espagne vacillant sur son
trône. Il disait savoir de bonne part que le comte de Gallas avait des
instructions et des pouvoirs fort étendus pour faire en sorte d'assurer
à l'empereur, dont il était ambassadeur à Rome, la succession du
grand-duc\,; qu'il devait faire de grandes offres aux parents du pape\,;
qu'il avait pouvoir de leur promettre un État en souveraineté dans la
Toscane\,; qu'il se flattait de conduire le pape jusqu'où il voudrait
par le cardinal Albane, tout autrichien, et par plusieurs autres
cardinaux\,; que l'empereur deviendrait ainsi aisément maître des États
de Toscane, où, Livourne étant compris, il se trouverait encore en état
d'avoir des forces maritimes et de se rendre maître de la Méditerranée
comme il le serait de l'Italie. À quoi le duc de Parme ajoutait des
raisonnements puissants et qui marquaient qu'il n'avait encore aucune
connaissance de ce que l'Espagne méditait sur la Sardaigne et ensuite à
l'égard de l'Italie.

Le courrier qui portait de l'Escurial à Rome l'accommodement entre les
deux cours arriva au commencement de juillet. Au lieu d'y causer de la
joie, il mit le pape dans une colère étrange, parce que l'Espagne
n'avait pas voulu annuler par un décret ceux qui avaient été
précédemment faits, et que le pape prétendait blesser l'honneur du
saint-siège. Il s'emporta contre Aldovrandi\,; dit qu'il lui avait menti
dans le fond et dans la forme\,; s'expliqua en termes très vifs à Santi,
envoyé de Parme\,; maintint qu'Aldovrandi lui avait offert la
satisfaction qui se trouvait refusée, dont il lui avait montré la minute
concertée avec Albéroni et Aubenton, sur quoi lui-même avait dressé un
nouveau projet de décret, dont Aldovrandi, qui le trahissait, avait
emporté la minute\,; lequel, malgré ses ordres les plus positifs
là-dessus, venait de conclure l'accommodement sans obtenir une pièce si
importante, et qu'il devait regarder comme principale. Mais ceux qui
connaissaient les mouvements impétueux de sa colère n'en prirent pas une
grande alarme.

Le Prétendant, prêt à quitter Rome, vint prendre congé du pape. Il
savait l'accommodement signé, il crut la conjoncture heureuse, et il
pressa le pape de tenir sa parole sur Albéroni, puisque les différends
étaient terminés. Le contretemps était complet. Le pape répondit
froidement qu'il exécuterait ses promesses, mais que les affaires
avaient été si mal digérées, qu'il n'était pas encore en état de le
faire. Les deux Albane déclamèrent contre Aldovrandi, et parlèrent
fortement contre lui à Acquaviva.

Ce cardinal, ayant appris qu'il y aurait consistoire le lundi suivant,
voulut avoir auparavant une audience du pape, qui la lui donna. Le pape
y parut content du roi et de la reine d'Espagne et d'Albéroni, mais
outré contre Aldovrandi. Acquaviva le défendit. Il fit convenir le pape
que l'écrit signé entre son nonce et Albéroni était le même qu'il avait
donné à ce nonce. Les plaintes les plus vives tombèrent sur l'omission
du décret. Plus le pape montra de colère, plus Acquaviva le pressa de
déclarer Albéroni cardinal au consistoire du lendemain. Le pape, pressé,
s'en tira par alléguer que le temps était trop court, et qu'il n'y
aurait point de consistoire. C'était ce qu'Acquaviva voulait, parce que,
n'espérant pas que la promotion d'Albéroni y fût faite, son but avait
été d'éloigner le consistoire, et cependant le pape s'engageait à n'en
point tenir sans contenter en même temps le roi d'Espagne.

Toutefois, il forma une congrégation de cardinaux pour avoir leur avis
sur l'accommodement. Ils conclurent que le roi d'Espagne avait fait tout
ce qui dépendait de lui pour satisfaire le pape, qui par conséquent ne
pouvait se dispenser d'accomplir la parole qu'il lui avait donnée\,;
mais, suivant la maxime des cours de flatter le maître aux dépens du
ministre absent et indéfendu, ils blâmèrent unanimement Aldovrandi. Ses
amis n'en furent pas fort émus, et moins encore de la colère du pape.
Ils connaissaient la légèreté des promesses et des menaces de Sa
Sainteté, et combien il les oubliait promptement et entièrement, et
consolèrent le nonce sur ce principe qu'il connaissait comme eux.

Quoique persuadé de cette vérité, Aldovrandi était inquiet des
résolutions que prendrait le pape quand il serait instruit que le roi
d'Espagne avait refusé de passer ce décret qu'il désirait. Un autre
sujet d'agitation était l'entreprise que l'escadre d'Espagne allait
faire, dont le public ignorait encore l'objet, et dont il parlait fort
diversement. Le nonce, à dessein de servir Albéroni, appuyait l'opinion
de ceux qui la croyaient destinée pour Oran, et se fondait sur une
lettre mystérieuse, mais consolante, qu'il avait reçue de lui sur
l'objet de cette escadre. Ainsi trompé par ce ministre tout puissant, ou
de concert avec lui, il donnait pour véritable tout ce qu'il paraissait
lui confier. Il assura le pape, sur sa parole, que si elle était
destinée contre la Sardaigne, ou si elle pouvait causer quelque
préjudice au repos de l'Italie, l'entreprise était certainement formée
contre le sentiment et l'avis d'Albéroni\,; qu'il s'y était
particulièrement opposé à cause du grand préjudice qu'en recevrait le
duc de Parme. Il ajoutait que, s'en étant voulu plus éclaircir, il
s'était adressé à Daubenton qui lui avait répondu qu'il ne s'était
jamais mêlé des vaisseaux du roi d'Espagne, qu'il avait seulement donné
toute son attention à l'accommodement entre les deux cours.

Quoique cet armement eût coûté fort cher, qu'on y eût embarqué un nombre
de troupes assez considérable, que dix galères l'eussent joint à
Barcelone, ces préparatifs ne suffisaient pas pour exécuter les grands
desseins qu'on attribuait à l'Espagne sans le secours d'autres princes
et la connivence de plusieurs. Cette vérité multipliait les
raisonnements des politiques. Les uns croyaient l'entreprise concertée
avec la Hollande, même avec l'Angleterre, fondés sur l'intimité qui se
remarquait entre Albéroni et les ministres que ces puissances tenaient à
Madrid. Avec cette supposition de leur jalousie des desseins de
l'empereur, ils jugeaient que l'Espagne, ou gagnerait un royaume, ou, ne
réussissant pas, se retrouverait au même état qu'auparavant. Le
ressentiment de l'empereur inutile contre elle ne pouvant retomber que
sur l'Italie, peu de gens pensaient que la France y prît part\,; on la
jugeait plus occupée de ses affaires domestiques qu'à se mêler
d'affaires qui lui étaient étrangères, et qui étaient capables de
l'entraîner dans une nouvelle guerre. Enfin, la plupart jugeaient que le
projet était communiqué au roi de Sicile, qui agirait de concert avec
d'autres princes d'Italie dans la même ligue.

L'ambassadeur de ce prince à Madrid en pensait bien différemment\,; il
était persuadé que l'entreprise regardait plus la Sicile que la
Sardaigne, et se fondait sur l'impénétrable secret qui en couvrait les
desseins, Patiño et don Miguel Durand, secrétaire d'État pour la guerre,
étant les deux seuls dont Albéroni se fût servi. Lorsque l'affaire
éclata Aldovrandi et Mocenigo, destiné ambassadeur de Venise, allèrent
trouver Albéroni au Prado à qui ils représentèrent fortement les
malheurs qu'il allait attirer sur l'Italie s'il donnait à l'empereur un
sujet légitime de rompre la neutralité. Albéroni leur répondit seulement
qu'il était étonné de voir deux hommes aussi consommés ajouter foi aux
chansons de Madrid, et les assura que l'escadre était destinée et serait
employée au service du pape et de la république. Tous deux se
contentèrent de cette réponse.

Enfin, la nouvelle de l'entreprise devenue publique, à n'en pouvoir plus
douter, elle fut universellement blâmée et ses suites prédites comme
funestes à l'Europe. Le secrétaire d'Angleterre s'éleva tellement
contre, à Madrid, qu'il effaça tout soupçon de concert avec
l'Angleterre. Riperda en fit autant d'abord, mais il changea depuis. Les
ministres étrangers disaient tout haut qu'Albéroni ne se souciait pas
d'allumer une nouvelle guerre pourvu qu'il rendît son nom glorieux.

Ce premier ministre aurait bien désiré que sa promotion eût précédé la
publicité de son entreprise\,; mais voyant qu'elle ne pouvait plus se
différer, il tâcha d'y préparer et de gagner des suffrages en se
plaignant hautement de l'arrêt de la personne de Molinez. On peut se
souvenir de l'indifférence qu'il avait eue là-dessus, du mépris qu'il
avait témoigné du grand inquisiteur, qu'il n'appelait que
\emph{solemnissima bestia}. Mais il lui convenait alors de se récrier
sur cette violence, comme de la continuation des outrages que les
Impériaux n'avaient cessé de faire au roi d'Espagne, dont il serait
enfin contraint de se venger malgré sa répugnance, par rapport au repos
de l'Europe. Il paraphrasait ce texte, et y ajoutait qu'il en
souffrirait en son particulier, parce qu'il prévoyait que les mesures
prises pour son chapeau en seraient rompues, sur quoi il s'expliquait en
style d'ancien Romain. Il se complaisait d'avoir rétabli la marine
d'Espagne en si bon état, n'en ayant trouvé aucune, surtout des magasins
de Cadix, qu'il publiait être plus remplis que ne l'étaient ceux de
Brest, Toulon et Marseille. À quoi il ajoutait toutes sortes d'utiles
vanteries.

Aldovrandi le servait à Rome en tâchant d'y persuader que l'entreprise
regardait Oran. Il trouvait les préparatifs trop grands pour la
Sardaigne, insuffisants pour Naples et la Sicile. Il en concluait pour
Alger, et se rabattre après sur Oran\,; et n'osant plus amuser le pape
que cette escadre irait au Levant, il le flattait au moins qu'elle
allait tomber sur les Barbaresques.

Del Maro, de plus en plus persuadé par la profondeur du secret que cet
orage regardait la Sicile, cherchait des voies détournées pour en
avertir son maître, persuadé que toutes ses lettres étaient
interceptées, et que sa maison était environnée d'espions. Il fit passer
un courrier à Turin, qui lui revint à Madrid malgré toutes les
précautions dont la nature, qui allait à la violence, confirma tous ses
soupçons.

Le duc de Parme méritait d'être distingué des autres princes, parce
qu'il était à la reine d'Espagne et par ce qu'Albéroni lui devait, qui
était encore son ministre à Madrid. Il sut donc enfin sous le dernier
secret la véritable destination de l'escadre d'Espagne. Il donna tous
les avis qu'il put pour en faciliter les desseins. Il avertit que les
préparatifs de Barcelone avaient jeté les ministres impériaux à Naples
dans la consternation\,; qu'ils connaissaient parfaitement leur
faiblesse si le royaume était attaqué, et le voeu général des grands et
des peuples d'être délivrés du joug des Allemands\,; qu'un des ces
ministres avait avoué que l'enlèvement de Molinez était insoutenable,
que c'était une infraction manifeste de la neutralité d'Italie, et
qu'elle aurait de fâcheuses suites. Le vice-roi, qui ne voulait pas
montrer leur agitation commune, avait donné des ordres secrets de
fortifier plusieurs places, et redoubla de soins pour la sûreté du
royaume. La justice y était abolie, le négoce cessé, l'administration et
les gouvernements en vente au plus offrant. Le désespoir y était, et les
voeux peu retenus de voir paraître l'escadre espagnole, et le roi
d'Espagne était fortement exhorté de profiter de cette conjoncture
pendant la campagne de Hongrie. Le duc de Parme appuyait de toutes ses
forces l'avis de la conquête de Naples, par la crainte qu'il avait de la
puissance et des desseins de l'empereur. Il prétendait qu'elle était
facile, et n'avoir qu'à s'y présenter pour opérer une révolution
subite\,; qu'une fois faite, elle se conserverait aisément parce que les
princes d'Italie, gémissants et tremblants sous l'autorité de
l'empereur, concourraient tous à la défense quand ils se verraient
soutenus, surtout le roi de Sicile, certain de la haine que l'empereur
lui avait jurée, et les Vénitiens enveloppés de tous côtés par les États
de l'empereur\,; que le pape serait le premier à s'engager, auquel il
exhortait le roi d'Espagne de donner promptement la satisfaction à
laquelle il se bornait. Ce n'était plus ce décret refusé par l'Espagne,
mais une simple lettre secrète du roi d'Espagne à lui, par laquelle il
désavouerait, non pas le livre que le duc d'Uzeda avait fait imprimer il
y avait quelques années, mais la partie seulement de ce livre qui
contenait des choses injurieuses à sa personne\,; et comme le duc de
Parme cherchait à plaire au pape et à lui faire voir son crédit à
Madrid, il demandait que cette lettre lui fût adressée pour la faire
passer entre les mains de Sa Sainteté.

Enfin le pape, ne pouvant plus résister aux menaces du roi d'Espagne et
à la frayeur de la vengeance d'Albéroni, le fit cardinal le 12 juillet.
Cette promotion ne fut approuvée de personne lorsqu'elle fut déclarée au
consistoire. Aucun cardinal ne loua le nouveau confrère. Quelques-uns la
désapprouvèrent ouvertement, entre autres Dadda, Barberin, Borromée,
Marini. Giudice y dit qu'il ne pouvait y consentir en sûreté de
conscience, et le cardinal de Schrottembach, ministre de l'empereur, ne
se trouva pas au consistoire. Toutes ces choses furent interprétées
diversement. Ce qui est vrai, c'est que Giudice avait dressé une partie
d'opposition qui dans la crise lui manqua tout net, et qu'Acquaviva, qui
ne l'aimait pas et qui voulait plaire en Espagne, n'y laissa pas
ignorer.

Le roi d'Angleterre était fort mal à son aise au milieu de sa cour.
Parmi tous ses ménagements pour l'empereur, on prétendait qu'il avait
personnellement plus d'éloignement que d'amitié pour lui\,; qu'il était
entraîné par ses ministres allemands, dévoués à la cour de Vienne pour
en obtenir des grâces pour eux et pour leurs familles, et en opposition
fréquente avec les ministres anglais, qui ne se contraignaient à leur
égard sur l'aversion et le mépris que lorsque quelque intérêt
particulier les engageait à vouloir plaire au roi leur maître. Ce prince
venait d'avoir le dégoût, malgré ses efforts, de voir sortir avec
honneur et justice le comte d'Oxford de l'accusation capitale intentée
contre lui, et la division s'accroître entre les gens qui lui étaient
les plus attachés. Elle augmentait sans cesse entre lui et le prince de
Galles, et s'il ne le pouvait ramener à lui par la douceur, il avait
résolu d'user de rigueur et d'éloigner de lui ceux qui, dans le
parlement, avaient voté contre le général Cadogan. C'était là un autre
point de discorde qui intéressait la nation, laquelle, aussi bien que le
prince, prétendait que la prérogative royale ne s'étendait pas
jusque-là.

La haine entre le père et le fils éclatait jusque dans les moindres
choses. Elle devint tout à fait publique à l'occasion d'une revue d'un
régiment qui portait le nom du prince, dont le roi ne voulut pas
s'approcher que le prince, qui était à la tête en habit uniforme, ne se
fût retiré. Il obéit et dit en s'en allant que ce coquin de Cadogan en
était cause.

Parmi ces inquiétudes Georges en avait beaucoup de l'entreprise de
l'escadre d'Espagne, dont il n'avait aucune connaissance, et dont il en
cherchait vainement par Monteléon, qui en était lui-même en parfaite
ignorance. On y était aussi très attentif en Hollande, mais avec moins
d'intérêt qu'en Angleterre, parce que la république n'en avait rien à
craindre et n'était obligée par aucun traité de secourir l'empereur, et
qu'il ne lui était pas inutile qu'il survint des embarras à ce prince
qui le rendissent plus traitable et plus facile à terminer ce qui
restait de différends à régler sur la Barrière. On s'y apercevait même
déjà d'un grand et prompt changement de ton là-dessus du baron de
d'Heems, envoyé de l'empereur à la Haye.

Beretti s'applaudissait de cette douceur nouvelle. Il l'attribuait aux
soins qu'il avait pris d'ouvrir les yeux aux Hollandais sur le danger
des desseins et de la puissance de l'empereur, et de seconder, au
contraire, ceux du roi d'Espagne. Il assurait ce prince que la moitié de
l'Angleterre lui désirait un bon succès, moins à la vérité par affection
que pour le plaisir de voir l'embarras du gouvernement d'Angleterre sur
le parti qu'il aurait à prendre, et Beretti se persuadait toute bonne
volonté de la part des États généraux\,; il les croyait même peu
contents de remarquer tant d'attachement du roi d'Angleterre pour
l'empereur, et il comptait que les plaintes qu'il s'attendait de
recevoir de leur part sur l'entreprise de l'Espagne ne seraient
qu'accordées à la bienséance et aux clameurs des Impériaux. Cet
ambassadeur d'Espagne n'oubliait rien pour donner à sa cour de la
confiance aux dispositions des Hollandais pour elle, et tout ce qu'il
pouvait de défiance de celles de la cour d'Angleterre pour détourner la
négociation d'être portée à Londres, où il craignait qu'elle tombât
entre les mains de Monteléon, et pour la faire ouvrir au contraire à la
Haye, dans l'espérance qu'elle n'y sortirait pas des siennes. Il
conseillait aussi de faire quelque réponse aux propositions que
l'Angleterre lui avait faites, pour éviter le reproche de ne vouloir
point de paix avec l'empereur, dont il était persuadé que les
prétentions paraîtraient si déraisonnables, qu'il serait très facile de
faire tomber sur lui ce même reproche.

Le silence de Madrid était mal interprété à Paris, à la Haye, à Londres.
L'envoyé d'Angleterre à la Haye s'en plaignit à Beretti et Duywenworde
aussi. Il pressait donc Albéroni de lui prescrire quelque réponse à
Stanhope, non plus en espérance de négocier, mais pour faire cesser le
démérite du refus de s'expliquer. Il ne comptait nullement sur le succès
de la négociation\,; il représentait, au contraire, que l'objet
principal de tout l'ouvrage était de travailler pour les intérêts du
régent, de l'Angleterre et de l'empereur, sous le nom du roi d'Espagne
et sous prétexte d'agir en sa faveur. Il était aussi très embarrassé des
questions sur la véritable destination de l'escadre espagnole, dont il
ne savait rien.

Monteléon n'était pas à Londres dans une moindre presse, ni dans une
moindre ignorance là-dessus. Il apprit par les ministres d'Angleterre
que le régent avait dit à Stairs et à Koenigseck que l'entreprise
regardait Naples, et que, la France étant garante de la neutralité
d'Italie, Son Altesse Royale avait dépêché à Madrid, pour savoir les
intentions de Sa Majesté Catholique. Wolckra, envoyé de l'empereur à
Londres, et Hoffmann, qui y était depuis longtemps de sa part en qualité
de résident, demandèrent tous deux l'assistance du roi d'Angleterre
comme garant de la neutralité d'Italie, et comme engagé par le dernier
traité à secourir l'empereur, s'il était attaqué dans ses États\,; mais
les ministres d'Angleterre suspendirent la réponse.

Peterborough se disposait alors à passer en Italie. Quelques-uns crurent
que ce voyage cachait quelque mystère\,; mais ni le roi d'Angleterre ni
pas un de ses ministres ne se fiaient en lui\,; pas un des partis
n'avait pour lui ni estime ni confiance. Bien des gens crurent que son
but était de se faire considérer par les cours de l'empereur et de
France, en les informant de ce qu'il pourrait pénétrer réciproquement de
chacune. On lui rendait justice sur l'esprit et le courage, dont il
avait beaucoup, même trop, et que toutes ses idées allaient à le mettre
dans l'embarras, lui et ceux qu'il pouvait engager dans ses vues.

Cependant on ignorait également à Paris, à Londres et à Vienne, le
véritable dessein du roi d'Espagne. Patiño était seul dans le secret du
cardinal Albéroni\,; et le marquis de Lede, chef des troupes embarquées,
ne devait ouvrir ses ordres qu'en mer. Ainsi les raisonnements étaient
infinis sur le but de cette expédition. Outre les propos généraux que
tenait Albéroni, e fort obscurs, il fit dire précisément au Pensionnaire
qu'il fallait que la Hollande choisît ou d'unir ses forces à celles de
l'empereur contre l'Espagne, ou au roi d'Espagne pour donner l'équilibre
à l'Europe, en commençant par l'Italie. Il avouait à ses amis que, si sa
promotion au cardinalat n'avait pas été déclarée le jour même qu'elle la
fut, il aurait lieu de la regarder comme fort éloignée\,; mais qu'ayant
obtenu ce qu'il désirait, les considérations particulières ne
l'empêcheraient plus d'agir pour la gloire et les intérêt du roi son
maître (vérité digne de servir de leçon aux rois). Acquaviva et d'autres
encore l'exhortaient à profiter de la conjoncture pour venger l'Espagne
du mépris et de la mauvaise foi de la maison d'Autriche, et de
l'enlèvement de Molinez.

Gallas, ambassadeur de l'empereur à Rome, ne tarda pas à se plaindre
fortement au pape que le roi d'Espagne employait l'indult qu'il lui
avait accordé sur le clergé, non contre les Turcs, mais pour faire la
guerre à l'empereur\,; et s'étendit sur des projets qui attentaient à la
neutralité de l'Italie. Le pape répondit qu'il n'avait point encore à se
plaindre du roi d'Espagne, qui lui avait promis un secours maritime
contre les Turcs\,; qu'il n'était pas en droit de trouver mauvais
qu'après avoir exécuté sa promesse, l'escadre s'employât à quelque chose
d'utile à son service\,; et qu'à l'égard de la neutralité d'Italie, il
n'en pouvait rien dire, parce que jamais on ne lui avait fait part du
traité pour l'établir\,; qu'il était vrai que le roi d'Espagne lui avait
offert de ne point inquiéter l'empereur pendant la guerre de Hongrie,
mais avec une condition réciproque, que l'empereur avait refusée.
Gallas, court de raisons, mais qui connaissait le terrain, répondit par
des menaces que l'empereur ferait incessamment une trêve avec les Turcs,
et qu'il enverrait quarante mille hommes en Italie, dont l'État
ecclésiastique et celui de Parme entendraient parler les premiers.

Il n'en fallait pas tant pour effrayer le pape. Aussitôt après
l'audience, il manda l'envoyé de Parme, et le conjura de dépêcher à
l'instant un courrier à Madrid, d'y représenter vivement le péril
imminent où le duc de Parme se trouvait exposé, et de n'y rien oublier
pour détourner toute entreprise capable de troubler le repos de
l'Italie.

Outre ces menaces, les projets de la cour de Vienne inquiétaient
cruellement les princes d'Italie, et faisaient trembler les Vénitiens,
environnés en terre ferme par les États et les troupes de l'empereur,
qui voulait encore se rendre maître de leurs mers par de nouveaux ports
dans le golfe Adriatique, et les assujettir par les forces maritimes
qu'il se proposait d'y établir. On disait de plus qu'il prétendait
mettre dans Livourne une garnison allemande, et qu'il avait fait
demander des subsides au grand-duc en des termes de la dernière hauteur.
D'autre part, les ministres du roi d'Espagne l'avertissaient que
l'empereur persistait toujours dans la maladie de retourner en Espagne,
par conséquent de la nécessité de le prévenir.

Au contraire, Rome redoublait ses instances pour détourner le roi
d'Espagne de toute entreprise sur l'Italie, et ri oubliait aucune raison
d'honneur, d'intérêt ni de conscience. Mais le pape parlait à un sourd
qui, ne craignant plus rien de sa part depuis qu'il en avait reçu le
chapeau, s'inquiétait peu de ses exhortations et de ses menaces.

Stairs s'était déchaîné à Paris contre Albéroni à l'occasion de
l'entreprise, quoique encore ignorée pour le lieu. Albéroni lui rendait
la pareille, et disait que le roi d'Espagne demanderait justice au roi
d'Angleterre de cet homme vendu à l'empereur. Albéroni ne voulait plus
écouter les sollicitations de l'Angleterre d'envoyer un ministre à
Londres travailler à la paix avec l'empereur, par la médiation de la
France et de l'Angleterre. Il trouvait que cette démarche ne se pouvait
faire avec honneur, que l'affaire était sans lueur ni apparence de
succès, vision ou piège de la cour de Vienne. Il disait que l'offre
d'assurer la succession de Parme aux enfants de la reine, tandis que le
duc de Parme et son frère n'étaient ni vieux, ni hors d'espérance
d'avoir des enfants, troublerait plutôt l'Italie qu'elle n'apporterait
d'avantage à ces princes collatéraux. On était à la fin d'août sans être
plus éclairci\,; mais on ne doutait plus qu'il ne s'agît de la
Sardaigne.

Aldovrandi, pour faire sa cour au cardinal Albéroni, publiait que
l'entreprise se faisait contre son avis, qu'il s'y était opposé en vain,
qu'il avait eu la sage précaution d'en conserver les preuves\,; que,
voyant enfin qu'il ne la pouvait empêcher, il avait au moins détourné le
plus grand mal, et fait résoudre la Sardaigne pour préserver l'Italie.
Il fallait nommer l'auteur d'un conseil dont Albéroni voulait se
défendre. Sur sa parole Aldovrandi répandit que c'était le conseil
d'État dont l'emportement avait été extrême. Sur la même foi, que ce
nonce prétendait très sincère, il donnait les Hollandais pour favoriser
sous main l'entreprise, pour occuper l'empereur loin des Pays-Bas.

L'Angleterre ne laissait pas seulement soupçonner ses intentions. Ses
embarras domestiques faisaient juger que son intérêt la portait à voir
avec beaucoup de peine l'Europe prête à s'embraser de nouveau.

Pour la France, elle s'était expliquée. Le duc de Saint-Aignan avait
représenté que le roi, garant de la neutralité d'Italie, ne pouvait
approuver une entreprise qui y contrevenait. Il avait excité le nonce de
solliciter le pape d'employer les offices de père commun\,; enfin il
avait essayé de toucher par la fâcheuse situation du duc de Parme, à qui
l'empereur demandait hautement de fortes contributions. Ce prince
manquait d'argent. Il avait inutilement recours à l'Espagne, qu'il
exhortait toujours, et avec aussi peu de succès, de donner au pape la
dernière satisfaction qu'il désirait, sur le livre du duc d'Uzeda dont
on a parlé. Del Maro ne cessait d'avertir son maître que l'entreprise
regardait la Sicile\,; et les ministres d'Angleterre, de Hollande et de
Venise à Madrid, s'épuisaient en inquiétudes et en raisonnements.

\hypertarget{chapitre-v.}{%
\chapter{CHAPITRE V.}\label{chapitre-v.}}

1717

~

{\textsc{L'Espagne publie un manifeste contre l'empereur.}} {\textsc{-
Déclaration vague de Cellamare au régent.}} {\textsc{- Efforts
d'Albéroni pour exciter toutes les puissances contre l'empereur\,; veut
acheter des vaisseaux dont il manque\,; en est refusé.}} {\textsc{- Ses
bassesses pour l'Angleterre inutiles.}} {\textsc{- Singulières
informations d'Albéroni sur Riperda.}} {\textsc{- Cet ambassadeur cru
vendu à Albéroni et soupçonné de vouloir s'attacher au service du roi
d'Espagne.}} {\textsc{- Aldovrandi cru, à Rome et ailleurs, vendu à
Albéroni.}} {\textsc{- Artifices de ce dernier sur son manque
d'alliés.}} {\textsc{- Ses offres à Ragotzi.}} {\textsc{- Fureur
d'Albéroni contre Giudice.}} {\textsc{- Crainte et bassesse de ses
neveux.}} {\textsc{- Le roi d'Espagne défend à ses sujets de voir
Giudice à Rome et tout commerce avec lui.}} {\textsc{- Point de la
succession de Toscane.}} {\textsc{- Manèges des ministres hanovriens
pour engager le régent à s'unir à l'empereur.}} {\textsc{- L'Angleterre
désire la paix de l'empereur et de l'Espagne, et veut envoyer faire des
efforts à Madrid.}} {\textsc{- Ruses à Londres avec Monteléon.}}
{\textsc{- Soupçons et vigilance de Koenigseck à Paris.}} {\textsc{-
Entreprise sur Ragotzi sans effet.}} {\textsc{- Les Impériaux lui
enlèvent des officiers à Hambourg.}} {\textsc{- Baron de Goertz mis en
liberté.}} {\textsc{- Le czar plus que froid aux propositions du roi
d'Angleterre, lequel rappelle ses vaisseaux de la mer Baltique.}}
{\textsc{- Situation personnelle du roi d'Angleterre avec les Anglais.}}
{\textsc{- Il choisit le colonel Stanhope, cousin du secrétaire d'État,
pour aller en Espagne.}} {\textsc{- Visite et singulier conseil de
Châteauneuf à Beretti.}} {\textsc{- Sentiment des ministres d'Angleterre
sur l'entreprise de l'Espagne en soi.}} {\textsc{- Wolckra rappelé à
Vienne\,; Penterrieder attendu à Londres en sa place pour y traiter la
paix entre l'empereur et l'Espagne avec l'abbé Dubois.}} {\textsc{-
Artifices de Saint-Saphorin auprès du régent de concert avec Stairs.}}
{\textsc{- Vaine tentative de l'empereur pour de nouveaux honneurs à son
ambassadeur en France.}} {\textsc{- Inquiétude de l'Angleterre\,; ses
soupçons du roi de Sicile.}} {\textsc{- Misérables flatteries à
Albéroni.}} {\textsc{- Cellamare excuse et confie le secret de
l'entreprise de l'Espagne au régent\,; dont la réponse nette ne le
satisfait pas.}} {\textsc{- Nouveau complot des Impériaux pour se
défaire de Ragotzi, inutile.}} {\textsc{- Sèche réponse des ministres
russiens aux propositions de l'Angleterre.}} {\textsc{- La flotte
espagnole en Sardaigne.}} {\textsc{- Le pape, effrayé des menaces de
Gallas, révoque les indults accordés au roi d'Espagne\,; lui écrit une
lettre à la satisfaction des Impériaux\,; désire au fond succès à
l'Espagne\,; offre sa médiation.}} {\textsc{- Misérables flatteries à
Albéroni.}} {\textsc{- Il fait ordonner à Giudice d'ôter les armes
d'Espagne de dessus la porte de son palais à Rome.}} {\textsc{- Sa
conduite et celle de ses neveux.}} {\textsc{- Victoire du prince Eugène
sur les Turcs.}} {\textsc{- Il prend Belgrade, etc.}} {\textsc{-
Soupçons de l'empereur à l'égard de la France.}} {\textsc{- Entreprise
inutile sur la vie du prince Ragotzi.}} {\textsc{- Deux François à lui
arrêtés à Staden.}} {\textsc{- Scélératesse de Welez.}} {\textsc{-
Artifices de l'Angleterre et de Saint-Saphorin pour lier le régent à
l'empereur, et en tirer des subsides contre les rois d'Espagne et de
Sicile.}} {\textsc{- Artifices du roi de Prusse auprès du régent sur la
paix du nord.}} {\textsc{- Goertz à Berlin\,; y attend le czar.}}
{\textsc{- Propositions de ce ministre pour faire la paix de la Suède.}}
{\textsc{- Soupçons du roi de Prusse à l'égard de la France, à qui il
cache les propositions de Goertz.}} {\textsc{- Hasard à Paris qui les
découvre.}} {\textsc{- L'Angleterre liée avec l'empereur par des traités
précis, et craignant pour son commerce de se brouiller avec l'Espagne, y
envoie par Paris le colonel Stanhope.}} {\textsc{- Objet de cet envoi,
et par Paris.}} {\textsc{- Artifices de l'Angleterre pour unir le régent
à l'empereur.}} {\textsc{- Georges et ses ministres en crainte du czar
et de la Prusse, en soupçon sur la France.}} {\textsc{- Leur haine pour
Châteauneuf.}} {\textsc{- Bolingbroke secrètement reçu en grâce par le
roi d'Angleterre.}} {\textsc{- Opiniâtreté d'Albéroni.}} {\textsc{-
Leurres sur la Hollande.}} {\textsc{- État et suite de la vie de
Riperda.}} {\textsc{- Venise se déclare pour l'empereur.}} {\textsc{-
Colère d'Albéroni.}} {\textsc{- Ses étranges vanteries et ses artifices
pour se faire un mérite de se borner à la Sardaigne cette année, sentant
l'impossibilité de faire davantage.}} {\textsc{- Sa fausseté insigne à
Rome.}} {\textsc{- Embarras et conduite artificieuse et opiniâtre
d'Albéroni.}} {\textsc{- Sa réponse à l'envoyé d'Angleterre.}}
{\textsc{- Albéroni se fait un bouclier d'un équilibre en Europe\,;
flatte bassement la Hollande\,; n'espère rien de l'Angleterre. Plan
qu'il se propose pour objet en Italie\,; il le confie à Beretti et lui
donne ses ordres en conséquence. Propos d'Albéroni\,; vanteries et
fourberies insignes et contradictoires. Conduite d'Aubenton et
d'Aldovrandi, qui lui sont vendus pour leur intérêt personnel. Les
Impériaux demandent qu'Aldovrandi soit puni\,; effrayent le pape. Il
révoque ses indults au roi d'Espagne\,; lui écrit au gré des
Impériaux\,; en même temps le fait ménager et adoucir par Aldovrandi, à
qui il écrit, et à Daubenton, de sa main. Frayeurs du duc de Parme, qui
implore vainement la protection du pape et le secours du roi d'Espagne.
Plaisant mot du cardinal del Giudice au pape. Le pape dépêche à Vienne
sur des propositions sauvages d'Acquaviva, comptant sur le crédit de
Stella qui voulait un chapeau pour son frère. Molinez transféré du
château de Milan dans un des collèges de la ville. Vastes projets
d'Albéroni, qui en même temps sent et avoue sa faiblesse. Propos
trompeurs entre del Maro et Albéroni. Ses divers artifices. La Hollande
inquiète est touchée de l'offre de l'Espagne de reconnaître sa
médiation. Cadogan à la Haye\,; son caractère. Ses plaintes, sa
conduite. Inquiétude de l'Angleterre sur le nord. Ses ministres
détrompés sur le régent, reprennent confiance en lui\,; font les
derniers efforts pour faire rappeler Châteauneuf. Substance et but du
traité entre la France, le czar et la Prusse. Abbé Dubois à Londres et
le colonel Stanhope à Madrid. Le czar parti de Berlin sans y avoir rien
fait ni voulu écouter sur la paix du nord. Le roi de Prusse réconcilié
avec le roi d'Angleterre, cherche à la tromper sur la paix du nord\,; se
plaint de la France, qui le contente. Poniatowski à Paris\,; confident
du roi de Suède, consulté par Kniphausen, lui trace le chemin de la paix
du nord. Ardeur du roi d'Angleterre, et sa cause, pour pacifier
l'empereur et l'Espagne qui ne s'en éloigne pas. Sentiment de Monteléon
sur les Anglais. Sa situation redevenue agréable avec eux. Caractère du
roi d'Angleterre et de ses ministres. Bassesse du roi de Sicile pour
l'Angleterre, inutile. Son envoyé à Londres forme une intrigue à Vienne
pour y réconcilier son maître. Opinion prétendue de l'empereur sur le
régent et sur le roi de Sicile. Crainte publique des princes d'Italie.
Sages pensées de Cellamare. Avis envenimés contre la France de Welez à
l'empereur. Conseils enragés de Bentivoglio au pape, qui fait entendre
qu'il ne donnera plus de bulles sans conditions et précautions.}}

~

Enfin le moment arriva d'éclaircir l'Europe. L'Espagne fit publier par
ses ministres dans les cours étrangères, un manifeste contenant les
raisons qui l'engageaient d'attaquer l'empereur, et de tourner ses armes
sur la Sardaigne, au lieu de joindre sa flotte à l'armée chrétienne,
comme elle avait fait l'année précédente, et comme elle l'avait promis
et résolu encore pour cette aimée. Ce manifeste rappelait tous les
manquements de parole, les déclamations injurieuses, le détail de tout
ce qui s'était passé depuis le traité d'Utrecht, jusqu'à l'enlèvement du
grand inquisiteur par les Impériaux. Il finissait en montrant la
nécessité où l'honneur et toutes sortes de raisons obligeaient le roi
d'Espagne de se venger. Cellamare, avec ce manifeste, reçut ordre de
déclarer au régent que la conquête de la Sardaigne n'empêcherait pas le
roi d'Espagne de donner à l'Europe l'équilibre nécessaire à sa sûreté,
lequel était impossible tant que l'empereur conserverait la supériorité
qu'il avait en Italie. Albéroni n'oubliait rien pour faire peur à toutes
les puissances de celle de l'empereur, qui voulait tout envahir, et qui
n'avait ni règle, ni parole, ni justice, et qui n'entrevoit jamais
sincèrement dans aucune négociation de paix, quoiqu'il en voulût amuser
l'Espagne par artifice, par l'intervention de la Hollande et de
l'Angleterre, et avec lequel il n'y avait plus d'autre parti que celui
de se bien préparer à faire la guerre. La Sardaigne, en effet, n'était
qu'un essai. Albéroni prétendait bien avoir une armée plus considérable
l'année suivante, et plus de forces sur mer. Mais le temps était court,
sa marine ne répondait pas à ses desseins. Il voulut acheter des navires
en Hollande et en Angleterre, et il en fut refusé. Néanmoins il la
ménageait beaucoup. Il lui offrit de cesser tout commerce avec le
Prétendant, et de faire incessamment avec les Anglais un traité de
commerce à leur satisfaction.

On le croyait sûr de la Hollande. Riperda eut la sotte vanité de laisser
croire qu'il avait eu part au secret de l'entreprise. Les traitements
qu'il recevait du roi d'Espagne confirmaient cette opinion. On savait
encore qu'Albéroni s'était exactement informé en Hollande du caractère
de cet ambassadeur, quoiqu'il le connût par lui-même, de son bien, de
ses charges, des distinctions dont il jouissait dans sa province\,; et
on en soupçonnait que, s'il agissait par ordre de ses maîtres, il
agissait encore plus pour son intérêt, et dans la vue de s'attacher au
service du roi d'Espagne.

Le nonce n'était pas moins soupçonné que lui d'être vendu à Albéroni.
Tout ce qui s'était passé de publiquement intime entre eux, depuis son
arrivée à l'Escurial, jusqu'à le faire loger dans son appartement, ces
circonstances faisaient croire à quelques-uns que le pape était
d'intelligence avec l'Espagne, à la plupart que son nonce était livré à
Albéroni. Cette dernière opinion régnait à Rome, d'où le nonce recevait
les reproches les plus durs.

Il était trop difficile au premier ministre d'imposer au monde sur les
sentiments de l'Angleterre et de la Hollande à l'égard de son
entreprise. Quoique sans alliés, il voulait pallier cette vérité,
espérant que {[}ce que{]} le roi de Suède pensait là-dessus était moins
démêlé. Il essaya d'en profiter pour laisser croire que ce prince était
de concert avec l'Espagne.

Pour la France, il était évident qu'elle ne voulait point de guerre, et
qu'elle ne prendrait point de part à celle que l'Espagne allait faire.
Mais on laissait entendre avec succès qu'elle ne serait pas fâchée de
voir les principales puissances en guerre entre elles, pour avoir le
temps de remédier à ses désordres domestiques.

Albéroni fut ravi du passage de Ragotzi en Turquie. Il lui promit un
vaisseau pour en faire le trajet, s'il n'en pouvait obtenir un en
France, et lui fit espérer des secours s'il en avait besoin dans la
suite. Cette négociation passa fort secrètement par Cellamare, qui était
d'autant plus attentif à plaire à Albéroni que ce cardinal était irrité
au dernier point de la manière dont Giudice avait parlé au consistoire
de sa promotion. Il faisait de son ressentiment celui de Leurs Majestés
Catholiques, voulait persuader que la conduite de ce cardinal était
également offensante pour elles et pour le pape même, protestait qu'elle
aurait perdu Cellamare si son amitié personnelle pour lui n'en avait
détourné le coup. Le prélat Giudice, frère de Cellamare, avait écrit
avec toute la bassesse possible à Albéroni, qui résolut de faire tomber
toute sa colère sur le cardinal leur oncle. Le roi d'Espagne manda donc
à Acquaviva qu'il regardait désormais ce cardinal comme livré à
l'empereur, et travaillant à la négociation pour assurer la possession
de la Toscane à l'empereur, et un État souverain en Toscane aux neveux
du pape\,; qu'il lui défendait de le voir, et tout commerce direct ou
indirect avec lui\,; et lui ordonnait d'intimer la même défense à tous
ses sujets et affectionnés à Rome.

Cette succession de Toscane faisait alors un grand point dans les
négociations entamées pour assurer le repos de l'Europe. Les ministres
hanovriens du roi d'Angleterre, étoient parvenus à faire exclure le roi
de Prusse dans le traité, jusqu'à ce que la négociation fût achevée. Ce
point gagné sur le régent, comme on l'a déjà vu, ces mêmes ministres,
dévoués à l'empereur pour leurs intérêts particuliers de famille, firent
entendre au régent, pour l'intimider, que, si la campagne de Hongrie
était heureuse, la négociation qu'il avait commencée serait bien plus
difficile\,; qu'il ne devait donc pas laisser échapper l'occasion de
s'assurer l'appui de l'empereur, parce que, étant uni avec lui et avec
le roi d'Angleterre, il se mettrait à couvert des entreprises des
malintentionnés de France. Ils lui rendaient suspects ceux qui le
détournaient de suivre cette route, comme étant des créatures de
l'Espagne. Ils voulaient persuader au régent que plus ces gens-là
s'acharnaient à traverser la négociation, plus il devait avoir
d'empressement de la conclure\,; qu'il pouvait aisément le faire jusqu'à
la signature, sans leur en donner connaissance, après quoi, sûr qu'il
serait des principales puissances de l'Europe, rien ne l'empêcherait
d'envoyer promener des ministres si opposés à une négociation si
avantageuse. Dans le désir de l'avancer, l'Angleterre pressait la cour
de Vienne d'envoyer à Londres le secrétaire Penterrieder, comme le seul
capable de la conduire à une bonne fin. Mais il ne suffisait pas de
traiter seulement avec l'empereur, il fallait obtenir le consentement de
l'Espagne, puisqu'il ne s'agissait pas d'exciter une nouvelle guerre,
mais d'assurer le repos de l'Europe.

Le roi d'Angleterre résolut donc d'envoyer à Madrid un homme de
confiance et de poids, pour représenter au roi d'Espagne que
l'Angleterre, engagée par son dernier traité avec l'empereur de lui
garantir généralement tous les domaines dont il était en possession, à
l'exception seulement de la Hongrie, ne pouvait s'empêcher de le
secourir lorsque les armes espagnoles l'attaqueraient en Italie. On
proposa pour cette commission le général Cadogan, en qui le roi
d'Angleterre avait une confiance particulière, et de faire passer en
même temps une escadre dans la Méditerranée, pour donner plus de force à
ses discours, ou pour contenir les Espagnols, s'ils voulaient faire
quelque entreprise en Italie. Stanhope, alors secrétaire d'État,
feignait d'être ami particulier de Monteléon, et, sous couleur d'amitié,
tous ses propos ne tendaient qu'à l'intimider sur les résolutions que le
roi d'Angleterre serait obligé de prendre, et par l'engagement du traité
et par les ménagements qu'il devait comme prince de l'empire, auxquels
ses ministres allemands étaient fort attentifs\,; que quelques Anglais,
des principaux même, s'y laissaient entraîner, se souciant peu du
préjudice que le commerce de la nation pourrait souffrir de la rupture
avec l'Espagne.

Tandis qu'il lui parlait comme ami, Sunderland lui disait les mêmes
choses avec la hauteur naturelle aux Anglais. Il reprochait en termes
durs à l'Espagne de vouloir allumer une guerre générale. Il l'assura
qu'elle ne serait suivie de personne\,; que le régent déclarait vouloir
maintenir la neutralité d'Italie\,; que l'Angleterre était dans les
mêmes sentiments, et particulièrement obligée par son traité de garantie
avec l'empereur\,; que la Hollande suivrait les traces de
l'Angleterre\,; que, si l'Espagne comptait sur des mouvements à Naples,
elle devait savoir qu'on y voudrait changer de gouvernement toutes les
semaines\,; et que, si le roi de Sicile avait quelque part aux desseins
de l'entreprise de l'Espagne, il aurait bientôt lieu de s'en repentir.
On soupçonnait beaucoup en effet cette prétendue intelligence, parce
qu'il n'entrait dans la tête de personne que l'Espagne seule et sans
alliés entreprît d'attaquer l'empereur.

Les Impériaux, plus persuadés que personne du mauvais état de l'Espagne,
travaillaient de tous côtés à en pénétrer les intelligences secrètes. La
France leur était toujours suspecte. Koenigseck y redoublait d'attention
pour découvrir s'il se faisait dans le royaume quelques mouvements de
troupes, quelques préparatifs capables d'augmenter les soupçons. Ne
trouvant rien, il se réduisait à veiller sur la conduite de Ragotzi et
sur les secours qu'il pouvait espérer. Un coquin, nommé Welez, qui avait
été envoyé de Ragotzi en France, s'offrit à Koenigseck. Son maître
l'avait disgracié. Il promit à l'ambassadeur de l'empereur de l'informer
de tout ce qu'il pourrait découvrir. Il lui donna une lettre de la
princesse Ragotzi à ce prince, qu'il prétendit avoir interceptée. Il
l'assura qu'il y avait un traité fait à Paris, entre le czar et Ragotzi,
où les rois de Suède et de Pologne étaient compris\,; et que le moyen le
plus sûr d'en empêcher l'effet était d'assassiner Ragotzi, passant dans
l'État d'Avignon, parce qu'il n'y avait rien à craindre dans la
souveraineté du pape. Il l'avertit aussi de faire arrêter à Hambourg un
officier, appelé Chavigny, que Ragotzi envoyait en Pologne, et cela fut
exécuté de l'autorité de l'empereur.

Les États de Gueldre, sans consulter les États généraux, rendirent, au
commencement d'août, la liberté au baron de Goertz, lassés d'être les
geôliers du roi d'Angleterre qui en fut très fâché, et encore plus d'une
course que le czar, encore en Hollande, fit alors au Texel, qu'on crut
moins de curiosité que pour conférer avec Goertz. Ce soupçon fut
confirmé par la froideur que Widword, envoyé d'Angleterre, trouva dans
ce monarque. L'amiral Norris, que le roi d'Angleterre lui crut agréable,
et par lequel il lui fit proposer un traité de commerce et quelques
projets pour la paix du nord, ne fut pas mieux reçu.

Les vaisseaux Anglais qui se trouvaient dans la mer Baltique eurent
ordre de revenir dans les ports d'Angleterre. Georges voulait se trouver
en état de les employer comme il le jugerait à propos, suivant les
mouvements de ceux d'Espagne, en continuant néanmoins d'assurer le roi
d'Espagne de la correspondance parfaite qu'il voulait entretenir avec
lui. Quelques ménagements qu'il eût pour l'empereur, ses plaintes contre
l'Espagne étaient froidement écoutées à Londres, d'où néanmoins, pour
apaiser un peu les Impériaux, on fit partir le colonel Guillaume
Stanhope, cousin du secrétaire d'État, pour aller en Espagne. Il devait
d'abord passer en Hollande avec Cadogan, et le mener peut-être en
Espagne\,; mais, outre que ce général y était fort suspect, le ministère
Anglais crut en avoir besoin à Londres pour manéger dans le parlement
qui devait bientôt se rassembler. Georges n'avait pu parvenir à se
concilier l'affection des Anglais depuis qu'il était monté sur le trône.
Ils le croyaient dévoué à l'empereur, eux l'étaient à leur commerce\,;
et on parlait haut à Londres, à la Bourse, contre la rupture avec
l'Espagne.

Châteauneuf, ambassadeur de France à la Haye, alla un soir trouver
Beretti. Il lui dit, sous le plus grand secret, qu'il avait un conseil à
lui donner, dont il était moins l'auteur que le canal. Ce conseil fut
que l'Espagne ne devait pas s'alarmer des raisons ni des menaces de
l'Angleterre pour l'engager à se désister de son entreprise, mais
témoigner son étonnement de voir que cette couronne, après avoir si
tranquillement souffert tant d'infractions de l'empereur au traité dont
elle était garante, tant pour la sortie des troupes allemandes de la
Catalogne que pour la neutralité d'Italie, rompît aujourd'hui le
silence, et prit un ton si différent de celui dont elle avait usé à
l'égard de l'empereur. Il ajouta que le roi d'Espagne devait dire que,
n'ayant jamais fait de paix avec la maison d'Autriche, il se lassait
enfin d'en recevoir tant d'insultes\,; qu'il s'étonnait de la protection
qu'il semblait que le roi d'Angleterre voulait donner à la conduite de
la cour de Vienne, après tous les avantages obtenus par les Anglais de
Sa Majesté Catholique pour leur commerce ; mais qu'il était aisé de
l'interdire, et de donner des marques de ressentiment, si cette nation
continuait à favoriser les seuls ennemis de l'Espagne, qui était un
argument bien fort pour les contenir.

Cela fut dit avec un air si naturel et si sincère que Beretti ne fut
embarrassé que sur l'auteur du conseil, entre des membres principaux des
États généraux, ou par ordre du régent. En ce dernier cas Beretti
conclut que la France serait bien aise de voir l'Italie délivrée du joug
de la maison d'Autriche, dont la puissance devenait formidable, et la
devenait encore davantage alors par les victoires que le prince Eugène
venait de remporter sur les Turcs et la prise de Belgrade. Néanmoins les
ministres d'Angleterre craignaient que l'empereur ne fût attaqué en
Italie. Ils dirent même à Monteléon que, si l'entreprise regardait la
Toscane, même {[}s'il s'agissait{]} de mettre garnison dans Livourne du
consentement du grand-duc, la conséquence en serait bien moins grande
pour l'Angleterre que si elle se faisait à Naples ou en d'autres États
appartenant à l'empereur. Les ministres de ce monarque à Londres ne
cessaient de presser l'exécution de la garantie par des secours
effectifs, avec peu de succès, soit qu'on y voulût voir celui de
l'entreprise d'Espagne, ou que leurs personnes ne fussent pas agréables.
Wolckra fut en ce temps-là rappelé à Vienne pour faire place à
Penterrieder pour traiter la paix de l'empereur avec le roi d'Espagne,
par la médiation de la France, de l'Angleterre et de la Hollande, sur le
fondement des propositions faites l'année précédente à Hanovre,
concertées avec l'abbé Dubois, qui depuis avait toujours suivi cette
négociation, et qui devait la venir reprendre jusqu'à son entière
décision.

Saint-Saphorin, qui la conduisait à Vienne pour le roi d'Angleterre,
cherchait plus à se faire valoir qu'à la mener au gré du régent. Il ne
chercha dans les commencements qu'à lui inspirer des défiances des
personnes qui l'environnaient et qu'il pouvait consulter. Il disait que
le comte de Zinzendorff lui avait souvent parlé des cabales qui se
formaient contre lui, et voulait, sur ce qu'il avait tiré de ce ministre
et de quelques autres à Vienne, qu'il était de l'intérêt de l'empereur
de soutenir ceux du régent, dont les ennemis attachés aux maximes du
gouvernement précédent voulaient exciter des brouilleries dans l'Europe,
et réunir ensemble les deux monarchies de France et d'Espagne\,; que
l'unique moyen de s'y opposer était une union étroite entre l'empereur
et le régent, qui lui donnât courage et force nécessaire d'anéantir ses
ennemis qui étaient aussi ceux de l'empereur, et c'était, disait-il,
l'avis de Zinzendoff. Stairs, sous une apparente affection, avait
souvent tenu les mêmes langages. Il s'étonnait de la douceur et de la
patience du régent, qui, à son avis, s'il avait un procès devant le
conseil de régence ne l'y gagnerait pas. Lui et Saint-Saphorin, par qui
la négociation passait, tâchaient d'inspirer, à Vienne, l'opinion du peu
d'autorité du régent, en quoi ils ne pouvaient se déguiser leur
mensonge, surtout Stairs qui était sur les lieux. Koenigseck n'était
chargé de rien que du cérémonial. L'empereur voulait qu'il obtint les
mêmes distinctions dont jouissait le nonce, mais avec un ordre secret de
s'en désister s'il ne pouvait soutenir cette prétention sans se mettre
hors d'état de traiter les affaires dont il pourrait être chargé.

La cour de Vienne, embarrassée dans la guerre de Hongrie, avait une
grande inquiétude que l'entreprise d'Espagne ne se bornât pas à la
Sardaigne. L'Angleterre, qui lui trouvait trop d'ennemis, ne se pouvait
persuader que le roi de Sicile ne fût du nombre par son intérêt et par
celui de l'Espagne, qu'on n'imaginait pas pouvoir s'en passer\,; et les
ministres du roi d'Angleterre ne se pouvaient rassurer sur les réponses
constantes que La Pérouse, ministre de ce prince à Londres, faisait à
leurs questions. Les ministres allemands de Georges, aussi ardents que
ceux de l'empereur, ne cessaient de le presser d'aider ce prince et de
hâter le départ du colonel Stanhope. Bothmar était le plus ardent, mais
Bernsdorff, plus modéré, concourait en tout avec lui.

Les flatteurs d'Albéroni le louaient particulièrement de son
impénétrable secret, inconnu depuis tant d'années en Espagne\,; mais il
avait été trop poussé à l'égard de la France\,; elle s'en plaignait.
Enfin, vers la fin d'août, Cellamare reçut ordre du roi d'Espagne de
rompre le silence, et de dire au régent que, s'il ne lui avait pas
communiqué plus tôt son projet, il ne le devait pas attribuer à manque
de confiance, mais à égard et à considération, pour ne l'exposer à aucun
embarras à l'égard de l'empereur, et ajouta Cellamare de lui-même, à
celui de mécontenter le conseil de régence en ne lui en faisant point
part, ou en la lui faisant d'en exposer le secret. Il n'oublia rien pour
faire goûter ce long mystère\,; mais il n'eut pas lieu d'être content de
trouver le régent persuadé de l'intérêt de la France à conserver la
paix, et que, loin d'entrer dans les vues du roi d'Espagne, il ne devait
rien oublier pour empêcher la moindre altération dans la tranquillité
publique. Cellamare attribua cette disposition à des vues futures et
personnelles. Cet ambassadeur, qui voulait faire sa cour, regardait
comme le point capital l'établissement des droits de sa reine sur la
succession de Toscane, et comme celle qui devait être soutenue avec le
plus de force, l'épée et la plume à la main. Mais il se plaignait du peu
de prévoyance qu'il trouvait en France, où le présent seul faisait
impression sur les esprits. En même temps des émissaires de l'empereur
tâchaient de lui faire accroire que la France agissait de concert avec
l'Espagne pour le dépouiller de ce qu'il possédait en Italie, ainsi que
le roi de Sicile.

Supposant aussi les mouvements des mécontents de Hongrie comme une
branche du projet, ils firent arrêter à Hambourg des officiers attachés
à Ragotzi, et prirent des mesures pour le faire enlever ou tuer
lui-même, soit qu'il voulût passer en Hongrie, ou joindre les Espagnols
en Italie\,; et on sut que l'un d'eux devait recevoir six livres par
jour, outre les dédommagements des frais de la suite de ce prince,
auquel on détacha aussi d'autres espions.

L'inquiétude des Impériaux était tellement étendue qu'un espèce d'agent
du czar, nommé Le Fort, étant parti alors de Paris pour Turin, ils en
inférèrent des liaisons secrètes de ce prince avec le roi de Sicile. Le
czar était très suspect aux Anglais. On a vu que Widword et l'amiral
Norris l'avaient inutilement caressé en Hollande sur le commerce et sur
les vues de la paix du nord, et sur l'amitié du roi d'Angleterre. Les
Moscovites, pour toute réponse, avaient insisté sur le projet agité
l'hiver précédent\,; que c'était uniquement sur ce pied-là, et d'une
garantie mutuelle, qu'ils traiteraient avec le roi d'Angleterre\,;
qu'ils ne l'engageraient pas à former un concert pour la paix, non plus
qu'à tenter aucune entreprise, quand l'engagement ne serait que pour un
an. Les Anglais, dans ce mécontentement du czar, s'en consolèrent sur
l'espérance, qu'ils commencèrent à prendre, que les dispositions du
régent étaient sincères, qu'il observerait la triple alliance, qu'il
agirait de bonne foi avec eux pour empêcher le renouvellement de la
guerre.

On sut enfin que la flotte d'Espagne ayant fait voile de Barcelone le 15
juillet, une partie était arrivée devant Cagliari le 10, l'autre le 21
août\,; que le marquis de Lede, général des troupes, ayant fait toutes
les dispositions nécessaires pour la descente, avait fait sommer le
marquis de Rubi, vice-roi pour l'empereur, que, sur son refus, dix-huit
mille hommes\footnote{Le manuscrit porte dix-huit mille hommes. La
  plupart des historiens disent que l'armée de débarquement n'était que
  de huit mille hommes.} avaient mis pied à terre\,; que le vice-roi,
sommé une seconde fois, avait répondu comme à la première\,; qu'il
n'avait que cinq cents hommes de garnison, et qu'on doutait qu'il pût se
défendre six ou sept jours au plus. Ce commencement de guerre conduisait
à un embrasement général de l'Europe, selon les raisonnements des
politiques.

Le vice-roi de Naples, craignant d'avoir bientôt les Espagnols sur les
bras, prenait toutes les mesures qui lui étaient possibles\,; et Gallas,
soupçonnant le pape d'être d'intelligence avec l'Espagne, ne se
contentait d'aucunes raisons. Il le menaçait et demandait qu'il se
justifiât par des déclarations publiques, en répandant dans Rome les
grands et imminents secours des princes engagés dans la triple alliance,
et à la garantie de la neutralité de l'Italie. Le pape, épouvanté,
résolut d'apaiser l'empereur. Il rassembla devant lui la congrégation
qui avait examiné l'accommodement des cours de Rome et de Madrid. Il y
résolut de révoquer les concessions qu'il avait faites au roi d'Espagne
pour lui donner moyen d'équiper la flotte destinée contre les Turcs,
qu'il employait contre l'empereur, et d'écrire au roi d'Espagne une
lettre dont les Impériaux fussent contents\,; cela fait, d'offrir sa
médiation à l'empereur pour calmer ces mouvements de guerre.

Ces mesures, et la nouvelle que reçut le pape en même temps
d'Aldovrandi, qu'il était en pleine possession de la nonciature, le
rendirent plus traitable dans l'audience qu'il donna à Acquaviva. Ce
cardinal crut même s'apercevoir qu'il craignait que l'entreprise de
Sardaigne ne réussît pas, ou que, si elle était heureuse, l'Espagne ne
s'en tînt là. Le pape voyait qu'il y en avait assez pour faire venir les
Impériaux en Italie, et pas assez pour les en chasser, parce qu'il
commençait à paraître clair que l'Espagne était seule, et s'était
embarquée sans aucuns alliés. Les flatteurs d'Albéroni le berçaient de
la jonction du pape, des Vénitiens et du roi de Sicile, dès que les
Espagnols auraient mis le pied en Italie. Il était pourtant difficile
que ces mêmes gens-là en crussent rien. Il semblait que, dans cette
conjoncture critique, il eût été du service du roi d'Espagne de réparer
par des attentions et des grâces l'avantage, qu'il avait perdu avec
l'Italie, d'avoir, comme ses prédécesseurs, beaucoup de cardinaux
dépendants, attachés et affectionnés. Au contraire d'y travailler,
l'animosité d'Albéroni et d'Acquaviva contre Giudice lui attira des
désagréments publics. Le roi d'Espagne lui fit ordonner d'ôter de dessus
sa porte à Rome les armes d'Espagne. Ses représentations furent
inutiles, ainsi que les offices du régent qu'il réclama, et que ce
prince lui accorda. Il protesta de son attachement pour la France, de
son empressement à le marquer. Il chercha à se lier au cardinal de La
Trémoille, son ancien ami, malgré tout ce qui s'était passé entre la
princesse des Ursins et lui. Il était de la congrégation du
saint-office. La Trémoille le ménagea par cette raison pour les affaires
de France, que Bentivoglio et ses adhérents embrasaient plus que jamais.

Ce fut en ce temps-ci que la position dangereuse de l'armée impériale,
enfermée entre celle du grand vizir, qui venait secourir Belgrade, et
cette place assiégée, tenait les amis et les ennemis de la maison
d'Autriche dans une merveilleuse attente. Elle ne dura pas, et la
victoire complète que le prince Eugène remporta sur les Turcs, la prise
de Belgrade, et tous les succès qui la suivirent rapidement, fut une
nouvelle incontinent répandue partout. Le régent, livré à l'Angleterre,
s'était rendu à ses instances sur son union avec l'empereur\,; mais ce
prince, malgré la situation heureuse dans laquelle il se trouvait, et
les propositions qu'il recevait de la part du régent, se défiait de ses
desseins cachés, qui est le caractère le plus facile, et en même temps
le plus de celui de la cour de Vienne.

On a vu les desseins de cette cour sur Ragotzi. Ses ministres
n'oubliaient rien pour veiller ses actions, et pour l'exécution de leurs
ordres. Son séjour était encore matière d'un continuel soupçon à l'égard
de la France. Welez, espion de l'empereur, dont on a déjà parlé, était
chargé de le défaire de cet ancien chef des mécontents de Hongrie, à
condition des plus grandes récompenses. Il avait ordre de communiquer à
Koenigseck tout ce qui regardait cette importante affaire. Sur les avis
qu'il donna, l'empereur fit arrêter à Staden deux François, qui étaient
à Ragotzi\,: Charrier, son écuyer\,; l'autre avait pris le nom de comte
de L'Hôpital. Welez informa Koenigseck du départ de Ragotzi, de la route
qu'il avait prise, et des détails les plus précis, avec des réflexions
qui donnaient au régent toute la part de ce dessein, et tous les secours
pour l'exécution. Ses preuves étaient que Ragotzi ayant permis au jeune
Berzini d'aller joindre son père dans l'armée des Turcs, son rang de
colonel et ses appointements lui étaient conservés au service de France.
Welez sut positivement le jour que Ragotzi arriva à Marseille, la maison
où il logeait, ses conférences avec l'envoyé turc, le vaisseau qu'il
devait monter, et qu'il lui avait été préparé par ordre du comte de
Toulouse, d'où il concluait qu'il n'y avait pas lieu de douter des
secours et des intentions de la France contre l'empereur. Cet homme se
persuada que le prince Ragotzi ne continuerait pas son voyage à
Constantinople, lorsqu'il apprendrait la victoire et les conquêtes des
Impériaux en Hongrie, et se flatta bien à son retour de ne pas manquer
son coup, pour en délivrer l'empereur, et se procurer les grâces sans
nombre qui lui étaient promises. Il crut en même temps que l'empereur
voudrait que le coup fût précédé ou suivi de quelques plaintes au
régent. Il offrit de fournir telles preuves qu'on pourrait désirer pour
justifier que le régent était pleinement informé des desseins de ce
prince, et par conséquent qu'il avait manqué à la parole qu'il avait
donnée là-dessus à Penterrieder, pendant que ce secrétaire était à
Paris.

Cependant l'empereur écoutait les propositions faites par l'Angleterre,
et avait promis de faire partir dans un mois Penterrieder, pourvu que
l'abbé Dubois se rendît en même temps à Londres. Il doutait néanmoins
toujours des véritables intentions du régent. Il se proposait de les
examiner de près, par la conduite qu'il tiendrait sur le mouvement des
Espagnols vers l'Italie. Il ne prétendait s'engager qu'autant qu'il
trouverait ses avantages, et ne se pas contenter de peu. Le roi
d'Angleterre, bien plus enclin à l'empereur qu'au régent, n'oubliait
rien pour se donner le mérite de ses services à la France, et
Saint-Saphorin vantait ses soins qui valaient au régent la considération
personnelle de l'empereur qui, à cause de lui, voulait bien laisser un
terme à l'Espagne pour accepter le traité, et qu'il consentait en cas de
refus qu'il fût libre à la France d'assister Sa Majesté Impériale
d'argent sans être obligée à prendre les armes contre le roi d'Espagne.
La même complaisance était accordée en cas qu'il fût question de faire
la guerre au roi de Sicile, pour l'obliger à céder cette île.

Saint-Saphorin relevait beaucoup cette modération de l'empereur, et les
soins et l'habileté qu'il avait mis en usage pour l'y conduire. Il
louait ce prince de donner cette marque du désir sincère qu'il avait de
concourir à l'affermissement du repos public. En même temps le roi
d'Angleterre avertissait le régent d'être fort sur ses gardes contre le
parti du roi d'Espagne en France, appuyé de toute l'ancienne cour,
lequel, suivant tous les avis de Hollande, était persuadé que, s'il
arrivait malheur au roi, le régent n'aurait pas assez d'amis pour le
porter sur le trône. Enfin on ajoutait que le czar offrait ses secours
au roi d'Espagne dans la vue de se conserver toujours une part
considérable dans les affaires de l'Europe, et un prétexte de renvoyer
et tenir de ses troupes en Allemagne. De tout cela Georges concluait
que, s'il s'élevait une guerre civile en France, le régent avait grand
intérêt d'acquérir, à quelque prix que ce fût, des amis assez puissants
pour maintenir ses droits contre ses ennemis. Mais pour une guerre
civile, il faut des chefs en premier et en divers ordres, une
subordination, des têtes et de l'argent. Il n'y avait rien de tout cela
en France. L'inanition était son grand mal\,; elle n'avait rien à
craindre de la réplétion. Nulle harmonie, nulle audace qu'au coin du
feu, une habitude servile qui dominait partout, et qui, au moindre
froncement de sourcil, faisait tout trembler, ceux qui pouvaient figurer
en premier et en second encore plus que les autres.

Chaque prince se croit habile de couvrir ses intérêts du prétexte de
zèle pour ceux de son allié. Ainsi dans ce même temps le roi de Prusse,
sous le même prétexte de l'intérêt de la France, la pressait d'agir
vivement pour la paix du nord, de peur que l'empereur n'en eût le
mérite, à l'exclusion de la France, parce que, depuis sa victoire de
Hongrie, les princes du nord paraissaient portés à recourir à sa
médiation préférablement à toute autre. Ensuite il se plaignait du peu
de secret gardé sur le traité que la France avait conclu avec lui. Il
priait le régent de lui faire savoir ce qu'il devait répondre aux
questions fréquentes des ministres de l'empereur, de l'Angleterre et du
czar, lequel il attendait à Berlin vers le 15 septembre, avec lequel il
espérait décider alors de la paix ou de la continuation de la guerre
avec la Suède.

Goertz, sorti des prisons de Hollande, retournant en Suède toujours
honoré de la confiance de son maître, s'était arrêté à Berlin, où il
avait promis d'attendre le czar, et en l'attendant avait agité avec les
ministres de Prusse quelques projets pour parvenir à la paix. Ils
auraient voulu le trouver plus facile. C'était selon eux une espèce
d'impossibilité de prétendre la restitution des États envahis par
l'Angleterre et le Danemark sur la Suède, dureté ou défiance à Goertz de
refuser, comme il faisait, de se contenter pour cela des simples offices
du roi de Prusse. Ce prince voulait traiter avec lui et le préférait à
Spaar, son ennemi, qui n'avait pas la même confiance du roi de Suède. Le
point capital du roi de Prusse était d'obtenir la cession de Stettin et
de son district. Goertz demandait pour conditions\,:

La restitution des provinces et des places conquises sur la Suède par le
czar, à l'exception de Riga\,;

Celle de Stralsund, Rügen et du reste de la Poméranie\,;

Celle de Brême et de Verden\,;

Que le roi de Prusse s'engageât par un traité particulier avec le roi de
Suède à faire rétablir le duc de Holstein dans son État\,;

Enfin, que le roi Stanislas fût appelé au trône de Pologne et assuré d'y
monter après la mort du roi Auguste, et qu'il jouît en attendant d'un
revenu sûr et convenable à son rang.

Quelque difficiles que fussent ces conditions, le roi de Prusse
craignait de laisser échapper un commencement de négociation directe
avec la Suède. La France lui devenait très suspecte parce qu'il la
croyait tout à l'Angleterre. Il trouvait les instances du comte de La
Marck lentes et froides auprès du roi de Suède. Il se tenait pour bien
averti que le landgrave de Hesse agissait pour obtenir de la Suède que
le roi d'Angleterre conservât Brême et Verden\,; qu'en ce cas les
intérêts de la Prusse seraient sacrifiés, et que le landgrave serait, en
récompense du succès de cette négociation, porté à la tête des
Provinces-Unies en qualité de stathouder. Ainsi le roi de Prusse se
contentait de continuer à solliciter les offices du roi auprès de la
Suède\,; mais il ordonna à Kniphausen, son ministre à Paris, d'y cacher
avec grand soin les propositions de Goertz et l'état de la négociation
commencée à Berlin. Ce ministre en avait entamé une à Paris pour
faciliter le payement des subsides dus à la Suède en vertu du traité
qu'elle avait fait avec le feu roi. Goertz s'était figuré un prompt et
facile payement s'il pouvait gagner le sieur Law, et lui avait fait
offrir une gratification de six pour cent. Le négociateur était un
secrétaire que Goertz avait envoyé exprès à Paris. Comme il agissait
indépendamment de l'envoyé de Suède\,; celui-ci se plaignait du
préjudice que cette négociation indépendante pouvait causer aux affaires
dont il était chargé, et de plus Law n'était pas homme à se prêter à des
choses de cette nature, et à n'en pas avertir. Les plaintes de cet
envoyé ne nuisirent pas aussi à découvrir la tentative infructueuse de
Goertz. Ce fut en ce temps-là que les Suédois découvrirent si à propos
l'entreprise d'enlever le roi Stanislas aux Deux-Ponts, qui fut sur le
point de réussir, comme on l'a déjà dit.

L'Angleterre, garante de la neutralité d'Italie, et de plus engagée avec
l'empereur, par leur traité de l'année précédente, à lui garantir les
États dont il était en possession, se plaignit vivement de l'infraction
de l'Espagne\,; mais comme il n'était pas de l'intérêt des Anglais de
rompre avec elle, ils protestèrent que leur roi maintiendrait toujours
une intelligence et une amitié constantes avec le roi d'Espagne\,; et
pour confirmer ces assurances, il fut résolu de faire partir
incessamment le colonel Stanhope pour Madrid, qui y était destiné depuis
longtemps. L'objet de cet envoi était de préparer de loin la cour
d'Espagne à concourir au traité que le roi d'Angleterre se proposait de
faire entre l'empereur et cette couronne. Georges pressait l'arrivée de
Penterrieder à Londres, et pria en même temps le régent de ne point
faire partir l'abbé Dubois pour s'y rendre, qu'il n'eût appris que
Penterrieder était en chemin. Ce prince ne cessait de représenter au
régent l'intérêt pressant qu'il avait de s'unir étroitement avec
l'empereur, et d'avoir de puissants amis qui maintinssent son autorité,
qu'il croyait fort ébranlée par les mouvements du parlement de Paris et
des cabales qui, selon lui, s'étendaient jusque dans le nord, et qui
avaient engagé le czar d'envoyer un ministre à Madrid et un autre à
Turin. Stairs eut ordre de lui tenir le même langage et de l'avertir que
le baron de Schemnitz, qui venait en France de la part du czar,
s'attacherait à la même cabale, surtout à d'Antin et aux maréchaux de
Tessé et d'Huxelles. Il n'y avait qu'à connaître les personnages pour
n'en avoir pas grand'peur.

Le ministère de Londres en avait beaucoup du czar, qui ne cachait point
ses mauvaises dispositions pour Georges. Ce dernier monarque et ses
ministres, surtout les Allemands, haïssaient le roi de Prusse et ses
ministres Ilghen et son gendre Kniphausen, lequel ils croyaient avoir
fabriqué une ligue avec le vice-chancelier du czar, fort contraire à
l'Angleterre, qu'ils niaient depuis la victoire de Hongrie, mais qui
leur faisait craindre des mouvements du Prétendant, qui avait des gens à
lui à Dantzig, peut-être même le duc d'Ormond. Ils crurent avoir trouvé
plus de froid dans le czar depuis que ses ministres avaient conféré avec
ceux de France et de Prusse. Leur inquiétude sur la France ne put être
rassurée par les assurances que Châteauneuf leur donna de n'avoir été à
Amsterdam que pour marquer son respect au czar, sans avoir eu la moindre
affaire à traiter avec lui. Châteauneuf avait été employé par le feu
roi, et c'en était assez pour mériter toute la haine du ministère de
Georges. Aussi n'oublièrent-ils rien pour le faire rappeler, et pour
engager le régent d'envoyer un autre ambassadeur en Hollande.

Ce fut en ce temps-ci que le vicomte de Bolingbroke fut reçu, mais
secrètement, en grâce, et que Stairs eut ordre de le dire au régent, et
de le prier de le regarder désormais comme un sujet que le roi
d'Angleterre honorait de sa protection. Stanhope, passant en France pour
aller en Espagne, eut ordre aussi de faire voir au régent les
instructions dont il était chargé. Le régent ne les ayant pas trouvées
assez fortes, le colonel offrit de recevoir celles qu'il lui voudrait
dicter, ayant ordre de se conformer d'agir avec un parfait concert en
Espagne avec l'ambassadeur de France. Stairs et lui eurent de longues
conférences avec l'abbé Dubois, et tous deux en parurent très contents.
Ils dirent même que le duc de Noailles et le maréchal d'Huxelles
semblaient se disputer à qui seconderait le mieux les vues du roi
d'Angleterre. C'est un éloge que je n'ai jamais mérité.

Albéroni, se flattant du succès immanquable de son entreprise et plus
encore des suites qu'il s'en promettait, éloignait toute proposition de
traités et de négociations, et s'il était forcé de les entendre, les
voulait remettre à l'hiver. Il comptait beaucoup sur la Hollande.
Beretti, pour le flatter et faire valoir ses services, ne doutait point
de l'en assurer. L'intimité avec laquelle Albéroni vivait avec Riperda
le faisait croire aussi au dehors. Cet ambassadeur était d'une maison
illustre de la province d'Over-Yssel, mais sans biens. Il ne subsistait
que des appointements de l'ambassade. Il avait été catholique, mais il
s'était perverti pour entrer dans les charges de son pays. Il n'avait pu
néanmoins en obtenir aucune, et comme il n'était nullement estimé, son
choix avait étonné tous ses compatriotes.

La république de Venise ne laissa pas le monde dans une si longue
incertitude. Le noble Mocenigo était, sans caractère à Madrid, chargé de
ses ordres\,; on y fut bien étonné de lui entendre dire que sa
république était obligée par son traité avec l'empereur de lui fournir
dix mille hommes, en cas d'infraction à la neutralité de l'Italie.

Albéroni entra dans une furieuse colère qu'il ne prit pas le soin de lui
déguiser. Ses vanteries étaient sans mesure sur les ressources et la
puissance que l'Espagne montrerait dans peu, et qui n'étaient dues qu'à
ses soins. L'entreprise de Sardaigne n'était qu'un coup d'essai. Il
promettait, pour l'année suivante, une telle irruption en Italie, où il
voulait engager tout le monde à l'aider à en chasser les barbares, que
l'empereur occupé en Hongrie, dont il fallait profiter, n'aurait pas le
temps d'y envoyer des troupes, et le tout pour mettre l'équilibre dans
l'Europe. Il n'était point touché de la conquête de Naples, qu'il ne
pouvait soutenir que par mer, tandis que l'empereur y pouvait envoyer
des secours de plain-pied, outre que ce royaume tomberait de soi-même,
si les succès étaient heureux en Italie.

Il était résolu à se borner cette année à la Sardaigne\,; mais il voulut
se faire en France, surtout à Rome, un mérite de cette modération forcée
par la saison qui n'en permettait pas davantage. Cellamare eut ordre de
la faire valoir comme une complaisance pour les instances du régent et
du pape, et la suspension de l'embarquement pour l'Italie comme une
marque de disposition à la paix\,; que le roi d'Espagne espérait aussi
que cette complaisance engagerait le régent et le pape de se joindre à
lui pour donner l'équilibre à l'Italie, et le repos, par conséquent, à
l'Europe. En même temps il eut l'audace d'écrire au pape qu'il se
représentait la joie qu'il aurait d'apprendre, par une lettre de la main
du premier ministre d'Espagne, que ses instances avaient eu le pouvoir
d'arrêter l'embarquement prêt à passer en Italie, satisfaction qu'il
n'aurait pas obtenue s'il n'avait pas eu en Espagne un cardinal sa
créature. Cette feinte complaisance n'abusa personne\,; elle fut
attribuée à Rome et à Paris, non à déférence, mais à nécessité.

Albéroni, qui, comme on l'a vu, s'était déjà servi d'Aldovrandi pour
faire accroire à Rome que l'entreprise était entièrement contre son avis
et sa volonté, persévérait si bien à vouloir persuader cette fausseté
insigne que peu s'en fallut qu'il n'obtînt une lettre de la main du roi
d'Espagne pour la lui certifier. Le premier ministre voyait et sentait
les suites que pouvait avoir l'engagement où il venait de se mettre, et
son propre péril, si l'Espagne venait à lui reprocher les conséquences
fatales de ses conseils. Il désirait donc ménager le pape, et faire en
sorte qu'il s'interposât pour concilier l'empereur et le roi d'Espagne,
et qu'il procurât une paix utile et nécessaire à l'Europe. La partie
était trop inégale.

La paix du Turc paraissait prochaine\,; les Allemands menaçaient déjà
l'Italie, et parlaient hautement de mettre des garnisons impériales dans
Parme et dans Plaisance. Dans cette situation, Albéroni, sans nul allié,
se montrait aussi opiniâtre aux représentations des princes amis de
l'Espagne que si toute l'Europe se fût déclarée pour elle.

Le roi d'Angleterre lui fit dire l'embarras où le mettrait l'engagement
qu'il avait pris avec la France et avec l'empereur, si ce prince lui
demandait en conséquence la garantie des États qu'il possédait en
Italie, ne voulant d'ailleurs rien faire qui pût troubler la bonne
intelligence qu'il avait, lui Georges, avec le roi d'Espagne, et qu'il
prétendait entretenir fidèlement. Sur ce fondement, l'envoyé
d'Angleterre à Madrid demanda l'explication précise des desseins du roi
d'Espagne, en sorte que le roi d'Angleterre pût juger certainement du
parti qu'il avait à prendre. Albéroni répondit que l'expédition de
Sardaigne n'avait d'autre motif que la juste vengeance des insultes
continuelles et des infractions des traités\,; qu'il ne voulait mettre
aucun trouble en Europe\,; qu'il était particulièrement éloigné de tout
ce qui pouvait altérer le repos et la tranquillité de l'Italie\,; qu'il
contribuerait de toutes ses forces à maintenir la paix, qui ne pouvait
être solidement établie que par un juste équilibre qu'il était
impossible de former, tant que la puissance de l'empereur serait
prédominante en Italie. Cet équilibre était le bouclier dont il couvrait
les entreprises qu'il méditait.

Comme il croyait le roi d'Angleterre trop étroitement lié avec
l'empereur pour en rien espérer, il se tournait tout entier vers la
Hollande, à qui, par Riperda, il faisait entrevoir les avantages qu'elle
pouvait attendre d'une amitié et d'une alliance particulière avec
l'Espagne, laquelle était disposée à faire ce qu'une aussi sage
république jugerait nécessaire pour le repos de l'Europe. En même temps,
il essayait de lui indiquer la route que lui-même y jugeait la
meilleure.

Il avait enfin confié à Beretti le plan qu'il s'était proposé de suivre,
qu'il fallait ménager adroitement, sans laisser entendre que ce fût un
projet véritablement formé en Espagne, en parler à propos et dans les
occasions, ne le pas expliquer d'abord entièrement, mais suivant les
conjonctures en découvrir une partie, ensuite une autre, exciter le
désir d'en savoir davantage et d'être admis à une plus grande confiance.
C'était par ces manèges que Beretti devait marquer les talents qu'il
prétendait avoir pour les négociations.

L'objet d'Albéroni était 1° de sauver l'honneur du roi d'Espagne\,; 2°
d'établir et confirmer le repos de l'Italie\,; 3° d'assurer les
successions de Toscane et de Parme aux fils de la reine d'Espagne. Le
projet, dressé sur ce fondement, était de partager les États d'Italie\,;

Obtenir pour le roi d'Espagne Naples et Sicile, et les ports de Toscane,
et l'assurance réelle des États du grand-duc et du duc de Parme pour un
des fils de la reine, si ces princes mouraient sans héritiers\,;

Diviser l'État de Mantoue eu donnant une partie du Mantouan au duc de
Guastalla, et l'autre partie, avec la ville de Mantoue, aux Vénitiens\,;

Le Milanais entier, avec le Montferrat, à l'empereur, et la Sardaigne,
au duc de Savoie, pour le dédommager de la Sicile, et lui conserver le
titre de roi qu'il aurait perdu avec la Sicile\,;

Enfin la restitution de Commachio au pape, pour faire acte de sa
créature.

À l'égard des Pays-Bas catholiques, il les partageait entre la France et
la Hollande.

Tel était le plan qu'Albéroni s'était fait. Il rejetait toute autre
proposition, principalement la simple assurance des successions de
Toscane et de Parme à un fils de la reine, qu'il appelait un appât
trompeur, un leurre des amis de l'empereur pour lui laisser loisir et
liberté de s'emparer de toute l'Italie en moins de deux mois. Il
représentait soigneusement ce prince comme en état d'imposer des lois à
toute la terre après ses victoires de Hongrie, mais dont il n'était pas
impossible d'arrêter les vastes desseins par de justes bornes, si toute
la terre ne se laissait pas saisir d'une terreur panique. Il voulait
persuader que les troupes impériales étaient fort diminuées par les
maladies, et que les Turcs reparaîtraient en Hongrie plus en force que
jamais. De tout cela on concluait que ce cardinal voulait allumer un
incendie en Italie qui embrasât toute l'Europe, et qui obligeât les
puissances les plus éloignées à s'unir pour donner des bornes à celle de
l'empereur, persuadé que, si le succès était heureux, la gloire et
l'avantage en demeureraient à l'Espagne, sinon qu'elle ne recevrait
aucun préjudice d'avoir fait une tentative inutile. De là, il disait que
l'Espagne se contenterait pour cette année de ce qu'elle n'avait pu
refuser à son honneur blessé, donnerait le temps de l'hiver aux
puissances de l'Europe de chercher à mettre l'Italie à couvert\,; que si
cela n'était pas, au printemps il y allumerait un incendie, qu'elles
seraient forcées d'y accourir, et de le venir éteindre. Il s'emportait
ensuite contre chacune d'elles, surtout contre l'Angleterre, en
plaintes, en reproches, en menaces.

Ainsi, il s'avouait partout l'auteur de la guerre, excepté à Rome, où il
voulait persuader au pape qu'il verrait clair quelque jour à tout ce
qu'il avait fait pour empêcher le mal\,; lui promettait de susciter tant
d'embarras au second convoi qu'il l'empêcherait de partir de Barcelone
(d'où en effet il ne pouvait ni ne voulait le faire partir)\,;
proposait, comme un expédient glorieux au pape, d'offrir sa médiation\,;
faisait l'embarrassé de parler au roi d'Espagne contre son goût et sa
volonté\,; se faisait valoir de s'occuper et de chercher à en saisir les
moments favorables, comme si tout n'eût pas dépendu de lui uniquement,
comme il l'avait tant de fois fait dire au pape par toutes sortes de
voies, lorsqu'il s'agissait de presser sa promotion, comme il était vrai
aussi, et comme personne n'en doutait en Europe. Il donnait pour témoins
de sa conduite contraire à cette entreprise le P. Daubenton et le nonce
Aldovrandi, tous deux en esclavage sous lui pour conserver leurs postes,
qui répétaient ce qu'il leur dictait, jusqu'aux particularités les plus
imaginaires, pour prouver que le conseil d'État l'avait emporté sur lui,
ce conseil qu'il avait anéanti, et de la destruction duquel il s'était
vanté à Rome et dans les autres cours. En un mot, selon eux., la capture
de Molinez avait tellement irrité le roi et le conseil d'État
qu'Albéroni n'avait pu faire que des efforts inutiles. Ainsi,
Aldovrandi, avouant que l'Espagne avait manqué de parole, en détournait
la faute sur le conseil d'État, exhortait le pape à ne pas prendre des
conseils violents, qui, par la rupture avec l'Espagne, seraient d'un
grand préjudice à la cour de Rome, et n'obtiendraient pas grande
reconnaissance de l'empereur\,; appuyait sur l'offre de sa médiation,
surtout à ménager Leurs Majestés Catholiques et leur premier ministre,
l'unique qui pût obtenir quelque chose d'elles. Ce même homme, qui ne
pouvait rien sur cette grande affaire, était pourtant le seul qui pût
tout, et cela dans la même bouche et dans les mêmes dépêches
d'Aldovrandi. C'est ainsi que l'artifice et l'imposture se trahissent,
même avec grossièreté.

Les Impériaux n'ignoraient pas la conduite de ce nonce. Maîtres de
l'Italie, rien n'était secret pour eux à Rome. Le pape, effrayé de leurs
menaces, n'était occupé qu'à se laver auprès d'eux de toute intelligence
avec l'Espagne\,; et eux répliquaient qu'il ne le pouvait que par le
châtiment d'un ministre ignorant, s'il n'avait rien découvert de cette
entreprise, infidèle si, l'ayant sue, il n'en avait pas averti le pape.
Ce pontife, qui croyait déjà voir l'État ecclésiastique en proie aux
Allemands, chercha à les apaiser par des brefs qu'il écrivit en Espagne,
et à en adoucir la dureté des expressions par le moyen d'Aldovrandi.

Celui qu'il adressa au roi d'Espagne était rempli de plaintes et de
reproches vifs de son entreprise. Il en attribuait le projet à ses
ministres\,; il lui demandait de réparer au plus tôt le mal qu'il
faisait à la chrétienté, par la diversion dés troupes de l'empereur,
occupées avec gloire et succès contre les infidèles. Ceux qui furent
adressés au premier ministre et au confesseur étaient de la main du
pape. Il faisait au premier l'exhortation la plus pathétique du côté de
Dieu et des hommes, pour employer tout son crédit à obtenir sur le repos
de l'Italie ce qu'Aldovrandi lui dirait, et les instances étaient
d'autant plus pressantes, que l'agitation était extrême à Rome sur la
prochaine paix du Turc, et une guerre imminente en Italie, où l'empereur
ne désirait qu'un prétexte de porter ses armes.

Le duc de Parme, qui comptait bien être exposé tout le premier à la
vengeance de ce prince, implorait vainement la protection du pape, comme
de son seigneur suzerain, pour mettre Parme et Plaisance à couvert à
l'ombre d'une garnison des troupes de l'Église, et celle d'Espagne en
représentant à Albéroni le triste état de sa situation.

Ce n'était plus le temps où ce premier ministre était le sien et son
sujet en Espagne\,; il n'avait plus besoin de lui pour hâter sa
promotion\,; elle était faite, et désormais il n'avait plus rien qui le
pût détourner de suivre ses vues et son entreprise, ni d'écouter aucune
représentation, encore moins les reproches\,: qu'il ne devait la pourpre
qu'aux promesses d'envoyer la flotte d'Espagne contre les Turcs,
{[}reproches{]} qui l'irritèrent, et qu'il crut devoir l'affranchir de
toute reconnaissance.

Le pape, outré de ne pouvoir rien gagner sur lui, eut la faiblesse de
dire au cardinal del Giudice qu'il savait bien qu'il se damnait en
élevant un tel sujet à la pourpre, mais qu'il s'était trouvé engagé si
fortement au roi et à la reine d'Espagne qu'il n'y avait pas eu moyen de
les refuser\,; sur quoi Giudice lui répondit plaisamment qu'il se ferait
toujours honneur de suivre Sa Sainteté partout où elle irait, hors à la
maison du diable.

Dans ces détresses, Acquaviva lui dit que l'Espagne bornerait ses
conquêtes à la Sardaigne, s'il pouvait promettre que l'empereur
observerait exactement la neutralité d'Italie, qu'il n'y enverrait point
de troupes au delà du nombre stipulé par les traités, qu'il n'y lèverait
point de contributions, qu'enfin il ne mettrait point de garnisons dans
les places de Toscane. Le pape fit mine de sacrifier avec peine son
ressentiment du manque de parole de l'Espagne au bien public. Il en
parla à Gallas, et tous deux dépêchèrent à Vienne en conséquence. Le
pape y comptait peu sur son crédit. Rien n'égalait le mépris où il était
dans cette cour, persuadée qu'il ne cherchait que les avantages de sa
maison, et d'envoyer, à l'occasion de cette négociation, son neveu
Alexandre à la cour impériale. Le pape en sentait le mépris, mais il
comptait aussi que le crédit de Stella sur l'esprit de l'empereur lui
obtiendrait ce qu'il n'osait espérer par lui-même, et qu'il disposerait
aisément de ce favori moyennant un chapeau pour son frère.

Molinez était sorti du château de Milan, et avait été conduit dans un
collège de la ville, où il était gardé par des soldats de l'Église. Cela
pouvait satisfaire les vastes prétentions de l'immunité ecclésiastique,
mais non pas l'Espagne, ni la violation en sa personne de la neutralité
de l'Italie. Son âge et sa santé le rendaient incapable de pouvoir plus
rendre aucun service\,; sa captivité était le dernier qu'il avait rendu
pour servir de prétexte aux vues et aux projets d'Albéroni, après
l'avoir d'abord si publiquement méprisée.

Il travaillait avec grand soin à la marine d'Espagne. Il se flattait
pour le printemps prochain de mettre en mer trente navires, tant grands
que petits, chargés de douze mille hommes. Mais il avouait en même temps
que, s'ils n'étaient pas soutenus des secours de France, d'Angleterre et
de Hollande, l'Espagne ne se pouvait rien promettre de ses efforts en
Italie. Il y fallait transporter non seulement les troupes et les vivres
par mer, mais généralement toutes les provisions nécessaires pour une
armée. C'étaient des frais immenses. Ceux de la Sardaigne, jusqu'au
temps du débarquement, allaient déjà à un million de piastres.
L'empereur, au contraire, envoyait des troupes en Italie de
plain-pied\,; il y trouvait partout des vivres\,; il en tirait de
l'argent, de gré ou de force, tout autant qu'il en voulait des princes
d'Italie. L'Espagne ne pouvait les garantir de ces vexations, ni même
d'une invasion totale, et elle était obligée de l'avouer au duc de
Parme. Albéroni, qui ne se pouvait flatter de réussir lui tout seul en
Italie par la force, lui faisait espérer le secours de la négociation.

Le seul allié considérable à envisager était le roi de Sicile, intéressé
autant que nul autre à borner la puissance de l'empereur\,; mais
Albéroni ne l'avait pas ménagé. Del Maro, son ambassadeur, lui avait
déplu par son application à pénétrer ses desseins, et par ses avis
réitérés à son maître qu'on en voulait à la Sicile. Albéroni s'en était
grièvement offensé. Le roi de Sicile s'était tenu dans une grande
réserve, et del Maro ne s'était pas montré au palais depuis l'expédition
de Sardaigne. On ne peut s'empêcher d'admirer jusqu'où les faux
raisonnements d'Albéroni l'emportèrent, en s'engageant seul dans une
guerre insoutenable, et l'ensorcellement des monarques abandonnés à un
premier ministre. Del Maro eut pourtant ordre de voir Albéroni après le
débarquement en Sardaigne, de l'assurer des voeux de son maître en
faveur de l'Espagne, mais de lui dire que tout était à craindre, surtout
après les victoires de Hongrie, s'il n'était assuré de la France, dont
il n'y avait que le secours qui pût arriver de plain-pied en Italie.

Albéroni répondit que le dessein de l'Espagne n'était pas de faire des
conquêtes en Italie, mais de réprimer les infractions et les violences
des Allemands contre les traités, et de montrer en même temps sa
sincérité, en se bornant à la conquête de la Sardaigne\,; que l'Espagne
ne craignait ni les desseins ni la puissance de l'empereur\,; que, si
les princes d'Italie voulaient traiter de concert avec elle, elle y
contribuerait de ses soins et de ses forces. Il ajouta des vanteries sur
la modération et la puissance de l'Espagne, et ne laissa pas d'appuyer
sur le droit des enfants de la reine sur la succession de Toscane. Son
prétexte était toujours l'équilibre en Italie, et de ne travailler que
pour le repos public. Il promit au régent et au roi d'Angleterre, comme
il avait fait au pape, de leur laisser tout l'hiver à travailler à un
accommodement convenable à tous les partis. Il ne leur donnait rien en
cela que la saison avancée ne lui prescrivît aussi bien que
l'impuissance actuelle. En attendant, il travailla sans relâche à
ramasser l'argent et toutes les choses nécessaires à une grande
expédition. Il reçut très mal un mémoire que le roi d'Angleterre lui fit
donner par son ministre, contenant des représentations très vives. Il se
plaignit avec emportement à Londres et à Paris dés discours que Stairs y
avait tenus.

Il ne comptait plus sur la cour de Londres, trop dévouée à celle de
Vienne\,; toute sa ressource était la Hollande, à qui il n'oubliait rien
pour rendre l'empereur odieux, et pour la persuader de prendre des
mesures avec lui pendant l'hiver, pour établir un juste équilibre en
Italie. Il était principalement touché de diviser ce que l'empereur et
le roi de Sicile y possédaient, et de partager cette partie de l'Europe,
comme il a déjà été dit. Il promettait aux Hollandais que l'Espagne
doublerait ses forces l'année prochaine, sans avoir besoin d'aucun
emprunt, et il donnait des commissions d'acheter des vaisseaux de guerre
en Angleterre et en Hollande. Riperda, tout dévoué au cardinal, y
écrivait ce qu'il lui dictait. Beretti mandait que la proposition de
prendre cette république pour médiatrice de la paix y avait beaucoup
plu\,; et, dans le dessein peut-être de s'attirer la négociation, il
soutenait qu'il la fallait traiter à la Haye, parce que le ministère du
roi d'Angleterre était tellement impérial, qu'on se défiait de lui en
Hollande, jusque-là que le Pensionnaire, quoique si autrichien de tout
temps, lui avait dit qu'on ne songeait à Londres qu'à entraîner la
Hollande en des engagements dont l'Angleterre aurait tout l'honneur, et
dont la dépense retomberait toute sur les Provinces-Unies. Ainsi Beretti
croyait que la seule démarche que feraient les Hollandais serait
d'employer leurs offices pour la paix. On pensait de même à la Haye du
régent. Il était vrai qu'on avait été fort touché en Hollande de la
confiance du roi d'Espagne sur la médiation.

Cadogan, arrivé depuis peu à la Haye de la part du roi d'Angleterre,
était d'un caractère à ne ménager personne. Il avait eu la guerre passée
toute là confiance du duc de Marlborough, et par lui du prince Eugène et
du Pensionnaire, et, comme eux, haïssait parfaitement la France, surtout
le gouvernement du feu roi et tous ceux dont il s'était servi. Il parla
à Beretti de l'entreprise de l'Espagne avec toute la fureur
autrichienne. Inquiet du traité fait depuis peu entre le régent, le czar
et le roi de Prusse, il se plaignit aigrement de n'en avoir point de
connaissance. Là-dessus Châteauneuf eut ordre de le lui communiquer. Il
prétendit qu'il ne l'avait fait qu'en termes généraux, et que, depuis la
triple alliance, le Pensionnaire et plusieurs autres membres des États
généraux s'étaient attendus qu'il le communiquerait en forme. Cela fit
courir le bruit que le roi d'Angleterre avait demandé le rappel de
Châteauneuf, pour avoir négocié et signé ce traité. Le fond était la
mésintelligence de Georges avec son gendre et le czar, son chagrin et
celui de ses ministres de les voir unis avec la France, et leur
inquiétude de leur voir faire une paix séparée avec la Suède, en se
détachant de la ligue du nord.

Goertz, principal ministre de Suède, était à Berlin. Le czar, plus animé
que jamais contre Georges et contre la personne de ses deux ministres
allemands, se trouvait aussi à Berlin, et il s'y était dressé un plan de
paix particulière avec la Suède, à l'exclusion des rois d'Angleterre et
de Danemark. Ce projet passait en Hollande pour être concerté avec la
France, et le régent pour en presser l'exécution. Cadogan et quelques
autres assuraient que le régent n'y avait point de part, mais un autre
parti en France qui empêchait souvent l'exécution des volontés de ce
prince, qui voulait borner son autorité, et pour cela embraser l'Europe,
pour y embarrasser la France et encore plus le régent, dont l'intérêt
personnel était de concourir avec l'Angleterre à rétablir le repos du
nord et à prévenir les troubles de l'Italie\,; et {[}il ajoutait{]} que
la Hollande était disposée à prendre les mesures nécessaires pour cela
contre l'opinion de Beretti. La haine des Anglais pour Châteauneuf était
extrême. Ils voulurent lui faire un crime personnel auprès du régent sur
une insolence de la gazette de Rotterdam, dont ils prétendirent avoir
découvert la trame venue de la vieille cour et du parti contraire au
régent. Ils ignoraient, même Stairs, que ce traité avec le czar et la
Prusse eût été communiqué par le régent au roi d'Angleterre. Ils
commencèrent à compter sur la sincérité de la conduite de Son Altesse
Royale avec leur roi\,; mais ils ne purent revenir sur Châteauneuf,
quoiqu'il eût enfin communiqué ce traité aux États généraux, où on vit
qu'il n'y avait que de simples assurances et liaisons d'amitié, et que
l'objet n'en était que d'engager les puissances engagées dans la guerre
du nord de reconnaître la France pour médiatrice de cette paix.

L'abbé Dubois était parti pour Londres le 20 septembre, et, deux jours
auparavant, le colonel Stanhope, que le roi d'Angleterre envoyait à
Madrid par Paris, en était parti pour s'y rendre. Penterrieder était sur
le point de partir de Vienne pour l'Angleterre. Ainsi la scène des
grandes négociations s'allait ouvrir de tous côtés.

On commençait aussi à parler de négociations secrètes prêtes à s'ouvrir
à Abo, entre les ministres de Suède, de Russie et de Prusse\,; mais le
czar était parti de Berlin sur la fin de septembre sans avoir pris de
nouvel engagement, et ses ministres disaient qu'à l'exception de la
Finlande, il ne voulait rien rendre à la Suède\,: ainsi les choses
étaient encore peu disposées à la paix. Le roi de Prusse ne le
paraissait pas plus par les protestations d'union à ses alliés du nord,
qu'il faisait au roi d'Angleterre, avec lequel il s'était réconcilié, et
dont il ne se départirait point, pour forcer la Suède à une paix
raisonnable, pourvu qu'il n'eût pas lieu de croire par des démarches
qu'on voulût traiter sans lui, et le laisser dans l'embarras. Pour
preuve de sa sincérité, il assura le roi d'Angleterre de ce qui vient
d'être dit du czar à son départ de Berlin, qu'on n'y était convenu
d'aucun projet avec Goertz, et que, dans la vérité, il aurait été
difficile à ce Suédois de traiter avec ce prince, qui s'était expliqué
avec tant de hauteur sur les conditions de la paix, qu'on ne les pouvait
entendre sans indignation. Cette confiance en son beau-père ne
l'empêchait pas de se plaindre que la France lui eût communiqué {[}le
traité{]} fait entre elle, le czar et lui sans concert. On lui répondit
qu'il avait été impossible de le tenir caché plus longtemps. L'article
séparé en était demeuré fort secret. Le roi de Prusse voulut aussi
savoir de quel ce il on voyait en France les prospérités de l'empereur
en Hongrie. Le maréchal d'Huxelles dit à son envoyé qu'elles méritaient
de sérieuses réflexions, dont on lui ferait bientôt part, ainsi que du
motif du voyage de l'abbé Dubois à Londres.

Nonobstant de si beaux propos et si clairs du roi de Prusse au roi
d'Angleterre son beau-père, il ne perdait point de vue sa paix
particulière avec la Suède. Kniphausen, son envoyé à Paris, reçut ordre
de s'informer du général Poniatowski, qui s'y trouvait aussi et qui
avait la confiance du roi de Suède, si le landgrave de Hesse-Cassel
était un bon canal pour ménager cette paix particulière, et si le roi de
Prusse pouvait prendre confiance en lui. Poniatowski lui répondit que
cette voie n'était pas bonne\,; que le landgrave avait perdu son crédit
depuis que le roi de Suède s'était aperçu qu'il avait des liaisons trop
étroites avec le roi d'Angleterre\,; que la maison de Holstein avait
plus d'amis en Suède que celle de Hesse, et Goertz beaucoup plus de part
en la confiance de son maître que le landgrave\,; que, si le roi de
Prusse voulait conduire sûrement une négociation particulière avec
succès, il fallait premièrement qu'il fit en sorte de suspendre la
démolition des fortifications de Wismar\,; hâter ensuite le retour du
baron de Goertz en Suède\,; enfin que, s'il était possible de trouver
quelque expédient au sujet de Revel, la paix serait bientôt conclue
entre la Suède, la Russie et la Prusse. Il s'en fallait bien qu'il y eût
une égale disposition à la paix entre les rois d'Angleterre et de Suède.
Malgré les instances de la France, les Suédois assuraient que jamais le
roi de Suède ne consentirait à la cession de Brême et de Verden. Ce
prince, dont les sujets étaient épuisés, sollicitait vivement en France
le payement de ses subsides, cherchait dans Paris, sous de bonnes
conditions, deux millions d'espèces réelles, et autorisa son envoyé en
France de donner des commissions à des armateurs qui voudraient faire la
course sous le pavillon de Suède.

Plus il y avait d'agitation dans le nord, plus le roi d'Angleterre se
croyait intéressé à pacifier l'empereur et l'Espagne, en procurant des
avantages à l'empereur. Il comptait s'en faire un puissant protecteur
pour conserver les États usurpés sur la Suède, et que néanmoins le roi
d'Espagne lui aurait obligation de l'avoir délivré du seul ennemi qu'il
eût, et de lui assurer ainsi la possession tranquille de ses États. Lui
et ses ministres redoublaient donc d'empressement, et l'Espagne alors ne
paraissait pas s'en éloigner. Monteléon eut ordre d'assurer Stanhope que
son cousin serait bien reçu à Madrid. Monteléon se persuadait que
l'extrême répugnance que la nation Anglaise avait à se brouiller avec
l'Espagne à cause de son commerce retiendrait Georges et ses ministres
sur la partialité, et les bornerait aux offices pour ménager la paix.

Il paraissait que cet ambassadeur avait regagné la confiance du roi
d'Angleterre et de ses principaux ministres, et qu'il avait eu en même
{[}temps{]} l'adresse de se conserver celle des principaux personnages
opposés à la cour. Stanhope l'employait comme son ami en des affaires
particulières, et il mena, en même temps, dans son carrosse à Hampton
Court le duc de Buckingham, qui n'avait pas vu le roi d'Angleterre
depuis qu'il lui avait ôté la place de président du conseil. Monteléon
avait toujours été attaché à la France, et fidèle dans ses principes et
dans sa conduite à l'union intime entre la France et l'Espagne, qu'il
croyait avec raison absolument nécessaire aux deux couronnes. Cette
maxime, qui n'était pas dans les vues ni dans les intérêts de la cour
d'Angleterre, y avait déplu. Elle en était moins choquée depuis qu'elle
ne pouvait plus douter des plaies que cette union recevait, ni de celle
que le régent voulait avoir avec elle, pour ne pas dire même dépendance
entière fondée sur les vues, l'intérêt et l'étrange crédit de l'abbé
Dubois.

Cette confiance néanmoins de la cour d'Angleterre en un ministre
étranger était d'autant plus marquée que le roi d'Angleterre était
défiant et parlait peu. Ce silence était moins attribué à politique qu'à
la crainte de parler mal à propos ou de parler contre le sentiment de
ses ministres, desquels le public prétendait que la principale
application était de se conserver dans leurs places, et d'être si
appliqués à leur intérêt particulier qu'ils n'écoutaient qu'avec
répugnance et dégoût ce qui pouvait regarder les intérêts étrangers.

C'était à ces dispositions que l'envoyé du roi de Sicile attribuait le
peu d'égards et d'effet de ses représentations et de ses protestations,
que son maître n'avait nulle part aux projets de l'Espagne, qu'il
observerait fidèlement les traités, surtout qu'il s'attacherait
constamment aux sentiments de l'Angleterre quand il s'agirait de prendre
parti\,; mais le ministère connaissait le caractère du roi de Sicile\,;
il croyait lui faire honneur d'écouter les propos de son ministre, et de
lui laisser croire par son silence, s'il voulait, qu'il les avait
persuadés. Cet envoyé se défiait de l'union de la France et de
l'Angleterre, et que plus attentives à leurs intérêts qu'à ceux du roi
de Sicile, elles ne traversassent même sa réunion avec l'empereur. Il
chercha donc à y travailler lui-même sans la participation des ministres
d'Angleterre. Il se servit pour cela de l'envoyé de Modène à Londres,
dont le frère était à Vienne, lequel prétendait traiter directement avec
l'empereur indépendamment de ses ministres, et qui assurait avoir bonne
opinion de cette négociation.

L'envoyé, son frère, fondait ses espérances sur ce que l'empereur savait
que le roi de Sicile avait constamment refusé toute ligue nouvelle avec
le régent, qu'il avait répondu que les engagements déjà pris
suffisaient, et que cette réponse lui avait attiré la haine et les
soupçons du régent\,; que de là l'empereur inférait que le régent lui
serait toujours contraire, et que, si ce prince témoignait tant
d'empressement pour empêcher le renouvellement de la guerre dans
l'Europe, ce n'était pas par aucun attachement pour lui qu'il craignait
et n'aimait point, mais pour empêcher la réunion que cette guerre
produirait infailliblement entre lui empereur et le roi de Sicile\,; que
c'était le motif du voyage de l'abbé Dubois à Londres\,; que
l'intelligence était parfaite entre le roi d'Angleterre et le régent\,;
qu'on savait que le projet du roi d'Espagne, qui venait de la reine,
était, pour assurer la Toscane à la maison de Parme, d'y joindre le
royaume de Sardaigne, et d'en tirer un titre pour faire porter au duc de
Parme celui de roi de Sardaigne.

Quel que fût le projet, tous les princes d'Italie craignaient également
d'être soupçonnés d'y participer. Leurs ministres en France le
désapprouvaient publiquement, et ne cessaient de dire que leurs maîtres
étaient bien éloignés d'entrer dans aucun projet capable de porter le
moindre préjudice à l'empereur.

Cellamare était témoin de ces apologies continuelles, et très inquiet du
voyage de l'abbé Dubois à Londres. Mais c'était un homme sage, qui
espérait peu de l'entreprise d'Espagne, et qui croyait que le mieux,
pour le roi son maître, serait de suivre la voie que la France et
l'Angleterre lui ouvraient pour entrer en négociation avec l'empereur.

Une guerre sans alliés lui paraissait téméraire, et c'était, à son sens,
un faible fondement que de compter uniquement sur la diversion des
Turcs. Ragotzi était le seul qui assurât qu'ils feraient la campagne
suivante, et dans cette confiance il avait fait voile de Marseille à
Constantinople.

Welez, cet espion de l'empereur, l'avait exactement informé de son
départ, des circonstances de son voyage, des voies dont ses amis se
servaient pour lui envoyer des lettres. Il prétendait avoir découvert
que quelques-unes passaient par le comte de Toulouse, d'autres par le
bureau des affaires étrangères, et nommait ses banquiers à Paris et à
Vienne. Welez offrit encore à l'empereur de faire enlever l'abbé Brenner
avec tous ses papiers. Il concluait que si Ragotzi n'avait eu d'autre
protection que celle des Turcs, il n'aurait pas trouvé en France toutes
les facilités qu'il y avait eues pour son départ et son embarquement\,;
qu'il était donc certain que la France et l'Espagne étaient
d'intelligence pour susciter à Sa Majesté Impériale un ennemi qu'elles
croyaient dangereux et redoutable.

Bentivoglio, toujours le plus violent ennemi de la France où il était
nonce, avait fait tous ses efforts pour empêcher le pape d'accorder
l'indult pour la nomination à l'archevêché de Besançon, duquel au fond
on pouvait très bien se passer et nommer\,; et outre les difficultés que
l'indécision du pape y apporta, il le persuada de faire entendre qu'il
n'accorderait plus de bulles sans des précautions et des conditions à
l'égard de ceux que le roi nommerait aux évêchés et aux autres
bénéfices. Bentivoglio reprit ses anciennes exhortations et les plus
vives pour engager le pape à se rendre le maître en France, par faire
avec l'empereur cette ligue dont le baron d'Hohendorff lui avait,
quelque temps auparavant, communiqué le projet. Il assurait le pape,
avec ses mensonges et sa hardiesse accoutumée, que tous les bons
catholiques de France désiraient cette union. Il ajoutait que ce serait
la preuve la plus forte pour dissiper les soupçons de l'empereur, et le
meilleur et le plus sûr moyen de s'attirer un respect nouveau de la part
de tous les princes. Mais il voulait attirer la république de Venise
dans cette ligue, qui, selon lui, ne la refuserait pas. Mais sa
politique raffinée voulait que le pape gardât un juste milieu entre
l'empereur et l'Espagne sans pencher de côté ni d'autre, pour être
toujours en état d'offrir sa médiation\,; et de là ce digne ministre de
paix pressait le pape, avec les plus étranges efforts, de prendre et
d'effectuer les plus violentes résolutions contre la France.

\hypertarget{chapitre-vi.}{%
\chapter{CHAPITRE VI.}\label{chapitre-vi.}}

1717

~

{\textsc{Saint-Albin coadjuteur de Saint-Martin des Champs.}} {\textsc{-
Infamie de l'abbé d'Auvergne.}} {\textsc{- Dispute encore entre le grand
et le premier écuyer.}} {\textsc{- Le duc de Noailles et Law, brouillés,
se raccommodent en apparence.}} {\textsc{- Noailles obtient le
gouvernement et capitainerie de Saint-Germain par la mort de Mornay.}}
{\textsc{- Plénoeuf, relaissé à Turin de peur de la chambre de justice,
imagine d'y traiter le mariage d'une fille de M. le duc d'Orléans avec
le prince de Piémont, pour se faire de fête.}} {\textsc{- Je suis chargé
de ce commerce malgré moi, et je m'en décharge sur l'abbé Dubois, à son
retour d'Angleterre.}} {\textsc{- Querelle entre le maréchal de Villeroy
et le duc de Mortemart, premier gentilhomme de la chambre en année, qui
la perd.}} {\textsc{- Autres disputes des premiers gentilshommes de la
chambre.}} {\textsc{- Le maréchal de Villeroy refuse la prolongation du
don de cinquante mille livres de rente sur Lyon.}} {\textsc{- Son
motif\,; sa conduite\,; explication de ce qu'il n'y perd rien.}}
{\textsc{- Quatre-vingt mille livres au duc de Tresmes.}} {\textsc{- Le
prince électoral de Saxe se déclare catholique à Vienne.}} {\textsc{-
Abbé de Louvois refuse l'évêché de Clermont\,; quel.}} {\textsc{- Rion
gouverneur de Cognac.}} {\textsc{- Mort d'Oppède, mari secret de
M\textsuperscript{me} d'Argenton, et de l'abbé de Langlée.}} {\textsc{-
Mort et famille de la comtesse de Soissons.}} {\textsc{- Appel du
cardinal de Noailles devenu public.}} {\textsc{- La Parisière, évêque de
Nîmes, exilé dans son diocèse.}} {\textsc{- Affaire du pays de Lalleu,
où je sers adroitement le duc de Boufflers.}} {\textsc{- Anecdote
singulière de l'étrange indécision du chancelier.}} {\textsc{- Capacité
singulière de d'Antin.}} {\textsc{- Reconnaissance des députés du pays
de Lalleu.}} {\textsc{- Les ducs de La Force et de Noailles brouillés.}}
{\textsc{- M\textsuperscript{me} d'Arpajon dame de M\textsuperscript{me}
la duchesse de Berry, et Bonivet maître de sa garde-robe.}} {\textsc{-
Mort du cardinal Arias, archevêque de Séville.}} {\textsc{- Mort de
M\textsuperscript{me} de Monjeu et de Richard Hamilton.}} {\textsc{-
Caractère de ce dernier.}} {\textsc{- Assassinats et vols.}} {\textsc{-
Teneurs de jeux de hasard mis en prison.}} {\textsc{- États de Bretagne
orageux et rompus.}} {\textsc{- M\textsuperscript{me} d'Alègre entre
avec moi en mystérieux commerce qui dure plus d'un an.}} {\textsc{- Abbé
Dubois revient pour peu de jours d'Angleterre à Paris\,; y laisse sa
correspondance à Nancré\,; trouve le roi d'Angleterre et le prince de
Galles fort brouillés.}} {\textsc{- Cause originelle de leur
éloignement.}}

~

Rome venait pourtant d'approuver, en faveur de M. le duc d'Orléans, la
coadjutorerie du riche prieuré de Saint-Martin des Champs dans Paris, et
qui a beaucoup de collations, pour l'abbé de Saint-Albin, bâtard non
reconnu de ce prince et de la comédienne Florence. Le cardinal de
Bouillon, comme abbé de Cluni, avait donné autrefois ce prieuré à l'abbé
de Lyonne, fils du célèbre ministre et secrétaire d'État des affaires
étrangères. Cet abbé de Lyonne, dont j'ai parlé ailleurs, était un homme
de moeurs, de vie, d'obscurité, de régime même, fort extraordinaires,
gouverné par un fripon que lui avaient donné les jésuites, qui s'y
enrichit au trafic de ses collations et à la régie de son bien, connu du
feu roi pour si scélérat, et de tout le monde, que le P. Tellier et
Pontchartrain, comme on l'a vu ailleurs, échouèrent à le faire évêque,
et qui l'est, depuis ceci, devenu de Boulogne. L'abbé de Lyonne fut donc
tonnelé pour cette coadjutorerie qui au fond ne lui faisait aucun tort,
et l'abbé d'Auvergne, comme abbé de Cluni, se fit un mérite auprès du
régent, non seulement d'y consentir, mais d'y contribuer de tout son
pouvoir. Il est vrai que ce prince n'eut pas plutôt les yeux fermés, que
l'abbé d'Auvergne ne rougit point d'attaquer son bâtard, devenu
archevêque de Cambrai, et qui, depuis deux ans, était en possession
paisible du prieuré, sans réclamation quelconque, par la mort de l'abbé
de Lyonne. L'abbé d'Auvergne, lors archevêque de Vienne, cria à la
violence, contre la notoriété publique, intenta un procès et le perdit
avec infamie. La vérité est qu'il n'y laissa point son honneur, parce
qu'il y avait longues années que, de ce côté-là, il n'avait plus rien à
perdre\,; ce qui n'a pas empêché que le cardinal Fleury ne l'ait fait
cardinal pour n'avoir point de similitude importune.

M. le Grand qui, comme on l'a vu en son lieu, avait perdu
contradictoirement toutes ses prétentions contre le premier écuyer, et à
qui M. le duc d'Orléans avait eu la faiblesse de permettre des
protestations, n'avait presque point cessé depuis de faire des
tentatives et des entreprises de fait, qui devinrent si fortes qu'il
fallut encore que M. le duc d'Orléans en fut importuné. Ce fut en vain.
Les \emph{mezzo-termine} lui plaisaient trop pour rien finir. Ce
harcelage dura longtemps encore et abrégea la vie du premier écuyer par
le chagrin et le dépit\,; mais sa charge n'y perdit pas un pouce de
terrain, jusqu'à ce que enfin le cardinal Fleury, qui avait été de ses
amis, se trouvant le maître, décida si nettement en faveur de son fils,
que le grand écuyer cessa pour toujours de le troubler et d'entreprendre
sur la petite écurie.

Le duc de Noailles, jaloux de la confiance du régent pour Law et du
succès de sa banque, la troublait tant qu'il pouvait. Law coulait et
quelquefois se plaignait modestement. Noailles, qui le voulait perdre
pour être pleinement maître de toutes les parties des finances, redoubla
de machines pour le culbuter. Cette banque était lors une des
principales ressources pour rouler. Le régent voulut qu'ils se
raccommodassent. Law s'y présenta de bonne foi, le duc de Noailles ne
put reculer\,; il fit le plus beau semblant du monde.

Précisément en ce moment heureux, Mornay mourut fort promptement. Il
était lieutenant général, et il était aussi gouverneur et capitaine de
Saint-Germain après Montchevreuil, son père. Le duc de Noailles, alerte
sur tout, l'apprit à son réveil et courut sur-le-champ demander cet
emploi à M. le duc d'Orléans, qui le lui donna à l'instant. Mon père
l'avait eu. Je ne sus la mort de Mornay que l'après-dînée et en même
temps la diligence du duc de Noailles. Il n'était pas aisé de se lever
plus matin que lui. Il y avait cent mille francs de brevet de retenue à
payer. M. de Noailles, grand politique et grand serviteur du parlement,
demanda aussitôt la distraction de Maisons et de Poissy de la
capitainerie de Saint-Germain, et s'en fit un grand mérite. La situation
des lieux en montre l'absurdité. Aussi y ont-ils été remis, à l'instance
du même duc de Noailles, à la mort du dernier président de Maisons.

M\textsuperscript{me} la duchesse d'Orléans me chargea vers ce temps-ci
d'un commerce fort peu de mon goût, et dans lequel M. le duc d'Orléans
me pria aussi d'entrer. Plénoeuf, dont la femme et la fille,
M\textsuperscript{me} de Prie, ont fait depuis, par leur jalousie de
beauté et leurs querelles, tant de fracas dans le monde, avait gagné des
monts d'or dans les partis\footnote{Dans les traités pour affaires de
  finances.}, et depuis dans les vivres. La chambre de justice l'avait
mis en fuite, et il s'était retiré à Turin.

Je n'avais jamais eu aucun commerce avec pas un de ces sortes de gens\,;
de celui-là en particulier, j'en étais mécontent, parce que, étant
devenu un des principaux commis du bureau de la guerre sous Voysin, dans
les derniers temps du feu roi, la majorité de Blaye vaqua, et
sur-le-champ il la fit donner à un de ses parents. Le roi m'avait
toujours conservé la distinction, après mon père, de ne remplir les
places de l'état-major de Blaye que de ceux que je demandais, et c'était
la première fois qu'on en remplissait une sans moi. Voysin en ce
temps-là était dans la plus haute faveur, et insolent à proportion.
C'était alors, comme on l'a vu, l'homme de M\textsuperscript{me} de
Maintenon et de M. du Maine, et le directeur et le rédacteur de
l'apothéose des bâtards et du testament du roi. Je compris donc que je
ne gagnerais que du dégoût à résister à contretemps, et que bientôt les
choses changeraient de face. En effet, la première chose que je fis
aussitôt après la mort du roi fut de chasser ce major et d'en mettre un
autre.

Plénoeuf avait de l'esprit et de l'intrigue\,; il voulait ne rien perdre
à sa déconfiture, et revenir à Paris riche et employé, s'il pouvait. Il
se fourra donc dans le subalterne de la cour de Turin\,; par là eut
quelque accès auprès des ministres, imagina de travailler au mariage
d'une fille de M. le duc d'Orléans avec le prince de Piémont. Sa femme
fort intrigante et de beaucoup d'esprit, manégea si bien qu'elle vit
M\textsuperscript{me} la duchesse d'Orléans plusieurs fois en
particulier, et lui donna tant d'espérance que la négociation ne pouvant
demeurer entre les mains du mari et de la femme avec décence aux yeux
des ministres de la cour de Turin, M\textsuperscript{me} la duchesse
d'Orléans proposa de m'en charger. M\textsuperscript{me} de Plénoeuf ne
me connaissait point\,; elle dit seulement à M\textsuperscript{me} la
duchesse d'Orléans que je n'aimais pas son mari, et lui conta ce qui
vient d'être expliqué. Cela ne rebuta point M\textsuperscript{me} la
duchesse d'Orléans\,: elle me pria de passer pour l'amour d'elle sur ce
mécontentement d'un homme de plus si infime, et de vouloir recevoir
M\textsuperscript{me} de Plénoeuf et entrer en commerce direct avec
Plénoeuf sur ce mariage.

Par ce qu'on a vu de la situation du régent et du roi de Sicile, l'un à
l'égard de l'autre, cette négociation de mariage était fort déplacée\,:
c'était ce qu'il ne m'était pas permis de dire à M\textsuperscript{me}
la duchesse d'Orléans\,; mais quand M. le duc d'Orléans m'en parla, deux
jours après, je ne lui cachai pas ce que j'en pensais, et ma surprise de
sa complaisance. Il en convint\,: «\,Mais, après tout, me dit-il, c'est
un coup d'épée dans l'eau\,; et, quoique sans apparence, il est des
choses bizarres qui réussissent quelquefois\,: ce ne sont que quelques
lettres perdues qu'il nous en coûtera à tout hasard.\,» Je ne pus donc
m'en défendre.

M\textsuperscript{me} de Plénoeuf vint chez moi bien parée, bien polie,
bien louangeuse, bien éloquente, et bien pleine de son affaire. Force
soumission sur son mari, et tout aussitôt les lettres mouchèrent. De
réalité, je n'en vis jamais ombre\,; mais force langage d'un homme qui
voulait plaire et se faire valoir. Ce commerce dura quelques mois\,;
mais sitôt que l'abbé Dubois fut revenu d'Angleterre, je priai M. le duc
d'Orléans de m'en décharger sur lui, et M\textsuperscript{me} la
duchesse d'Orléans de le trouver bon, sous prétexte que je ne voulais
point choquer un homme si jaloux d'affaires, qui traverserait celle-là
entre mes mains, et qui pouvait réussir entre les siennes. Je la lui
remis donc, et il convint avec moi que c'était une vision en la
situation où étaient les choses entre les deux princes. Aussi n'eut-elle
point de suite et je n'en entendis plus parler depuis.

Un amusement de l'âge du roi fit une querelle sérieuse. On lui avait
tendu une tente sur la terrasse des Tuileries, devant son appartement et
de plain-pied. Les jeux des rois sentent toujours la distinction. Il
imagina des médailles pour les donner aux courtisans de son âge qu'il
voudrait distinguer, et ces médailles, qu'ils devaient porter, leur
donnaient le droit d'entrer dans cette tente sans y être appelés\,: cela
s'appela l'ordre du Pavillon. Le maréchal de Villeroy donna l'ordre à
Lefèvre de les faire faire. Il obéit, et les apporta au maréchal, qui
les présenta au roi. Lefèvre était argentier de la maison du roi, et,
comme tel, sous la charge des premiers gentilshommes de la chambre. Le
duc de Mortemart était en année. Il avait déjà eu des démêlés sur le
maréchal de Villeroy. Il prétendit que ç'avait été à lui à commander les
médailles, et à lui de les présenter au roi. Il se fâcha que le tout se
fût fait à son insu, et le voilà aux champs et en plaintes à M. le duc
d'Orléans. C'était une bagatelle qui ne valait pas la relever, et à
laquelle aussi les trois autres premiers gentilshommes de la chambre ne
prirent point de part. Ainsi seul vis-à-vis du maréchal de Villeroy, la
partie ne fut pas égale. M. le duc d'Orléans, avec ses
\emph{mezzo-termine} ordinaires, dit que Lefèvre ne les avait point fait
faire ni portées au maréchal comme argentier, mais comme ayant reçu par
lui l'ordre du roi, et qu'il n'en fallait pas parler davantage. Le duc
de Mortemart fut outré, et ne s'en contraignit pas sur le maréchal.

Une autre querelle combla celle-ci. Le duc de Mortemart prétendit une
place derrière le roi, et l'ôter à un chef de brigade des gardes du
corps qui la prenait. Les capitaines des gardes soutinrent leur
officier, et M. de Mortemart ôta des entrées qu'avaient les officiers
des gardes du corps. Les trois autres gentilshommes de la chambre se
joignirent au duc de Mortemart. Ils plaidèrent tous huit devant M. le
duc d'Orléans plusieurs fois, à cause de la pièce du trône différemment
placée qu'à Versailles, où M. de Mortemart renouvela la défense aux
huissiers de laisser entrer les officiers des gardes du corps.
Là-dessus, autre \emph{mezzo-termine}. M. le duc d'Orléans fit ôter le
trône, pour ôter ce sujet de contestation. M. de Mortemart, piqué de
cette décision, cessa d'aller chez le roi, quoique en année, et les
premiers gentilshommes de la chambre firent un mémoire et le
présentèrent à M. le duc d'Orléans.

L'affaire en demeura là jusqu'à une autre qui arriva un mois après entre
le duc de Mortemart et le maréchal de Villeroy, pour des bagatelles de
service. Les autres premiers gentilshommes de la chambre prirent fait et
cause, et pas un d'eux ne se présenta plus chez le roi. Cela dura huit
ou dix jours, après lesquels ils y retournèrent. Le régent ne put se
résoudre à prononcer\,; mais le maréchal, battu de l'oiseau, s'abstint
depuis d'entreprises pour quelque temps. Néanmoins, M. de Mortemart
piqué voulut envoyer la démission de sa charge. M. le duc d'Orléans m'en
parla fort en colère\,; et en effet c'était tous les jours quelque chose
de nouveau avec lui. J'apaisai le régent comme je pus par le souvenir de
M. de Beauvilliers, et je détournai l'orage.

Les premiers gentilshommes de la chambre eurent encore une dispute avec
les maîtres d'hôtel du roi, à qui l'avertirait que sa viande était
servie\,; et comme les maîtres d'hôtel sont sous le grand maître, M. le
Duc les soutenait\,; car tout était en prétention et en entreprises. Au
dîner du feu roi, j'ai vu toute ma vie le maître d'hôtel avertir le
premier gentilhomme de la chambre, et celui-ci entrer dans le cabinet du
roi seul, et l'avertir\,; et le soir que le roi était chez
M\textsuperscript{me} de Maintenon, le maître d'hôtel avertir le
capitaine des gardes qui entrait seul dans la pièce où le roi était, et
l'avertissait que son souper était servi.

Le maréchal de Villeroy, mal dans ses affaires par une magnificence sans
règle ni mesure, avait obtenu du feu roi cinquante mille livres par an,
sur la ville de Lyon, pendant six ans, et une continuation encore
pendant autres six années, qui se renouvela de six en six ans. Jamais le
feu roi ne pensa à les lui accorder pour toujours, et on ne lui a vu
donner de tout son règne cinquante mille livres de rente à personne à
prendre sur lui pour toujours, excepté des appointements de
gouvernements ou de charges dont le taux y était attaché\,; et à l'égard
des pensions, personne, hors le premier prince du sang et ses bâtardes
en les mariant, n'eut jamais de pensions approchantes, sinon, comme on
l'a remarqué, Chamillart qui en eut une de soixante mille livres en le
renvoyant, ce qui fut une chose unique en tout son règne. C'était en
cette année et dans ce temps-ci, que les six années du don au maréchal
de Villeroy finissaient\,; M. le duc d'Orléans le voulut renouveler,
même pour toute sa vie. Le maréchal fit le généreux, s'excusa de
l'accepter pour toujours, ni même par aucun renouvellement, dit qu'il
était riche par les successions et les bienfaits qui lui étaient
arrivés, et qu'il n'était pas juste que, dans un temps où tant de gens
souffraient, il abusât des bontés qui lui étaient offertes. Il fut
pressé, résista constamment, mais pour s'en vanter publiquement et se
parer dans le monde de la faveur de la considération et du
désintéressement. Le bout de cela est que lui personnellement est mort
ruiné, et que son fils a été obligé de payer ses dettes qui étaient
grandes, et sur les fins de le faire subsister. Ce n'est pas qu'avec de
l'économie du fils et du petit-fils il ne leur soit demeuré des biens
immenses des successions de Lesdiguières et de Retz\,; mais ce n'a pas
été la faute des désordres du maréchal.

C'était un homme qui n'avait point de sens, et qui n'avait d'esprit que
celui que lui en avait donné l'usage du grand monde, au milieu duquel il
était né et avait passé une très longue vie. On a eu si souvent occasion
de parler de lui, qu'il suffit ici de faire souvenir de ce caractère, de
l'orgueil dont il était pétri, que ses fréquentes et cruelles
déconvenues, toutes arrivées par faute de sens, n'avaient pu émousser,
et de l'éclat où les passions et l'intérêt de M\textsuperscript{me} de
Maintenon et de M. du Maine l'avaient mis dans les derniers temps de la
vie du feu roi, surtout à sa mort, qui avait porté cet orgueil à son
comble. Depuis qu'il se vit dans les places où cette mort l'établit et
dans la considération qui en était une suite, la tête lui tourna\,: il
se crut le père, le protecteur du roi, l'ange tutélaire de la France, et
l'homme unique en devoir et en situation de faire en tout contre au
régent.

Sa fatuité lui avait fabriqué un autre devoir qui fut d'épouser contre
ce prince toute la haine de la Maintenon, sa patronne, et toute la
mauvaise volonté qu'elle avait arrachée contre lui du roi mourant. Il
s'applaudit sans cesse des démarches infatigables que le régent faisait
vers lui, qui ne faisaient que rehausser son courage à lui nuire\,; il
abusait continuellement de la confiance et de la facilité à condescendre
à tout ce qu'il voulait d'un régent doux, timide, qui redoutait les
éclats, à qui ses grands airs avec feu Monsieur, et en commandant les
armées où M. le duc d'Orléans avait commencé à servir, lui avaient
imposé au point qu'il lui imposait toujours. Ainsi ce prince voulait et
croyait le gagner à force de flatter son incroyable vanité, et d'aller
au-devant de tout ce qui lui pouvait plaire, sans jamais lui rien
refuser pour les siens ni pour personne\,; tandis que, déterminé à
figurer en grand aux dépens du régent, ce qu'il ne croyait pas possible
autrement, il s'unissait à tous ses ennemis, à ceux que l'ambition ou
l'amour des nouveautés rendaient tels, les excitait, les encourageait,
les grossissait pour se former un parti\,; et pour cela, très attentif à
un apparent désintéressement qui augmentât sa réputation et la
confiance, tellement que, par principes, il était incapable d'être
arrêté par les grâces et les bienfaits de M. le duc d'Orléans. En le
refusant des cinquante mille livres de rente sur Lyon, il ne refusait
rien en effet\,; mais il suivait son plan\,: il se donnait un éclat
propre à éblouir la multitude, surtout le parlement en particulier et la
robe en général qu'il cultivait soigneusement, à s'attacher des
partisans, à augmenter la confiance de ceux qu'il voulait capter, à
blâmer avec l'autorité de ce refus et de la manière la plus publique, et
en apparence la plus innocente, la facile prodigalité du régent, et sans
en demeurer plus pauvre.

De tout temps ses pères, son oncle et lui étaient maîtres absolus et
uniques à Lyon. Dès les temps du feu roi les intendants n'y avaient pas
la plus légère inspection. L'autorité du maréchal y était encore plus
devenue sans bornes dans une régence qui ne songeait qu'à lui plaire, et
à aller au-devant de tout à son égard. De tout temps il était, après ses
pères et son oncle, en possession de nommer seul le prévôt des marchands
de Lyon, qui avait tout le pouvoir bursal dans la ville, sans inspecteur
ni conseiller. Il disposait seul sous le maréchal de Villeroy des
immenses revenus de la ville, d'en diriger de même tout le commerce, et
d'y être le maître des commerçants. Il ne comptait de la recette et de
la dépense de ces immenses revenus, qu'avec le maréchal de Villeroy
seul, et les comptes ainsi arrêtés entre eux deux seuls, où le maréchal
était de droit le maître, ne se trouvaient plus, et ne se voyaient
jamais plus, tellement que c'est parler exactement que dire que le
maréchal de Villeroy était le seul roi de Lyon, que le prévôt des
marchands y était son vice-roi \emph{ad nutum}, et qu'ils mettaient en
poche tout ce qu'il leur plaisait de prendre, sans le moindre embarras,
sans formalité aucune, et sans la moindre crainte d'aucune suite pour
l'avenir, ni même qu'on pût jamais savoir ce qui se passait là-dessus
entre eux deux. Il est donc clair que, maître tous les ans de ces
prodigieux revenus et de tout le commerce de la plus florissante place
du royaume en ce genre, le maréchal de Villeroy prenait en toute liberté
tout ce qu'il voulait, et qu'en refusant le don que le régent lui
voulait continuer, il ne refusa rien en effet. Aussi ceux de Lyon
savaient bien qu'en dire, malgré toute la protection qu'il leur donnait
à tous. Mais pas un d'eux n'osa jamais se plaindre ni branler devant lui
sous le dernier règne\,; combien moins pendant cette régence, à la
posture où se trouvait leur gouverneur. Son fils, qui l'a peu survécu,
soutint encore cette puissance, mais plus faiblement. Enfin le duc de
Villeroy d'aujourd'hui en a sauvé de grandes bribes, mais les finances y
ont mis la main, et ont fort borné ce pouvoir si pécunieux et si fort
illimité.

Le duc de Tresmes ne fut pas si délicat que le maréchal de Villeroy\,:
aussi était-ce un honnête homme qui était bien éloigné des mêmes
projets. Il eut quatre-vingt mille livres en dédommagement du deuil,
dont il devait et n'avait pas profité à la mort du roi, où il était en
année de premier gentilhomme de la chambre.

Le prince électoral de Saxe, catholique dès qu'il était à Rome, avec une
permission du pape de le demeurer caché, le déclara en ce temps-ci à
Vienne, où il était allé voyager et voir l'empereur\,; le roi de Pologne
son père était du secret et avait fort contribué à le faire catholique,
pour lui frayer le chemin à lui succéder en Pologne. Mais la mère et
l'épouse de ce roi, qui étaient des piliers de leur religion, y étaient
si opposées, que le roi de Pologne ne put, depuis qu'il fut catholique,
avoir presque de commerce avec l'électrice sa femme que des moments
rares quand il allait en Saxe, où même ce n'était qu'en visite, sans
qu'elle voulût demeurer dans le même lieu que lui, ni qu'elle voulût
ouïr parler d'aller en Pologne, ni souffrir le titre, ni aucun des
honneurs, ni des traitements de reine. Le roi son mari supportait cela
avec toujours beaucoup de considération pour elle, mais il s'en
consolait avec ses maîtresses. L'électrice sa mère étant morte, il ne
fit plus difficulté de laisser déclarer son fils catholique.

L'abbé de Louvois refusa l'évêché de Clermont, sous prétexte de sa
santé, en effet parce qu'il s'était attendu longtemps aux plus grands
postes, et qu'il se trouvait vieux pour en accepter un si médiocre. Il
n'était pas sans mérite, il avait de l'esprit, du monde et du savoir, et
remplissait, par lui-même et avec réputation, la belle place dans les
lettres de bibliothécaire du roi. À peine commençait-il à poindre lors
de la mort de son père, qui était perdu bien auparavant. Barbezieux,
crossé par le roi comme un jeune homme des débauches et des disparates
duquel il était très souvent mécontent, n'eut pas loisir de mûrir et de
s'accréditer assez pour vaincre auprès du roi les soupçons que les
jésuites et M\textsuperscript{me} de Maintenon, par Saint-Sulpice, lui
donnaient sans cesse de l'éducation ecclésiastique du neveu de
l'archevêque de Reims, que les jésuites avaient toujours regardé comme
leur ennemi, et donné, par conséquent, pour un dangereux janséniste. Ce
manège avait perdu l'abbé de Louvois dans l'esprit du roi, et quelques
bagatelles de première jeunesse, qu'en ce genre il ne pardonnait jamais.
Ainsi l'abbé de Louvois avait vu les premiers postes lui échapper. Mais
il n'avait pu s'accoutumer à en perdre l'espérance, depuis même que sa
situation était devenue ordinaire par la perte du ministère de son frère
et de son oncle. Il était demeuré assez de crédit et d'établissements
parmi ses frères et soeurs pour la nourrir, et tout attendre de la
facilité du régent. Quand il vit ses espérances trompées par l'évêché de
Clermont, il ne put en digérer l'humiliation, et il aima mieux hasarder
de ne sortir point du second ordre. Le P. Massillon, père de l'Oratoire,
célèbre par ses sermons, en profita. Crosat, le cadet, paya pieusement
et noblement ses bulles.

M\textsuperscript{me} la duchesse de Berry fit donner au vieux
Saint-Viance, très galant homme, qui avait été lieutenant des gardes du
corps, et lieutenant général, cinquante mille livres, et deux mille
livres de pension pour son gouvernement de Cognac, de douze mille livres
de rente, sans obliger à résidence, et fit présent de ce gouvernement à
Rion.

M\textsuperscript{me} d'Argenton, longtemps depuis que M. le duc
d'Orléans l'eut quittée, avait vécu avec le chevalier d'Oppède, jeune et
bien fait, qui était dans les gardes du corps, et dont le nom était
Janson, fort proche du feu cardinal de Janson. Ensuite elle pensa à
accommoder ses plaisirs à sa conscience, lui fit des avantages pour un
cadet qui n'avait rien, l'obligea à quitter le service et l'épousa. Mais
tous deux, par honneur, voulurent que ce fût secrètement. Elle n'en eut
point d'enfants, et le perdit en ce temps-ci. Il la traitait avec grande
rudesse, et lui donna tout lieu de se consoler. L'abbé de Langlée,
singulier ecclésiastique, frère de Langlée dont il a été quelquefois
parlé, mourut aussi. Il n'avait presque rien qu'une pension de six mille
livres que lui donnait M\textsuperscript{me} de Villequier, fille de sa
soeur, M\textsuperscript{me} de Guiscard.

La comtesse de Soissons mourut en même temps à Paris, point vieille, et
belle encore comme le jour. On n'a rien à en dire de plus que ce qui
s'en trouve t. VI, p.~124. Elle fut depuis pauvre, malheureuse,
errante\footnote{Voy. à la fin du tome IV, la note rectificative de M.
  Chantérac.}. De fois à autre M. le duc d'Orléans lui faisait donner
quelque gratification. Elle laissa deux fils qui moururent jeunes, sans
alliances, dont le prince Eugène leur oncle prenait soin. Il avait
destiné l'aîné à être son héritier, et avait arrêté son mariage avec
l'unique héritière de la maison Cybo, qui a depuis porté les petits
États de Massa et Carrara, avec d'autres grands biens, au fils aîné du
duc de Modène et d'une fille de M. le duc d'Orléans, qui l'a épousée. La
comtesse de Soissons laissa aussi une fille dont le roi de Sicile
prenait soin, dans un couvent à Turin, que le prince Eugène, qui à
survécu ses deux neveux, a fait son héritière, et qui a épousé à Vienne
le prince de Saxe-Hilbourghausen, et qui a tant fait parler de lui, plus
en partisan hasardeux qu'en officier principal, dans l'armée impériale
en Italie, contre les troupes unies de France, Espagne et Savoie, dont
les maréchaux de Coigny et de Broglio eurent le commandement sous le roi
de Sicile, après la mort du maréchal de Villars. Ainsi finit là branche
de Soissons de la maison de Savoie.

L'appel du cardinal {[}de Noailles{]} devint public, et fut imprimé avec
une instruction admirable, dont il n'a paru que la première partie par
ce qui arriva depuis, dont il eut tout lieu de se repentir, ainsi que de
n'avoir pas fait paraître son appel bien plus tôt, dans le temps que je
l'en pressai, comme je l'ai raconté en son lieu. Je n'en dis pas
davantage pour ne pas effleurer une matière si étendue et qui se trouve
traitée exprès.

La Parisière, évêque de Nîmes, qui écrivait à tous les prélats et aux
universités étrangères pour avoir leur adhésion à là constitution, eut
ordre de se retirer dans son diocèse\,; mais la cabale le fit rappeler
au bout de huit ou dix mois. On a vu ailleurs que, pigeon privé du P.
Tellier, il s'éleva en Languedoc contre la constitution\,; dans les
commencements gagna peu à peu la confiance des prélats, des communautés
et des principaux ecclésiastiques\,; et, pour se l'acquérir entièrement,
poussa les choses si loin, de concert avec le P. Tellier, qu'étant nommé
député des états de Languedoc pour en venir apporter les cahiers, il y
eut un ordre du roi d'en choisir un autre. Quand il se fut bien instruit
de tout ce qu'il voulait découvrir, qu'il en eut rendu compte au P.
Tellier, et qu'il n'eut plus rien à apprendre, il chanta la palinodie
dès qu'il fut retourné à Nîmes, y monta en chaire et fit amende
honorable à la constitution. Aussitôt le roi lui fit rendre la
députation, et il vint triomphant jouir de son crime dans les caresses
et les promesses du P. Tellier, qui ne l'empêcha pas de devenir
l'horreur du monde. Il avait bien d'autres choses encore sur son compte,
et est mort enfin escroc et banqueroutier, et d'une façon déplorable.

Il se présenta une affaire au conseil de régence qui me donna lieu à un
petit trait qu'il faut que je m'amuse un moment à rapporter. M.
d'Elboeuf était gouverneur de Picardie et d'Artois, où il ne tenait pas
ses mains dans ses poches, et se moquait des intendants. M. le duc
d'Orléans le considérait et le ménageait, et il en abusa au point qu'il
le força d'y mettre quelque ordre. Il y a un petit canton riche et
abondant, entré l'Artois et la Flandre, qui s'appelle le pays de Lalleu,
qui de tout temps était du gouvernement de Flandre et des états de
Lille. M. d'Elboeuf qui était bien aise d'y allonger ses mains et
l'étendue aussi de son gouvernement, demanda que ce pays de Lalleu fût
incorporé aux états d'Artois, et ne fût plus de ceux de Lille. Je
supprime les raisons de part et d'autre, qui ne feraient qu'ennuyer.

La maréchale de Boufflers vint m'apprendre cette prétention qui devait
être incessamment jugée au conseil du dedans du royaume, puis rapportée
par d'Antin au conseil de régence pour l'être définitivement. Peu
importait à la maréchale de quels états serait ce petit pays, mais elle
sentait que la prétention du duc d'Elboeuf était un chausse-pied s'il la
gagnait, pour les états d'Artois, de le prétendre après de son
gouvernement, quoiqu'il ne s'en agît pas encore. Je lui conseillai d'en
faire parler par son frère à M. le duc d'Orléans. Mais depuis l'affaire
du régiment des gardes, il n'y avait plus guère que de l'extérieur entre
eux, et elle me le laissa bien sentir. Je voulus lui persuader de parler
elle-même sans l'y pouvoir résoudre. Elle me dit qu'elle mettait toute
sa confiance en moi pour conserver au gouvernement de Flandre, qu'avait
son fils, toute son intégrité. Elle avait raison, car j'étais fort de
ses amis, et on a pu voir que je l'étais intimement de son vertueux
mari. Je ne lui dis point ce que je ferais, car je l'ignorais encore, et
après toute réflexion faite je crus plus à propos de ne faire rien, dans
la connaissance de la faiblesse de M. le duc d'Orléans, qui ne tiendrait
jamais, pour un petit garçon de l'âge du duc de Boufflers, à
l'audacieuse ardeur du duc d'Elboeuf, soutenue de celle de M. le Grand,
dont le fils avait la survivance du gouvernement de Picardie. J'attendis
donc sans dire mot à personne et sans voir depuis la maréchale de
Boufflers, que l'affaire se rapportât au conseil de régence, où les
chefs ou présidents des autres conseils furent appelés.

Dès que nous fûmes en place, d'Antin mit les papiers sur la table et
voulut commencer son rapport. «\,Un moment, monsieur,\,» lui dis-je. Et
me tournant vers le régent, je lui dis que, s'il le trouvait bon, il
fallait, avant de commencer l'affaire, savoir si au cas que les états
d'Artois la gagnassent, M. d'Elboeuf prétendait distraire du
gouvernement de Flandre le pays de Lalleu et le joindre à celui
d'Artois, parce que, en ce cas, nous étions plusieurs qui étions trop
proches de M. d'Elboeuf pour être ses juges, à commencer par M. d'Antin,
son cousin germain, moi, issu de germain, M. le maréchal d'Estrées et
d'autres encore.

Ce n'était pas que j'ignorasse qu'en ce conseil les parentés ne font
rien, parce que devant le roi, qui à tout âge y est censé présent, on
n'a que voix consultative pour débattre et l'informer, et que sa seule
voix décide, et que sur cette question que le chancelier d'Aguesseau,
tout au commencement qu'il le fut, avait voulu remuer sous prétexte de
l'âge et de l'absence réelle du roi, il avait passé en plein conseil
qu'il demeurerait de la sorte, et comme le roi âgé et présent\,; mais
j'espérais qu'on n'y songerait plus, et cela arriva comme je l'avais
pensé et à tout hasard tenté.

M. le duc d'Orléans dit que j'avais raison, et tout de suite demanda à
d'Antin ce qui en était. Il répondit qu'il n'en était point question\,;
que M. d'Elboeuf ne lui avait point parlé de gouvernement, et que
sûrement il ne demandait rien là-dessus. Je repris la parole, et dis au
régent que, puisque cela était, la chose méritait d'être constatée à
cause de la proche parenté des juges, et que dès que M. d'Elboeuf ne
songeait point, quoiqu'il fût jugé, à demander que le pays de Lalleu fût
mis de son gouvernement, il serait bon que Son Altesse Royale voulût
bien ordonner à M. d'Antin d'écrire présentement sur son dossier qu'en
cas que le pays de Lalleu fût jugé séparé des états de Lille et joint à
ceux d'Artois, ce jugement n'aurait aucune influence à l'égard de l'état
du gouverneur du pays de Lalleu, qui demeurait toujours à l'avenir du
gouvernement de Flandre comme par le passé. Le régent regarda la
compagnie, disant qu'il n'y trouvait point d'inconvénient. D'Antin dit
que l'écrire ou ne l'écrire pas était de même, parce que M. d'Elboeuf ne
demandait rien. «\,Mais, monsieur, repris-je, cela sera plus régulier,
et Son Altesse Royale l'approuve. --- À la bonne heure,\,» dit d'Antin,
et se mit à l'écrire. Un moment après, tandis qu'il écrivait, je dis au
régent qu'il me semblait à propos aussi, puisque M. d'Antin en mettait
la note sur le dossier du procès, que M. de La Vrillière l'écrivît en
même temps sur le registre du conseil, pour que cela fût uniforme. Cela
parut si simple que le régent, sans regarder la compagnie comme la
première fois, répondit\,: «\,A la bonne heure, il n'a qu'à l'écrire.\,»
À l'instant je regardai La Vrillière, qui aussitôt prit la plume et
l'écrivit sur le registre du conseil. Dès que cela fut fait, d'Antin
commença le rapport. J'y reviendrai pour une anecdote singulière.

Le soir la maréchale de Boufflers vint chez moi, bien en peine de ce que
les états d'Artois avaient gagné, et s'il n'y avait eu rien de fait sur
le gouvernement. «\,Pardonnez-moi, madame, lui dis-je, il a été question
du gouvernement, et on y a fait quelque chose.\,» Et tout de suite,
après lui avoir donné la souleur\footnote{Frayeur, saisissement.}, je
lui contai ce qui s'était passé. Elle m'en embrassa bien et fut ravie.

Tandis qu'elle était chez moi, M. d'Elboeuf était chez La Vrillière, à
qui il dit, sans seulement paraître en douter\,; que puisque le pays de
Lalleu était adjugé membre des états d'Artois, et ne l'être plus de ceux
de Lille, il était de son gouvernement aussi, et que l'un emportait
l'autre. Sur la mine que fit La Vrillière\,: «\,Comment, lui dit-il,
monsieur, avec l'air de la plus grande surprise du monde, est-ce que
vous en pouvez douter\,? eh\,! ce pays n'a été du gouvernement de
Flandre que comme membre des états de Lille, et l'arrêt d'aujourd'hui,
qui l'en distrait pour le faire membre des états d'Artois, décide la
question et n'y laisse pas l'ombre de difficulté.\,» La Vrillière lui
répondit modestement que le conseil ne l'avait pas entendu ainsi, et
qu'il croyait qu'il ferait bien de n'y pas songer. M. d'Elboeuf lui
demanda, avec émotion, où il avait pris cette intention du conseil qui
ne pouvait être avec l'arrêt qu'il avait rendu et qui décidait tout
seul. Alors La Vrillière lui montra le registre, et lui dit de lire ce
qu'il avait écrit en plein conseil par ordre de M. le duc d'Orléans et
du conseil. Voilà le duc d'Elboeuf en furie, qui dit qu'il allait parler
à M. le duc d'Orléans, et qu'il ferait bien changer cette belle
décision. Il y fut en effet, mais comme il s'agissait d'effacer ce qui
avait été écrit sur le dossier et sur le registre en plein conseil, et
de l'avis de tout le conseil, ou explicite ou tacite, sans opposition
d'aucun, et en changer la disposition du blanc au noir, le régent se
défendit d'y pouvoir toucher et de pouvoir reporter au conseil une chose
qu'il avait décidée. M. d'Elboeuf tempêta et cria, mais ce fut tout,
l'affaire était bridée, et le pays de Lalleu demeura du gouvernement de
Flandre, et en est encore aujourd'hui.

Je m'étais bien attendu au but et au vacarme de M. d'Elboeuf contre
lequel la faiblesse du régent aurait besoin d'une barrière, et je me sus
bon gré de l'avoir adroitement su introduire, et poser si forte, sans
que personne se fût aperçu ni douté de mon but, qu'elle ne pût après
recevoir d'atteinte. La maréchale de Boufflers alla le lendemain
remercier le régent.

Je reviens maintenant à l'anecdote qui confirmera pleinement ce que j'ai
marqué du caractère indécis, à l'extrême, du chancelier d'Aguesseau. M.
le duc d'Orléans avait ordonné que cette affaire de Lalleu, qui était
longue, serait rapportée en deux conseils, le même jour, le matin et
l'après-dînée\,; que le matin serait pour le rapport uniquement, sans
que d'Antin s'ouvrît en rien de son opinion\,; que l'après-dînée il
commencerait par opiner\,; que tout le conseil opinerait après et que
l'arrêt serait rendu. D'Antin fit un très long rapport qui tint jusqu'à
une heure après midi. Comme on sortait du conseil le chancelier me prit
auprès de la porte, et me dit tout bas qu'il mourait d'envie de prendre
avec moi une liberté qu'il ne voudrait pas prendre avec un autre, et
qu'il espérait que je ne trouverais pas mauvaise, c'était de me demander
l'avis que j'avais pris sur le rapport, et que j'opinerais
l'après-dînée. Je lui répondis qu'en effet je ne m'en ouvrirais pas à un
autre, et après quelques compliments je le lui dis, et, aussi
sommairement que le temps et le lieu l'exigeaient, les raisons
principales qui m'y déterminaient. Il m'embrassa et me dit, plus que
très obligeamment, que je lui faisais le plus grand plaisir du monde
d'avoir bien voulu le lui dire, parce que c'était le sien aussi, et que
le mien l'y confirmait, avec force compliments flatteurs. Nous nous
séparâmes de la sorte.

Cette affaire, dans laquelle je n'entrerai pas ici, était susceptible de
trois sortes d'opinions\,: laisser le pays de Lalleu comme il était,
membre des états de Lille\,; l'en distraire et l'adjoindre à ceux
d'Artois\,; enfin, laisser ce petit pays indépendant de ces deux états,
et qu'il en eût pour lui tout seul. C'est ce que ce petit pays
demandait, consentant toutefois à demeurer comme il était, uni si on le
voulait aux états de Lille, mais se défendant d'être uni à ceux
d'Artois. Mon avis était qu'il eût des états particuliers pour lui, et
qu'il ne fût membre ni de ceux de Lille ni de ceux d'Artois. C'était
aussi celui du chancelier quand nous sortîmes du conseil du matin, comme
je viens de le dire.

Nous n'eûmes que le temps de dîner. À trois heures le conseil commença.
Quoiqu'on y fût fort accoutumé aux beaux rapports de d'Antin,
l'exactitude, la précision, l'explication foncière, la netteté, la
force, l'agrément de son rapport avait enlevé la compagnie, qui ne la
fut pas moins de sa belle, longue et forte opinion l'après-dînée. Il se
peut dire qu'il excellait en ce genre sur tous les magistrats\,; avec
cela une mémoire qui n'oubliait pas les plus petites choses\,; qui
ramenait tout avec ordre, justesse et clarté, qui rie se méprenait
jamais en aucun fait, circonstances, nom propre, date, et qui, à mesure
qu'il en citait, disait à l'évêque de Troyes, devant qui d'ordinaire il
mettait la pile de ses papiers, le cahier, la liasse, la page par numéro
et par chiffre, où il trouverait ce qu'il citait, et\,; dans le moment
même, M. de Troyes le trouvait et le lisait tout haut. D'Antin, qui
n'opinait jamais pour soi-même, et qui ne faisait que rapporter l'avis
du conseil du dedans, ainsi que tous les autres chefs des autres
conseils sur les affaires qu'ils en rapportaient au conseil de régence,
fut pour les états d'Artois. Presque tous le suivirent, le peu d'autres
furent pour ceux de Lille.

Mon rang d'opiner était immédiatement avant le chancelier, après lequel
il n'y avait plus que les deux bâtards et les princes du sang. Je vis
donc que j'allais ouvrir un avis, et, comme je savais que le chancelier
serait du même, je ne voulus pas en épuiser les raisons pour en laisser
de nouvelles à dire au chancelier, qui donnassent lieu aux préopinants
de s'y accrocher pour revenir à son avis avec moins de répugnance qu'ils
n'en auraient eu à revenir au mien, et de couvrir leur petite vanité du
poids de la place, de l'état et de la capacité du premier magistrat.
Néanmoins, comme il fallait des raisons pour soutenir un avis tout neuf,
je ne laissai pas de parler assez longtemps tant {[}pour{]} le faire
bien entendre et valoir, que pour affaiblir et réfuter les deux autres
avis. Je fus surpris d'y être souvent interrompu par des voix qui
disaient tout haut\,: «\,Mais M. de Saint-Simon a raison.\,» Cela arriva
si souvent et par tant de personnes, que je me tournai à la fin vers le
conseil, car on opinait un peu tourné vers le régent, et je dis que,
puisqu'on trouvait que j'avais raison, rien n'empêchait de revenir à mon
avis, ceux qui le trouvaient le meilleur, puisque l'arrêt n'était point
fait. Des voix dirent\,: «\, Cela est vrai,\,» et encore, pendant le
reste de mon opinion, que j'avais raison\,; cependant elles s'en tinrent
là, et personne ne prit la parole pour se rendre à mon avis. Je compris
la petite faiblesse, et je m'en sus plus de gré de laisser quelques
raisons nouvelles au chancelier à dire et à appuyer, qui donneraient
lieu aux préopinants de revenir à son avis avec moins de peine qu'au
mien.

Le chancelier, quand j'eus fini, débuta par l'éloge de mon avis, dont il
loua en détail la justice, les raisons et la force. Il balança ensuite
les trois avis en avocat général\,; puis, se rabattant sur la politique
et les événements fâcheux de la dernière guerre du feu roi en Flandre,
il s'étendit sur son regret d'être obligé de faire taire le droit, la
raison, l'équité devant les motifs majeurs et pressants de l'intérêt de
l'État, paraphrasa longuement et gauchement, quoique éloquemment, cette
politique, protesta encore de sa répugnance et de son regret d'être
entraîné par des considérations si fortes, nonobstant le droit et
l'équité, et conclut pour les états d'Artois. Je l'écoutais avec une
attention extrême. Je ne pouvais comprendre d'abord qu'il eût changé
d'avis depuis qu'il m'avait parlé en sortant du conseil deux heures
auparavant, et ma surprise fut extrême quand à la fin je n'en pus
douter. J'oublie de dire qu'en finissant il loua encore mon avis, et me
fit un petit compliment direct sur la peine où il était de n'en pouvoir
être par la seule raison d'État.

Dès que je m'aperçus qu'il avait tourné, je dis tout bas au comte de
Toulouse que je ne pouvais revenir d'un étonnement dont je lui dirais la
cause en sortant\,; mais que je le priais de ne pas prendre la parole
après le chancelier, parce que je voulais parler encore. Ce n'était pas
que j'espérasse faire revenir personne à ce que je voyais, mais je ne
crus pas juste de taire les raisons que je n'avais retenues que pour les
laisser neuves dans la bouche du chancelier, par la raison que j'en ai
dite. Ainsi, quand il eut fini, je priai le régent de me permettre
d'ajouter un mot à mon opinion. Je le fis donc avec étendue et avec les
mêmes applaudissements que j'avais raison, mais sans autre succès. Le
surplus des opinions se conforma au chancelier, et l'arrêt suivit de
même.

En sortant du conseil, le comte de Toulouse me prit à part, curieux de
savoir la cause de mon extrême surprise, et fut étonné au dernier point,
lorsque je la lui dis. Le chancelier et moi ne nous cherchâmes point en
sortant de ce second conseil, et jamais depuis nous ne nous en sommes
parlé.

Le pays de Lalleu, qui est riche, mais qui n'a que de gros laboureurs,
mais gens de bon sens et de bon gros raisonnement, en avaient député à
la suite de cette affaire qui les intéressait beaucoup. On me les
annonça pour la première fois comme j'allais sortir pour le conseil du
matin, où leur affaire fut rapportée. Ils voulurent me parler et me
présenter leur mémoire\,; je l'avais eu d'ailleurs avec ceux des états,
et je les avais tous fort étudiés. Je voyais que ces paysans avaient
raison, et j'étais fâché qu'ils vissent et instruisissent si tard leurs
juges. Je n'avais pas alors le temps de les entendre\,: c'était l'heure
du conseil. Je les rabrouai donc au lieu de les écouter, et je montai
devant eux en carrosse. Je fus tout étonné de les voir revenir le
surlendemain matin, avec deux prodigieuses mannes du plus beau linge de
table que j'aie jamais vu et en la plus grande quantité. Ils avaient su
que j'avais été seul pour eux au conseil, et que j'avais longuement
opiné. Ils venaient avec ce présent me témoigner leur reconnaissance.
J'eus beau leur dire ce que je devais là-dessus, je ne pus les empêcher
de déployer quelques nappes et quelques serviettes\,; mais quand ils
virent qu'il leur fallait les remporter, ils se mirent à pleurer et à
dire que je les méprisais, quoique je leur eusse parlé avec toute
l'honnêteté possible. Je fus si touché de leur douleur de si bonne foi,
que je leur dis enfin que, pour leur montrer combien j'étais éloigné de
mépris et touché de leur sentiment pour moi, ils me feraient faire ce
que je n'avais jamais fait et ne ferais jamais pour personne. Je pris
donc une nappe et une douzaine de serviettes\,; cela les consola un peu.
Ils remportèrent tout le reste en me comblant de bénédictions. Je le dis
à M. le duc d'Orléans. Pour l'histoire du chancelier, je n'en parlai
qu'au comte de Toulouse.

Il y eut une assez forte brouillerie entre les ducs de Noailles et de La
Force sur quelques affaires de finances. La Force avait été mis dans le
conseil de finances à l'insu, puis malgré le duc de Noailles, contre
tout ce que j'avais pu lui dire d'une place en troisième, après le
maréchal de Villeroy et le duc de Noailles, dont il était si fort
l'ancien en dignité, sans compter la naissance, et place subalterne
encore pour le travail et le détail, et qui, sous le nom personnel de
vice-président, n'était pas supérieure en effet aux emplois des autres
de ce conseil, qui, plus rompus aux affaires de finances que lui et
appuyés du duc de Noailles, lui feraient passer sans cesse la plume par
le bec, et avec force révérences se moqueraient de lui. Il fut en effet
exposé à toutes les niches que le duc de Noailles ne lui épargna pas.
L'esprit et la capacité, joints à sa qualité, le soutinrent, mais
n'empêchèrent pas tous les effets de la jalousie du duc de Noailles
contre un seigneur qui pour le moins le valait et lui était égal, et
qu'il voyait lié avec Law, qui était sa bête. Ces démêlés finirent avec
beaucoup d'autres qui avaient moins éclaté, mais ce ne fut qu'en
apparence, par un département fort étendu qui fut donné à M. de La
Force, avec assez d'autorité\,; mais à quelque sauce que cela pût se
mettre, ce n'était être, en bon français, qu'intendant des finances un
peu renforcé, et par conséquent être fort déplacé, comme il n'en pouvait
être autrement, dès qu'il avait bien voulu se fourrer si bassement dans
le conseil des finances.

J'avais oublié deux bagatelles sur M\textsuperscript{me} la duchesse de
Berry. Elle choisit M\textsuperscript{me} d'Arpajon pour la place d'une
de ses dames qui vaquait par la mort de M\textsuperscript{me} d'Aydie,
soeur de Rion. Arpajon, l'un des plus sots hommes de France, sans
contredit, et des plus avares, avait acheté le gouvernement de Berry du
duc de Noailles, et obtenu assez légèrement la Toison en Espagne, où il
avait servi longtemps avec les troupes de France. Il était lieutenant
général et petit-fils du bonhomme Arpajon, duc à brevet, chevalier de
l'ordre, et distingué en son temps par son mérite et ses emplois, la
naissance ancienne et fort bonne. M\textsuperscript{me} d'Arpajon avait
une figure extrêmement noble et agréable, peu d'esprit, beaucoup de
douceur et de politesse\,; très vertueuse et d'une piété qui n'a
toujours fait qu'augmenter. Elle était fille de Le Bas de Montargis, un
des trésoriers de l'extraordinaire des guerres, et d'une fille de
Mansart, qui avait les bâtiments. Elle était extrêmement riche et peu
heureuse avec un mari qui ne la méritait pas\,; mais elle le cachot avec
grand soin, et lui redait des devoirs infinis. Ils n'ont eu qu'une
fille, qui a épousé, avec de grands biens, le second fils du duc de
Noailles. M\textsuperscript{me} la duchesse de Berry la choisissait
volontiers, avec la marquise de La Rochefoucauld, fille de Prondre, pour
aller avec elle coucher aux Carmélites, et leur disait toujours\,: «\,Je
vous amène mes deux bourgeoises.\,»

Cette princesse si haute et si fière, avec qui les seuls princes du sang
pouvaient manger, et encore point à l'ordinaire ni en public, hors à des
mariages, mais à la campagne et en particulier, mangeait avec tous les
roués de M. le duc d'Orléans, et chez elle avec des hommes de peu de
chose, et avec un jésuite d'esprit et de manège, qui s'appelait le P.
Riglet, qu'elle avait connu de jeunesse par ses femmes, et qui en disait
des meilleures.

Elle imagina aussi d'avoir un maître de la garde-robe. C'est une charge
de valet. Joyeux, mort premier valet de chambre de Monseigneur, l'avait
été de la reine. Ceux de la reine mère et des deux Dauphines ne valaient
pas mieux. Elle trouva une manière de chevalier d'industrie, grand
spadassin de son métier, bâtard d'un Gouffier, qui se faisait appeler
Bonivet, qui ne voulait point être bâtard, et qui pourtant n'a pu être
autre chose ni reconnu comme légitime de pas un de la maison de
Gouffier. Il trouva là quelques petits gages dont il avait besoin, et y
espéra quelque fortune par son manège. M\textsuperscript{me} la duchesse
de Berry le prit, et dit en confidence à M\textsuperscript{me} de
Saint-Simon, qui ne lui en parlait point, que c'était une espèce de nom
qu'elle mettait dans sa maison, de plus un homme de main qu'elle était
bien aise d'avoir, parce que, bien aujourd'hui avec M. le duc d'Orléans,
cela pouvait changer, et qu'il fallait avoir chez soi de quoi se faire
compter. Tels étaient la tête et le coeur de cette princesse.

On apprit la mort du cardinal Arias, archevêque de Séville, un des plus
honnêtes hommes et des meilleures têtes d'Espagne, et qui avait le plus
contribué au testament de Charles II, étant conseiller d'État et
commandeur dans l'ordre de Malte. On a vu quel il était lorsqu'on a
parlé ici de l'avènement de Philippe V à la couronne, la part qu'Arias
eut au gouvernement, et comme la princesse des Ursins sut s'en défaire,
ainsi que du cardinal Portocarrero et de tous les autres, pour demeurer
seule maîtresse du gouvernement. Arias fut aussi bon prêtre et évêque,
qu'il avait été bon ministre d'État, ravi de n'avoir plus à se mêler de
rien, uniquement appliqué à son diocèse, d'où il ne sortit plus, et à
s'occuper de son salut sous la pourpre romaine, qu'il n'avait point
briguée, mais que la pudeur lui fit donner par le roi d'Espagne, pour
une marque de son estime et de sa satisfaction de ses services, qui fut
universellement applaudie. Arias méprisa le monde et la cour, et se
trouva mieux à Séville qu'il n'avait fait à Madrid, quoique ce grand
archevêché ne lui eût été donné que comme un exil honorable et pour se
défaire de lui. Il était assez vieux, et fut regretté de toute
l'Espagne, et infiniment dans son diocèse.

La comtesse d'Harcourt, qui se fit appeler depuis comtesse de Guise,
comme on l'a vu ailleurs, perdit M\textsuperscript{me} de Monjeu sa
mère, qui était Dauvet des Marests.

En même temps mourut aussi Richard Hamilton. C'était un homme de
beaucoup d'esprit, qui savait, qui amusait, qui avait des grâces et
beaucoup d'ornement dans l'esprit, qui avait eu une très aimable figure
et beaucoup de bonnes fortunes en Angleterre et en France, où la
catastrophe du roi Jacques II l'avait ramené. Il avait servi avec
distinction, et la comtesse de Grammont, sa soeur, l'avait initié dans
les compagnies de la cour les plus choisies\,; mais elles ne lui
procurèrent aucune fortune, pas même le moindre abri à la pauvreté. Il
était catholique, et sa soeur l'avait mis dans une grande piété qui
l'avait fait renoncer aux dames, pour qu'il avait souvent fait de très
jolis vers et des historiettes élégantes. Sa demeure était à
Saint-Germain. Il alla mourir à Poussay chez sa nièce, qui en était
abbesse, pauvre elle-même, mais moins pauvre que lui pour ne pas mourir
de faim.

Vers le milieu de décembre, l'abbé de Bonnoeil fut trouvé tué dans sa
chambre de coups de bâton sur la tête, et de coups d'épée dans le corps
par devant et par derrière, et son valet de chambre, qui était son seul
domestique, au même état près de lui, son épée nue auprès de lui, et un
couteau de chasse nu auprès de l'abbé. Il était grand joueur, avait
beaucoup gagné depuis peu et voyait assez bonne compagnie. On le trouva
volé. La femme du valet de chambre fut arrêtée sur quelques indices.
Elle avoua qu'elle était en commerce avec un soldat aux gardes, qui
entra dans la maison pour tuer le valet de chambre et voler le maître,
qui, pour son malheur, rentra chez lui bien plus tôt qu'à l'ordinaire,
comme l'expédition s'achevait. Le soldat fut arrêté à Bar un matin dans
son lit, qui, se voyant pris, se tua tout roide d'un poignard qu'il
avait sous son chevet. On prit aussi un laquais de M\textsuperscript{me}
du Guesclin, chanoinesse, qui voulut tuer sa maîtresse. Elle eut le
courage de lui arracher son épée et la charité de lui dire de se sauver.
Sa femme de chambre, qui était du complot, l'ut prise aussi. Ces
tragiques aventures firent redoubler les défenses des jeux de hasard, et
mettre en prison une trentaine de tailleurs au pharaon, qui continuaient
leur métier malgré les premières défenses.

Les états de Bretagne s'ouvrirent de façon à ne pas laisser douter qu'il
n'y eût du bruit, et qu'on ne s'y fût préparé dans la province. La
noblesse qui vint au-devant du maréchal de Montesquiou arrivant à Rennes
pour les tenir, se formalisa de ce qu'il ne sortit point de sa chaise de
poste pour monter à cheval avec elle, et de ce qu'au lieu d'aller aux
états de son logis à pied, avec une foule de noblesse venue chez lui
pour l'y accompagner, il s'y fit porter en chaise. En ces deux points la
noblesse n'avait pas tort\,; mais elle en prit occasion de traiter fort
mal le maréchal de Montesquiou, à qui ils disputèrent tout, et de là,
non contents de refuser le don gratuit par acclamations, comme ils
l'avaient toujours fait depuis 1672, et peu satisfaits d'un million de
diminution qui leur avait été accordé dessus, ils ne parlèrent que de
leurs privilèges du temps de leurs ducs, et voulurent changer une
infinité de choses, sans que le prince de Léon, qui présidait à la
noblesse, et qui y était considéré, pût, rien gagner. On y envoya neuf
bataillons, outre deux qui y étaient déjà, et on y fit marcher en même
temps dix-huit escadrons. On s'attendait depuis quelque temps à y voir
arriver du désordre. Le maréchal de Montesquiou avait été chargé de
séparer les états s'il les voyait disposés à ne pas obéir à la volonté
du roi. Il différa quelques jours\,; mais les états ayant déclaré qu'ils
ne changeraient point d'avis, il congédia l'assemblée. Ce fut le
commencement des troubles de ce pays-là, et le fruit des pratiques de M.
et de M\textsuperscript{me} du Maine.

Il y avait quelque temps que j'étais dans un commerce secret et encore
plus obscur qui, en voulant me mettre le doigt sur la lettre, m'en
montrait assez pour me faire voir en gros de dangereuses cabales, et me
faisait une énigme suivie de tout ce qui m'en pouvait éclaircir.
M\textsuperscript{me} d'Alègre, dont le mari a été longtemps depuis
maréchal de France, m'envoya un prêtre un matin me demander chez moi une
audience fort secrète, et me prier surtout de ne point aller chez elle.
Je ne la connaissais en façon du monde, et je n'avais jamais été en
aucun commerce avec son mari. L'aventure me parut fort singulière, aussi
cette femme l'était-elle beaucoup. J'en ai parlé assez pour la faire
connaître, à l'occasion du mariage de sa fille M\textsuperscript{me} de
Barbezieux, et des suites de ce mariage. M\textsuperscript{me} d'Alègre
vint donc chez moi à l'heure marquée.

Ce fut d'abord des compliments sans fin et des louanges merveilleuses\,;
je répondais courtement et voulais venir au fait\,; mais je reconnus
bientôt que l'embarras d'y entrer multipliait la préface. De là elle
vint aux louanges de M. le duc d'Orléans, à celles de mon attachement
pour lui, à la constitution, au gouvernement. Elle épuisa tous les
entours et les environs avec une impatience de ma part inexprimable.
Enfin elle se mit sur le ton des oracles, serrant la bouche, tournant
les yeux, accommodant sa coiffe, frottant son manchon, tantôt me
regardant à me pénétrer, puis baissant les yeux et jouant de l'éventail,
disant deux mots coupés et laissant le sens suspendu, tombant dans un
morne silence. Ce manège fut constant dans toutes les visites que j'en
reçus depuis, et qui furent assez fréquentes pendant quatre ou cinq
mois. Enfin elle me fit entendre qu'il se brassait beaucoup de choses
très importantes contre M. le duc d'Orléans et contre son gouvernement,
qu'elle n'en pouvait douter, et sans rien spécifier ni nommer lieux ou
gens, elle ne cessait d'appuyer sur la certitude de ses connaissances,
et de m'exhorter d'y prendre garde, et d'avertir M. le duc d'Orléans
pour qui elle me dit merveilles de son attachement et de l'obligation
qu'elle se croyait en conscience de venir à moi par mon attachement pour
lui, et la confiance qu'il avait en moi. J'eus beau lui dire que, dans
les avis qu'elle avait la bonté de me donner, je ne voyais qu'une
inquiétude inutile à prendre, sans aucune lumière qui pût conduire aux
précautions nécessaires, je n'en pus jamais tirer davantage, sinon
qu'elle me reverrait quelquefois avec le même mystère, qu'elle verrait
quand et comment elle m'en pourrait dire davantage\,; revint à appuyer
la certitude de ses connaissances, revint aux compliments et aux
protestations, et surtout exigea le plus entier secret de M. le duc
d'Orléans et de moi, et que je n'allasse, jamais chez elle, parce que le
moindre soupçon qu'on aurait d'elle la perdrait. Tout ce verbiage dura
près de deux heures, et, le mystère fut poussé jusqu'à exiger que je
fermerais la porte de mon cabinet sur elle sans la conduire un pas.

Je savais bien qu'il se brassait quelque chose en Bretagne, où les états
n'étaient point encore assemblés. Mais M\textsuperscript{me} d'Alègre
était de Toulouse, son mari d'Auvergne. Je ne leur voyais point
d'entours bretons. Sa singularité, sa vie dévote et assez retirée, son
esprit, car elle en avait, qui assait pour tourné à la chimère, me fit
soupçonner qu'elle cherchait à s'intriguer. Je ne fis donc pas grand cas
de tout ce qu'elle me dit, et comme il n'y eut rien que de fort vague,
je ne crus pas en devoir alarmer le régent.

Après l'éclat des états de Bretagne, elle revint, me dit qu'elle était
bien informée d'avance de ce qui venait d'arriver, et encore par quels
ressorts\,; que le régent se trompait s'il pensait que l'affaire fût
finie, ou que les prétentions des états en fussent l'objet\,; et me
prenant les mains et les appuyant sur mes genoux avec des roulis
d'yeux\,: «\,Tout cela, monsieur, assurez-vous-en bien et ne le laissez
pas ignorer au régent, n'est que le chausse-pied, vous en verrez bien
d'autres\,; mais\ldots. et\ldots. car\ldots.\,» Et d'autres mots coupés,
comme une femme qui sait et qui se retient, et tout de suite se lève
pour s'en aller. J'eus beau faire, je n'en pus rien tirer de plus. En
passant la porte\,: «\,Il n'est pas temps encore, me dit-elle, mais je
vous reverrai, mais ne vous endormez pas, ni M. le duc d'Orléans.\,» En
disant cela, elle ferme la porte et s'en va.

Quelque obscure que fût cette seconde visite, je crus devoir pourtant en
rendre compte à M. le duc d'Orléans. Quoiqu'il connût bien ce que
c'était que M\textsuperscript{me} d'Alègre, et qu'il ne vît pas plus
clair dans ses langages que moi, il me parut en faire plus de cas que je
n'aurais pensé. Il voulut que je suivisse ce commerce, c'est-à-dire que
je me tinsse toujours prêt à la recevoir et à l'entendre, puisque sa
maison m'était interdite\,; que je lui témoignasse reconnaissance de sa
part, et que je fisse de mon mieux pour en tirer tout ce qu'il serait
possible. J'aurai à revenir à ce commerce plus d'une fois.

L'abbé Dubois revint d'Angleterre les premiers jours de décembre, et y
retourna avant la fin du même mois. C'était Nancré qu'il avait établi
son correspondant et par qui ses lettres passaient au régent et du
régent à lui. Par ce qu'on a vu ici en quelques endroits de Nancré, on
comprend qu'il était très propre à vouloir être et à devenir en effet
l'homme de confiance de l'abbé Dubois. Nocé l'avait été un temps, mais
il était trop singulier et trop roide pour que cette liaison pût
durer\,; elle se tourna depuis en froideur et puis en haine ouverte.
Nancré avait tout le liant, le ployant, la patience, l'intelligence et
la conformité d'âme, qui l'y redait merveilleusement propre. Il était
souple et flatteur avec Canillac et admirateur avec Noailles, valet à
tout faire avec Law pour en tirer et pour plaire, et grand courtisan de
Stairs. J'ai parlé de lui ailleurs plus en détail. En un mot, il voulait
être et surtout s'enrichir et faire encore fortune.

L'abbé Dubois trouva le prince de Galles en arrêt dans son appartement,
sans pouvoir être vu que de son plus nécessaire service. Il écrivit de
là deux lettres au roi son père, qui l'irritèrent encore plus. Il eut
ordre ensuite de sortir du palais. Il fut loger chez le lord Lumley à
Londres, puis s'établit à une lieue de Londres au village de Richmont.
Toute l'Europe a su l'horrible catastrophe du comte de Koenigsmarck que
Georges, n'étant que duc de Hanovre, fit jeter dans un four chaud, et
mit la duchesse sa femme dans un château bien gardé, où elle n'a eu un
peu de liberté que depuis que Georges a été roi d'Angleterre. Ce prince
ne pouvait souffrir son fils dans la persuasion qu'il n'était pas de
lui, et le fils ne pouvait souffrir le père dans le dépit de cette
persuasion continuellement marquée, et des mauvais traitements faits à
sa mère. Charlotte de Brandebourg-Anspach, sa femme, était une princesse
d'esprit, liante, sage, aimée extrêmement en Angleterre, fort bien avec
son mari et son beau-père, qui se mettait sans cesse entre-deux. Le roi
d'Angleterre lui offrit de demeurer au palais avec ses enfants, mais
elle voulut suivre son mari.

\hypertarget{chapitre-vii.}{%
\chapter{CHAPITRE VII.}\label{chapitre-vii.}}

1717

~

{\textsc{Idées et précautions d'Albéroni.}} {\textsc{- État embarrassant
du roi d'Espagne.}} {\textsc{- Capacité de del Maro odieuse à Albéroni,
qui le décrie partout.}} {\textsc{- Ses exhortations et ses menaces au
pape en faveur d'Aldovrandi.}} {\textsc{- Manèges d'Aldovrandi.}}
{\textsc{- Sagacité de del Maro.}} {\textsc{- Première audience du
colonel Stanhope peu satisfaisante.}} {\textsc{- Chimères d'Albéroni.}}
{\textsc{- Craintes d'Albéroni parmi sa fermeté.}} {\textsc{- Son
espérance en la Hollande fomentée par Beretti.}} {\textsc{- Découverte
de ce dernier sur le roi de Sicile.}} {\textsc{- Faux raisonnements de
Beretti sur les Hollandais.}} {\textsc{- Abbé Dubois à Londres.}}
{\textsc{- Monteléon y est leurré\,; cherche à pénétrer et à se faire
valoir.}} {\textsc{- Audacieux avis des Anglais au régent sur son
gouvernement intérieur\,; qu'ils voudraient changer à leur gré.}}
{\textsc{- Réflexions.}} {\textsc{- Projets du czar à l'égard de la
Suède, et ceux du roi de Prusse.}} {\textsc{- Offres de la Suède.}}
{\textsc{- Conduite suspecte de Goertz, et celle du czar et du roi de
Prusse en conséquence.}} {\textsc{- Avis de ce dernier au régent.}}
{\textsc{- Ses chimères.}} {\textsc{- Objet du roi d'Angleterre dans son
désir de moyenner la paix entre l'empereur et l'Espagne à Londres.}}
{\textsc{- Penterrieder y arrive.}} {\textsc{- Divers sentiments en
Hollande.}} {\textsc{- Conditions fondamentales proposées à l'Espagne
pour la paix.}} {\textsc{- Albéroni, aigri contre Stairs, est contenté
par Stanhope, qui l'amuse sur l'affaire principale par une équivoque.}}
{\textsc{- Grande maladie du roi d'Espagne.}} {\textsc{- Solitude de sa
vie.}} {\textsc{- Albéroni veut interdire toute entrée à Villena,
majordome-major, qui, dans la chambre du roi d'Espagne, la reine
présente, donne des coups de bâton au cardinal, et est exilé pour peu de
temps.}} {\textsc{- Le roi d'Espagne fait un testament.}}

~

Albéroni ne voulait ouïr parler d'aucun accommodement avec l'empereur.
Il se forgeait des chimères que lui-même n'espérait pas, et qu'il ne
laissait pas de faire proposer pour attaquer l'Italie et les Pays-Bas à
la fois par la France `d'un côté avec les Hollandais, et de l'autre avec
le roi de Sicile, tandis que l'Espagne attaquerait le royaume de Naples,
et en chasser ainsi l'empereur. Mais se voyant seul, il n'oubliait rien
pour avoir de grandes forces pour faire craindre l'Espagne, et obtenir
de meilleures conditions quand il faudrait traiter.

Il comptait tellement sur la mauvaise disposition de l'Angleterre, qu'il
voulait, pour premier point préliminaire, que la négociation ne se
traitât point à Londres. Il se flattait qu'il y aurait bientôt des
mouvements considérables en ce pays-là. Il se mit à caresser le
Prétendant, sans toutefois lui donner le plus petit secours, et il lui
fit passer par le cardinal Gualterio l'avis de se marier, comme étant
celui de tous les Anglais, même les plus opposés à son rétablissement,
et la chose la plus agréable à toute l'Angleterre, comme le sachant d'un
homme principal et fort mêlé dans le gouvernement.

Albéroni n'avait laissé que le nom et les places aux conseillers d'État,
qui est ce que nous appelons ici les ministres. Il ne leur communiquait
que des choses indifférentes\,; les secrétaires d'État n'avaient même
aucune part à rien de son entreprise. Il en avait écrit et signé de sa
main tous les ordres. Patiño seul en conduisait l'exécution sous lui. Il
voulait le même secret dans toutes les affaires, et que les ministres
d'Espagne dans les cours étrangères ne rendissent compte qu'à lui tout
seul. Il avait de plus la raison de l'État du roi d'Espagne, accablé de
vapeurs qui le faisaient juger plus mal qu'il n'était. Sa mélancolie
était extrême, et quoique extérieurement soumis à la reine et aux
volontés du cardinal, qui disposait seul en effet de toutes les
affaires, il y en avait néanmoins de particulières, où la mauvaise
humeur du roi éclatait au-dehors assez pour y être connue et remarquée
par les ministres étrangers.

L'abbé del Maro, ambassadeur du roi de Sicile à Madrid, était celui
{[}qui, par{]} la vigilance à être des mieux informés et la pénétration
qu'Albéroni ne pouvait tromper, lui était le plus odieux comme un
surveillant insupportable\,: Il prit aussi un soin particulier de le
décrier dans sa cour, et dans les autres où cet abbé pouvait avoir
quelque relation, et à le faire passer à Rome pour le plus grand fourbe
du monde et le plus grand ennemi du pape.

Il en tirait avantage pour exhorter le pape à la patience, à la
dissimulation, et à se mesurer en sorte qu'il ne le mît pas hors d'état
de lui rendre le moindre service. Il consentait qu'il criât, qu'il se
plaignît de l'Espagne pour contenter les Impériaux, mais à condition
qu'il ne laisserait jamais imprimer le bref qu'il avait écrit au roi
d'Espagne, parce que, s'il le permettait, on ne pourrait plus répondre
d'empêcher les grands désordres qui en arriveraient\,; que c'était pour
les prévenir qu'il avait empêché Aldovrandi de le présenter au roi
d'Espagne, déférence et prudence dont il voulait que le pape louât son
nonce et lui en sût gré. Comme le cardinal jugeait que cette
complaisance d'Aldovrandi exciterait puissamment les Allemands à le
perdre, il protestait au pape que, s'il le rappelait, il pouvait assurer
de voir la nonciature fermée pour longtemps, et le roi d'Espagne marcher
sans mesure avec la hauteur et la dignité qui lui convenait. Il lui
disait que le seul moyen de travailler utilement pour l'un et pour
l'autre était que le roi d'Espagne fût puissamment armé par mer et par
terre. Aussi le cardinal y travaillait-il de toutes ses forces.

Il trouvait inutile d'acquérir pour l'Espagne des partisans à Rome par
des grâces pécuniaires, dont elle ne tirerait nul service, si les
affaires demeuraient en l'état où elles étaient, qui, venant à changer,
on verrait bien des gens principaux de cette cour briguer à genoux la
protection de cette couronne. Il menaçait ceux de cette cour qui
recevaient des grâces de celle de Vienne. Il prétendait que le cardinal
d'Albane en touchait vingt mille écus de pension, que l'empereur l'avait
menacé de lui ôter sur le soupçon du pape avec l'Espagne depuis le
mouvement de ses troupes. Là-dessus, il déclamait contre ce cardinal
neveu qui vendait son honneur et son oncle.

Il avertissait le pape de tenir la balance égale entre le roi d'Espagne
et l'empereur, de l'indignité de se rendre l'esclave des Allemands, en
consentant de retirer qui leur déplairait des emplois, et Acquaviva fut
chargé de déclarer de la part du roi et de la reine d'Espagne que si les
mauvais offices qu'à l'instigation de Gallas Albane rendait
continuellement à Aldovrandi faisaient rappeler ce nonce, on n'en
recevrait point d'autre en sa place, et que la nonciature demeurerait
fermée pendant tout ce pontificat, aussi bien qu'une bonne partie de la
daterie.

Albéroni en effet ne pouvait avoir un nonce plus à sa main, ni plus
souple à ses volontés qu'Aldovrandi. Celui-ci était persuadé de la
nécessité de l'union des deux cours\,; qu'elle ne pouvait subsister
qu'autant qu'il se rendrait agréable à celle où il était envoyé. C'est
ce qui l'avait rendu si docile à remettre les brefs d'indults avant
l'accommodement, et à ne les point retirer contre les ordres positifs du
pape. Le désir de profiter de sa nonciature le fit insister auprès du
pape à ne plus parler de ces deux griefs. Les brefs en dépôt entre les
mains d'Albéroni et du confesseur y étaient en sûreté\,; on n'en pouvait
de plus faire usage que de l'autorité de la nonciature, par conséquent
sans la permission du pape\,; et de plus le roi s'en pouvait passer, en
demandant à son clergé le même don gratuit, qui aimerait mieux se faire
un mérite de l'accorder que d'y être forcé par les bulles.

Ce nonce tâchait de persuader au pape que la conquête de la Sardaigne
pouvait devenir un moyen de paix par les offices commencés de la France
et de l'Angleterre. Il reconnaissait que le roi de Sicile y pouvait
contribuer\,; mais il ne jugeait pas qu'on pût se fier à un prince aussi
capable que lui de faire les mêmes manèges à Vienne et à Madrid.

L'abbé del Maro y paraissait, depuis quelque temps, plus souvent à la
cour, et Albéroni moins aigre à son égard.

Ce changement qui, mal à propos, fit soupçonner quelque négociation
entre les deux cours n'en fit aucun dans l'esprit de cet abbé. Il crut
toujours que le projet d'Albéroni avait été la Sicile\,; que le roi
d'Espagne s'y était opposé\,; que la Sardaigne n'avait été qu'un
amusement pour occuper et ne pas laisser la flotte et les troupes
inutiles.

Le colonel Stanhope arriva cependant à Madrid, où il trouva Bubb,
secrétaire, chargé jusqu'alors des affaires d'Angleterre. Tous deux
virent ensemble Albéroni. Ils l'assurèrent d'abord de l'amitié du roi
d'Angleterre pour le roi d'Espagne, motivèrent après ses plaintes de
l'infraction de la neutralité d'Italie, dirent qu'il espérait que le roi
d'Espagne, acceptant sa médiation, enverrait incessamment un ministre à
Londres pour y travailler à un bon accommodement pour prévenir un
embrasement en Europe\,; ils ajoutèrent qu'en ce cas le roi d'Angleterre
avait les pouvoirs nécessaires pour entamer un traité à des conditions
avantageuses et honorables à l'Espagne et utiles pour assurer le repos
de l'Europe. Albéroni s'emporta d'abord, invectiva contre le traité
d'Utrecht qui, en donnant tant à l'empereur, avait ôté la balance\,; dit
qu'il était contre toute politique et contre l'intérêt général de
permettre que l'empereur se rendît maître de l'Italie, et conclut que le
roi d'Espagne n'entrerait en aucune négociation, et n'enverrait personne
à Londres s'il n'était auparavant informé des conditions qu'on proposait
pour l'accommodement. Les Anglais répondirent qu'il s'expliquait d'une
manière bien opposée à l'avis du régent, qui de concert avec le roi
d'Angleterre avait déjà envoyé l'abbé Dubois à Londres\,; qu'à l'égard
des conditions de l'accommodement ils étaient prêts de les lui
expliquer.

Le cardinal les interrompit, et dit que si leurs propositions
regardaient les successions de Toscane et de Parme, il en était
suffisamment instruit\,; que le roi d'Espagne ne faisait nul cas de
pareilles offres\,; que, si on prenait de telles mesures, il faudrait
que le roi d'Espagne mît une garnison dans Livourne dans le moment que
le traité serait conclu\,; en un mot, qu'il était impossible de rien
déterminer si les puissances de l'Europe ne s'accordaient à diminuer et
à borner le pouvoir excessif de l'empereur. Les Anglais représentèrent
que l'Europe ne s'armerait pas pour dépouiller l'empereur des États
qu'il possédait, que les principales puissances s'étaient obligées à lui
garantir\,; que l'unique moyen d'empêcher qu'il fît de nouveaux progrès
serait de s'obliger nouvellement par un traité à se déclarer contre ce
prince s'il voulait faire quelque entreprise. Ils soutinrent que rien ne
serait plus désagréable au grand-duc que de mettre une garnison dans
Livourne, mais que cette difficulté ne devait pas rompre un traité si
nécessaire à la tranquillité publique.

Malgré ces facilités, les Anglais ne trouvèrent qu'emportements et
fureurs. Albéroni protesta que le roi d'Espagne n'aurait jamais
l'infamie de faire à sa postérité le tort de céder pour rien ses justes
prétentions en Italie\,: qu'il n'y avait ni confiance ni sûreté à
prendre en toutes les garanties du monde, qui n'empêcheraient pas
l'empereur de se saisir de ce qu'il voudrait envahir. La conférence
finit ainsi sans se persuader.

Albéroni, néanmoins, assez satisfait de la modestie des Anglais, en
conclut que le roi d'Angleterre se trouvait embarrassé de s'être trop
engagé, et que l'intérêt du commerce ne permettrait pas à son parlement
de lui fournir de quoi faire la guerre à l'Espagne pour l'empereur. Il
ne doutait pas d'une autre campagne encore en Hongrie\,; il comptait sur
une grande diminution des troupes impériales en ce pays-là, et sur un
grand désordre dans ses finances. Sur ce ruineux fondement il résolut de
répondre, en général, que le roi d'Espagne serait toujours disposé à la
paix quand le traité produirait la sûreté de l'Italie et un juste
équilibre en Europe, et qu'il ne pouvait envoyer à Londres que de
concert avec le régent qui avait offert ses offices, dont il fallait
savoir les sentiments avant de répondre positivement.

Le cardinal avait déjà laissé pénétrer ses mauvaises intentions à
l'égard du roi de Sicile. Del Maro avait remarqué son affectation à
retarder l'accommodement de quelques différends de peu de conséquence
avec ce prince. Il jugea qu'il les réservait peut-être pour servir un
jour de prétextes aux projets qu'il méditait.

Le mauvais état de la santé du roi d'Espagne et sa mélancolie profonde
n'arrêtait point Albéroni. Il insista toujours sur l'impossibilité de
compter sur aucunes garanties par l'exemple de Majorque et de
l'évacuation de la Catalogne. Il en concluait que l'empereur, maître de
l'Italie, le serait de s'emparer des successions de Toscane et de Parme,
et de fomenter encore des troubles dans l'intérieur de l'Espagne\,;
enfin, il déclara que le roi d'Espagne ne consentirait à aucun
accommodement, si l'empereur n'était auparavant dépouillé d'une partie
des États qu'il possédait en Italie, seul moyen d'assurer la balance
absolument nécessaire au bien public de l'Europe. Quand les Anglais
opposaient la parole et la garantie de leur maître, il répondait que la
parole des princes n'avait lieu qu'autant qu'elle n'était pas contraire
au bien public et au bien particulier de leurs peuples, et prétendait
faire voir que rien n'était plus préjudiciable aux intérêts de
l'Angleterre que de faire la guerre à l'Espagne, même que toute alliance
avec l'empereur. Si les Anglais lui représentaient que l'intention de
leur maître n'était pas de porter la guerre en Espagne, mais d'accorder
à l'empereur un secours de vaisseaux pour garder les côtes d'Italie, il
répondait qu'il serait bien singulier de voir l'Angleterre tenir une
armée navale dans la Méditerranée, uniquement pour le service de
l'empereur, et que si les puissances souffraient l'usurpation de Parme
et de Ferrare, que l'empereur projetait, elles commettraient une
indignité dont elles auraient tout lieu de se repentir. Les ministres
d'Espagne au dehors eurent ordre de s'expliquer dans les mêmes
sentiments du premier ministre, avec toute la confiance qu'il faisait
paraître dans les forces de l'Espagne. Il craignait les desseins de
l'Angleterre et les effets de sa partialité pour l'empereur. Les
discours que Stairs tint là-dessus à Paris lui déplurent tellement,
qu'il voulut que les ambassadeurs d'Espagne en France et en Angleterre
s'en plaignissent formellement.

Celui de Hollande à Madrid tenait une conduite très opposée à celle des
Anglais. Il fut le seul des ministres étrangers qui illumina sa maison
pour la prise de Cagliari. Ses démonstrations différentes de joie firent
soupçonner faussement que sa république avait approuvé cette entreprise
sur la Sardaigne. Beretti se flattait d'y trouver beaucoup de penchant
pour l'Espagne, et l'Espagne affectait une grande confiance pour la
république. Cadogan même en marqua obligeamment sa jalousie à Beretti.

Ce dernier prétendait avoir appris du baron de Rensworde qu'en même
temps que la flotte espagnole mit à la voile pour la Sardaigne, le roi
de Sicile envoya secrètement déclarer à Vienne qu'il n'avait point de
part à l'entreprise\,; qu'il avait ordonné à tous les ports de Sicile de
tout refuser aux Espagnols, et qu'il prendrait avec l'empereur tous les
engagements qu'il pourrait désirer, s'il voulait le reconnaître comme
roi de Sicile, et approuver le traité fait et ratifié avec l'empereur
Léopold en 1703. Beretti se faisait un mérite de ces avis, et se paraît
de la confiance intime qu'avaient en lui les principaux membres des
États généraux. Il assurait qu'ils ne permettraient point que l'empereur
se rendît maître de Livourne, et que l'Angleterre même concourrait à
l'empêcher. Il se fondait en raisonnements pour persuader en Espagne que
les Hollandais craignaient qu'on traitât à Londres, et voulaient que ce
fût dans une ville de leur État. Il inférait de la route que
Penterrieder devait prendre pour se rendre à Londres en évitant la Haye,
malgré les instances de la république, que l'empereur craignait la
partialité des Hollandais, et que les Anglais voulaient se réserver à
eux seuls la négociation, et n'en donner connaissance aux Hollandais que
lorsque toutes choses seraient absolument réglées. Il se trompait en
tous points. Ceux qui étaient au timon de la république étaient
dépendants de l'Angleterre et n'osaient s'écarter de ses intérêts. Il
était donc nécessaire qu'elle agît de concert avec l'Angleterre. Cadogan
en était bien persuadé, et il attendait Penterrieder à la Haye, qu'il
avait prié d'y passer.

L'abbé Dubois prévint par son arrivée à Londres celle de Penterrieder.
Il y guérit les ministres de la crainte qu'ils avaient conçue que le
maréchal d'Huxelles ne fût contraire à la négociation qui s'allait
commencer. Le roi d'Angleterre et ses ministres ne cessaient d'assurer
Monteléon d'une amitié et d'une correspondance entière avec le roi
d'Espagne, et que la négociation tournerait à sa satisfaction, et cet
ambassadeur s'en flatta plus encore sur la réponse du roi d'Angleterre à
l'envoyé de l'empereur. Ce ministre, en prenant congé de lui, insista
sur la garantie, et lui demanda pressement et avec hauteur s'il voulait
ou non satisfaire aux traités et donner à l'empereur les secours de
troupes et de vaisseaux nécessaires pour conserver les États qu'il
possédait en Italie. À quoi le roi d'Angleterre répondit qu'en l'état où
se trouvaient les affaires générales de l'Europe, il avait besoin de
plus de temps pour faire ses réflexions, avant de prendre une
résolution.

D'autre part, l'abbé Dubois assurait Monteléon d'une manière qui lui
paraissait sincère que ses ordres du régent étaient très positifs en
faveur de l'Espagne, dont il regardait les intérêts comme inséparables
de ceux de la France, et l'avait expressément chargé d'y veiller avec
une égale attention. Néanmoins Monteléon cherchait à pénétrer s'il
disait bien vrai, et si sa mission ne regardait que la négociation qui
paraissait, et peut-être en même temps pacifier les troubles du nord, ou
s'il y avait quelque cause secrète et quelque mystère plus profond. La
confiance qui paraissait entre Stanhope et lui donnait celle de pénétrer
ce qu'il en était, parce que Stanhope était le principal acteur. Mais,
jusqu'à l'arrivée de Penterrieder, il ne se pouvait agir que de propos
généraux. En attendant, Monteléon vantait en Espagne ses services et ses
soins, et au duc de Parme, qu'il y croyait tout-puissant, l'attention
qu'il apportait à ses intérêts, les conseils qu'il donnait en sa faveur
à l'abbé Dubois et les démarches qu'il continuerait de faire en sa
faveur, sans que l'Espagne fût instruite de tout ce qu'il faisait à cet
égard.

On croyait à Londres que le duc de Parme et les autres princes d'Italie
désiraient la conclusion du traité qui allait s'y négocier, pour éviter
la guerre dont ils étaient menacés\,; mais quelques-uns d'eux, qui
étaient dans la confiance d'Angleterre, doutaient de la sincère
intention de la France, non du régent, dont ils croyaient l'intérêt
d'être uni au roi d'Angleterre, mais d'une puissante cabale, fort
contraire au régent, et fort attachée au roi d'Espagne.

Bernsdorff, celui des ministres hanovriens qui avait le plus de crédit
auprès du roi d'Angleterre, et Robeton, réfugié français, imaginèrent et
prièrent Stairs de conseiller au régent de choisir cinq ou six bonnes
têtes dévouées à ses intérêts, de se conduire par leur conseil, de les
prendre parmi les évêques et les ecclésiastiques réputés jansénistes, où
il y avait des gens habiles\,; qu'étant sans famille, ils seraient plus
hardis que les laïques, et qu'ennemis des jésuites, ils tiendraient tête
à cette canaille, auteurs de libelles répandus contre lui, en dernier
lieu de celui de la \emph{Gazette de Rotterdam} très certainement sorti
de leur boutique. Ces deux hommes attribuaient à cette cabale d'avoir
mis et de soutenir en place Châteauneuf en Hollande, Rottembourg en
Prusse, Poussin à Hambourg, le comte de La Marck en Suède, Bonac à
Constantinople. Ce dernier, disaient-ils, instruit par la cabale, avait
proposé une alliance entre la Suède et la Porte pour continuer la guerre
en Hongrie, et averti Ragotzi de ce qu'il devait faire pour détourner
les Turcs de faire la paix. Châteauneuf menaçait les Hollandais du
ressentiment de la France, s'ils accordaient à leurs sujets des lettres
de représailles contre les Suédois. La Marck travaillait à une paix
particulière contre le czar et la Suède, avec un zèle et une partialité
extrême pour celle-ci, tandis qu'elle se plaignait amèrement de
l'engagement pris par la France de lui refuser tout subside et tout
autre secours après le terme expiré porté par le dernier traité
d'alliance. Rottembourg était l'entremetteur d'une négociation secrète
entre Ilghen, ministre du roi de Prusse et Goertz, ministre de Suède,
auquel il avait offert de le tenir caché dans sa maison. On voit avec
surprise et avec quelque chose de plus, jusqu'à quel point l'intérêt et
le crédit de l'abbé Dubois et celui de ses croupiers pour le leur à lui
plaire, jusqu'à quel point, dis-je, se portait la hardiesse des Anglais
dans l'intérieur du régent, d'essayer de lui donner un nouveau conseil à
leur gré, et de lui faire changer tous les ministres au dehors,
c'est-à-dire de faire de ce prince leur vice-roi en France, et d'y
montrer à tous les François qu'aucun ne pouvait espérer aucune de toutes
les places du gouvernement au dedans, ni au dehors, ni de conserver dans
aucune que par leur choix ou par leur permission. Les imputations faites
à ceux du dehors portaient encore cette hardiesse au delà de tout ce qui
se pouvait comprendre.

Quelque insensée que fût l'entreprise d'Albéroni sans alliés, le
fourreau était jeté, et il était véritable que, si contre toute
apparence, elle eût pu réussir, il était de l'intérêt de la France que
l'empereur devînt moins puissant en Italie, et que l'Espagne s'y accrût
de partie de ce qu'elle y avait perdu. Bonac servait donc utilement de
chercher les moyens de prolonger la guerre de Hongrie, de laquelle
uniquement l'Espagne pouvait espérer des succès en Italie, et d'obliger
l'empereur à se prêter à des conditions de paix désirables.

À l'égard de la Suède, il n'y avait que le désespoir de la pouvoir
rétablir, aussi démontré qu'il l'était alors, qui pût faire cesser les
efforts de la France en faveur d'un ancien allié, dont la descente en
Allemagne avait été la première borne de l'énorme vol que la puissance
de la maison d'Autriche avait pris en Europe, et que les possessions
demeurées en Allemagne à la Suède avaient sans cesse empêchées de
reprendre. Le rétablissement de cette couronne devait donc être
infiniment cher à la France, si, dans la ruine des malheurs de Charles
XII, elle avait pu la procurer. À ce défaut, l'intérêt de la France, qui
l'empêchait de se commettre seule avec toutes les puissances conjurées
contre la Suède, était de procurer avec adresse et sagesse une paix qui
sauvât à la Suède tout ce qu'il serait possible de ses débris pour la
laisser respirer, et en situation d'oser songer à se rétablir un jour
dans l'état d'où elle était déchue. C'est ce qui ne se pouvait espérer
qu'en travaillant à des paix particulières qui rompissent la ligue qui
l'accablait, qui en missent, s'il était possible, les membres aux mains
les uns contre les autres, qui intéressassent contre les opiniâtres ceux
qui auraient fait leur paix particulière, à soutenir la Suède contre
eux, et par ce moyen lui sauver enfin des provinces en Allemagne qui lui
laissassent un pied dans l'empire, une voix dans les diètes, et les
occasions d'y contracter des alliances et d'y figurer encore, de
cheminer vers son rétablissement, et d'y balancer à la fin la puissance
de la maison d'Autriche, et la grandeur naissante de la maison de
Hanovre.

Ainsi le comte de La Marck et Rottembourg servaient très utilement
l'état de travailler à séparer et à brouiller cette ligue du nord, si
utile aux vues et à la puissance de l'empereur et de la maison de
Hanovre, qui était si occupée de se conserver ses usurpations de Brême
et de Verden sur la Suède\,; et ces ministres ne pouvaient mieux s'y
prendre qu'en procurant à la Suède des paix particulières. Châteauneuf
aussi avait grande raison d'empêcher, tant qu'il pouvait, la Hollande de
se joindre aux ennemis de la Suède, en troublant et infestant le peu de
commerce qui lui restait. On ne peut donc assez admirer que l'Angleterre
osât vouloir, à visage découvert et sous prétexte d'avis d'amitié,
tourner la France à un intérêt si contradictoire à ceux de cette
couronne, tonneler\footnote{Terme de chasse\,: prendre piège. (Y. B.)}
le régent en l'effrayant de cabales, et l'obliger à se défaire de ceux
qui servaient le mieux les vrais intérêts de leur maître, pour leur en
substituer d'autres qui ne prendraient ordres ni instructions que des
ministres de Georges, comme on l'a vu depuis pratiquer à découvert après
que l'abbé Dubois eut totalement subjugué le régent et par lui tout le
royaume.

La paix du nord, sans l'intervention de Georges, aurait été l'événement
qui l'aurait le plus sensiblement touché. Il comptait les intérêts et
son établissement sur le trône d'Angleterre sujet aux caprices et aux
révolutions pour rien en comparaison de ses États d'Allemagne et de leur
agrandissement. Le czar désirait sa paix particulière avec la Suède par
les avantages qu'il y trouvait, et par la difficulté pécuniaire d'en
soutenir plus longtemps la guerre. La base du traité était le
rétablissement de Stanislas, de s'emparer de Dantzig, d'y mettre des
troupes moscovites et de l'y faire régner pendant la vie de l'électeur
de Saxe, dont il aurait été le successeur à la couronne de Pologne,
moyennant quoi le czar espérait faire beaucoup relâcher le roi de Suède
sur les conditions de sa paix.

Le roi de Prusse entrait dans ce projet\,; mais, se défiant du czar, il
traitait séparément avec la Suède. Il y eut divers projets proposés à
Berlin pendant le séjour que le baron de Goertz, ministre confident du
roi de Suède, fit en cette ville. Quoique le traité ne fût qu'entre la
Suède et la Prusse, ce dernier prince affectait de veiller aux intérêts
du czar, son allié. Goertz offrit de laisser au czar Pétersbourg, une
lisière des deux côtés du golfe de Finlande avec tous les ports et
havres qui en dépendaient en l'état qu'ils se trouvaient alors, et
promettait sur Revel qu'on trouverait des expédients pour aplanir la
difficulté de cet article. La cession de Stettin et de son district
était ce qui touchait le plus le roi de Prusse. Goertz disait qu'il
n'avait pouvoir d'y consentir qu'à condition que le roi de Prusse
promettrait en même temps la restitution de toutes les conquêtes de ses
alliés, excepté Pétersbourg. C'était un engagement qu'il était
impossible que le roi de Prusse pût prendre. Le czar avait déclaré
positivement qu'à l'exception de la Finlande, il ne restituerait
absolument rien. Il avait particulièrement dit qu'il voulait la Livonie
et qu'il ferait la guerre plutôt vingt ans encore que de changer quoi
que ce soit à la résolution qu'il avait prise.

Goertz augmenta les difficultés en déclarant qu'il ne ferait pas un seul
pas dans la négociation si la démolition des fortifications de Wismar
n'était suspendue. Le roi de Prusse, qui le connaissait bien, lui fit
offrir cent mille écus\,; mais pour cette fois ce moyen si sûr avec lui
ne réussit pas, et on jugea que Goertz avait pris ailleurs des
engagements dont il croyait tirer davantage\,; le soupçon fut que
c'était avec le roi de Pologne. En effet, Goertz demeurait en Pologne
pendant la négociation\,; il refusa de se trouver à une conférence avec
les ministres de Russie et de Prusse, qui devait se tenir près de
Berlin. Il partit brusquement sans dire adieu, sans avertir, sans
déclarer où il voulait aller, se rendit à Breslau, terre fort suspecte
dans ses conjonctures pour le roi de Prusse et pour ses alliés, parce
que le roi d'Angleterre avait averti l'empereur que le czar avait offert
à la France d'attaquer les États héréditaires de la maison d'Autriche,
si le régent voulait donner des subsides pour entreprendre et pour
soutenir cette guerre, qui auraient été bien mieux employés que ceux que
l'abbé Dubois lui fit donner bientôt après contre l'Espagne. L'avis
ajoutait que la proposition s'était faite avant la prise de Belgrade qui
avait fait changer de ton au czar. Mais c'en était bien plus qu'il ne
fallait pour le rendre suspect à Vienne, et pour faire craindre à ce
prince et à ses alliés que cette cour ne fût informée de l'état de la
négociation entamée pour la paix du nord.

Le roi de Prusse, irrité de l'infidélité de Goertz, ne songea plus qu'à
se lier plus intimement avec le czar. Il résolut d'envoyer un ministre
aux conférences qu'on parlait de tenir aux environs de Pétersbourg, où
le czar et lui souhaitèrent également qu'il n'y vînt personne de la part
de la France, qui traverserait sûrement le traité si le roi d'Angleterre
n'y était compris, avec lequel elle s'était si étroitement liée, et
qu'ils accusaient sans doute de l'avoir averti des propositions que le
czar lui avait faites, qu'on vient de voir être allées par Georges
jusqu'à l'empereur\,: autre ouvrage de l'abbé Dubois, si le fait était
vrai.

Toutefois, il n'y avait pas lors un mois que le roi de Prusse avait
exhorté le régent à penser sérieusement à former un parti dans l'empire
capable de borner l'autorité de l'empereur\,; il avait offert d'y donner
ses soins et ses offices\,; il se disait sûr du landgrave de Hesse et du
duc de Wurtemberg\,; il travaillait à s'unir plus étroitement au duc de
Mecklembourg qui avait dix ou douze mille hommes\,; il espérait d'y
attirer les Hollandais qui voulaient traiter avec lui\,; il demandait à
la France de travailler à une harmonie parfaite entre le roi
d'Angleterre et lui, chose bien contradictoire à tout le reste. Le
landgrave était fort lié avec Georges, de l'appui duquel, en Hollande,
il espérait procurer au prince de Nassau, gouverneur de Frise, son
petit-fils, la charge de stathouder des Sept-Provinces, et celle de
capitaine général au prince Guillaume son fils. Le roi de Prusse
attribuait le défaut d'intelligence entre le roi son beau-père et lui à
l'intérêt particulier de Bernsdorff, et croyait que l'abbé Dubois
pourrait terminer ces difficultés particulières\,; mais la base de tout
ce projet était la fin de la guerre du nord ou de celle en particulier
du roi de Prusse, pendant la durée de laquelle il ne pouvait rien
entreprendre, et se trouvait obligé de ménager l'empereur.

Il n'était pas aisé de faire revenir la cour de Londres sur le roi de
Prusse, dont la légèreté et le peu de fidélité ne permettaient pas de
compter sur lui avant que les mesures projetées entre l'empereur et la
France fussent réglées, et les Anglais mêmes se plaignaient de
Rottembourg comme dévoué à Ilghen et à la cour de Berlin. Ils étaient
fort attentifs à la négociation commencée entre la Suède et le czar
qu'ils croyaient en désir d'une paix avantageuse en abandonnant ses
alliés, et qui, haïssant le roi Georges et parlant de lui sans mesure,
pourrait former une liaison intime avec la Suède, et faire dans leur
traité une condition principale de soutenir les droits du Prétendant et
de concourir à son rétablissement. Ces considérations vivement imprimées
dans l'esprit des ministres Anglais attachés à Georges, leur faisaient
sentir la nécessité de lui attacher les principales puissances de
l'Europe, pour s'en assurer contre de nouvelles entreprises de ce
malheureux prince, et pour cela même l'importance de procurer par sa
médiation la paix entre l'Espagne et l'empereur que, comme chef de
l'empire où Georges avait ses plus précieux États, il avait plus besoin
d'obliger et de s'acquérir. C'est ce qui avait engagé ces ministres
Anglais à ne rien omettre pour engager cette négociation à Londres.

Penterrieder y arriva à la fin d'octobre, fort content des dispositions
qu'il croyait avoir remarquées à la Haye d'entrer sans peine dans toutes
les mesures que la France et l'Angleterre jugeraient nécessaires pour
affermir le repos de l'Europe. Cadogan, qui connaissait mieux que
Penterrieder les Hollandais, desquels il avait un long usage, n'en
jugeait pas si favorablement que lui\,; il comptait bien sur leur
principe d'intelligence parfaite avec l'Angleterre, et d'entrer autant
qu'il serait possible dans les mêmes alliances. Mais, quoique tous les
particuliers convinssent en cela, ils différaient dans les voies pour
arriver au même but. La Hollande, comme les autres pays, était partagée
en partis, en divisions et en subdivisions, et ces différents sentiments
se portaient aux États généraux. Cadogan remarquait aussi que
Châteauneuf, plus fidèle à ses anciens préjugés qu'à ses derniers
ordres, travaillait à détruire plutôt qu'à fortifier la confiance entre
l'Angleterre et la Hollande. Beretti, mal instruit des démarches de
Penterrieder à la Haye, crut qu'il n'avait traité d'affaires qu'avec
Cadogan et Widword, et qu'il s'était contenté de se plaindre aux. États
généraux de l'entreprise de l'Espagne en termes fort aigres et fort
hauts, que les Anglais avaient approuvés\,; sur quoi il s'étendit en
grands raisonnements en Espagne sur la partialité de Georges et de ses
ministres pour l'empereur, à qui Cadogan avait un ancien attachement
personnel, et sur la sagesse de la résolution de ne point traiter à
Londres, mais à la Haye, où la partialité pour l'empereur serait
infiniment moins dangereuse. Cadogan n'avait point caché à Beretti que
le roi d'Angleterre travaillait fortement à la paix de l'empereur avec
l'Espagne, ni les conditions qui en étaient le fondement.

Elles étaient que l'empereur reconnaîtrait Philippe V roi des Espagnes
et des Indes\,; qu'il donnerait à un des fils de son second mariage
l'investiture des États de Toscane et de Parme quand les successions en
seraient ouvertes\,; que les mesures seraient si bien prises que la
Toscane n'appartiendrait jamais à l'empereur, ni spécialement
Livourne\,; moyennant quoi l'Espagne y trouverait ses avantages, les
princes d'Italie leur sûreté, l'équilibre serait conservé, et la
tranquillité publique.

Cadogan, loin d'en demander le secret, dit à Beretti que le régent avait
chargé le duc de Saint-Aignan de communiquer ce projet en Espagne, et
qu'étant avantageux, il y avait lieu d'en espérer des réponses
favorables, dont dépendait tout le succès de la négociation. Beretti en
jugea de même, mais il n'osa s'en déclarer, en attendant d'être informé
des sentiments d'Albéroni. Ce cardinal, comme on l'a vu, s'était offensé
des propos que Stairs avait tenus sur l'entreprise de Sardaigne. Il s'en
était plaint en forme par un mémoire qu'il remit au secrétaire
d'Angleterre. Stairs, à son tour, se plaignit de la vivacité du mémoire.
Cellamare, sans ordre, mais dans l'opinion du grand crédit de Stairs
auprès du régent, s'entremit pour le calmer. Stanhope écrivit là-dessus
à Albéroni d'une manière respectueuse pour le roi d'Espagne, tendre pour
lui, par laquelle il l'assurait que l'Angleterre ne donnerait jamais de
secours à l'empereur pour faire la guerre à l'Espagne. Le cardinal goba
l'équivoque, triompha, brava, et s'engoua de ses idées plus que jamais.

Parmi tous ces soins, le roi d'Espagne tomba véritablement et
dangereusement malade. Albéroni avait eu grand soin de le conserver dans
l'habitude que M\textsuperscript{me} des Ursins lui avait fait prendre
d'être continuellement enfermé avec la reine et elle, et de le rendre
inaccessible, non seulement à sa cour et aux seigneurs les plus
distingués, mais à ceux même dont les charges étaient les plus
intérieures. C'était par là qu'elle s'était mise seule en possession du
gouvernement de l'État et de disposer de toutes les affaires et de
toutes les grâces. Albéroni, qui en avait été témoin du temps de M. de
Vendôme et depuis sa mort, comme envoyé de Parme, et de cette sorte de
prison du roi, encore plus resserrée depuis la mort de la reine, où il
ne voyait que la princesse des Ursins avec qui il passait sa vie
perpétuellement enfermé, profita de la leçon pour la nouvelle reine et
pour sa propre fortune. Comme l'habitude était prise, il n'eut pas de
peine à la faire continuer\,; mais il resserra le roi bien plus
étroitement qu'il ne l'avait été du temps de la première reine, dont
l'habitude a duré autant que la vie de Philippe V. C'est un détail que
j'aurai lieu de faire à l'occasion de mon ambassade, si Dieu permet que
j'achève ces Mémoires. Je me contenterai de dire ici ce qui fait à la
matière présente.

Qui que ce soit n'approchait de l'intérieur indispensable du roi
d'Espagne, c'est-à-dire lever, coucher, repas\,; car cet intérieur
nécessaire se bornait à trois ou quatre valets français et deux seuls
gentilshommes de la chambre\,; aucun ministre qu'Albéroni, le confesseur
un quart d'heure tous les matins à la suite du lever, le duc de Popoli
et les autres gouverneurs ou sous-gouverneurs des infants à leur suite,
mais un quart d'heure à la toilette de la reine le matin, où le roi
allait après avoir congédié son confesseur\,; le cardinal Borgia,
patriarche des Indes, rarement le marquis de Villena, majordome-major,
les deux gentilshommes de la chambre, seuls en exercice\,; les mêmes,
excepté les infants et leurs gouverneurs, pouvaient entrer au dîner et
au souper sans y rester longtemps. Les soirs, les infants et leurs
gouverneurs venaient voir le roi et la reine seuls\,; leur visite ne
durait qu'un moment. Les premiers médecin, chirurgien et apothicaire
avaient ces mêmes entrées dont, à l'exception du lever, ils usaient
sobrement. De femmes, la nourrice seule voyait la reine au lit quand le
roi en sortait, et la chaussait. C'était là le seul moment qu'elle eût
seule avec elle, qui s'allongeait tant qu'on pouvait, à la mesure de
l'habiller du roi, qui se faisait dans une pièce joignante. La reine
passait à sa toilette, où elle trouvait la camarera-mayor, trois ou
quatre dames du palais, autant de \emph{señoras de honor}, et quelques
femmes de chambre. À dîner et à souper, la camarera-mayor, deux dames du
palais de jour et deux \emph{señoras de honor} de jour servaient, et les
femmes de chambre apportaient de la porte les plats et à boire, et les y
rendaient aux officiers. La bouche du roi ne lui préparait rien et était
absolument inutile. Il n'était servi que de celle de la reine. Le
majordome-major était donc exclu, ainsi que le sommelier du corps, qui
est de tous les grands officiers le plus intérieur, et tous les
gentilshommes de la chambre, dont il y a une vingtaine, desquels
auparavant deux étaient de service par semaine tour à tour. Ainsi le
service intérieur était réduit à ce très court nombre de valets et
d'officiers de santé, aux deux gentilshommes de la chambre seuls,
toujours en fonction, et au majordome-major de la reine, qui était aussi
l'un de ces deux gentilshommes de la chambre toujours en service, à ce
peu de dames de la reine tour à tour et à ces deux ou trois autres que
j'ai nommées, qui, sans service, entraient quelquefois à la toilette ou
au dîner\,: le duc d'Escalope, qu'on appelait toujours marquis de
Villena, était majordome-major du roi, un des plus grands seigneurs
d'Espagne en tout genre, et le plus respecté et révéré de tous, avec
grande raison, par sa vertu, ses emplois et ses services. J'en ai parlé
t. III, p.~333, et t. VI, p.~103, et de son fils aîné, le comte de San
Estavan de Gormaz, grand d'Espagne aussi, gendre de la camarera-mayor,
et premier capitaine des gardes du corps alors.

La maladie du roi fit réduire ce court intérieur dont je viens de
parler, à la reine unique de femme et à sa nourrice, aux deux
gentilshommes de la chambre toujours en service, aux officiers de santé,
qui n'étaient que quatre parce que le premier médecin de la reine y fut
admis, et aux quatre ou cinq valets intérieurs, Albéroni sur le tout. Le
reste sans exception fut exclus\,; le P. Daubenton même n'y était admis
qu'avec discrétion.

La médecine du roi est tout entière sous la charge de son
majordome-major. Elle lui doit rendre compte de tout, il doit être
présent à toutes les consultations, et le roi ne doit prendre aucun
remède qu'il ne sache, qu'il n'approuve et qu'il ne soit présent.
Villena voulut faire sa charge. Albéroni lui fit insinuer que le roi
voulait être en liberté, et qu'il ferait mieux sa cour de se tenir chez
lui, ou d'avoir la discrétion et la complaisance de ne point entrer où
il était, et d'apprendre de ses nouvelles à la porte. Ce fut un langage
que le marquis ne voulut point entendre.

On avait tendu au fond du grand cabinet des Miroirs un lit en face de la
porte où on avait mis le roi, et comme la pièce est vaste et longue, il
y a loin de cette porte, qui donne dans l'extérieur, jusqu'au fond où
était le lit. Albéroni fit encore avertir le marquis que ses soins
importunaient, qui ne laissa pas d'entrer toujours. À la fin, de concert
avec la reine, le cardinal résolut de lui fermer la porte. Le marquis
s'y étant présenté une après-dînée, un de ces valets intérieurs
l'entrebâilla et lui dit avec beaucoup d'embarras qu'il lui était
défendu de le laisser entrer. «\,Vous êtes un insolent, répondit le
marquis, cela ne peut pas être\,;» pousse la porte sur le valet et
entre. Il eut en face la reine, assise au chevet du lit du roi. Le
cardinal, debout auprès d'elle, et ce peu d'admis qui n'y étaient pas
même tous, fort éloignés du lit. Le marquis, qui était avec beaucoup de
gloire fort mal sur ses jambes, comme on l'a vu dans ce que j'ai dit de
lui, s'avance à petits pas, appuyé sur son petit bâton. La reine et le
cardinal le voient et se regardent. Le roi était trop mal pour prendre
garde à rien, et ses rideaux étaient fermés, excepté du côté où était la
reine. Voyant approcher le marquis, le cardinal fit signe avec
impatience à un des valets de lui dire de s'en aller, et tout de suite,
voyant que le marquis sans répondre avançait toujours, il alla à lui, et
lui remontra que le roi voulait être seul et le priait de s'en aller.
«\,Cela n'est pas vrai, lui dit le marquis, je vous ai toujours regardé,
vous ne vous êtes point approché du lit, et le roi ne vous a rien
dit.\,» Le cardinal, insistant et ne réussissant pas, le prit par le
bras pour le faire retourner. Le marquis lui dit qu'il était bien
insolent de vouloir l'empêcher de voir le roi et de faire sa charge. Le
cardinal, plus fort que lui, le retourna, l'entraînant vers la porte, et
se disant mots nouveaux, toutefois le cardinal avec mesure, mais le
marquis ne l'épargnant pas. Lassé d'être tiraillé de la sorte, il se
débattit, lui dit qu'il n'était qu'un petit faquin, à qui il saurait
apprendre le respect qu'il lui devait\,; et dans cette chaleur et cette
pousserie le marquis, qui était faible, tombe heureusement dans un
fauteuil qui se trouva là. De colère de sa chute il lève son petit bâton
et le laisse tomber de toute sa force dru et menu sur les oreilles et
sur les épaules du cardinal, en l'appelant petit coquin, petit faquin,
petit impudent qui ne méritait que les étrivières. Le cardinal, qu'il
tenait d'une main à son tour, s'en débarrassa comme il put et s'éloigna,
le marquis continuant tout haut ses injures, le menaçant avec son bâton.
Un des valets vint lui aider à se lever du fauteuil, à gagner la porte,
car, après cette expédition, il ne songea plus qu'à s'en aller. La reine
regarda de son siège toute cette aventure en plein, sans branler ni mot
dire\,; et le peu qui était dans la chambre, sans oser remuer. Je l'ai
su de tout le monde en Espagne, et de plus j'en ai demandé l'histoire et
tout le plus exact détail au marquis de Villena, qui était la droiture
et la vérité même, qui avait pris de l'amitié pour moi, et qui me l'a
contée avec plaisir toute telle que je l'écris. Santa Cruz et l'Arco,
les deux gentilshommes de la chambre, qui me l'ont aussi contée, riaient
sous cape. Le premier avait refusé de lui aller dire de sortir\,; et
après l'accompagnèrent à la porte. Le rare est que le cardinal, furieux,
mais saisi de la dernière surprise des coups de bâton, ne se défendit
point, et ne songea qu'à se dépêtrer. Le marquis lui cria de loin que,
sans le respect du roi et de l'état où il était, il lui donnerait cent
coups de pied dans le ventre et le mettrait dehors par les oreilles.
J'oubliais encore cela. Le roi était si mal qu'il ne s'aperçut de rien.

Un quart d'heure après que le marquis fut rentré chez lui, il reçut un
ordre de se rendre en une de ses terres à trente lieues de Madrid. Le
reste du jour sa maison ne désemplit pas de tout ce qu'il {[}y{]} avait
de plus considérable à Madrid, à mesure qu'on apprenait l'aventure, qui
fit un furieux bruit. Il partit le lendemain avec ses enfants. Le
cardinal toutefois demeura si effrayé que, content de l'exil du marquis
et de s'en être défait, il n'osa passer aux censures pour en avoir été
frappé. Cinq ou six mois après il lui envoya ordre de revenir, sans
qu'il en eût fait la plus légère démarche. L'incroyable est que
l'aventure, l'exil, le retour ont été entièrement ignorés du roi
d'Espagne jusqu'à la chute du cardinal. Le marquis n'a jamais voulu le
voir ni ouïr parler de lui, pour quoi que ce pût être, depuis qu'il fut
revenu, quoique le cardinal fût absolument le maître, dont l'orgueil fut
fort humilié de cette digne et juste hauteur, et d'autant plus piqué
qu'il n'oublia rien pour se replâtrer avec lui, sans autre succès qu'en
recueillir les mépris, qui accrurent beaucoup encore la considération
publique où était ce sage et vertueux seigneur.

Le roi fut assez mal pour faire son testament, dicté par le cardinal et
concerté avec la reine. Personne n'en eut connaissance et ne douta que
la régence et toute autorité ne lui fût donnée, avec le cardinal pour
conseil. Tout fut en trouble, et peu de gens étaient persuadés que la
régence d'une belle-mère du successeur fût reconnue si le roi venait à
mourir, et une belle-mère aussi haïe que celle-là l'était de toute
l'Espagne, et qui n'avait d'appuis que le duc de Parme et Albéroni si
parfaitement détesté.

\hypertarget{chapitre-viii.}{%
\chapter{CHAPITRE VIII.}\label{chapitre-viii.}}

1717

~

{\textsc{Opiniâtreté d'Albéroni contre la paix.}} {\textsc{- Le pape
fait imprimer son bref injurieux au roi d'Espagne, qu'Aldovrandi n'avait
osé lui présenter.}} {\textsc{- Ce nonce fait recevoir la constitution
aux évêques d'Espagne.}} {\textsc{- Anecdote différée.}} {\textsc{-
Servitude du pape pour l'empereur, qui le méprise et fait Czaki
cardinal.}} {\textsc{- Le pape fait arrêter le comte de Peterborough\,;
et, menacé par les Anglais, le relâche avec force excuses.}} {\textsc{-
Sa frayeur, et celle du duc de Parme, de l'empereur.}} {\textsc{-
Conseils furieux et fous contre la France de Bentivoglio au pape.}}
{\textsc{- Son extrême embarras entre l'empereur et l'Espagne.}}
{\textsc{- Ses tremblantes mesures.}} {\textsc{- Le pape avoue son
impuissance pour la paix.}} {\textsc{- Avis à l'Espagne et raisonnements
sur Naples.}} {\textsc{- Mesures militaires d'Albéroni, et sur la paix
qu'il ne veut point.}} {\textsc{- Mystère du testament du roi
d'Espagne.}} {\textsc{- Faiblesse d'esprit du roi d'Espagne guéri.}}
{\textsc{- Vanteries des forces d'Espagne, et conduite d'Albéroni.}}
{\textsc{- Ses mesures.}} {\textsc{- L'Angleterre arme une escadre.}}
{\textsc{- Forts propos entre le duc de Saint-Aignan et Albéroni.}}
{\textsc{- Chimères de ce cardinal.}} {\textsc{- Riperda, tout à
Albéroni, tient à del Maro d'étranges propos.}} {\textsc{- Dons faits au
cardinal Albéroni, qui est nommé à l'évêché de Malaga, puis à
l'archevêché de Séville.}} {\textsc{- Il montre à del Maro son
éloignement de la paix, qui en avertit le roi de Sicile.}} {\textsc{- Le
cardinalat prédit à Albéroni.}} {\textsc{- Aldovrandi, pensant bien
faire d'engager les prélats d'Espagne d'accepter la constitution, est
tancé avec ordre de détruire cet ouvrage comme contraire à
l'infaillibilité.}} {\textsc{- Aldovrandi fort malmené.}} {\textsc{-
Griefs du pape contre lui.}} {\textsc{- Demandes énormes de l'empereur
au pape.}} {\textsc{- Hauteur incroyable de l'empereur avec le pape qui
tremble devant lui et qui est pressé par l'Espagne.}} {\textsc{-
Reproches entre le cardinal Acquaviva et le prélat Alamanni de la part
du pape.}} {\textsc{- Mouvements inutiles dans le royaume de Naples.}}
{\textsc{- Soupçons sur le roi de Sicile, qui envoie le comte de Provane
à Paris.}} {\textsc{- Le duc de Modène n'ose donner sa fille au
Prétendant qui est pressé de tous côtés de se marier.}} {\textsc{- Les
neveux du pape vendus à l'empereur.}} {\textsc{- Faiblesse entière du
pape pour le cardinal Albane, sans l'aimer ni l'estimer.}} {\textsc{-
Crainte de ce neveu à l'égard d'Aldovrandi.}} {\textsc{- Gallas et
Acquaviva également bien informés par l'intérieur du palais du pape.}}
{\textsc{- {[}Le pape{]} veut se mêler de la paix entre l'empereur et
l'Espagne.}} {\textsc{- Hauteur et menaces des Impériaux sur la paix,
qui déplaisent eu Hollande.}} {\textsc{- Manèges intéressés de
Beretti.}} {\textsc{- Friponnerie de l'abbé Dubois.}} {\textsc{- Manèges
intéressés de Monteléon, qui compte sur Chavigny, amené par l'abbé
Dubois à Londres, et en est trompé.}} {\textsc{- Inquiétude chimérique
des Anglais d'un mariage du prince de Piémont avec une fille du
régent.}} {\textsc{- Même inquiétude, et personnelle, de La Pérouse.}}
{\textsc{- Il apprend de Penterrieder que l'empereur veut absolument la
Sicile, avec force propos hauts et caressants.}} {\textsc{- Il l'assure
de tout l'éloignement de la France pour le roi de Sicile.}} {\textsc{-
Court voyage de l'abbé Dubois à Paris.}} {\textsc{- Cajoleries du roi
d'Angleterre à la reine d'Espagne et à Albéroni, en cas de mort du roi
d'Espagne.}} {\textsc{- Proposition du roi d'Espagne pour entrer en
traité avec l'empereur par l'Angleterre.}} {\textsc{- Manège des
ministres du roi d'Angleterre.}} {\textsc{- Ils n'ont point de secret
pour Penterrieder.}} {\textsc{- Résolution du régent sur le traité,
mandée par l'abbé Dubois en Angleterre.}} {\textsc{- Inquiétude des
ministres de Sicile à Londres et à Paris.}} {\textsc{- Éclat entre le
roi d'Angleterre et le prince de Galles.}} {\textsc{- Manège et embarras
de La Pérouse.}} {\textsc{- L'Angleterre arme doucement une escadre pour
la Méditerranée.}} {\textsc{- Plaintes de Monteléon.}} {\textsc{-
Réponse honnête, mais claire, des ministres anglais.}} {\textsc{-
Chimère imaginée par les ennemis du régent, qu'il voulait obtenir de
l'empereur la succession de la Toscane pour M. son fils.}} {\textsc{-
Beretti, trompé par de faux avis, compte avec grande complaisance sur la
Hollande, dont il écrit merveilles en Espagne, et de la partialité
impériale des Anglais.}}

~

Au milieu de ces confusions et du péril où était la vie du roi
d'Espagne, le cardinal déclara qu'il ne trouvait pas les propositions
des Anglais suffisantes pour assurer le repos de l'Italie, et qu'il
n'enverrait point de ministre à Londres. Il dit à ses amis qu'il ne se
laisserait point endormir par des négociations apparentes\,; qu'il avait
tout l'hiver devant lui pour prendre ses mesures\,; qu'il fallait
marcher à pas lents, et voir si les nuages du nord ne produiraient pas
des tonnerres et des grêles\,; que, si le roi d'Espagne pouvait armer
une bonne flotte, plusieurs pourraient changer de ton. Il comptait sur
les assurances que Riperda lui donnait que l'intérêt du commerce ne
permettrait point à ses maîtres de s'opposer à l'Espagne\,; et dans
cette confiance Albéroni parlait plus haut même au pape, dont le bref au
roi d'Espagne, dont on a parlé, et qu'Aldovrandi n'avait osé lui
présenter, avait été imprimé en Hollande par ordre du nonce de Cologne.
Aldovrandi, fort embarrassé, chercha à faire sa cour au pape par engager
les évêques d'Espagne, à qui il écrivit, d'accepter la constitution. Je
n'en dirai pas davantage ici sur cette matière. On verra à l'occasion de
mon ambassade en Espagne, ce que l'archevêque de Tolède, qui était lors
et qui était le même {[}à l'époque de mon ambassade{]}, m'en dit
lui-même sous le dernier secret. Il est mort depuis cardinal.

Le pape, tremblant devant l'empereur, n'en usait pas avec lui comme il
faisait avec la France et l'Espagne, qui avaient une plus dommageable
simplicité. Non seulement il faisait à l'instant tout ce qu'il plaisait
à l'empereur, mais sans attendre qu'il le demandât, et sans que ce
prince daignât même le remercier. Ainsi l'empereur ayant voulu la
promotion de Czaki, archevêque de Colveza et évêque de Varadin, et sans
nomination aucune de sa part, ce prélat fut déclaré cardinal aussitôt,
malgré tant de paroles données du premier chapeau à Gesvres, archevêque
de Bourges, qui languissait après depuis si longtemps, et que le pape
amusa encore de discours pathétiques.

Une autre affaire embarrassa davantage le pape. Il eut avis que
Peterborough, se promenant en Italie, avait de mauvais desseins sur la
vie du Prétendant. Il le fit arrêter et garder étroitement dans le fort
Urbin. Peterborough était comte d'Angleterre, pair du royaume, chevalier
de la Jarretière. Les Anglais prirent feu sur cet affront, et le roi
d'Angleterre éclata en menaces de bombarder Civita-Vecchia. Le duc de
Parme s'entremit. Le pape eut grand'peur, fit force compliments à
Peterborough, le mit en liberté, et l'orage se dissipa. Le duc de Parme
était encore bien plus alarmé pour lui-même\,: il comptait sur
l'indignation de l'empereur qui ne demanderait qu'un prétexte pour
l'accabler. La proposition d'assurer à un fils de la reine d'Espagne la
succession de Toscane, de Parme et de Plaisance lui faisait déjà voir
une garnison impériale dans ces deux places, et se croire perdu sous le
joug des Allemands. Il eut recours au cardinal Albéroni, et conseilla au
roi d'Espagne de s'armer au commencement de l'hiver, et avec éclat, pour
tenir les Allemands en crainte.

Cellamare donnait les mêmes conseils, surtout depuis la prise de
Cagliari. Le pape était dans les mêmes frayeurs. Il souhaitait ardemment
la neutralité de l'Italie\,; il ne l'espérait que de l'établissement de
la paix entre l'empereur et l'Espagne. Il ordonna à son nonce à Paris de
presser le régent d'agir pour la procurer, mais par insinuations
seulement, tant il redoutait de choquer la cour de Vienne, et
d'entretenir sur cette affaire une correspondance exacte avec son nonce
à Madrid.

Il se trouvait alors en d'étranges embarras entre les cours de Madrid et
de Vienne, par les engagements où la frayeur de la dernière l'avait fait
entrer. Bentivoglio, tout nouvellement, n'avait rien oublié pour
l'épouvanter des alliances que la France faisait avec les protestants,
et pour le presser de se lier avec l'empereur. Il voulait aussi qu'il
travaillât au rétablissement du Prétendant, avec son peu de sens et de
jugement ordinaire, comme si ce projet avait pu être compatible avec une
alliance étroite avec l'empereur, si lié avec le roi d'Angleterre. Les
Impériaux, maîtres en Italie, et qui savaient que la frayeur était le
seul moyen d'obtenir tout du pape, l'effrayèrent tellement, par la
persuasion et la colère qu'ils feignirent de ce qu'il était de concert
de l'entreprise de l'Espagne, que, pour s'en laver, il avait écrit ce
bref au roi d'Espagne, dont on a parlé, et qu'il avait depuis approuvé
son nonce de ne l'avoir pas rendu. Mais menacé de plus en plus, il le
fit imprimer, comme on l'a dit, en distribua des copies à tous ses
nonces, exigea non seulement de celui d'Espagne de le remettre enfin au
roi, mais prétendit encore qu'il en tirât réponse, qu'il se croyait due,
pour démentir aux yeux de toute l'Europe l'énorme calomnie qu'on lui
imputait d'être de concert de son entreprise contre l'empereur, dont il
paraphrasait la nécessité de se laver. Il écrivit d'une manière
pathétique et personnelle à Albéroni, dont la promotion n'avait été
faite que sur une parole à laquelle il avait si cruellement manqué\,; et
comme les indults qu'il avait accordés au roi d'Espagne sur le clergé
d'Espagne et des Indes, qu'il avait révoqués, comme on l'a dit, en même
temps qu'il avait écrit ce bref au roi d'Espagne, mais que ces indults
étaient entre les mains d'Albéroni et d'Aubenton, il ordonna à
Aldovrandi, qu'au cas qu'ils refusassent de les lui remettre, d'écrire à
tous les prélats d'Espagne qu'ils étaient révoqués, de leur défendre
d'en rien payer, et de montrer à Albéroni la lettre par laquelle il lui
ordonnait de le faire. Le pape ne put tellement se couvrir, et se parer
du devoir d'impartialité de père commun, et de l'obligation de
manifester la pureté de sa conduite, qu'il n'avouât à Aldovrandi sa
crainte des plaintes que l'empereur faisait des indults qu'il avait
accordés, et de ses menaces, qui suivaient toujours les moindres
complaisances de Rome pour l'Espagne. Il était d'autant plus embarrassé
que ses différends avec cette cour n'étaient pas terminés\,: il ne
prétendait rien moins que d'obliger le roi d'Espagne d'annuler par un
décret tous ceux qu'il avait faits depuis neuf ans contre les
prétentions de la juridiction ecclésiastique, et il comptait pour
l'obtenir sur la reconnaissance d'Albéroni de sa promotion si nouvelle,
sur l'attachement pour lui d'Aubenton, et sur le crédit de tous les
deux.

En même temps il fit voir à l'empereur, par son nonce à Vienne, ce bref
si offensant qu'il avait écrit au roi d'Espagne, et depuis fait imprimer
et répandre, et il espérait par là se laver du soupçon d'intelligence
avec l'Espagne, et détourner l'orage qu'il craignait, peut-être même
faire accepter sa médiation. Mais la froideur et la sécheresse de la
cour de Vienne répondait peu et souvent point à tant de prostitution. La
suspension d'armes en Italie, que le pape lui avait proposée de concert
avec l'Espagne, ne reçut pas la moindre réponse. Les uns crurent que
l'empereur n'y consentirait point par la médiation du pape\,; d'autres
qu'il avait dessein d'envahir l'Italie, dont il ne voulait point perdre
l'occasion. Le pape avoua au cardinal Acquaviva que ses démarches
n'avaient et n'auraient aucun succès, qu'il n'en fallait attendre que
par la France et l'Angleterre, mais que l'empereur était prévenu au
dernier point contre tous ceux qui lui parlaient de paix, et qu'il
protestait tous les jours qu'il renoncerait plutôt à la couronne
impériale qu'à ses prétentions sur celle d'Espagne.

Acquaviva, autant pour son intérêt que pour celui du roi d'Espagne, le
sollicitait de profiter du désordre et de la consternation où étaient
les Allemands du royaume de Naples, de l'empressement que tous les
habitants témoignaient de changer de domination\,; d'y accorder un
pardon général, et l'abolition, non de tout impôt, mais de tous ceux que
les Allemands y avaient mis, parce qu'on n'y pouvait rien espérer de la
force, mais de la seule bonne volonté des nombreux habitants\,; de ne
pas laisser le temps aux Impériaux de finir la guerre de Hongrie\,;
enfin d'envoyer au commencement du printemps une forte escadre en
Italie, et une puissante armée pour y maintenir l'équilibre et protéger
le duc de Parme. Mais rien n'était disposé pour entreprendre sur Naples,
de sorte qu'Acquaviva ne voulut pas risquer beaucoup de seigneurs
napolitains qui s'étaient offerts à lui d'exposer leur vie en se
déclarant, et les maintint seulement dans les bonnes dispositions où ils
étaient. Acquaviva ajoutait à ses conseils au roi d'Espagne que, s'il
n'était pas en état de secourir les princes d'Italie et qu'il voulût
faire la paix avec l'empereur, il ne la pouvait obtenir que par la
France et l'Angleterre, et ne {[}devait{]} point compter sur les offices
du pape, que Vienne méprisait parfaitement.

Albéroni jugeait, comme Acquaviva, des propositions que les Napolitains
lui faisaient. Il aurait pourtant voulu que le mécontentement général se
fit sentir quelquefois pour exciter le châtiment, et par conséquent
aliéner encore plus les peuples. Il faisait ses dispositions pour avoir
au printemps une escadre de trente navires de guerre, vingt mille hommes
de débarquement, un train d'artillerie de cent cinquante pièces de
canon. Il envoya en Hollande le chef d'escadre Castañeda pour acheter
sept vaisseaux équipés et armés en guerre, et à Ragotzi un François
nommé Boissimieux, bien instruit de tout ce que le roi d'Espagne pouvait
et voulait faire pour entretenir la guerre en Hongrie, et pour l'être
lui-même en quel état elle était et quel fondement il y pouvait faire.
Il ne voulait point de paix\,; mais, comme il ne le pouvait témoigner
avec bienséance, il fit part aux cours étrangères de ce qui s'était
passé en gros entre le colonel Stanhope et lui sur les propositions de
paix. Il y fit entendre que le colonel Stanhope et le sieur Bubb avaient
trouvé ses réponses raisonnables, et dépêché en Angleterre. Il se paraît
en même temps de la suspension du second embarquement en considération
des offices de la France et de l'Angleterre, insistait sur l'équilibre,
et sur être en liberté d'agir si la négociation ne réussissait pas. Son
but était de ne prendre aucun engagement et de conserver la liberté de
prendre, suivant les conjonctures, les partis qu'il jugerait à propos.
L'état dangereux du roi d'Espagne les pouvait bientôt changer.

On le crut, ou on le voulut croire si mal qu'on lui fit faire, comme on
l'a dit, un testament sur la fin d'octobre, duquel, outre la reine et
Albéroni, il n'y eut que le P. Daubenton et le duc de Popoli qui en
eussent connaissance. Il fut signé par un notaire de Madrid très obscur.
Six grands furent appelés ensuite, qui signèrent que c'était la
signature du roi et son testament, mais sans qu'ils sussent rien de ce
qu'il contenait. Cela renouvela les bruits ci-devant remarqués sur la
reine, et on fit plusieurs réflexions sur la confiance du contenu du
testament, dont Popoli était le seul seigneur qui en eût le secret, à
l'exclusion même des ministres, ce qui surprit d'autant plus qu'il était
gouverneur du prince des Asturies, et publiquement mal avec le cardinal,
qu'il se piquait de mépriser.

Ce triste état du roi d'Espagne servit au cardinal à éluder les
nouvelles instances du pape, dont on vient de parler, mais il ne parut
pas abattre le courage du premier ministre. Ses discours ne témoignèrent
ni frayeur ni faiblesse. Il brava même, et fort en détail, sur la
puissance qu'on voulait attribuer à l'empereur, en entretenant
l'ambassadeur de Sicile. Celui de Hollande parlait comme le cardinal, ce
qui faisait croire la Hollande unie avec l'Espagne. La même confiance ne
paraissait pas à l'égard de la France, beaucoup moins encore pour
l'Angleterre. On ne doutait pas que le cardinal ne choisît la médiation
des États généraux.

Vers la fin de novembre, la santé du roi d'Espagne fut tout à fait
rétablie\,: le sommeil, l'appétit, les forces, l'embonpoint\,; mais
l'esprit demeura si frappé de sa fin comme imminente qu'il voulait sans
cesse son confesseur auprès de lui. Il le retenait souvent jusqu'au
moment qu'il se mettait au lit avec la reine. Souvent encore il
l'envoyait chercher au milieu de la nuit\,; mais cette faiblesse ne
s'étendait pas sur d'autres choses, et il ne paraissait pas au dehors
qu'il eût été malade.

Albéroni ne pensait qu'à ses préparatifs de guerre. Il publiait qu'en
mai suivant le roi d'Espagne aurait cinquante mille hommes de pied
effectifs, et quinze mille chevaux, et trente vaisseaux de guerre bien
armés, non pour faire aucunes conquêtes, mais pour maintenir ses droits
et ses amis, si aucun était molesté en haine de cette amitié. Mais il ne
persuadait personne, parce que personne ne pouvait croire que tant de
dépense n'eût d'objet que celui qu'Albéroni publiait. Le colonel
Stanhope en fut d'autant plus inquiet qu'il le pressait souvent de lui
apprendre le motif de l'armement d'une escadre qui se faisait en
Angleterre pour la Méditerranée. On disait à Vienne que c'était contre
l'Espagne. Monteléon mandait que c'était contre le pape, sur l'affaire
de Peterborough. Mais Albéroni avait si peu de confiance en ce ministre
qu'on ne doutait pas que, s'il consentait enfin que la paix fût traitée
à Londres, il n'y fît passer Beretti.

C'était à quoi ce cardinal pensait bien moins qu'à conserver ses
conquêtes, et à en faire de nouvelles. Il fit laisser en Sardaigne neuf
bataillons et huit cents chevaux, prit ses mesures pour faire croiser
tout l'hiver des frégates depuis les côtes de Toscane jusqu'au phare de
Messine, envoya de Gènes à Cagliari trente-cinq mille pistoles, pourvut
toutes les places du roi d'Espagne de tout en abondance. Il refusa de
traiter, en s'expliquant différemment à l'Angleterre et à la France. Il
s'excusa au général Stanhope sur ce qu'il attendait les réponses du
régent, sans lesquelles l'union inséparable des deux couronnes
l'empêchait de rien faire\,; au duc de Saint-Aignan que, si le régent
tenait le même langage sur l'union des deux couronnes, il jouerait dans
le monde un rôle différent de celui qu'il y jouait. Il paraphrasa
l'indignité de sa servitude pour l'Angleterre, la terreur panique qu'on
prenait de l'empereur, les grandes choses qui résulteraient, à
l'avantage des deux couronnes, d'une union effective et stable. Il avait
raison sans doute, mais pour cela il aurait fallu chasser Albéroni et
Dubois dans les pays les plus éloignés de la France et de l'Espagne, qui
toutes les deux n'eussent jamais tant gagné.

Saint-Aignan lui représenta que les choses étaient déjà bien avancées\,;
que le régent était d'accord avec l'Angleterre sur les conditions de la
paix\,; que, si l'Espagne était attaquée, la France ne pourrait la
secourir, l'état du royaume obligeant à conserver la paix dont il
jouissait. Albéroni répondit que le roi d'Espagne ne s'éloignerait
jamais d'un accommodement à des conditions équitables\,; qu'il se
défendrait jusqu'à la dernière goutte de son sang si l'empereur était
injuste dans ses demandes\,; finit en disant qu'il ne pouvait croire
que, si le roi d'Espagne était attaqué dans le continent de son royaume,
une nation qui l'avait porté et maintenu sur ce trône le voulût voir
retourner en France simple duc d'Anjou\,; que si ce prodige arrivait, il
faudrait bien s'accommoder à la nécessité.

Ce discours fit un grand bruit, et fut interprété fort diversement. Ce
qui est certain, c'est qu'Albéroni éloigna toujours la négociation\,;
qu'il avait des motifs cachés d'espérance qu'on ne pénétra point\,;
qu'il croyait se faire une ressource d'une ligue qu'il formerait entre
le czar et la Suède\,; qu'il comptait qu'il pouvait naître de jour en
jour des événements favorables à l'Espagne. Il jugeait pouvoir faire
agir les armées au dehors sans avoir rien à craindre pour les provinces
de l'Espagne, et se repaissait ainsi de chimères.

Il désirait sur toutes choses de ménager les Hollandais, de les aigrir
contre l'empereur, et de profiter de l'occasion de se délivrer de sa
crainte et de ses desseins en modérant sa puissance. Mais ses
exhortations étaient vaines. Les Hollandais sentaient la nécessité du
repos pour le rétablissement de leur État, et quoiqu'il y eût différents
partis dans la république, tous se réunissaient à conserver la paix.
Ceux qui y avaient le plus de part aux affaires ne pouvaient sortir de
leurs maximes\,: que l'intérêt de la république était de s'attacher
indissolublement à suivre les résolutions de l'Angleterre, et suivre ses
mouvements, même avec dépendance.

Rien n'était plus éloigné des sentiments de la république que le concert
avec l'Espagne, que les discours de Riperda, tout à Albéroni, faisaient
plus que soupçonner. Il parla un jour à l'ambassadeur de Sicile de la
formidable puissance que l'Espagne aurait la campagne suivante,
supérieure aux forces délabrées de l'empereur, qui ne pouvait faire sa
paix avec les Turcs\,; lui vanta le bonheur de la conjoncture pour
établir un équilibre\,; proposa l'union du roi de Sicile avec le roi
d'Espagne, pour attaquer à la fois\,: l'un l'État de Milan, l'autre le
royaume de Naples. Del Maro, étonné d'un pareil propos de l'ambassadeur
de Hollande, répondit qu'il faudrait, avant de prendre un engagement
dont les suites pouvaient être si périlleuses, être bien assuré des
secours que pourraient et voudraient donner la France, l'Angleterre et
la Hollande. Riperda osa l'assurer que la France favoriserait
secrètement l'exécution de ce qu'il proposait. Sur l'Angleterre, il
avoua qu'il n'y fallait pas compter\,; mais il assura que, outre qu'il
ne convenait pas aux Anglais, par l'intérêt de leur commerce, de se
brouiller avec l'Espagne, il prévoyait tant d'embarras à Londres, que
Georges n'aurait ni le temps ni le moyen de songer ni de se mêler des
affaires des autres. À l'égard de sa république, il dit que, encore
qu'il ne fût pas de la bonne politique de rompre avec l'empereur dans
l'état où elle se trouvait alors, cette extrémité était encore moins
fâcheuse que de {[}se{]} brouiller avec l'Espagne, son commerce avec
elle étant ce que ses maîtres avaient de plus capital à conserver. Son
objet à lui était que la Hollande se maintînt neutre, mais en aidant
l'Espagne de tout ce qu'il serait possible sans se déclarer. Avec de
tels propos de l'ambassadeur de Hollande, il n'est pas surprenant que
les soupçons d'intelligence de sa république avec l'Espagne ne
grossissent\,; à. quoi en effet beaucoup furent trompés.

La mort de l'évêque de Malaga donna lieu de nommer Albéroni à cet évêché
de trente mille écus de rente, qu'il ne reçut que comme l'introduction
aux plus grands et aux plus riches sièges de l'Espagne, quand ils
viendraient à vaquer. Le roi d'Espagne lui donna encore vingt mille
ducats, à prendre sur les confiscations de ceux qui avaient suivi le
parti de l'empereur, et tous les meubles qui avaient appartenu au duc
d'Uzeda. Peu de temps après, le cardinal Arias, archevêque de Séville,
étant mort, Albéroni fut nommé à ce riche archevêché.

Il s'expliqua, sur la fin de cette année, avec tant d'emportement sur la
négociation de Londres pour la paix, à l'abbé del Maro, que ce dernier
assura le roi de Sicile qu'il n'y aurait point de paix\,; que l'Espagne,
peu disposée à jeter tant d'argent rial à propos, et qui ne pouvait
craindre d'invasion de la part de l'empereur, ne ferait pas des
préparatifs si considérables, si ce n'était pour entreprendre\,; et que
ces vues étaient conformes au caractère d'esprit d'Albéroni, dont
l'ambition était d'atteindre à la gloire des cardinaux Ximénès et de
Richelieu. Il prétendait qu'un nommé Zanchini, qui demeurait à Gênes,
lui avait prédit son cardinalat. Quelque temps après y être parvenu, il
l'envoya chercher, mais il ne put jamais le retrouver.

Aldovrandi, croyant faire sa cour à Rome de procurer l'acceptation
formelle de la constitution par les évêques d'Espagne, y avait
souverainement déplu. La folie de l'infaillibilité était souverainement
blessée qu'on pût imaginer qu'elle eût besoin d'autre autorité que de la
sienne, ni du concours de soumission explicite des évêques, pour donner
toute la force nécessaire aux bulles dogmatiques. La seule pensée était
un abus si terrible qu'il ne pouvait être compensé par aucune utilité
qu'Aldovrandi eût pu imaginer. Il eut donc ordre de détruire son propre
ouvrage, et d'empêcher les évêques d'Espagne d'accepter ce qu'ils
devaient adorer d'adoration de latrie, les yeux bandés et les oreilles
bouchées, \emph{provoluti ad pedes}, expression si chérie à Rome et si
barbare dans l'Église. Ce pauvre nonce était depuis quelque temps si
malmené de sa cour que le cardinal Paulucci, secrétaire d'État, en prit
honte et pitié, le consolait et lui en faisait comme des excuses. Le
manquement de parole d'Albéroni sur la flotte, celui de n'avoir pas
présenté ce bref injurieux au roi d'Espagne, la complaisance d'avoir
remis au premier ministre et au confesseur les brefs de révocation des
indults, les soins du nonce d'excuser toujours Albéroni et les procédés
de cette cour, étaient les griefs qui irritaient le pape, dans l'extrême
dépit et l'embarras où le jetait la hauteur sans mesure de l'empereur.

Ce monarque, qui sentait ses forces en Italie, et qui connaissait bien à
qui il avait affaire, écrivit moins une instruction d'un prince
catholique à Gallas, son ambassadeur auprès du souverain pontife, qu'une
déclaration de guerre et des lois d'un vainqueur sans ménagement pour le
vaincu, et parfaitement impossibles. Il manda à Gallas qu'il voulait
bien croire que le pape n'avait point de part à l'entreprise de
l'Espagne contre lui\,; mais qu'il ne suffisait pas qu'il voulût bien
avoir pour lui cette complaisance, que ses actions en devaient aussi
persuader le monde\,; que pour y réussir l'empereur demandait ce que le
pape prétendait faire contre le roi d'Espagne\,; mais prévoyant qu'il
aurait peine à se porter à des partis extrêmes, Sa Majesté Impériale
voulait bien se contenter de lui demander\,:

Qu'Aldovrandi fût rappelé et privé de tous ses emplois, pour avoir été
l'instrument de l'intelligence secrète entre le pape et le roi
d'Espagne\,;

Qu'Albéroni fût cité à Rome pour y rendre compte de sa conduite, ou que
le pape fît passer un de ses ministres en Espagne pour lui faire son
procès\,;

Que le roi d'Espagne fût privé de toutes les grâces que le saint-siège
avait accordées non seulement à lui, mais à tous ses prédécesseurs\,;

Que la croisade fût levée au profit de Sa Majesté Impériale dans le
royaume de Naples et le duché de Milan\,;

La promotion au cardinalat du comte d'Althan sur-le-champ et sans aucun
délai\,;

Des quartiers d'hiver dans l'État ecclésiastique pour ses troupes qu'il
voulait faire passer incessamment en Italie. Véritablement on voit bien
qu'il était difficile de rien demander de plus modeste.

La pape pria Gallas de lui laisser ces demandes par écrit. Il voulait
répondre dans le premier mouvement que, si l'empereur en venait à la
violence, il irait le recevoir le crucifix à la main. Son nonce en même
temps n'était plus admis chez l'empereur. Il eut grande peine à en
obtenir audience pour l'informer de la promotion de Czaki. Elle ne lui
fut accordée qu'à condition qu'il n'y parlerait d'aucune autre affaire.
Quoique l'empereur eût fort désiré et pressé cette promotion, il
répondit dédaigneusement au nonce qu'il ne savait encore s'il
accepterait la grâce que le pape faisait à cet archevêque. Ainsi la cour
de Vienne exigeait avec empire les grâces qu'elle voulait obtenir de
Rome, les méprisait après les avoir obtenues, la gouvernait par cette
politique, et la tenait toujours tremblante devant le prince qu'elle
regardait comme le maître de l'Italie, toujours prête à suivre et à
prévenir même ses désirs. Néanmoins les choses s'adoucirent de manière
qu'il y eut lieu de soupçonner qu'il y avait eu du concert.

Quoique l'Espagne, en perdant l'Italie, eût perdu en même temps son
poids et son ressort principal auprès du pape, ses ministres ne
laissaient pas de s'y expliquer avec assez de hauteur pour que le pape
s'en trouvât souvent embarrassé. Dès qu'Acquaviva eut appris les
demandes que Gallas avait faites, il écrivit au pape pour le presser de
répondre enfin au roi d'Espagne sur la médiation qu'il lui avait
offerte, de lui mander s'il y avait quelque apparence à cette médiation,
ou de lui laisser la liberté d'agir, puisque la cour de Vienne ne
songeait qu'à l'amuser, pendant qu'elle prenait ses mesures, et qu'elle
faisait les dispositions nécessaires pour envahir l'Italie.

Sur ce billet, le pape envoya à Alamanni, secrétaire des chiffres, dire
à Acquaviva qu'il n'avait pu proposer à Vienne la suspension d'armes,
parce qu'il n'avait point reçu de réponse du roi d'Espagne, quoiqu'il
l'eût prié de lui mander ce qu'il pensait sur cet article\,; que, dans
cette incertitude, il n'avait pu donner aucun projet, d'autant plus que
l'empereur avait demandé pour première condition la restitution de la
Sardaigne, ce que le pape ne pouvait assurer sans savoir les intentions
du roi d'Espagne.

Acquaviva témoigna sa surprise que depuis deux mois que le pape lui
faisait accroire qu'il avait proposé sa médiation à Vienne, fondée sur
le consentement du roi d'Espagne, il n'eût encore fait aucune démarche à
Vienne. Alamanni répondit à cette plainte par celle de l'offre du roi
d'Espagne de la médiation aux États généraux, déplora la malheureuse
situation du pape. Acquaviva riposta par celle de l'impression du bref
injurieux au roi d'Espagne, qui paraissait même dans toutes les
gazettes. Ainsi la visite se passa en reproches. Quelle que fût la
faiblesse du pape, Acquaviva ne pouvait se persuader qu'il se laissât
aller à quelque démarche violente contre le roi d'Espagne, mais bien que
ce prince n'avait rien à attendre de Sa Sainteté. Ce cardinal fut en
même temps averti de l'intérieur du palais qu'on avait vu sur la table
du pape une lettre d'Albéroni, contenant que le roi d'Espagne était
suffisamment pourvu de troupes et de vaisseaux pour faire par mer toutes
sortes de débarquements et toutes sortes d'entreprises par terre, et que
le traité en question serait bientôt conclu.

Acquaviva, bien servi de cet intérieur du palais, en apprit en même
temps qu'il s'était trouvé sur la table du pape une lettre du cardinal
Pignatelli, archevêque de Naples, qui lui mandait les mouvements de la
ville et des provinces, où les partisans d'Espagne étaient partout fort
supérieurs à ceux de l'empereur, et que tout était à craindre d'une
subite révolution. Acquaviva recevait lui-même souvent les mêmes avis et
des sollicitations pressantes d'assistance d'Espagne. Mais cette
couronne n'étant pas en état ni préparée à en pouvoir donner, on s'en
tint à l'avis déjà pris de n'exposer pas les bien intentionnés pour son
service.

On ne pouvait comprendre que l'Espagne pût soutenir la guerre sans
alliés, ni qu'à commencer par le pape, aucun prince d'Italie eût le
courage ni les forces d'entrer dans cette ligue, ni d'y apporter quelque
poids. Ils étaient tous environnés des États de l'empereur dont les
derniers progrès en Hongrie fortifiaient leurs chaînes. Il n'y avait que
le roi de Sicile qui pût faire pencher la balance du côté qu'il voudrait
embrasser. Il envoya le comte de Provane à Paris, et fit en même temps
des dispositions pour prendre un corps de Suisses à son service, ce qui
fit croire qu'il avait dessein d'entrer dans une alliance avec la France
et l'Espagne pour affranchir l'Italie du joug des Allemands.

On a déjà vu les justes frayeurs du duc de Parme, à qui l'empereur ne
pardonnait pas son inclination française dans la dernière guerre du feu
roi en Italie, et l'attachement naturel que lui donnait le second
mariage du roi d'Espagne.

Le duc de Modène, qui avait toujours fort ménagé la cour de Vienne et
qui avait eu l'honneur d'être beau-frère de l'empereur Joseph, refusa
par cette considération de donner sa fille au Prétendant, qu'Albéroni,
le faible parti de ce prince et ses amis pressaient de se marier. Les
Anglais même, et protestants, et les plus aliénés de sa maison, le
désiraient aussi pour avoir toujours un droit légitime à montrer à leur
roi, le faire souvenir de leur choix, et le contenir par cette
perspective. Le pape était entré dans ce mariage de Modène, et voulait
aller lui-même le célébrer à Lorette, et donner la bénédiction nuptiale,
honneur peu conforme aux intérêts du Prétendant en Angleterre, et à un
triste état qu'il ne cherchait qu'à cacher.

Outre le pouvoir que donnait à l'empereur sa situation de maître de
l'Italie, il y pouvait tout encore par le moyen des neveux du pape. On
doutait qu'il fût informé de leurs engagements secrets et des grâces
qu'ils en retiraient, mais on parlait tout haut à Rome et avec le
dernier scandale de la dépendance du cardinal Albane de la cour de
Vienne, et des sommes considérables qu'il touchait sur Naples, dont le
payement était régulier ou interrompu, selon que Gallas était satisfait
ou mal content de sa conduite. Il avait été suspendu à la promotion
d'Albéroni, parce que Gallas trouva qu'Albane ne s'y était pas assez
opposé. Dans la suite, ils se raccommodèrent, et le robinet de Naples
fut rouvert. On croyait communément que personne n'osait instruire le
pape de la vénalité de ses neveux\,; on voyait sa nonchalance sur un
désordre dont l'évidence ne pouvait lui être inconnue. Ceux qui étaient
le plus à portée de lui parler savaient certainement qu'ils se
perdraient s'ils touchaient cette corde, parce que le cardinal Albane
était le maître de les ruiner dans l'esprit de son oncle, quoiqu'il
n'eût pour lui ni estime ni tendresse. Ce neveu en était lui-même si
persuadé qu'il craignait la vengeance d'Aldovrandi qui, dans la
persécution qu'il souffrait des neveux, pour plaire à l'empereur, et
soutenu du roi d'Espagne, avait menacé de publier bien des choses, s'il
était pressé de faire connaître que ses ennemis étaient ceux qui
trahissaient le pape, parce qu'ils étaient vendus à l'empereur. Le
cardinal Albane, qui se reconnut aisément à ce portrait, et fort en
peine des dénonciations qu'Aldovrandi pouvait produire, fit divers
manèges pour l'adoucir, sans toutefois risquer de déplaire aux Allemands
qu'il informait des affaires les plus secrètes, que la faiblesse du pape
lui confiait sans réserve. Quand il était nécessaire de les instruire
avec plus de détail, il ne se faisait aucun scrupule de prendre sur les
tables du pape les mémoires qu'on lui donnait et de les remettre à
Gallas.

Cet ambassadeur n'était pas le seul bien informé de l'intérieur du
palais. Acquaviva l'était fort bien aussi. Il sut qu'Aldovrandi mandait
au pape que le roi d'Espagne pourrait consentir à la restitution de la
Sardaigne, non comme préliminaire, mais comme acte de concorde, si
d'ailleurs il recevait les satisfactions qu'il demandait. Malgré
l'obscurité de cette expression, le pape crut avoir beaucoup obtenu. Il
s'en servit avec art auprès des Allemands\,; il dit à Gallas qu'il
s'excuserait auprès du roi d'Espagne de se charger de la médiation parce
qu'il voyait qu'on se défiait à Vienne des offices qu'il s'était
proposés pour la pacification entre les deux cours. Gallas, surpris de
la proposition et n'ayant point d'ordre de son maître, n'osa prendre sur
soi de la rejeter. Il pria le pape de lui permettre de lui en écrire. Le
pape y consentit, et donna ses ordres en même temps à son nonce à
Vienne. Mais ces propositions de paix ne suspendirent pas les instances
que Gallas faisait au pape de rompre ouvertement avec l'Espagne. Ceux
qui connaissaient bien le pape n'étaient pas surpris de l'entendre
menacer de se porter à des résolutions extrêmes, et parler
imprudemment\,; mais ils étaient bien persuadés qu'il n'exécuterait rien
du tout, et qu'il ne prendrait jamais d'engagements à craindre, tant
qu'il serait maître de suivre sa pente naturelle et sa propre volonté.

La restitution de la Sardaigne était en effet la condition que
l'empereur posait pour base du traité à faire, s'il y en avait de
possible entre lui et l'Espagne. Ses ministres le disaient ainsi
partout. Ils comptaient que l'intérêt, personnel du roi d'Angleterre
l'emporterait sur celui du commerce des Anglais, et qu'ils ne pourraient
l'empêcher d'employer la force pour procurer la restitution de la
Sardaigne. Ils ne laissaient pas de craindre l'inquiétude que la nation
Anglaise pourrait prendre de cette violence, et que les Hollandais
n'eussent le bon sens de profiter de la division de l'Angleterre et de
l'Espagne.

Les ministres d'Angleterre tenaient un langage uniforme à celui des
Impériaux. Cadogan, prêt à partir de la Haye, dit à Beretti que
Penterrieder était à Londres uniquement pour écouter les propositions
qui seraient faites à l'empereur, non pour en faire aucune\,; qu'il
n'entrevoit point en négociation, si la restitution de la Sardaigne
n'était accordée comme une condition préliminaire du traité, et se jeta
de là en reproches mal fondés et en menaces d'invasion facile de
l'Italie, où le duc de Parme serait la première victime de l'indignation
de l'empereur.

Les propos si impériaux de Cadogan ne plurent pas à Heinsius, qui ne le
cacha pas à Beretti. Celui-ci crut voir de la jalousie sur la médiation
et Duywenworde, qui se flattait de l'aller exercer à Londres pour les
États généraux, en fut encore plus mécontent. Beretti, qui pour que la
négociation ne lui échappait pas, la souhaitait à la Haye, n'oublia pas
d'insister en Espagne sur la partialité déclarée du roi d'Angleterre et
de ses ministres, et sur le danger de traiter à Londres sous leurs yeux.
L'abbé Dubois écrivit de Londres à ses amis que ce serait un grand bien,
si le roi d'Espagne voulait bien envoyer promptement Beretti en
Angleterre, parce que certainement le ministère Anglais travaillerait
pour ses intérêts\,; que les ordres du régent étaient de les soutenir\,;
qu'il le ferait aussi de bonne sorte, et que Beretti en serait convaincu
s'il passait la mer. Ce sincère abbé en écrivit autant à Basnage, en
Hollande, de manière que Beretti qui avait toujours crié en Espagne
contre toute négociation qui se ferait à Londres, n'osa changer
subitement d'avis. Mais croyant sur cette lettre de l'abbé Dubois voir
jour à y être employé, ce qu'il n'espérait plus, il se contenta de
s'offrir en Espagne, si on voulait s'y servir de lui, quoiqu'il fût
toujours dans la même opinion sur une négociation traitée à Londres.

Monteléon, que cet emploi regardait si naturellement comme ambassadeur
d'Espagne en Angleterre depuis si longtemps, n'en voulait pas manquer
l'honneur. Il fit donc entendre qu'outre la confiance des ministres
d'Angleterre qu'il avait intimement, il était encore particulièrement
instruit des sentiments des ministres de France.

Il prétendait avoir tiré des lumières de Chavigny, que l'abbé Dubois
avait amené avec lui à Londres. C'est ce même Chavigny dont j'ai raconté
l'impudente et célèbre imposture, et l'éclatante punition qui le
déshonora à jamais, l'expatria jusqu'après la mort du roi, et fut sue de
toute l'Europe. Quoique ses aventures ne pussent être ignorées de
Monteléon, il crut en pouvoir faire usage. Il l'avait vu en Hollande, il
le cajola sur ce qu'il le voyait employé dans les affaires étrangères.
Il sut de lui que le maréchal d'Huxelles était entièrement pour
s'opposer à l'agrandissement et aux entreprises de l'empereur, et que
sur ce principe Chavigny prétendait que le maréchal avait soutenu que,
si l'empereur refusait de contenter le roi d'Espagne, {[}ce{]} qui
devait être la première condition du traité, il fallait se préparer à la
guerre offensive et défensive en union avec l'Espagne et le roi de
Sicile, et que c'était l'avis de presque tous ceux qui composaient le
conseil de régence, surtout depuis l'arrivée à Paris du comte de
Provane.

Sur cette friponnerie, Monteléon se donnait en Espagne comme pleinement
instruit des intentions de la France et de celles de l'Angleterre.
Stanhope lui avait dit en confidence que l'empereur ne s'éloignerait pas
d'un accommodement, à condition de reconnaissance et de renonciations
réciproques\,; qu'il consentirait à donner des sûretés pour la
succession de Toscane, et qu'il entrerait encore en d'autres
tempéraments, mais qu'il voulait la cession de la Sicile, et des secours
pour la conquérir. Monteléon avertissait l'Espagne que c'était sur ces
conditions qu'elle devait régler ses résolutions et ses mesures. Mais
cet ambassadeur ne réussissait pas à pénétrer, comme il le croyait, le
véritable état de la négociation de l'abbé Dubois et de Stanhope.

Elle était peu avancée avec Penterrieder à la fin de novembre.
L'empereur avait personnellement une telle répugnance à renoncer à la
monarchie d'Espagne pour toujours, que ses ministres, même Espagnols,
n'osaient lui en parler. À peine laissait-il entendre qu'il pourrait
renoncer à l'Espagne et aux Indes, en faveur de Philippe V et de sa
postérité\,; mais il ne voulait pas aller plus loin, ni ouïr parler de
la postérité d'Anne d'Autriche, quelque juste que cela fût, par les
traités et les renonciations. Il voulait bien accorder l'investiture de
Parme et de Plaisance à un fils de la reine d'Espagne, mais avec un
refus absolu de celle de Toscane. On faisait valoir comme une grande
complaisance qu'elle ne pût tomber à la maison d'Autriche, et qu'elle
fût assurée au duc de Lorraine. Toutes sortes de manèges étaient
employés pour faire consentir à de si déraisonnables articles. Toutefois
les Anglais assurèrent l'abbé Dubois qu'il pouvait absolument compter
sur la fermeté du roi d'Angleterre, s'il se pouvait promettre celle du
régent, et qu'il ne se laisserait point ébranler par la cabale du roi
d'Espagne en France. C'était le galimatias que cet abbé écrivit.

Les Anglais étaient en peine du voyage du comte de Provane à Paris, et
d'une liaison entre le roi de Sicile qui prenait confiance en ce
ministre, et le régent dont le mariage du prince de Piémont avec une
fille du régent serait le lien. Le ministre de Sicile à Londres en prit
une vive alarme. On a vu qu'il avait lié une négociation directe avec
l'empereur, même par le frère de l'envoyé de Modène à Londres qui était
à Vienne, et à portée de cette confiance avec l'empereur à ce qu'il
prétendait. Un des points de cette négociation était le mariage d'une
archiduchesse avec le prince de Piémont, ce qui aurait été renversé si
ce qu'on disait de celle du comte de Provane se trouvait véritable. La
Pérouse ne cessait d'aliéner son maître du régent\,; il se défiait
beaucoup de l'abbé Dubois, et n'était pas plus content de Penterrieder.
Ce dernier parla à l'envoyé de Modène\,: il ne le laissa en aucun doute
qu'il ne fût instruit de la négociation dont La Pérouse avait chargé son
frère à Vienne. Il ne lui déguisa point que l'empereur voulait avoir la
Sicile de gré ou de force\,; que, s'il était possible de convenir de
cette condition par un traité, il faudrait qu'il y eût un ministre
piémontais à Vienne\,; mais qu'il savait qu'il n'y serait pas reçu s'il
n'avait le pouvoir de faire cette cession\,; que l'empereur avait des
moyens sûrs de conquérir cette île, mais qu'il aimait mieux en avoir
l'obligation au roi de Sicile, aussi instruit qu'il l'était de la
situation des affaires de l'Europe\,; qu'on prendrait après les mesures
nécessaires pour lui conserver les titres d'honneur et d'autres
avantages encore dont il aurait lieu d'être content. L'envoyé de Modène
eut curiosité de savoir quel serait l'échange, et s'il se prendrait dans
le Milanais. Penterrieder répondit que l'empereur ne pouvait céder dans
tout cet État un seul pouce de terre, mais qu'en un mot le roi de Sicile
serait satisfait. La Pérouse, fort inquiet d'une réponse si générale,
pressa son ami de lui en dire davantage. Soit que l'envoyé de Modène en
sût plus en effet, ou que ce ne fût qu'un soupçon, il lui fit entendre
qu'on proposerait la Sardaigne. Cela fut soutenu de tous les langages
fermes, mais caressants et flatteurs, que Penterrieder sut tenir à La
Pérouse, en l'assurant bien surtout des mauvaises dispositions de la
France pour le roi de Sicile, dont lui-même se citait pour témoin
lorsqu'il était à Paris.

L'abbé Dubois s'était embarqué à la fin de novembre pour aller chercher,
disait-il, de nouvelles instructions, avec promesse d'un très prompt
retour. On le savait trop instruit des intentions du régent pour les
croire le motif de son voyage. On crut donc qu'il ne le faisait que pour
concilier les différents sentiments de ceux qui composaient le conseil
de régence. Comme j'en étais un, je puis assurer que ceux qui le crurent
ne rencontrèrent pas mieux.

Pendant cet intervalle de négociation, le colonel Stanhope eut ordre de
faire entendre par Albéroni à la reine d'Espagne que si Dieu disposait
du roi d'Espagne, qu'on croyait alors très mal, cet événement
n'apporterait aucun changement aux dispositions favorables du roi
d'Angleterre pour elle et pour lui, et qu'ils devaient compter tous deux
sur un appui solide et sur des assistances effectives de sa part\,;
qu'il maintiendrait les dispositions que le roi son mari aurait faites
en sa faveur, et pour gage de cette bonne volonté, Stanhope devait citer
ce que son maître faisait actuellement pour procurer par le traité de
paix les avantages des infants du second lit.

Pendant ce temps-là le roi d'Espagne fit dire à Bubb et au colonel
Stanhope, que, pour complaire au roi d'Angleterre, il entrerait dans la
négociation qu'il proposait, si l'empereur promettait pour préliminaire
de ne point envoyer de troupes en Italie, et de n'y point demander de
contributions. Le colonel Stanhope tâcha de persuader à Monteléon son
désir que la proposition fût acceptée à Vienne, où Penterrieder venait
de l'envoyer par un courrier. Il le prépara aux réponses hautaines de
cette cour\,; mais il ajouta que Georges étant content des bonnes
intentions du roi d'Espagne, il faudrait nécessairement que la médiation
d'Angleterre, soutenue de celle de France, réduisît les parties
intéressées à la raison. Bernsdorff vendu à l'empereur dont il attendait
tout, voulut tourner en poison la réponse du roi d'Espagne\,; dit
qu'elle était concertée avec la cabale de France opposée au régent,
laquelle voulait traîner la négociation en longueur, en représentant à
ce prince que, puisque le roi d'Espagne voulait bien entrer en traité,
Son Altesse Royale ne devait rien conclure sans la participation et
l'intervention de Sa Majesté Catholique. Bernsdorff savait peut-être que
les Impériaux, peu disposés à traiter, se rendraient encore plus
difficiles quand ils sauraient cette réponse, et insisteraient plus
fortement sur la restitution préliminaire de la Sardaigne. Les Allemands
du conseil de l'empereur souhaitaient et lui conseillaient d'accorder la
renonciation que le roi d'Angleterre lui demandait comme base du traité.
Mais le conseil destiné aux affaires d'Espagne, tout d'Espagnols et
d'Italiens rebelles et réfugiés à Vienne, s'y opposaient de toutes leurs
forces, et entretenaient l'opiniâtreté de l'empereur là-dessus. Le
ministre d'Angleterre relevait toutes ces circonstances, l'embarras et
la difficulté de la négociation que leur maître entreprenait, par
conséquent le mérite de ses bonnes intentions et de ses peines.

Stanhope, dont la conduite parut toujours la plus franche dans tout le
cours de cette affaire, témoigna beaucoup de joie d'apprendre par une
lettre que l'abbé Dubois lui écrivit, immédiatement après son arrivée à
Paris, que le régent était ferme dans sa résolution de conclure et de
signer le traité, même sans l'intervention du roi d'Espagne, pourvu que
l'empereur fit la renonciation dans les termes convenables, et qu'il
accordât la satisfaction demandée pour le roi d'Espagne sur l'article de
la Toscane. Le roi d'Angleterre promit d'appuyer fortement à Vienne des
demandes si raisonnables. Les ministres d'Angleterre en usaient avec
tant de confiance à l'égard de Penterrieder, qu'elle allait à lui
montrer les lettres qu'ils écrivaient et celles qui leur étaient
écrites.

Cette union alarmait beaucoup La Pérouse. Plus il voyait ce ministère
appliqué à plaire à l'empereur, plus il sentait le danger de remettre la
médiation des intérêts du roi de Sicile entre des mains qui les
sacrifieraient au désir qu'ils ne cachaient pas de procurer tous les
avantages à la maison d'Autriche. Provane n'était pas moins inquiet à
Paris. Il n'oubliait rien pour découvrir l'état de la négociation,
voyait souvent le régent, hasardait de lui faire des questions.
L'arrivée de l'abbé Dubois redoubla sa vigilance. Le régent lui promit
que, lorsqu'il renverrait Dubois à Londres, il lui donnerait ordre
précis de communiquer à l'envoyé de Sicile tout ce qui, dans la
négociation aurait rapport aux intérêts de ce prince. Provane n'en
pouvait pas demander davantage\,; mais sortant de la cour de Turin, il
comptait peu sur les promesses et sur la sincérité des princes.

Ce fut en ce temps-ci qu'arriva l'éclat dont on a parlé ailleurs entre
le roi d'Angleterre et le prince de Galles, à qui il était né un fils,
et qui, mécontent de ce que le roi son père avait nommé le duc de
New-Castle pour en être le parrain, s'emporta contre ce seigneur jusqu'à
le traiter fort injurieusement. Cette affaire, précédée de la
continuelle mésintelligence entre le père et le fils, dont la cause a
été aussi expliquée, fit augurer des troubles en Angleterre et des
révolutions qui inquiétèrent fort les étrangers sur la possibilité de
prendre des liaisons solides avec cette couronne. La Pérouse, qui le
pensait comme les autres, était persuadé aussi avec le public du peu de
sincérité des négociateurs entre le père et le fils, conseillait au roi
de Sicile de ne pas compter sur les offices ni sur la médiation de
l'Angleterre, mais de négocier directement à Vienne, et se flattait que,
persuadé de la solidité de ce conseil, il en estimerait davantage la
négociation directe qu'il y avait entamée par le frère de l'envoyé de
Modène à Londres, lequel frère était, comme on l'a vu, à Vienne.
L'envoyé, son frère, qui de son côté s'entremettait à Londres entre
Penterrieder et La Pérouse, mourut, dans cette conjoncture. Il fallut
chercher un autre canal en attendant le retour de l'abbé Dubois, dont
l'absence suspendait toutes ces négociations.

L'opinion qu'elles auraient un bon succès engagea le gouvernement
d'Angleterre à commencer doucement les dispositions nécessaires pour
obliger le roi d'Espagne à souscrire au traité dont la conclusion
paraissait prochaine. On travailla donc, quoique lentement, à l'armement
d'une escadre pour la Méditerranée. Monteléon, informé de cette
destination, déclara à Sunderland que le roi d'Espagne regardait avec
raison cet armement comme fait contre ses intérêts. Sunderland répondit
que jusqu'alors le roi d'Angleterre n'avait nulle intention d'envoyer
cette escadre dans la Méditerranée\,; qu'on ne l'armait que pour
intimider la cour de Rome et la forcer à donner une juste satisfaction
sur l'arrêt du comte de Peterborough dans le fort Urbin\,; que le roi
d'Angleterre espérait si bien de la négociation pour la paix qu'il n'y
aurait point lieu d'employer aucunes forces maritimes, ce qu'il était
bien résolu de ne faire que lorsqu'il verrait toutes voies fermées à la
conciliation, parce qu'alors il serait obligé de ne pas laisser allumer
en Italie une guerre qui embraserait toute l'Europe. Stanhope tint le
même langage à Monteléon\,; il lui dit de plus que l'abbé Dubois ne
différait son retour à Londres que pour savoir les dernières intentions
de la cour d'Espagne et pour attendre aussi les réponses de la cour de
Vienne. Il lui fit valoir la ferme résistance du roi d'Angleterre aux
instances continuelles des Impériaux qui ne cessaient de lui demander la
garantie du traité de 1716. Mais le roi d'Angleterre voulait attendre
l'effet de l'offre qu'il avait faite à Madrid de sa médiation,
conjointement avec celle de la France, et qu'il souhaite que l'Espagne
contribue de son côté à un accommodement raisonnable et que la haine du
refus retombe sur la cour de Vienne, en sorte que, par ce moyen,
l'Angleterre se trouve libre et dégagée de la garantie si répétée et si
sollicitée par les Impériaux. Les deux ministres firent fort valoir à
Monteléon les peines infinies qu'ils avaient à obtenir de l'empereur la
renonciation qu'il avait en horreur, dont néanmoins ils espéraient bien
venir à bout, mais qu'ils ne se flattaient pas d'un succès égal sur
l'article de la Toscane.

Comme les difficultés augmentaient à Vienne sur cette succession, les
ennemis du régent imaginèrent de persuader les Espagnols que ce prince
les faisait naître secrètement. Beretti fut averti que le régent
ménageait le refus de l'expectative pour l'infant don Carlos, dans la
vue de l'obtenir pour le duc de Chartres, et comme Beretti n'avait
jamais pu tirer de Stanhope, dans tout leur commerce, sur quel prince le
roi d'Angleterre jetait les yeux pour la Toscane, il se confirmait dans
ce soupçon. Il cherchait donc avec encore plus d'inquiétude à découvrir
les véritables projets. Duywenworde lui dit un jour que la cour de
Vienne proposerait bientôt un second plan, qui serait d'ajouter, en
faveur de l'empereur, la Sicile à Naples, et Mantoue, avec le petit État
de Guastalla, au Milanais\,; donner la Toscane au duc de Guastalla et la
Sardaigne à M. de Savoie. Soit que ce fût de bonne foi ou dans le
dessein de pénétrer mieux les pensées de Beretti, il déclama contre la
mauvaise volonté des Anglais, dit qu'il savait de bon lieu que le régent
appuierait les raisons du roi d'Espagne, que l'abbé Dubois avait ordre
de parler de manière à réussir et que, quand ce ne serait pas même le
sentiment du régent, il y avait dans le conseil de régence des hommes
assez courageux pour lui résister.

Beretti, flatté de ces dispositions de la France, se tenait encore plus
assuré de celles de la Hollande. Il les regardait comme son ouvrage,
assurait que {[}les États{]} ne se laisseraient point entraîner par
l'Angleterre contre l'Espagne, laquelle ils serviraient même s'ils
pouvaient. Il vantait le changement entier du Pensionnaire à cet égard,
qui trouvait très raisonnables les conditions que le roi d'Espagne avait
demandées, qui lors de la maladie de ce prince avait marqué beaucoup de
tendresse, et qui lui témoignait à lui une confiance entière, au lieu
qu'à Londres, où il n'était pas, tout était partial pour l'empereur.
Beretti attribuait à cette partialité les plaintes que l'Angleterre
avait portées aux États généraux du refus qu'avait fait Riperda de se
joindre aux envoyés d'Angleterre, pour faire de concert les
représentations que les Anglais avaient faites seuls sur l'entreprise de
Sardaigne. Il ajoutait que les principaux de la république, et qui
toujours avaient été les plus Anglais, comme Duywenworde et d'autres, ne
pouvaient souffrir l'ingratitude de l'Angleterre, qui voulait exclure la
Hollande de la négociation. Il répondait de l'inutilité des cabales des
Impériaux, qui ne pourraient rien opérer par l'Angleterre sans le
concours de la Hollande, et que sûrement Riperda, haï à Londres et à
Vienne, parce que ses relations étaient favorables à l'Espagne, n'aurait
point d'ordre d'adhérer aux instances ni aux menaces des Anglais qui,
dans la bouche de Cadogan, à la Haye, y avaient fort gâté les affaires
de l'Angleterre.

Beretti prétendait que les Hollandais ne pardonnaient point aux Anglais
la hauteur de vouloir que les ministres de Hollande dans les pays
étrangers fussent choisis, envoyés et rappelés suivant le caprice de la
cour d'Angleterre, comme ils le voulaient pour Riperda et même pour
Châteauneuf, ambassadeur de France à la Haye, qui ne se conduisait pas
selon leurs sentiments\,; et qu'ils disaient qu'il fallait savoir s'il
agissait par ceux des mécontents de France ou par ceux du régent, pour
s'éclaircir des véritables intentions de ce prince. Widword pourtant,
qui semblait plus modéré à Beretti, avouait que ce qu'il y avait de plus
sensé dans la république était cordialement disposé à maintenir le
régent suivant le traité de la triple alliance, et persuadé que tant que
ce prince agirait avec amitié et confiance à l'égard de l'Angleterre et
de la Hollande, il n'aurait rien à craindre du dedans ni du dehors.

\hypertarget{chapitre-ix.}{%
\chapter{CHAPITRE IX.}\label{chapitre-ix.}}

1717

~

{\textsc{Mouvements du roi de Prusse à divers égards.}} {\textsc{- Son
caractère et ses embarras.}} {\textsc{- Tentatives pleines d'illusions
de Cellamare, qui découvre avec art la vraie disposition du régent sur
les affaires présentes.}} {\textsc{- Mouvements en Bretagne.}}
{\textsc{- Idées d'Albéroni.}} {\textsc{- Il s'emporte contre les
demandes de l'empereur au pape, surtout sur celle qui le regarde
personnellement.}} {\textsc{- Déclaration du roi d'Espagne sur la
paix.}} {\textsc{- Propos, sentiment, conduite d'Albéroni.}} {\textsc{-
Ses préparatifs.}} {\textsc{- Son profond secret.}} {\textsc{- Sa
toute-puissance en Espagne.}} {\textsc{- Monti à Madrid.}} {\textsc{- Le
roi d'Espagne inaccessible.}} {\textsc{- Souverain mépris d'Albéroni
pour Rome.}} {\textsc{- Sa conduite sur le bref injurieux au roi
d'Espagne.}} {\textsc{- Aldovrandi occupé de rapprocher les deux cours
et de se justifier à Rome sur ce qu'il avait fait à l'égard de
l'acceptation de la constitution en Espagne.}} {\textsc{- Délicatesse de
Rome étrangement erronée.}} {\textsc{- Anecdote importante sur la
constitution entre l'archevêque de Tolède et moi.}} {\textsc{- Son
caractère.}} {\textsc{- La nonciature chassée de Naples.}} {\textsc{- Le
pape, n'osant rien contre l'empereur, s'en prend à l'Espagne.}}
{\textsc{- Rare expédient du pape sur la non-résidence d'Albéroni en son
évêché de Malaga.}} {\textsc{- Réflexion.}} {\textsc{- Délicatesse
horrible de Rome.}} {\textsc{- Fureurs de Bentivoglio qui dégoûtent de
lui les siens mêmes.}} {\textsc{- Il donne au pape des conseils
extravagants sur les affaires temporelles.}} {\textsc{- D. Alexandre
Albane passe pour vendu à l'Espagne.}} {\textsc{- Mauvais gouvernement
du pape.}} {\textsc{- Il refuse les bulles de Séville à Albéroni.}}
{\textsc{- Frayeur du duc de Parme et ses conseils à l'Espagne.}}
{\textsc{- Conduite et sentiments d'Albéroni.}} {\textsc{- Forces de
l'Espagne diversement regardées.}} {\textsc{- Sage avis de del Maro au
roi de Sicile.}} {\textsc{- Riperda, vendu à Albéroni, lui propose
l'union du roi de Sicile au roi d'Espagne.}} {\textsc{- Singulière
aventure d'argent entre Bubb, Riperda et Albéroni.}} {\textsc{- Triste
état personnel du roi d'Espagne et du futur {[}roi{]}.}} {\textsc{-
Insolentes vanteries d'Albéroni.}} {\textsc{- Ses efforts auprès des
Hollandais.}} {\textsc{- Son opinion de l'Angleterre.}} {\textsc{- Ses
bravades.}} {\textsc{- Riche arrivée des galions.}} {\textsc{- Haute
déclaration des ambassadeurs d'Espagne en France, en Angleterre et
ailleurs.}} {\textsc{- Propos d'Albéroni sur l'Angleterre et la
Hollande.}} {\textsc{- Mesures militaires d'Albéroni.}} {\textsc{- Il
veut engager une guerre générale.}} {\textsc{- Les Anglais ne laissent
pas de le ménager.}} {\textsc{- Triste état personnel du roi d'Espagne,
quoique rétabli.}} {\textsc{- Mesures d'Albéroni pour être seul et bien
le maître de sa personne.}} {\textsc{- Docteur Servi, médecin
parmesan.}} {\textsc{- Proposition en l'air de marier le prince des
Asturies à une fille du prince de Galles.}} {\textsc{- Roideur de
l'empereur soutenu des Anglais.}} {\textsc{- Inquiétude du roi de
Sicile.}} {\textsc{- Propos de son envoyé en Angleterre avec Stanhope,
qui l'augmente.}} {\textsc{- La Pérouse est la dupe de Penterrieder sur
la France.}} {\textsc{- Le czar prend la protection du duc de
Mecklembourg, et rassure le roi de Prusse sur un traité particulier avec
la Suède.}} {\textsc{- Mort de la maréchale de Duras.}} {\textsc{-
Quatre gentilshommes de Bretagne mandés par lettre de cachet pour venir
rendre compte de leur conduite.}}

~

M. le duc d'Orléans travaillait alors à réunir le roi de Prusse avec les
États généraux. Il se faisait un mérite auprès de Son Altesse Royale de
presser la république, par déférence pour lui, de conclure l'alliance
avec lui, où il aurait désiré d'attirer le roi d'Angleterre. Mais
Georges en paraissant éloigné, il priait le régent de presser la
Hollande de conclure avec lui sans le roi d'Angleterre. Le roi de Prusse
était encore plus agité des affaires du nord. Il souhaitait faire sa
paix particulière avec la Suède, et craignait l'abandon de ses alliés,
s'ils découvraient ses démarches là-dessus. Le désir d'acquérir et la
crainte de perdre ne s'accordaient en lui ni avec ses lumières ni avec
son courage. Il ne savait ni se résoudre ni soutenir ses résolutions. Il
était, comme on l'a déjà dit, léger, changeant, facile à regarder les
mauvaises finesses comme un trait d'habileté, et la mauvaise foi comme
la politique la plus fine. Le roi de Pologne avait découvert et publié
les propositions qu'il avait faites à l'insu de ses alliés. Lui, avait
donné de fausses interprétations à sa négociation. Il n'avait persuadé
personne, mais ses alliés ne voulaient pas le perdre, pour ne pas
affaiblir le nom et l'apparence de la ligue du nord. Eux-mêmes, chacun à
part, se sentaient coupables du même crime.

Le roi de Prusse se plaignit d'avoir été trahi par Goertz, ministre de
Suède, voulant faire entendre que, s'il avait voulu traiter secrètement,
il n'eût fait que suivre l'exemple du roi d'Angleterre\,; il fit avertir
que le comte de La Mark s'était rendu suspect au roi de Suède, en
traitant avec trop de chaleur pour les intérêts de la maison d'Hanovre,
et qu'il eût mieux réussi s'il eût commencé à traiter sa paix à lui. Il
demanda même qu'en vertu des obligations secrètes, la France cessât de
payer des subsides à la Suède. Il représentait le danger de
l'agrandissement de l'empereur, et des alliances qu'il contractait dans
l'empire, celle surtout avec la maison de Saxe. Il offrait de prendre
des mesures contre cette énorme supériorité de l'empereur, la nécessité
d'y faire entrer la Suède, et pour cela celle de sa paix avec lui, parce
qu'il protestait qu'il ne pouvait rien faire tant qu'il serait occupé de
la guerre du nord. On voyait ainsi le caractère du roi de Prusse, qui
était tremblant devant l'empereur, bien éloigné d'oser rien entreprendre
qui lui pût déplaire, et qui, parlant à la France, déclamait et
proposait tout contre lui.

Cellamare, par d'autres motifs, fit à peu près les mêmes représentations
au régent. Il le pressa d'agir de concert avec l'Angleterre, pour mettre
un frein à l'ambition des Impériaux. Il se flatta de mettre l'abbé
Dubois, arrivant de Londres, dans ses intérêts là-dessus. Il voulait
persuader que la France, pour trop désirer de conserver la paix, se
verrait entraînée à la guerre. S'il trouva l'abbé trop dévoué au
ministère d'Angleterre pour le persuader, il gagna du moins à acquérir
assez de lumières dans une longue conversation qu'il eut avec lui, pour
les communiquer à Madrid, par un courrier exprès. Il voulut voir si les
sentiments étaient uniformes entre les principaux du gouvernement. Il
mit le maréchal d'Huxelles sur la matière du traité, le contredit,
l'opiniâtra exprès, et en tira qu'il ne s'éloignait point des sentiments
de l'abbé Dubois. Le maréchal convint de la nécessité de borner
l'ambition et l'orgueil des Allemands\,; mais il soutint que la France
et l'Espagne unies, mais seules ensemble, n'étaient pas bastantes pour
arrêter les entreprises des Impériaux\,; que la France était trop
épuisée et hors d'état de s'exposer au péril de faire renouveler la
dernière ligue contre les deux couronnes. Cellamare combattit ce
raisonnement, moins pour convaincre que pour découvrir de plus en plus.
Le maréchal demeura ferme dans l'opinion que la France se tînt dans une
indifférence apparente, qu'elle achevât de gagner le roi d'Angleterre et
ses ministres, déjà bien disposés\,; que ce serait du même coup gagner
la Hollande, inséparable de l'Angleterre\,; que le roi d'Espagne devait
marquer beaucoup de promptitude et de docilité à tout accommodement
raisonnable\,; s'accréditer par quelque démonstration extérieure, comme
d'envoyer un ministre à Londres, pour assister à la négociation, avec
des instructions secrètes pour faire avec adresse tomber sur les
Allemands la haine de former des prétentions déraisonnables. Il n'en
fallut pas davantage à Cellamare pour se convaincre des maximes
présentes que le gouvernement de France se proposait de suivre. Il
conclut que son unique objet était d'éviter une guerre qu'on croyait
généralement que la France ne pourrait soutenir, que Cellamare traitait
de terreur panique, ce que les mouvements de la Bretagne imprimaient
encore plus fortement. Cellamare, qui en voyait un apparent mépris dans
le gouvernement, ne les crut ni si méprisables ni si indifférents qu'on
les voulait donner. Ils n'étaient pas non plus si considérables ni si
pernicieux que les malintentionnés le voulaient persuader. Le plus grand
mal, selon cet ambassadeur, était la faiblesse du gouvernement, agité
par la diversité des intérêts et des passions, manquant d'argent, et
accablé par les dettes de l'État.

Albéroni, véritable roi d'Espagne absolu, et seul, était persuadé que
les négociations de Londres seraient sans effet, que l'intérêt du roi
d'Espagne était de les regarder avec grande indifférence, et d'attendre
du temps les avantages qui seraient refusés par un traité. Il croyait
avoir beaucoup fait que d'accepter la médiation du régent et d'y
persister\,; il se faisait un grand mérite, à son égard, d'avoir
suspendu le second embarquement, ce qu'il n'avait fait que par
impuissance\,; il comptait que l'Italie ne serait jamais tranquille tant
que l'empereur y posséderait un pouce de terre\,; il se flattait que la
conquête de la Sardaigne encouragerait les Turcs à continuer la
guerre\,; il se moquait et se plaignait de la faiblesse du pape, qui
était une des sources de la fierté des Allemands et de l'insupportable
hauteur de leurs demandes, surtout de celle d'envoyer un commissaire
pour lui faire son procès à Madrid\,; il s'exhala en injures et en
épithètes, dit qu'il ne conseillerait pas au pape de le hasarder, parce
qu'il ne serait pas sûr que son commissaire fût bien reçu\,; qu'à
l'égard de la citation il pourrait se rendre à Rome si le roi d'Espagne
y consentait, mais que ce serait avec une telle compagnie qu'elle
pourrait déplaire au pape, et plus encore à l'auteur de la demande, dont
il prit occasion de déclamer contre la domination tyrannique que les
Allemands entreprenaient d'étendre sur le genre humain, et la nécessité
et l'intérêt pressant de toutes les nations de s'unir contre leur
ambition. Loin de croire que la négociation de Londres fût propre à la
borner, il la décriait comme un artifice concerté entre l'empereur et le
roi d'Angleterre pour tenir en panne la France et l'Espagne, et se
moquer après de toutes les deux. Mais pour éviter l'odieux de ne vouloir
entendre à rien qui pût conduire à la paix, il déclara que le roi
d'Espagne était prêt à intervenir dans la négociation par un ministre,
quand le régent jugerait que l'empereur se porterait véritablement à une
paix solide et sûre pour le repos de l'Italie\,; mais s'il se voyait
obligé d'envoyer un ministre à Londres, Albéroni comptait bien d'y
prolonger la négociation, de la suspendre, d'en arrêter la conclusion,
suivant qu'il le jugerait à propos, et d'armer pour cela son ministre de
propositions équivalentes à celles des Impériaux, comme de prétendre,
pour condition préliminaire, le remboursement des dépenses de la
conquête de la Catalogne et de Minorque, que l'empereur, contre ses
promesses, avait longtemps défendues, même le remboursement de
l'expédition de la Sardaigne. Mais son intention, disait-il, était de
les tenir secrètes, de laisser à la France et à l'Angleterre le soin de
rédiger et de faire les propositions qui pouvaient conduire à la paix,
surtout au repos de l'Italie, et de se réserver la faculté de les
approuver ou non, selon ce qui conviendrait le mieux aux intérêts du roi
d'Espagne. Il ordonna donc à tous les ministres d'Espagne, dans les
cours étrangères, de les assurer que Sa Majesté Catholique ne
s'éloignerait jamais de contribuer de sa part au repos de l'Europe.

En même temps il songeait à faire acheter en Hollande des vaisseaux de
guerre, de la poudre, des boulets, des munitions de marine. Il se
flattait de trouver toute facilité dans la république par son intérêt de
commerce à l'égard de l'Espagne. Il se répandit un bruit que le roi
d'Espagne avait offert aux États généraux de leur céder les Pays-Bas ou
la meilleure partie, s'ils voulaient entrer avec lui dans une alliance
particulière, et on prétendit que le Pensionnaire en avait averti
l'empereur. Albéroni nia le fait avec aigreur, et dit que, si l'Espagne
voulait adhérer à de certaines propositions, la Hollande n'y trouverait
peut-être pas son compte. Il ne s'expliqua pas davantage\,; mais il
gémissait de voir l'amour de la patrie éteint dans les républiques,
leurs divisions, leurs factions, leurs principaux membres sordidement
vendus aux puissances étrangères. Il assurait en même temps le colonel
Stanhope et Bubb que le roi d'Angleterre connaîtrait bientôt par
expérience que la cour de Vienne ne songeait qu'à ses intérêts, et
qu'elle n'avait d'égard pour personne.

Il pressait cependant tous les préparatifs pour la campagne et les
recrues de l'infanterie, et disposait toutes choses pour embarquer les
troupes dès que la saison le permettrait. On disait que le roi d'Espagne
voulait avoir des troupes étrangères, engager à son service celles que
les Hollandais réformaient, principalement les bataillons suisses. On
parlait fort aussi des négociations secrètes d'Albéroni pour engager les
Turcs, par le moyen de Ragotzi, à ne faire ni paix ni trêve avec
l'empereur.

Mais le secret de ce premier ministre était réservé à lui tout seul. Qui
que ce soit n'avait sa confiance, ses accès très difficiles\,; les
ministres étrangers ne lui parlaient que par audiences qu'il leur
fallait demander par écrit. Tout le gouvernement était renfermé dans sa
seule personne. Chaque secrétaire d'État venait lui rendre compte de son
département et recevoir ses ordres. La stampille\footnote{Il a été
  question de la stampille ou sceau, tome III, p. 117-118.} même était
entre ses mains, par lesquelles passaient toutes les expéditions et les
ordres secrets du roi d'Espagne, qui était inaccessible, qu'on ne voyait
que le moment qu'il s'habillait, et qui ne disait jamais mot à personne.
Monti même, l'ami intime d'Albéroni de tous les temps, allé à Madrid
pour le plaisir de le voir revêtu de la pourpre, et logeant chez lui,
eut peine à voir le roi et la reine d'Espagne. On n'a point su s'il y
eut entre ces deux amis quelque affaire particulière et quelque mesure
prise par rapport aux affaires de France on remarqua seulement
qu'Albéroni affecta de répandre qu'il ne voyait Monti qu'à dîner qui,
accoutumé aux sociétés de Paris, s'ennuierait bientôt de la solitude de
Madrid. Chalais y arriva alors rappelé par le roi d'Espagne\,; on crut
que c'était pour l'employer dans la marine. Albéroni triomphait du bon
et glorieux état où il avait remis l'Espagne, et en insultait au
cardinal del Giudice et aux précédents ministères, qui n'avaient pu la
tirer de son abattement.

Il témoignait à ses amis que rien ne le surprenait de ce qui se passait
à Rome. La reine et lui avaient pour cette cour le plus profond mépris.
Il fit déclarer dans toutes les cours étrangères que ce bref injurieux
que le pape avait fait imprimer n'avait jamais été présenté au roi
d'Espagne, et fit valoir au pape cette déclaration comme un moyen le
plus doux qui se pût proposer dans une matière si grave, où à peine la
grande piété du roi d'Espagne l'avait empêché d'user des remèdes
proportionnés à l'affront qu'il recevait, mais qui deviendraient
inévitables si le pape, non content de ce qu'il avait fait, se portait à
passer à de nouvelles explications. Albéroni profitait de la commodité
d'avoir un nonce persuadé que sa fortune dépendait de l'union entre les
deux cours, et qui en écartait autant qu'il le pouvait tout sujet de
mésintelligence, et qui représentait sans cesse au pape la nécessité,
pour l'intérêt du saint-siège, de ménager le zèle et les bonnes
intentions du roi d'Espagne. Il voulut aussi s'excuser sur ce qu'il
avait fait pour l'acceptation des prélats d'Espagne de la
constitution\,; il fit entendre que l'Espagne avait aussi ses novateurs,
contre lesquels la vigilance des évêques et l'autorité même de
l'inquisition ne suffisaient pas, et qui n'étaient retenus que par la
crainte du châtiment\,: galimatias faux dans son principe, faux dans sa
conséquence, parce que rien n'est plus redouté en Espagne que
l'inquisition, ni plus redoutable, en effet, que sa toute puissance, et
que sa cruauté sur laquelle, comme je l'ai vu moi-même, les
sollicitations ni l'autorité du roi ne peut rien.

Aldovrandi continuait à tirer de cette prétendue situation de l'Espagne
qu'il fallait pour y remédier des choses extraordinaires. Il représenta
au pape qu'en partant de Rome le cardinal Fabroni, moteur principal, et
le prélat Alamanni, spécialement chargé de l'affaire de la constitution,
lui avaient dit tous deux qu'il serait bon, à son arrivée en Espagne, de
porter les évêques de marquer leur obéissance au saint-siège par un acte
public et par une acceptation formelle de la bulle\,; que là-dessus il
s'était adressé aux universités d'Espagne\,; que le pape avait approuvé
les insinuations qu'il leur avait faites par une lettre qu'il avait
reçue de sa part du cardinal Paulucci, dont il lui envoyait copie, et
qu'il avait eu une attention particulière à bien mesurer les termes de
sa lettre aux évêques pour prévenir les conséquences que les
malintentionnés pourraient tirer de la recherche de l'acceptation des
évêques d'Espagne, comme si Rome croyait qu'une acceptation de tous les
évêques de la chrétienté pût donner la force aux constitutions
apostoliques qu'elles avaient par elles-mêmes ou que cette acceptation y
fût le moins du monde nécessaire, supposition la plus mal fondée.
L'énormité de cette chimère saute aux yeux et porte l'indignation avec
elle. C'est à elle néanmoins que Rome sacrifie tout\,; habile à écarter
tout ce qui lui peut porter préjudice et à se parer de tout avantage
qu'elle peut usurper.

Elle ne répliqua rien aux raisons du nonce, mais elle lui fit savait
qu'il y avait quelques expressions dans la lettre de l'archevêque de
Tolède au pape qui lui déplaisaient. Celle-ci surtout\,: \emph{Comme le
nonce de Votre Sainteté nous a fait exposer depuis peu}. La délicatesse
de l'infaillibilité et de l'indépendance du consentement même de
l'approbation de l'Église, assemblée ou séparée, était blessée de ce
qu'on pouvait inférer de ces ternies que l'archevêque eût été sollicité
d'accepter la constitution. Le fond de la lettre plut tellement au pape
qu'il promit, si l'archevêque lui écrivit une autre lettre pareille où
ces mots fussent omis, non seulement de lui répondre, mais de lui donner
toutes les louanges qui lui convenaient. Ainsi se débite l'orviétan de
Rome pour en masquer la tyrannie. Le pape suspendit donc sa réponse,
parce qu'il s'assurait que l'archevêque de Tolède la mériterait
incessamment par une prompte obéissance. Je ne puis mieux placer qu'en
cet endroit l'anecdote que j'ai promise, où elle se trouvera plus à
propos et plus naturellement que si je la différais au temps de mon
ambassade en Espagne, quatre ans après ceci.

Diegue d'Astorga y Cespedez, gentilhomme espagnol, né en 1666, est le
prélat duquel il vient d'être parlé. D'inquisiteur de Murcie il fut fait
évêque de Barcelone, à la mort de ce furieux cardinal Sala, en 1715,
dont j'ai parlé en son lieu, et pour son mérite et ses services signalés
à Barcelone, transféré cinq ans après, sans qu'il pût s'en douter, à
l'archevêché de Tolède, où je le trouvai placé à mon arrivée à Madrid,
qui est du diocèse de Tolède et le séjour ordinaire de ses archevêques.
Il fut cardinal de la promotion du 27 novembre 1727, de la nomination du
roi d'Espagne. Il n'a point été à Rome, et est mort en 17..\footnote{La
  date n'a pas été complétée par Saint-Simon.}. C'était un homme plein
de partout, de taille médiocre, qui, ressemblait parfaitement à tous les
portraits de saint François de Sales, dont il avait toute la douceur,
l'onction et l'affabilité. Il fréquentait peu la cour, n'{[}y{]} allait
que par nécessité ou bienséance\,; fort appliqué à son diocèse, à
l'étude, car il était savant, à la prière, aux bonnes oeuvres, étudiait
et travaillait toujours\,; si modeste dans une si grande place qu'il
n'en avait d'extérieur que ce qui en était indispensable. Son palais,
beau et vaste, dans Madrid, appartenant à son siège, était sans
tapisseries ni ornement, que quelques estampes de dévotion, le reste des
meubles dans la même simplicité. Il jouissait de plus de huit cent mille
livres de rente et ne dépensait pas cent mille francs par an, en toute
espèce de dépense. Tout le reste était distribué aux pauvres du diocèse
avec tant de promptitude qu'il était rare qu'il ne fût pas réduit aux
expédients pour achever chaque année. Il joignait avec aisance la
dignité avec l'humilité et il était adoré à la cour et dans tout son
diocèse, et dans une singulière vénération. Nous nous visitâmes en
cérémonie\,; bientôt après nous nous vîmes plus librement et nous nous
plûmes réciproquement. Un de ses aumôniers nous servait d'interprète.
Étant un jour chez lui, il me demanda s'il n'y aurait pas moyen de nous
parler latin, pour parler plus librement et nous passer d'interprète. Je
lui répondis que je l'entendais passablement, mais qu'il y avait longues
années que je ne m'étais avisé de le parler. Il me témoigna tant d'envie
de l'essai, que je lui dis que le plaisir de l'entretenir plus librement
me ferait passer sur la honte du mauvais latin et de tous les
solécismes. Nous renvoyâmes l'interprète, et depuis nous nous vîmes
toujours seuls et parlions latin.

Après plusieurs discours sur la cour, le gouvernement d'Espagne, et
quelques-uns aussi sur celui de France et sur les personnages, où nous
parlions avec confiance, il me mit sur la constitution, et ne pouvait
revenir de la frénésie française qui là-dessus l'étonnait au dernier
point\,: «\,Hélas\,! me dit-il, que vos évêques se gardent bien de faire
comme nous. Peu à peu Rome nous a, non pas subjugués, mais anéantis au
point que nous ne sommes plus rien dans nos diocèses. De simples prêtres
inquisiteurs nous font la leçon\,: ils se sont emparés de la doctrine et
de l'autorité. Un valet nous apprend tous les jours qu'il y a une
ordonnance de doctrine ou de discipline affichée à la porte de nos
cathédrales, sans que nous en ayons la moindre connaissance. Il faut
obéir sans réplique. Ce qui regarde la correction des moeurs est encore
de l'inquisition. Les matières de l'officialité, il ne tient qu'à ceux
qui y ont affaire de laisser les officialités et d'aller au tribunal de
la nonciature, ou s'ils ne sont pas contents des officialités, d'appeler
de leurs jugements au nonce, en sorte qu'il ne nous reste que
l'ordination et la confirmation sans aucune sorte d'autorité, et que
nous ne sommes plus évêques diocésains. Le pape est diocésain immédiat
de tous nos diocèses, et nous n'en sommes que des vicaires sacrés et
mitrés uniquement pour faire des prêtres et des fonctions manuelles,
sans oser nous mêler que d'être aveuglément soumis à l'inquisition, à la
nonciature, à tout ce qui vient de Rome, et s'il arrivait à un évêque de
leur déplaire en la moindre chose, le châtiment suit incontinent, sans
qu'aucune allégation ni excuse puisse être reçue, parce qu'il faut une
soumission muette et de bête. La prison, l'envoi liés et garrottés à
l'inquisition, souvent à Rome, sont des exemples devenus rares, parce
qu'ils ont été fréquents et qu'on n'ose plus s'exposer à la moindre
chose, quoiqu'il y en ait encore eu de récents en cette dernière sorte.
Voyez donc, monsieur, ajouta-t-il, quelle force peut donner à la
constitution l'acceptation des évêques des pays réduits dans cette
soumission d'esclaves tels que nous sommes en Espagne, et en Portugal,
et en Italie, à plus forte raison les universités et les docteurs
particuliers, et les corps séculiers, réguliers et monastiques. Mais je
vous dirai bien pis, ajouta-t-il avec un air pénétré. Croyez-vous que
pas un de nous eût osé accepter la constitution, si le pape ne nous
l'eût pas fait commander par son nonce\,? l'accepter eût été un crime
qui eût été très sévèrement châtié\,; c'eût été entreprendre sur
l'autorité infaillible et unique du pape dans l'Église, parce que oser
accepter ce qu'il décide, c'est juger qu'il décide bien. Or, qui
sommes-nous pour joindre notre jugement à celui du pape\,? Ce serait un
attentat dès qu'il parle, nous n'avons que le silence en partage.
L'obéissance et la soumission muette et aveugle, baisser la tête sans
voir, sans lire, sans nous informer de rien, en pure adoration. Ainsi,
même bien loin d'oser contredire, proposer quelque chose, demander
quelque explication, il nous est interdit d'approuver, de louer,
d'accepter en un mot toute action, tout mouvement, toute marque de
sentiment et de vie. Voilà, monsieur, la valeur des acceptations de
toutes les Espagnes, le Portugal, l'Italie, dont j'apprends qu'on fait
tant de bruit en France, et qu'on y donne comme un jugement libre de
toutes les Églises et de toutes les écoles. Ce ne sont que des esclaves
à qui leur maître a ouvert la bouche par permission spéciale pour cette
fois, qui leur a prescrit les paroles qu'ils devaient prononcer, et qui,
sans s'en écarter d'un iota, les ont servilement et littéralement
prononcées. Voilà ce que c'est que ce prétendu jugement qu'on fait tant
sonner en France que nous avons tous unanimement rendu, parce qu'on nous
a prescrit à tous la même chose.\,» Il s'attendrit sur un malheur si
funeste à l'Église et si contraire à la vérité et à la pratique de tous
les siècles, et me demanda un secret tel qu'on peut se l'imaginer, que
je lui ai fidèlement gardé tant qu'il a vécu, mais que je me suis cru
obligé aussi de révéler dès que son passage à une meilleure vie, auquel
toute la sienne ne fut qu'une continuelle préparation, l'eut mis hors
d'état de rien craindre de m'avoir parlé selon la vérité et la religion.

L'empereur commençait à faire sentir son mécontentement au pape. Le
vice-roi de Naples trouva mauvais, par son ordre, que le collecteur
apostolique usurpât la qualité de nonce. Il le fit sortir de Naples en
vingt-quatre heures, et en quarante-huit de tout le royaume, et avec lui
tous les officiers de la nonciature. Rien n'en put retarder l'exécution.
Rome, qui la traita d'attentat, n'osa s'en plaindre qu'à l'Espagne comme
la partie la plus faible, et déclara que c'était à elle à qui elle
attribuait cette offense, pour lui avoir manqué de parole sur l'usage de
sa flotte, et donné lieu de croire que le pape était d'intelligence avec
elle pour enlever la Sardaigne à l'empereur. Aldovrandi eut ordre de se
fonder sur un si beau raisonnement pour demander que les choses fussent
remises dans leur ancien état, à faute de quoi le pape déclarait le roi
d'Espagne redevable à Dieu et au monde de toutes les vexations où Sa
Sainteté se trouverait exposée, laquelle gardait en même temps un
silence de frayeur à l'égard de l'empereur.

L'évêché de Malaga avait été proposé en consistoire pour Albéroni par le
pape. Il en avait reçu de sanglants reproches des Allemands. Il chercha
donc à les apaiser à la première occasion. Elle se présenta bientôt, et
la sagacité du pontife y parut incomparable, aussi bien que la
délicatesse de la conscience d'Albéroni. Il avait voulu être évêque,
bien que cardinal, et avoir quatre-vingt-dix mille livres de rente de
l'évêché de Malaga, mais il n'y voulait pas s'ennuyer et perdre sa
toute-puissance. Il demanda donc une dispense de ne point résider. Le
pape le refusa. Il dit que les motifs qu'il alléguait n'étaient pas
suffisants\,; que, pour l'amour de lui, il avait essuyé tant de
désastres, surtout pour sa promotion au cardinalat, qu'il n'avait pas
résolu d'exposer davantage sa conscience pour le favoriser. Mais comme
il sentait qu'il n'était pas politique de perdre le fruit de tout ce
qu'il avait fait pour lui, et de s'aliéner le maître et le dispensateur
de toutes choses en Espagne, content d'un refus pour plaire à
l'empereur, il fit dire à Albéroni que tout ce qu'il pouvait faire était
de lui accorder la permission de s'absenter six mois l'année de son
église\,; que la disposition des conciles lui en permettait l'absence
autres six mois, et que, par cet expédient si heureusement trouvé, il
aurait ce qu'il demandait de n'y point aller du tout. Ainsi, dans ce
temps, on pouvait alléguer les conciles pour dispenser un évêque de six
mois par an de résidence\,; mais Rome regardait comme une erreur et
comme une offense à la personne et à la dignité du pape de parler de
concile quand il s'agissait de la constitution.

Quelque sujet qu'il eût d'être satisfait du zèle aveugle et emporté que
témoignaient pour son autorité et pour la plénitude de sa
toute-puissance plusieurs évêques français, il craignait toujours dans
leurs écrits quelque marque de leur prévention pour l'autorité de
l'Église universelle, soit assemblée, soit dispersée. Rome eût regardé
comme un grand manque de respect et comme une erreur punissable si les
évêques eussent dit que la constitution faisait loi et obligeait les
fidèles parce qu'elle avait été reçue dans l'Église, comme si, disait
cette cour, la cause nécessaire qui produisait cet effet était
l'acceptation de l'Église. Rome craignait toujours ce qu'elle appelait
les maximes et les phrases françaises, et plus encore la frayeur des
prélats français vendus à Rome de s'exposer aux attaques des parlements.

Bentivoglio, dont les furieuses folies pour mettre tout à feu et à sang
en France pour hâter sa promotion faisaient demander aux plus attachés à
Rome un nonce plus traitable et moins enragé, ne put se contenter de
parler au pape des choses de France\,; il voulut lui donner ses conseils
sur l'événement de la nonciature de Naples, et après l'avoir si souvent
et si fortement importuné de faire une ligue étroite avec l'empereur
pour se soumettre la France, il le pressa de chercher à borner
l'insupportable ambition et puissance de l'empereur, qui voulait mettre
toute l'Europe aux fers. Son jugement parut également en ces deux
conseils si contradictoires. Il pressa le pape de former une ligue avec
l'Espagne, le roi de Sicile et les Vénitiens également intéressés à
diminuer la puissance de l'empereur. Il lui recommanda le secret et la
diligence, lui dit que les hérétiques s'armaient contre lui, tandis que
ses enfants l'insultaient. Il chercha à l'effrayer de l'escadre que
l'Angleterre armait.

Don Alexandre, frère du cardinal Albane, passait pour l'espion secret
des Espagnols dans l'intérieur du pape son oncle, et pour avoir reçu
d'eux quinze mille pistoles à la fois, sans compter d'autres grâces. Le
pape mécontentait tous les princes, n'en ramenait pas un, n'avait encore
terminé, aucun de tous les différends nés sous son pontificat. Il
semblait éloigner tout accommodement sitôt qu'il était proposé\,; la
France et l'Espagne en fournissaient continuellement des exemples. Il
refusa les bulles de Séville à Albéroni. Acquaviva, qui haïssait
personnellement Giudice, l'accusa d'y fortifier le pape, qui faisait
valoir la prompte expédition des bulles de Malaga, qui lui avait attiré
les reproches de faire des grâces à qui méritait des châtiments. Il
assurait qu'il essuierait bien pis, s'il accordait les bulles de Séville
dans un temps où les soupçons de l'empereur étaient sans bornes, et où
il ne cherchait que des prétextes d'opprimer les terres de l'Église. Il
tremblait de se voir enlever l'État de Ferrare. Il imputait tousses
malheurs à la promotion d'Albéroni, et à sa facilité pour l'Espagne, et
se plaignait amèrement que le roi d'Espagne ni ses ministres n'eussent
seulement pas pris l'absolution de tant d'entreprises faites contre
l'autorité du saint-siège\,: c'était plutôt de s'être défendus des
siennes, et de n'avoir pas la bêtise de croire avoir besoin
d'absolution, forge si principale des fers romains.

L'empereur ne menaçait pas moins tous les princes d'Italie que le pape.
Le duc de Parme, le plus exposé de tous à sa vengeance, ne cessait
d'exhorter l'Espagne de hâter son escadre, et d'augmenter ses troupes de
vingt mille hommes, parce que l'empereur augmentait tous les jours
celles qu'il avait en Italie. Albéroni affectait d'en douter, de croire
une grande diminution dans les troupes impériales, et les Turcs éloignés
de faire la paix. Mais il ne laissait pas d'appliquer tous ses soins à
hâter tout ce qui était nécessaire pour attaquer les Allemands en
Italie, toujours persuadé qu'il n'y avait point de traité à faire avec
eux, et que l'Europe ne serait jamais tranquille, tandis que l'empereur
aurait un soldat et un pouce de terre en Italie. Son dessein était
d'avoir trente vaisseaux de guerre en mer, avec tous les bâtiments
nécessaires pour le service de cette flotte, et d'avoir des forces de
terre proportionnées. Les ministres étrangers résidents à Madrid étaient
étonnés, et quelques-uns bien aises de voir l'Espagne sortir comme par
miracle de sa faiblesse et de sa léthargie\,; d'autres en craignaient
les effets, persuadés que si les premiers succès de ces forces
répondaient aux désirs du premier ministre, il ne s'y bornerait pas,
autant pour son intérêt particulier que pour celui de son maître.

L'abbé del Maro ne cessait d'avertir le roi de Sicile qu'il avait tout à
craindre des projets d'Espagne\,: que tout concourait à croire qu'ils
regardaient le royaume de Naples\,; que s'ils en faisaient la conquête,
ils attaqueraient après la Sicile, ces deux royaumes étant nécessaires
l'un à l'autre, surtout à l'Espagne, pour s'assurer les successions de
Toscane et de Parme, le plus cher objet des vues de la reine d'Espagne.
Riperda était l'émissaire le plus secret d'Albéroni auprès des ministres
étrangers à Madrid, il alla trouver del Maro, et raisonnant avec lui sur
les préparatifs qui faisaient alors la matière de toutes les
conversations, il lui fit entendre que le dessein était de faire passer
le printemps prochain quarante mille hommes en Italie, pour attaquer le
royaume de Naples, et que si le roi de Sicile voulait s'unir au roi
d'Espagne pour attaquer le Milanais en même temps, ils chasseraient
infailliblement les Allemands de l'Italie. L'ambassadeur de Hollande
était connu pour trop partial pour persuader celui de Sicile. D'autres
soupçons tombaient encore sur lui. Bubb, résident d'Angleterre, s'était
adressé à Riperda pour engager Albéroni à recevoir du roi d'Angleterre
une gratification très considérable. Riperda s'était chargé de la
commission, à condition que Bubb n'en parlerait jamais directement ni
indirectement au cardinal. La somme avait été remise entre les mains de
Riperda, mais loin qu'Albéroni en donnât quelques marques indirectes de
reconnaissance, il avait, en différentes occasions, et d'un air assez
naturel, traité d'infâmes les ministres qui recevaient de l'argent des
princes étrangers. Ainsi Riperda, suspect au peu de gens qui surent
cette aventure secrète, n'était guère propre à les persuader. Mais qui
pouvait répondre qu'Albéroni ne fût pas assez fourbe pour avoir su
profiter de l'argent sans y laisser de sa réputation, et sans être tenu
de reconnaissance, et que Riperda, trop enfourné avec lui, et mal dans
son pays où il ne voulait pas retourner, n'en ait été la dupe, et forcé
de se laisser affubler du soupçon d'avoir profité de l'argent\,?

On doutait alors de la vie du roi d'Espagne, quelque soin qu'Albéroni
prit de publier le rétablissement parfait de sa santé. Ses anciennes
vapeurs le reprirent sur la fin de décembre, et lui causèrent des
faiblesses. On sut que sa tète était ébranlée au point de ne pouvoir
ranger un discours\,; en sorte que, supposé qu'il vécût, il serait
incapable de gouverner, et que toute l'autorité demeurerait au cardinal
et à la reine, et que la même chose arriverait s'il venait à mourir,
parce que le testament qu'il avait fait leur était en tout favorable.
Les grands et les peuples anéantis, les conseils pour le moins autant,
sans talents, sans moyens, sans courage pour s'affranchir du joug
d'Albéroni, maître des troupes et des finances\,; d'ailleurs, nulle
espérance du prince des Asturies, tendrement aimé des Espagnols, qui se
flattaient d'apercevoir en lui de bonnes qualités. Mais c'était un
enfant, élevé dans la crainte, tenu de fort court par un gouverneur
italien perdu d'honneur et de réputation sur tous chapitres, dont le
plus grand mérite était d'empêcher que qui que ce soit ne pût parler ni
même approcher du prince\,; capable de tout pour augmenter sa fortune,
et qu'on ne doutait pas qu'il ne fût vendu à la reine, même au cardinal,
quoique faisant profession de le mépriser. Ce gouverneur était le duc de
Popoli, dont j'aurai lieu de parler davantage si j'ai le temps d'écrire
jusqu'à mon ambassade. Albéroni, en attendant, se plaignait
audacieusement de son sort, disait qu'il n'était retenu d'abandonner le
chaos des affaires que par sa tendresse pour le roi et la reine
d'Espagne\,; qu'il trouvait à la vérité des ressources dans la
monarchie, et se livrait à des comparaisons pompeuses, et à se donner de
l'encens, et jusque de l'encensoir.

Les galions arrivèrent tout à la fin de cette année 1717, fort richement
chargés, et apportèrent pour le compte du roi d'Espagne dix-huit cent
mille piastres, secours arrivé fort à propos dans une conjoncture où on
ne voyait point d'alliés à l'Espagne, pour les entreprises qu'elle
méditait.

Albéroni s'épuisait en vain pour s'attirer l'union des Hollandais. Il
les prenait par l'intérêt de leur commerce, par la crainte de la
puissance et des desseins de l'empereur, par la honte de leur servitude
des Anglais, par leur opinion que Georges ne se pouvait maintenir sur le
trône sans l'assistance de la France et la leur. Ce même roi, il le
regardait comme le plus grand ennemi du roi d'Espagne, qui, par son
intérêt de duc d'Hanovre, n'emploierait jamais les forces de
l'Angleterre qu'en faveur de l'empereur, ce qui ne se pouvait selon lui
empêcher qu'en excitant les troubles dans son royaume et dans
l'intérieur de sa cour, qui lui ferait quitter le soin des affaires
étrangères, et terminer bientôt les négociations de Londres. Sur quoi il
disait que la bonté et la modération excessive du roi d'Espagne,
jusqu'alors si peu utile, lui devait servir de leçon pour en changer, et
en devait servir aussi aux autres princes à l'égard des Anglais, que
cette douceur rendait si insolents. De là à braver, à se vanter, à se
louer, à soutenir qu'une conduite tout, opposée était le seul chemin de
la paix, non à la mode de l'empereur et de Georges, mais d'une paix
raisonnable, sûre et solide, telle que le roi d'Espagne l'offrait, et
que la demandaient sa dignité, le bien de ses peuples et celui de toute
la chrétienté.

Ce fut en ces termes que les ministres d'Espagne au dehors eurent ordre
de s'expliquer aux cours où ils résidaient, Cellamare surtout\,;
Monteléon de renouveler à Londres les protestations du désir d'une paix
solide, mais dont la condition principale devait être l'engagement pris
par l'empereur de ne plus tirer de contributions d'aucun prince ni Étai
d'Italie, et de n'y plus envoyer de troupes\,; que le mal devenait tel,
qu'il ne pourrait plus trouver de frein si la paix se faisait en
Hongrie\,; qu'il ne fallait donc pas perdre un moment pour assurer le
bien et le repos de l'Europe. Quoique Albéroni fût bien persuadé de la
partialité du roi d'Angleterre, il affectait de répandre qu'il ne
pouvait croire que la nation Anglaise prît les intérêts de l'empereur
assez à coeur pour se déclarer contre l'Espagne.

Il parlait des Hollandais avec plus d'assurance, se fondant sur
l'intérêt de leur commerce\,; mais il se plaignait qu'ils pussent
compter que l'Espagne leur saurait gré de leurs ménagements et de leur
neutralité, tandis qu'il fallait agir pour assurer la tranquillité de
l'Europe, et prendre des mesures sages telles que l'Espagne se les
proposait, non par des négociations, pour arrêter l'ambition de la cour
de Vienne, sur laquelle il ne ménageait pas les expressions.

Les mesures qu'il prenait consistaient à faire payer les troupes
exactement, à fournir abondamment l'argent pour les recrues, les
remontes, les habits, les armes, l'approvisionnement des places, les
magasins\,; quatre fonderies pour des canons de bronze. On en fabriquait
en même temps de fer, des fusils et toutes sortes d'armes, six vaisseaux
de ligne au Passage, que les constructeurs s'obligèrent à livrer tout
prêts en avril 1719, en attendant une remise envoyée en Hollande de
quatre cent mille piastres pour acheter six navires. Les seuls revenus
du roi d'Espagne suffisaient à ces dépenses sans recourir à aucune voie
extraordinaire. Albéroni se faisait honneur d'avoir connu que le malheur
de l'Espagne venait d'avoir jusqu'alors dépensé prodiguement en choses
inutiles, et de manquer de tout pour les nécessaires. Il s'épuisait sur
ses propres louanges\,; disait que l'Espagne ne se pouvait flatter d'un
accommodement raisonnable si elle ne se montrait armée, espérant
d'obliger les plus indifférents à entrer en danse, et de faire venir à
chacun l'envie de danser par les bons instruments qu'on accordait à
Madrid. Ainsi il était évident qu'il ne songeait qu'à la guerre et point
à traiter\,; que sa répugnance était entière pour la médiation
d'Angleterre\,; qu'il ne traiterait même pas par celle des États
généraux malgré sa prédilection pour eux. Nonobstant ces notions
claires, les Anglais ne laissaient pas de le ménager, et ne
désespéraient pas encore de parvenir à leurs fins. Georges fit
renouveler à la reine et au cardinal tout ce qu'il leur avait déjà fait
promettre en cas de mort du roi d'Espagne.

Sa santé se rétablissait, mais il était plongé dans une mélancolie
profonde, et tellement dévoré de scrupules, qu'il ne pouvait se passer
un moment de son confesseur, quelquefois même au milieu de la nuit.
Albéroni, qui voulait être maître absolu de tous ceux qui approchaient
familièrement du roi d'Espagne, fit venir un médecin de Parme, nommé le
docteur Servi. Il se défiait du premier médecin, chirurgien et
apothicaire du roi, tous trois François, tous trois fort bien dans
l'esprit du roi et de la reine\,; mais le cardinal les trouvait trop
rusés et trop adroits pour les laisser en place. Tous les premiers
ministres se ressemblent en tous pays. La principale qualité d'un
médecin, selon celui-ci et tous les premiers ministres, était de n'être
point intrigant\,; l'intrigue, selon eux, est la peste des cours. Tout
est cabale, et en est qui ils veulent en accuser. Le cardinal prétendait
que celle d'Espagne en était pleine, et se mettait peu en peine de la
capacité du médecin. Celle de Servi était des plus médiocres\,; mais le
hasard y devait suppléer. Le point était qu'il eût du flegme, de la
patience, du courage pour éluder les panneaux et les traits des trois
François, qui ne manqueraient pas de le tourner en ridicule, et s'ils
pouvaient, de le dégoûter assez pour lui faire reprendre le chemin
d'Italie. Il s'en est bien gardé. Il a figuré depuis, et a été premier
médecin de la reine, et puis du roi jusqu'à sa mort, et l'est encore de
la reine sa veuve.

Ces dispositions faites, Albéroni, voyant la santé du roi d'Espagne
rétablie, sentit l'inutilité des offres du roi d'Angleterre. Il y
répondit comme il devait pour la reine et pour lui, mais sans donner au
fond à ces compliments plus de valeur qu'ils n'en méritaient. Il ne
parla pas même au colonel Stanhope d'une proposition que le P. Daubenton
lui avait faite, et à laquelle il n'aurait eu garde de s'avancer sans
l'ordre du cardinal\,: c'était le mariage du prince des Asturies avec
une fille du prince de Galles. Le colonel, qui n'était pas instruit des
intentions du roi son maître, n'osa répondre précisément sur une matière
dont il sentait les difficultés et les conséquences par rapport à la
religion, et à la jalousie que le régent d'une part, et l'empereur de
l'autre, en pourraient prendre. Albéroni donc n'en ouvrit pas la
bouche\,; il se contenta dans ses conférences avec le colonel Stanhope
de lui faire quelques questions sur la personne et le caractère de la
princesse. Ainsi la défiance était mutuelle parmi tous ces témoignages
d'amitié. L'escadre qui s'armait en Angleterre l'augmentait beaucoup.
Monteléon ne le cacha pas au roi d'Angleterre, qui protesta toujours de
son désir de venir à bout de la paix, et que l'escadre ne regardait
point le roi d'Espagne, mais l'insulte que la nation Anglaise avait
reçue en la personne du comte de Peterborough.

Il paraissait plus d'union et de sincérité entre la France et
l'Angleterre. Néanmoins, les ministres de Georges, surtout les
Hanovriens, trouvaient mauvais que le régent se montrât si opiniâtre à
vouloir la renonciation absolue de l'empereur à la monarchie d'Espagne,
et l'assurance des successions de Parme et de Toscane en faveur d'un
fils de la reine d'Espagne. Penterrieder assurait que jamais l'empereur
ne consentirait à l'une ni à l'autre de ces conditions\,; que c'était
une nouveauté directement contraire au plan dont l'abbé Dubois était
convenu lorsqu'il était à Hanovre. Bernsdorff et ceux qui dépendaient de
lui secondaient Penterrieder. Ils traitaient la fermeté et les instances
du régent de dispositions équivoques de la France, et d'irrésolutions
sans fin du régent. Robeton, ce réfugié que Bernsdorff avait insinué
dans les affaires, décidait et déclarait que, si le régent ne se
relâchait sur ces deux articles, il était inutile de négocier\,; que ce
n'était que par des tempéraments qu'on pouvait conduire les choses à une
heureuse fin.

Si les principales puissances intéressées dans la négociation étaient
dans une telle défiance réciproque, le roi de Sicile, plus soupçonneux
et plus persuadé que qui que ce fût que la défiance est une partie
essentielle de la politique, craignait à proportion de son caractère les
effets d'une négociation commencée et conduite à son insu, dont
vraisemblablement une des premières conditions serait de le dépouiller
de la Sicile. On ne lui en avait pas fait encore la moindre ouverture
tout à la fin de cette année. Il se plaignit à l'Angleterre d'un mystère
si long à son égard, qui ne pouvait lui annoncer rien que de mauvais.
Stanhope y répondit qu'il était vrai qu'on avait quelques espérances de
procurer le repos à l'Europe, en particulier à l'Italie, mais si faibles
jusqu'alors et si incertaines, qu'il était impossible de faire aucun
plan et de ne rien dire. La Pérouse représenta que son maître, plein de
confiance pour le roi d'Angleterre, aurait dû en espérer un retour
réciproque. Il assura que ce prince ne plierait jamais mal à propos,
qu'il hasarderait tout plutôt que de souffrir une injustice\,; que
l'Angleterre était garante des avantages qu'elle lui avait procurés par
le traité d'Utrecht\,; qu'ils étaient proprement le fruit des services
qu'il avait rendus pendant la grande alliance\,; qu'ainsi les deux
partis tory et whig étaient également engagés à le maintenir dans la
possession de la Sicile, qu'il avait acquise par la protection de
l'Angleterre. Stanhope répondit en homme embarrassé et qui craignait de
s'engager. Il mit des révérences à la place des raisons\,; dit que
pendant le séjour du roi d'Angleterre à Hanovre il avait agi auprès de
l'empereur pour procurer la paix au roi de Sicile, inutilement à la
vérité, mais que les ministres piémontais en avaient été avertis. Il ne
voulut rien dire de plus précis, et moyennant cette circonspection, il
laissa La Pérouse pleinement persuadé que la France et l'Angleterre
avaient une égale intention de donner atteinte aux traités d'Utrecht. Il
jugea même que le roi d'Espagne ne serait pas fâché que ces traités
fussent enfreints, pour avoir la liberté de recouvrer les États
autrefois dépendants de sa couronne, et pour revenir contre ses
renonciations à celle de France. Enfin La Pérouse, soufflé d'ailleurs
par les émissaires de Penterrieder, se persuada que la France et
l'Espagne s'entendaient ensemble et que le régent n'avait laissé aller
Monti à Madrid que pour gagner Albéroni, et qu'il y avait réussi. Cette
opinion néanmoins contredisait un autre discours tenu quelques jours
auparavant. On disait qu'Albéroni assurait la cour d'Angleterre que si
l'empereur voulait renoncer à l'Espagne et promettre pour un fils de la
reine d'Espagne l'expectative de Toscane et de Parme, le roi d'Espagne
unirait ses forces à celles de l'empereur pour le mettre en possession
de la Sicile.

Ainsi tout conspirait, selon l'opinion publique, à l'agrandissement de
l'empereur. Toutefois ses ministres prétendaient, mais sans faire pitié
à personne, que chacun voulait alors lui faire la loi dans l'empire.
Penterrieder le dit ainsi à Londres à l'occasion d'une déclaration que
le ministre de Moscovie fit à Bernsdorff. Elle portait que le czar ne
pourrait s'empêcher de protéger le duc de Mecklembourg son parent, si on
entreprenait de l'opprimer sous de vains prétextés. On croyait alors que
la paix entre la Suède et la Moscovie serait incessamment conclue, et
comme il n'était question que d'un accommodement particulier, le roi de
Prusse avait lieu de se croire abandonné. Mais le czar démentit les
bruits publics. Il écrivit au roi de Prusse, et l'assura positivement
qu'il détestait les traités secrets, et qu'il n'avait jamais pensé à en
conclure.

C'est en cet état que se trouvaient, à la fin de cette année 1717, les
affaires générales de l'Europe. Elle finit en France par la mort de la
maréchale de Duras à soixante-quinze ou soixante-seize ans, soeur du
dernier duc de Ventadour, fort retirée dans une terre près d'Orléans.
C'était une femme singulière, boiteuse, fort grosse et de beaucoup
d'esprit. J'avais oublié d'en faire mention\,; car elle mourut dès le
mois de septembre. Mais tout à la fin de l'année, on envoya en Bretagne
quatre lettres de cachet, pour ordonner à quatre gentilshommes de
Bretagne qui y avaient paru les plus opposés aux volontés de la cour,
d'y venir rendre compte de leur conduite. Leur nom était MM. de Piré,
Bonamour, Noyan et Guesclairs.

\hypertarget{chapitre-x.}{%
\chapter{CHAPITRE X.}\label{chapitre-x.}}

1718

~

{\textsc{Manèges du duc de Noailles à l'égard de Law.}} {\textsc{- Mort
de Mornay.}} {\textsc{- Duc de Noailles obtient sur-le-champ le
gouvernement et la capitainerie de Saint-Germain.}} {\textsc{- Liaison
de l'abbé Dubois et de Law, et sa cause.}} {\textsc{- Duc de Noailles,
agité de crainte pour sa place, veut me regagner, et me propose de
rétablir le temporel ruiné de La Trappe.}} {\textsc{- Sourds préparatifs
à déposter le duc de Noailles et son ami le chancelier.}} {\textsc{-
Édit en faveur de la compagnie d'Occident\,; quel.}} {\textsc{- Le
régent travaille à la Raquette avec Law, le chancelier et le duc de
Noailles, sur lequel il achève de s'indisposer.}} {\textsc{- La Raquette
et les Biron.}} {\textsc{- Grâce pécuniaire au Languedoc, d'où Bâville
se retire avec douze mille livres de pension.}} {\textsc{- Inondations
vers le nord.}} {\textsc{- M\textsuperscript{me} la Duchesse enlève à
M\textsuperscript{me} la maréchale d'Estrées une loge à l'Opéra.}}
{\textsc{- Morville ambassadeur en Hollande.}} {\textsc{- Mariage de
Chauvelin depuis si haut et si bas.}} {\textsc{- Grâces pécuniaires aux
comtes de Roncy et de Médavy.}} {\textsc{- Le comte de Rieux s'excuse au
régent de ses pratiques.}} {\textsc{- Son caractère.}} {\textsc{-
Mouvements, lettres et députation de Bretagne.}} {\textsc{- Incidents du
maréchal de Montesquiou.}} {\textsc{- Gentilshommes bretons, mandés,
puis exilés.}} {\textsc{- Embarras et projets sur les tailles.}}
{\textsc{- On me fait\,; par deux différentes fois, manquer la
suppression de la gabelle.}} {\textsc{- Tout bien impossible en
France.}} {\textsc{- Manèges d'Effiat et du premier président.}}
{\textsc{- Duperie du régent.}} {\textsc{- Conspiration très organisée
pour le culbuter.}} {\textsc{- Mouvements du parlement.}} {\textsc{-
Singulière colère et propos entre M. le duc d'Orléans et moi sur les
entreprises du parlement.}} {\textsc{- Manèges contre Law du duc de
Noailles et du chancelier.}} {\textsc{- Ma conduite à cet égard.}}
{\textsc{- Abbé Dubois lié de plus en plus avec Law contre le duc de
Noailles.}} {\textsc{- Son double intérêt.}} {\textsc{- Caractère
d'Argenson.}} {\textsc{- Raisons qui me déterminent pour Argenson, à qui
je fais donner les sceaux et les finances.}} {\textsc{- Je l'en avertis
la veille, et tâche de le capter en faveur du cardinal de Noailles.}}
{\textsc{- Le chancelier perd les sceaux\,; est exilé à Fresnes.}}
{\textsc{- Le duc de Noailles se démet des finances\,; entre au conseil
de régence.}} {\textsc{- Argenson a les finances et les sceaux.}}
{\textsc{- Politesse fort marquée d'Argenson à mon égard.}} {\textsc{-
Courte digression sur le chancelier.}} {\textsc{- Survivance de la
charge et des gouvernements du duc de Noailles donnée à son fils enfant,
sans l'avoir demandée.}} {\textsc{- Rouillé quitte les finances avec
douze mille livres de pension.}} {\textsc{- Marchault lieutenant de
police\,; son caractère.}} {\textsc{- Grâces faites à Châteauneuf\,; à
Torcy, qui marie sa fille à Duplessy-Châtillon\,; au duc d'Albret, qui
veut épouser la fille de Barbezieux.}}

~

Un événement, que nous verrons bientôt, puisqu'il arriva le 28 janvier
de cette année 1718, en laquelle nous allons entrer, m'a paru mériter
d'en approcher les choses un peu précédentes qui l'ont préparé, et de
préférer pour cette fois une suite plus éclaircissante des choses qui
l'ont amené, à un scrupule trop exact des temps même peu éloignés, et
qui aurait fait perdre de vue ce qui peu à peu a produit l'événement,
lorsqu'il sera temps de le raconter.

On a vu (ci-dessus, p.~129), la brouillerie du duc de Noailles et de
Law, le replâtrage qui s'y fit, le gré sensible que M. le duc d'Orléans
sut au duc de Noailles de sa complaisance et de ses protestations à cet
égard, et l'âpreté avec laquelle il en sut profiter pour en tirer le
gouvernement et la capitainerie de Saint-Germain, qu'il avait toute sa
vie muguetée, et que la fortune lui livra précisément dans ce favorable
instant par la prompte mort de Mornay sans enfants. Il y avait longtemps
que Noailles\,; jaloux de Law, troublait sa banque et ses desseins. Non
seulement il le barrait en tout par les manoeuvres et l'autorité de sa
place dans les finances\,; mais il lui suscitait dans les conseils et
dans le parlement tous les contradicteurs qu'il pouvait, et qui très
souvent arrêtaient et faisaient même échouer ses propositions les plus
raisonnables. Law, qui, comme je l'ai expliqué, venait chez moi tous les
mardis matin, m'en faisait continuellement ses plaintes, et m'en
prouvait d'autant plus aisément la raison et le mal que faisait aux
affaires cette contradiction perpétuelle, qu'on a vu, d'une part,
comment j'étais avec le duc de Noailles, et, d'autre part, mon
incapacité souvent avouée sur la matière des finances. Mais il y a
pourtant des choses qui dépendent quelquefois plus du bon sens que de la
science\,; et de plus Law, avec un langage fort écossais, avait le rare
don de s'expliquer d'une façon si nette, si claire, si intelligible,
qu'il ne laissait rien à désirer pour se faire parfaitement entendre et
comprendre.

M. le duc d'Orléans l'aimait et le goûtait. Il le regardait et tout ce
qu'il faisait comme l'ouvrage de sa création. Il aimait de plus les
voies extraordinaires et détournées, et il s'y attachait d'autant plus
volontiers, qu'il voyait échapper les ressources devenues si nécessaires
à l'État, et toutes les opérations ordinaires des finances. Ce goût du
régent blessait Noailles comme étant pris à ses dépens. Il voulait être
seul maître dans les finances. Law y avait une partie indépendante.
Cette partie plaisait au régent, et Noailles qui le prétendait gouverner
et atteindre par là au premier ministère, dont il ne perdit jamais les
vues ni l'espérance, trouvait en Law un obstacle dans sa propre gestion,
lui qui empiétait tant qu'il pouvait sur toutes celles des autres.
Toutes ses bassesses sans fin et sans mesure prodiguées au maréchal de
Villeroy n'avaient pu l'accoutumer à n'être que de nom à la tête du
conseil des finances. Ainsi il protégeait souvent Law contre lui, encore
qu'il n'aimât pas au fond ce que le régent pouvait rendre utile, et
qu'il fomentât sous main les mouvements sourdement commencés du
parlement, à qui il fallait des prétextes, et qui se proposait bien de
s'en faire un de la gestion des finances et de la singularité de celle
de cet étranger.

L'abbé Dubois, qui, pour regagner l'esprit de M. le duc d'Orléans, avait
eu besoin d'entours, ne se fut pas plutôt emparé de lui par ses
négociations avec l'Angleterre et la Hollande, que ceux dont il s'était
servi lui devinrent suspects dès que son crédit n'eut plus besoin du
leur. Son plan allait aussi au premier ministère. Il n'y voulait point
de concurrents ni de contradicteurs. Celui de tous qu'il craignait
davantage était le duc de Noailles, parce qu'il avait le même dessein et
bien d'autres moyens que lui pour s'y porter. Il résolut donc de
l'écarter de bonne heure sans rien marquer de personnel. La partie eût
été trop inégale, et d'ailleurs la soumission du duc de Noailles, qui
augmentait pour lui à la mesure du crédit qu'il reprenait auprès de son
maître lui en ôtait jusqu'au prétexte. On a vu combien pour lui plaire
il avait mérité les louanges des Anglais. Dubois se lia donc avec Law.
Leurs intérêts à former cette union étaient pareils. Un étranger, aboyé
d'un nombre de gens autorisés par leur être, par leur état, par leurs
places, avait à chaque instant tout à craindre de la faiblesse du
régent. En le favorisant Dubois flattait le goût de son maître et
portait indirectement des bottes à Noailles qu'il voulait perdre, sans
oser le montrer et sans que Noailles s'en doutât lui-même, ni dans ces
commencements le régent non plus avec tous ses soupçons. Tout se passait
à cet égard dans un intérieur que tout l'art de Noailles ne pouvait
percer.

Law ne me cacha point cette liaison naissante et l'usage qu'il
commençait à en tirer, mais il ne me disait pas ce qu'il lui en coûtait
pour l'accroître et pour la rendre tout à fait solide. Il commençait à
avoir de l'argent à répandre par ce négoce naissant, si connu depuis et
si fatal par l'abus qui s'en fit sous le nom de Mississipi. Il était
doux à l'abbé Dubois de trouver une ressource secrète dont il n'eût
obligation à personne qu'à celui qui avait autant d'intérêt, pour sa
propre défense, d'acheter sa protection, que de lui l'accorder à ce prix
et les moyens en même temps d'énerver de bonne heure un compétiteur à la
première place de toute autorité et de toute grandeur, à la cheville du
pied duquel il ne pouvait encore atteindre.

Telle fut la chaîne qui serra l'amitié entre ces deux hommes et qui les
a portés si haut ou si loin l'un et l'autre. Je ne sais si, à travers
les ruses et les caresses de Dubois, Noailles s'aperçut de quelque
chose, car l'odorat de tous les deux était bien fin. Ce qui me l'a fait
soupçonner, c'est ce qui m'arriva et qui, à la façon dont j'étais avec
le duc de Noailles, ne lui parlant et ne le saluant jamais et ne lui
épargnant pas, comme on l'a vu, les algarades publiques, me jeta dans le
dernier étonnement.

Vers la fin de l'été de 1717, étant un samedi après dîner au conseil de
régence pour finance, assis, à mon ordinaire, entre le comte de Toulouse
et le duc de Noailles, il se mit la bouche dans mon oreille tandis qu'on
commençait à opiner sur une affaire qu'il venait de rapporter et me
demanda si je n'étais pas toujours fort ami de l'abbaye de la Trappe\,;
un oui tout court, et sans plus que ce monosyllabe, fut toute ma
réponse. «\,Mais, ajouta-t-il, ne sont-ils pas fort mal dans leurs
affaires\,? --- On ne saurait plus, répondis-je. --- Mais seriez-vous
bien aise, continua-t-il, de les rétablir\,? --- Il n'y a rien, dis-je,
que je ne souhaitasse davantage. --- Oh bien, monsieur, me dit-il,
j'aime aussi beaucoup l'abbaye des Septfonts, qui n'est pas mieux dans
ses affaires\,; ayez la bonté de demander à la Trappe un état de leurs
dettes et de me le donner, et j'espère trouver moyen de les raccommoder
l'une et l'autre.\,» Je lui dis, mais sans aucune sorte de remerciement,
que j'en serais fort aise et que j'écrirais à la Trappe. Les opinions
vinrent à nous et il n'en fut pas dit davantage, même en nous levant du
conseil.

Le samedi au soir était justement le jour d'y écrire. Je reçus en
réponse l'état que je demandais, et je le donnai le samedi suivant au
duc de Noailles. En le recevant, assis en place, il me dit de ne rien
faire, et qu'il m'avertirait. Le samedi d'après, étant en place, il me
dit qu'il avait prévenu M. le duc d'Orléans, et que je ferais bien de
lui parler. Je le fis et avec succès, tant la voie se trouva aplanie.
Quinze jours après les payements commencèrent à couler par Law. C'était
la chose qui me tenait le plus au coeur, et sur laquelle je savais le
moins comment m'y prendre avec un homme fait comme l'était M. le duc
d'Orléans. La Providence y pourvut de la sorte d'une façon bien
singulièrement marquée\,: il n'est pas temps d'aller plus loin
là-dessus.

Le reste de l'année 1717 s'écoula en démêlés continuels entre Law et les
finances, c'est-à-dire le duc de Noailles, Rouillé et ceux dont ils se
servaient le plus, et en plaidoyers que Law était forcé d'aller faire
chez les principaux des conseils et du parlement. L'abbé Dubois, revenu
de Londres à Paris où il passa jusqu'au mois de janvier, en sut
profiter.

Le chancelier n'avait pas réussi dans cette grande place. Sa servitude
pour le duc de Noailles fit peur à tout le monde, jusqu'à M. le duc
d'Orléans. Son louche et son gauche en matière d'État le déprisa
beaucoup. Son esprit incertain, esclave des formes, puant le parquet en
matière de justice et de finance, ennuya et souvent impatienta\,; ses
hoquets continuels à arrêter les opérations de Law déplurent et
donnèrent beau jeu à l'abbé Dubois de s'espacer. Comme il connaissait le
terrain, il parla au maréchal de Villeroy, à qui il faisait extrêmement
sa cour, et l'aiguillonna à parler au régent. Il me montra aussi assez
où il en voulait venir sur le duc de Noailles pour m'exciter à en
profiter, et Law m'y exhortait pour la nécessité et le bien des
affaires, qui, indépendamment de celles que Noailles gâtait entre ses
mains, périssaient entre les siennes. Le publie, indigné de la dureté de
sa gestion, de l'insolence et des indécences brutales de Rouillé, criait
bien haut\,; les travailleurs effectifs du conseil des finances n'en
louaient pas la besogne. Dubois et Law cavaient en dessous auprès du
régent et faisaient tout valoir. Villeroy, avec un air d'autorité
modeste, se mesurait par eux auprès de lui, et frappait ses coups. Le
régent m'en parlait quelquefois, quoique en garde contre ma haine. Je
fus peut-être celui de tous qui lui fis le moins de mal, mais je savais
par Law et par le maréchal de Villeroy tout ce qui se faisait jour par
jour, et quelquefois, quoique avec plus de réserve, par l'abbé Dubois.
En voilà assez pour la préparation et pour servir de préface à l'année
1718 dans laquelle nous allons maintenant entrer.

Cette année 1718 s'ouvrit, dès le premier jour, par la publication de
l'édit en faveur de la compagnie d'Occident. Son fonds y fut fixé à cent
millions, et tout y fut déclaré non saisissable, excepté les cas de
banqueroute ou de décès des actionnaires. C'est ce nom qui fut enfin
substitué à celui de Mississipi, qui ne laissa pas de prévaloir, dont
les actions ruinèrent et enrichirent tant de gens, et où les princes et
les princesses du sang, surtout M\textsuperscript{me} la Duchesse, M. le
Duc et M. le prince de Conti trouvèrent plus que les mines du Potosi,
dont la durée entre leurs mains a fait celle de cette compagnie si
funeste à l'État dont elle a détruit tout le commerce. La protection
qu'ils lui ont toute leur vie donnée et publique, envers et contre tous,
pareille aux profits immenses qu'ils en ont tirés sans partage d'aucune
perte, l'a maintenue à tous risques et périls, et après eux les
puissants magistrats des finances qui en ont eu la conduite et l'engrais
jusqu'à présent.

Le régent de plus en plus aiguillonné et importuné des entraves
continuelles que le duc de Noailles mettait aux opérations de Law, et
des points sur les \emph{i} qu'y mettait son ami le chancelier, qui
ajoutait un poids qui accablait Law par l'autorité de sa charge et par
celle de sa personne, dont la réputation était lors tout entière, le
régent, dis-je, embarrassé à l'excès de ces deux adversaires qui
arrêtaient tout, l'un pour le fond, l'autre pour la forme, et malgré ces
obstacles déterminé aux vues et aux routes de cet Écossais voulut faire
un dernier effort pour les rapprocher de Law et pénétrer lui-même ce
qu'il y avait devrai et de bon de part et d'autre. Ce fut pour y
travailler sans distraction, avec plus d'application et de loisir, qu'il
voulut aller passer avec eux toute une après-dînée à la
Raquette\footnote{Le manuscrit porte \emph{la Raquette} et non \emph{la
  Roquette}, comme ont lu les précédents éditeurs. D'anciens plans de
  Paris désignent sous le nom de \emph{Raquette}, une maison de
  plaisance située au faubourg Saint-Antoine et entourée de vastes
  jardins. Le nom de Roquette a fini par prévaloir et s'applique encore
  aujourd'hui à une maison d'hospitalières du faubourg Saint-Antoine et
  à une rue qui aboutit à la place de la Bastille.}, où le duc de
Noailles lui donna ensuite à souper. Ce fut le 6 janvier.

La Raquette est une dépendance du faubourg Saint-Antoine, où le duc de
Noailles avait emprunté une fort jolie maison d'un financier appelé du
Noyer, recrépi d'une charge de greffier du parlement. Ce richard, pour
ses péchés, s'était dévoué à la protection des Biron qui, en bref, le
sucèrent si parfaitement qu'il est mort sur un fumier, sans que pas un
d'eux en ait eu souci ni cure. C'était leur coutume\,; plusieurs autres
les ont enrichis de toute leur substance, et en ont éprouvé le même
sort. M\textsuperscript{me} de Biron en riait comme d'une fine
souplesse, et comptait leur avoir fait encore trop d'honneur.

Le chancelier et Law se rendirent de bonne heure à la Raquette. La
séance y fut longue et appliquée de tous côtés\,; mais elle fut
l'extrême-onction des deux amis. Le régent prétendit n'avoir trouvé que
mauvaise foi dans le duc de Noailles, aheurtement aveugle dans le
chancelier esclave de toutes formes contre des raisons péremptoires et
les ressources évidentes de Law. Je l'ai déjà dit, cet Écossais, avec
une énonciation de langue peu facile, avait une netteté de raisonnement
et un lumineux séduisant, avec beaucoup d'esprit naturel qui, sous une
surface de simplicité, mettait souvent hors de garde. Il prétendait que
les obstacles qui l'arrêtaient à chaque pas faisaient perdre tout le
fruit de son système, et il en sut si bien persuader le régent, que ce
prince les força tous pour s'abandonner à lui.

Les esprits qui commençaient à s'échauffer en plus d'une province, par
les pratiques sourdes qui s'y faisaient, eurent part à une diminution de
huit cent mille livres sur la capitation, et à quelques autres grâces
accordées aux états de Languedoc. Bâville, depuis trente ans roi et
tyran de cette grande province sous le nom d'intendant, y contribua
beaucoup\,; il en était la terreur et l'horreur, si on en excepte un
bien petit nombre de personnes. Sa surdité était venue à un point qu'on
ne pouvait presque plus s'en faire entendre. Il voulut quitter un emploi
qu'il ne pouvait plus exercer, et il désira en sortir avec une apparence
de bonne grâce de la province en lui procurant ce soulagement. Il
revint, en effet, quelque temps après avec une pension de douze mille
livres, et vécut le reste de sa carrière à Paris sans aucune fonction,
dont ses oreilles le rendaient incapable, fort retiré dans sa famille,
et ne voyant que quelques amis particuliers. C'était un dangereux homme,
que les ministres avaient toujours tenu éloigné en le consolant par une
autorité absolue, et une des meilleures têtes qu'il y eût en France,
dont la capacité et le naturel absolu, avec beaucoup d'esprit, se fit
également craindre de tous les gens successivement en place.

On apprit que la mer avait rompu les digues de la Nort-Hollande et
inondé beaucoup de pays, et que les environs de Hambourg avaient essuyé
une pareille disgrâce.

M\textsuperscript{me} la Duchesse enleva de haute lutte une petite loge
à l'Opéra, qu'avait la maréchale d'Estrées, quoique amie de toute sa vie
et dans le commerce le plus intime avec les soeurs du maréchal, et fort
bien avec les Noailles. Cela fit grand bruit, et tout ce qui tenait aux
Estrées cessa de voir M\textsuperscript{me} la Duchesse. On eut recours
au régent pour décider, qui ne voulut point s'en mêler. Pareille chose
avait toute la grâce de la nouveauté, même de n'avoir jamais été
imaginée. Mais ce qu'on n'eût osé sous le feu roi, quelque indulgent
qu'il fût à ses filles et au respect des princes du sang, se hasarda
après d'autres essais de la patience et de la timidité du monde.
M\textsuperscript{me} la duchesse laissa crier et garda sa conquête. Peu
à peu ceux qui avaient cessé de la voir y retournèrent, et le maréchal
et la maréchale d'Estrées, après s'être assez longtemps soutenus,
lâchèrent pied comme les autres. Ainsi la hauteur des princes du sang
monta fort au-dessus de celle du feu roi même, qui se piqua toujours
d'être fort considéré, jusque dans les choses de cette nature, pour
contenir tout dans l'ordre et la raison, et qui ne souffrait ces
entreprises dans qui que ce pût être, au point que les plus grands de
son sang ne s'y hasardèrent jamais.

Morville, procureur général du grand conseil, fils d'Armenonville,
vendit sa charge à Héraut, avocat du roi au Châtelet, et fut nommé
ambassadeur en Hollande à la place de Châteauneuf, qui déplaisait aux
Anglais, et qui demandait son retour. Je parle de la vente de cette
charge parce qu'on a vu depuis Morville secrétaire d'État des affaires
étrangères, et Héraut, lieutenant de police, se signaler par son
inquisition.

Chauvelin, avocat général, si connu depuis par l'essor de sa fortune et
la profondeur de sa chute\footnote{Voy., sur ce Chauvelin, les notes à
  la fin du t. XII. On y trouvera, avec le nom de sa femme, une
  appréciation assez piquante des moyens par lesquels Chauvelin s'éleva
  à la dignité de garde des sceaux et de ministre des affaires
  étrangères.}, épousa la fille et nièce des plus riches marchands
d'Orléans, belle et bien et noblement faite. Elle avait été promise à un
autre, qu'elle-même aurait voulu épouser. L'autorité de magistrature
s'en mêla et l'emporta. Mais la peur qu'ils eurent de quelque parti
violent fit garnir par le guet tout le chemin de la maison à la
paroisse, ce qui parut fort étrange\,: autre entreprise qui ne se serait
pas tentée sous le feu roi. M\textsuperscript{me} Chauvelin s'est fait
considérer par sa conduite et sa vertu, et a eu à la cour un maintien
qui l'a fait estimer, et qui s'est bien soutenu dans la disgrâce en
vivant également bien avec un mari qu'elle n'avait pas choisi.

Le comte de Roucy, fort mal dans ses affaires, arracha cinquante mille
écus du régent en billets d'État, et Médavy cinquante mille livres sur
une vieille prétention d'un brevet de retenue du maréchal de Grancey,
son grand-père, sur le gouvernement de Thionville.

Le comte de Rieux eut une audience du régent, pour se justifier d'avoir
animé la noblesse de Bretagne. Il y avait conservé, malgré sa pauvreté,
beaucoup de considération et de crédit, qu'il entretenait par beaucoup
d'esprit et de manèges. Homme obscur, très glorieux de sa grande
naissance, toujours travaillant en dessous sans se commettre, lié
sourdement avec des personnages et avec la maison de Lorraine, plein des
plus hautes pensées et des plus grands projets, heureux à se faire des
dupes par son langage, ennemi de tout gouvernement, désireux de faire
jouer des mines, et peu retenu par l'honneur, la probité, la vérité,
sous le masque des plus vertueux propos. Tout se cuisait de loin en
Bretagne. On y flattait les Bretons d'une conquête d'indépendance qui ne
serait due qu'à leur union et à leur fermeté. Rieux était à Paris leur
homme de confiance\,; ils ne pouvaient la placer mieux, par l'intérêt
qu'il avait, et qu'il se proposait de faire tout à coup une grande
figure, et il avait assez d'esprit pour y parvenir, quoiqu'il n'eût
jamais vu la guerre, ni la cour, ni le grand monde, si l'affaire eût
réussi.

La noblesse de Bretagne écrivit une lettre au régent, soumise et
respectueuse en apparence, plus que forte en effet, dont les copies
inondèrent Paris. Deux présidents et quatre conseillers, députés du
parlement de Bretagne, arrivèrent avec une lettre de, ce parlement au
régent, en même sens que celle de la noblesse. Ces députés furent admis,
après plusieurs jours, à faire la révérence au régent, mais sans lui
parler d'aucune affaire. Le maréchal de Montesquiou, commandant en
Bretagne, en avait plusieurs de procédés avec ce parlement, qui en
cherchait et entreprenait. Le maréchal, de son côté, avait très mal
débuté avec la noblesse. Quatre ou cinq cents gentilshommes étaient
allés au-devant de lui à quelque distance de Dinan. Au lieu de s'arrêter
à eux, et de monter à cheval pour entrer avec eux à Rennes, il se
contenta de mettre la tête hors sa chaise de poste, et de continuer son
chemin. La noblesse, avec raison, en fut extrêmement choquée. Néanmoins
il en alla un grand nombre le prendre chez lui pour l'accompagner au
lieu des états pour leur ouverture. Au lieu d'y aller à pied avec eux,
il monta dans sa chaise à porteurs, et acheva ainsi de les offenser,
tellement que tout se tourna en procédés, et presque en `insultes. MM.
de Piré, Noyan, Bonamour et du Guesclairs, venus par lettre de cachet à
la cour rendre compte de leur conduite, furent exilés séparément en
Bourgogne, Champagne et Picardie. Piré, demeuré malade en Bretagne,
évita le voyage de Paris et l'exil.

Les désordres inévitables de la manière de lever les tailles occupaient
d'autant plus le régent, que la fermentation devenait palpable, dans le
parlement et dans quelques provinces. On avait voulu établir la taille
proportionnelle dans la généralité de Paris. Plusieurs personnes y
travaillaient depuis un an, sans autre succès qu'une dépense de huit
cent mille livres. On pensa ensuite à la dîme royale du maréchal de
Vauban, qu'on donna à rectifier à l'abbé Bignon et au petit Renault, qui
s'offrit d'aller à ses dépens en faire des essais dans quelques
élections, et qui dans la suite y alla en effet. Tous ces essais furent
funestes par la dépense qu'ils causèrent sans aucun succès. Soit que les
projets fussent vicieux en eux-mêmes, soit qu'ils le devinssent par la
manière de les exécuter, peut-être encore par les obstacles qu'y
semèrent l'intérêt et la jalousie de la cruelle gent financière,
toujours appuyée des magistrats des finances, il est certain que les
bonnes intentions du régent, qui en cela ne cherchait que le soulagement
du peuple, furent entièrement trompées, et il en fallut revenir à la
manière ordinaire de lever les tailles.

Quoique je n'aie jamais voulu me mêler de finances, je n'ai pas laissé
d'avoir une expérience personnelle de ce que je viens de dire des
financiers, et des intendants et autres magistrats des finances. J'étais
demeuré frappé de ce que le président de Maisons m'avait expliqué et
montré sur la gabelle, de l'énormité de quatre-vingt mille hommes
employés à sa perception, et des horreurs qui se pratiquent là-dessus
aux dépens du peuple. Je l'étais encore de cette différence de provinces
également sujettes du roi, dans une partie desquelles la gabelle est
rigoureusement établie, tandis que le sel est franc dans les autres,
dont le roi ne tire pas moins pourtant, et qui jouissent d'une liberté à
cet égard qui fait regarder avec raison les autres comme étant dans la
plus arbitraire servitude de tous les fripons de gabeleurs, qui ne
vivent et ne s'enrichissent que de leurs rapines. Je conçus donc le
dessein d'ôter la gabelle, de rendre le sel libre et marchand, et pour
cela de faire acheter par le roi, un tiers plus que leur valeur, le peu
de salines qui se trouvent appartenir à des particuliers\,; que le roi
les eût toutes\,; qu'il vendît tout le sel à ses sujets, au taux qui y
serait mis, sans obliger personne d'en acheter plus qu'il ne voudrait.
Il n'y avait guère que les salines de Brouage à acquérir. Le roi
gagnait, par la décharge des frais de cette odieuse ferme, et outre tout
ce que le peuple y gagnait par la liberté, et l'affranchissement des
pillages sans nombre qu'il souffre de ce nombre monstrueux d'employés,
qui mourraient de faim s'ils s'en tenaient à leurs gages\,; l'État y
aurait considérablement profité du côté des bestiaux, comme il se voit à
l'oeil, par la différence de ceux à qui on donne un peu de sel, dans les
pays qui n'ont point de gabelle, d'avec ceux à qui la cherté de la
contrainte du sel empêche d'en donner.

Je le proposai au régent qui y entra avec joie. L'affaire, mise sur le
tapis, allait passer, quand Fagon et d'autres magistrats des finances
qui n'avaient pu s'y opposer d'abord, prirent si bien leurs mesures
qu'ils firent échouer le projet. Quelque temps après j'y voulus revenir,
et j'eus tout lieu de croire la chose assurée et qu'elle serait faite
dans la huitaine. Les mêmes, qui en eurent le vent, la firent encore
avorter. Outre les avantages que je viens d'expliquer, c'en eût été un
autre bien essentiel de réduire cette armée de gabeleurs, vivant du sang
du peuple, à devenir soldats, artisans ou laboureurs.

Cette occasion m'arrache une vérité que j'ai reconnue pendant que j'ai
été dans le conseil, et que je n'aurais pu croire, si une triste
expérience ne me l'avait apprise, c'est que tout bien à faire est
impossible. Si peu de gens le veulent de bonne foi, tant d'autres ont un
intérêt contraire à chaque sorte de bien qu'on peut se proposer. Ceux
qui le désirent ignorent les contours, sans quoi rien ne réussit, et ne
peuvent parer aux adresses ni au crédit qu'on leur oppose, et ces
adresses appuyées de tout le crédit des gens de maniement supérieur et
d'autorité, sont tellement multipliées et ténébreuses, que tout le bien
possible à faire avorte nécessairement toujours. Cette affligeante
vérité, et qui sera toujours telle dans un gouvernement comme est le
nôtre, depuis le cardinal Mazarin, devient infiniment consolante pour
ceux qui sentent et qui pensent, et qui n'ont plus à se mêler de rien.

La fermentation du parlement augmentait à mesure que les espérances
augmentaient du côté de la Bretagne. Cette compagnie, qui a toujours
voulu troubler et se mêler du gouvernement avec autorité pendant les
régences, avait un chef qui voulait figurer, qui était également
nécessiteux et prodigue, qui, dans son ignorance parfaite de son métier
de magistrat, avait les propos à la main, l'art de plaire quand il
voulait, et la science du grand monde\,; que les paroles les plus
positivement données, que l'honneur, que la probité ne retenaient
jamais, et qui regardait la fausseté et l'art de jouer les hommes comme
une habileté, même comme une vertu dont on ne se pouvait passer dans les
places\,: en ce dernier point malheureusement homogène au régent jusqu'à
lui avoir su plaire par un endroit qui aurait dû lui ôter toute
confiance.

Livré, comme on l'a vu, pieds et poings liés au duc et à la duchesse du
Maine, il était informé des progrès de ce qu'ils brassaient en Bretagne
et partout, et il mettait tout son art à se conduire au parlement en
conséquence, mais avec les précautions nécessaires pour se le rendre au
régent et tout à la fois le rançonner et le trahir. Il y avait d'autant
plus de facilité. que d'Effiat était toujours l'entremetteur dont le
régent se servait sur tout ce qui regardait le parlement, d'Effiat,
dis-je, tout dévoué de longue main au duc du Maine, et accoutumé à
trahir son maître dès le temps du feu roi, de concert avec le duc du
Maine, comme on l'a vu lors de la mort de Mgr {[}le Dauphin{]} et de
M\textsuperscript{me} la Dauphine, et toujours depuis. Ainsi le régent,
avec tout son esprit, avait mis toute sa confiance en deux scélérats qui
s'entendaient pour le trahir et le jouer sans qu'il s'en voulût douter
le moins du monde, persuadé que l'argent immense que le premier
président tira de lui à maintes fortes reprises était un intérêt
supérieur à tout, qui l'attachait à lui en effet, en ne gardant pour M.
du Maine que les apparences nécessaires de l'ancienne amitié. D'Effiat,
intime du premier président et du duc du Maine, l'entretenait dans cette
duperie pour continuer la pluie d'or dans la bourse du premier président
et une confiance nécessaire aux desseins de ses deux amis. Tel fut
l'aveuglement d'un prince qui se persuadait que tout était fripon,
excepté le très petit nombre de ceux que l'éducation avait trompés et
raccourcis, et qui aimait mieux se servir de fripons connus pour tels
que d'autres, persuadé qu'il les manierait mieux et qu'il s'en
laisserait moins tromper. Cette préface est nécessaire à ce qui est
raconté ici entré le régent et le parlement. Tout se préparait ainsi à
donner bien des affaires au régent et à le culbuter.

Les menaces au dedans et au dehors par l'Espagne s'avançaient vers le
but que l'ambition et la vengeance se proposaient, et que les prestiges
répandus avec art parmi les fous, les ignorants et les sots, qui font
toujours le très grand nombre, avançaient à souhait. L'intelligence
entre Albéroni et M. et M\textsuperscript{me} du Maine était parfaite.
Leurs liaisons prises dès le temps du feu roi, de M. de Vendôme, de la
campagne de Lille, avaient toujours subsisté. L'art employé alors contre
Mgr. le duc de Bourgogne, et depuis, à sa mort, contre M. le duc
d'Orléans, fut toujours le même et toujours soutenu, et plus ou moins
entretenu. On a vu, en parlant des affaires étrangères, quel était le
génie d'Albéroni, sa toute puissance en Espagne, sa haine personnelle
pour M. le duc d'Orléans, qui avait encore la simplicité de faire
entretenir commerce avec lui par d'Effiat, son ancien ami, par les
bâtards, enfin la passion du roi et de la reine d'Espagne de venir
régner en France s'il arrivait faute du roi, et celle d'Albéroni de leur
plaire en flattant ces idées, en en préparant les voies, et en servant
la haine qu'il entretenait en eux contre le régent, tant sur les choses
personnelles et anciennes, que sur les modernes, en empoisonnant les
démarches les plus innocentes du régent, même les plus favorables à
l'Espagne.

Cellamare, tout occupé de sa fortune, pour laquelle la haine déclarée et
sans mesure des cardinaux del Giudice, son oncle, et Albéroni, le
faisait trembler continuellement, et qu'on a vu lui avoir fait faire
tant de bassesses, n'en était que plus occupé à plaire au formidable
ennemi de son oncle dans le point qui lui était le plus sensible et sur
lequel il était éclairé de si près par le duc et la duchesse du Maine,
l'âme et les inventeurs et promoteurs de tout ce qui se tramait.

Le maréchal de Villeroy, Villars, et bien d'autres gens qui se donnaient
pour fort importants, y donnaient tête baissée par une soif de
considération et de figurer que rien de tout ce que le régent faisait
sans cesse en leur faveur ne pouvait rassasier ni gagner. Le maréchal de
Villeroy, pour marcher mieux en cadence, n'oubliait aucune des plus
énormes messéances pour renouveler et autoriser les anciens bruits. Il
tenait sous la clef le linge du roi, son pain et diverses autres choses
à son usage. Cette clef ne le quittait ni jour ni nuit. Il affectait de
faire attendre après pour qu'on remarquât son soin et son exactitude à
enfermer ces choses et faire sottement admirer de si sages précautions
pour conserver la vie du roi, comme si les viandes et leurs
assaisonnements, sa boisson et mille autres choses dont il se servait
nécessairement, qui ne pouvaient être sous sa clef, n'eussent pu
suppléer au crime. Mais cela faisait et entretenait le bruit, les
soupçons, les discours, augmentait les prestiges et tendait toujours au
but qu'on se proposait. Villeroy, ayant toujours M. de Beaufort dans la
tête et sa royauté des halles, se tenait trop nécessaire pour en essuyer
le sort et le court règne, étant, comme il l'était, soutenu du gros du
public, trop appuyé du parlement qu'il courtisait avec servitude et qui
réciproquement s'appuyait sur lui pour inculquer au roi de bonne heure
toutes ses prétentions et pour faire contre au régent, comme il faisait
tant qu'il pouvait\,; il osait le mépriser d'autant plus qu'il en tirait
plus de grâces et qu'il s'en trouvait plus considéré et, si je l'ose
dire, infatigablement courtisé.

Je voyais clair, dès lors, en la plupart de ces choses, c'est-à-dire au
but de M. du Maine, du parlement, du maréchal de Villeroy, en
éloignement confus encore l'Espagne, et je gémissais en silence de la
mollesse et de l'aveuglement de M. le duc d'Orléans. Outre qu'elle ne
lui était que trop naturelle, la misérable crainte du parlement qui de
longue main l'avait saisi, comme on l'a vu, lui avait toujours depuis
été de plus en plus inculquée par l'intérêt de Canillac, qui s'était
figuré de gouverner cette compagnie par le crédit qu'il croyait avoir
hérité de Maisons et par celui dont se parait sa veuve qui en tenait
chez elle de petites assemblées\,; par la perfidie d'Effiat, qui servait
ses deux amis et qui se rendait un personnage par ses entremises entre
son maître et le parlement auquel il le vendait\,; par la frayeur du duc
de Noailles, si longtemps son instrument pour tout et dont les transes
l'avaient, comme on l'a vu, jeté dans la bassesse de compter des
finances devant des commissaires du parlement, en présence du régent
qu'il y avait entraîné avec lui\,; enfin, par l'écho d'un gros de valets
et de bas courtisans qui voulaient plaire à la mode ou qui connaissaient
la faiblesse de leur maître. Ce prince, dont la confiance en moi n'était
point refroidie, était pourtant en garde contre moi sur tout ce qui
regardait le duc de Noailles, d'Effiat, le premier président et le
parlement et comme je m'en étais bien aperçu depuis longtemps et que
cette prévention rendrait tous mes conseils à ces égards inutiles,
depuis longtemps aussi j'évitais avec grand soin de lui en jamais rien
dire, et si quelquefois il m'en parlait, je répondais vaguement,
courtement, avec une transition prompte et affectée à d'autres choses.

La pièce principale pour l'exécution pourpensée et projetée de toute
cette cabale, était le parlement. Il le fallait remuer par les vues du
bien public, l'exciter par les profusions et les moeurs du régent. Le
système de Law et sa qualité d'étranger de nation et de religion furent
d'un grand usage pour en imposer aux honnêtes gens du parlement et au
gros de cette compagnie. La vanité de devenir les modérateurs de l'État
l'aiguillonnait tout entière. Il fallait cheminer par degrés pour
accoutumer le parlement à une résistance qui aigrît le régent ou qui
l'abattît, dont on pût tirer de grands avantages et se conduire peu à
peu où on tendait, sans que presque personne de ce très grand nombre
qu'on pratiquait partout sût jusqu'où on le voulait mener, et le forcer
après par la nécessité où on l'aurait poussé, des conjonctures et des
engagements. L'autorité des lois et du parlement était un abri
nécessaire à qui voulait le plus les enfreindre. Il en fallait
nécessairement rendre cette compagnie complice pour les violer
impunément\,: tel fut le projet bien suivi et avec toute apparence du
plus grand succès, mais que la Providence, protectrice des États et des
rois faibles et enfants, sut confondre.

Ils trouvèrent donc qu'il était temps de commencer. Le parlement sema
force plaintes pour préparer le public, tant sur les finances et sur
Law, que sur la forme du gouvernement, par les conseils qu'il prétendit
allonger fort les affaires et les rendre beaucoup plus coûteuses
qu'elles n'étaient avant leur établissement. Ces précautions prises, le
parlement s'assembla le matin et le soir du 14 janvier, sous le prétexte
d'enregistrer l'édit de création des deux charges, l'une de trésorier
des bâtiments, l'autre d'argentier de l'écurie, qu'ils avaient longtemps
suspendue, et où ils firent plusieurs modifications. En ces deux
assemblées, qui continuèrent le matin et l'après-dînée du lendemain, ils
résolurent des remontrances et force demandes des plus hardies, et
mandèrent le prévôt des marchands à leur venir rendre compte de l'état
des affaires de l'hôtel de ville. Le premier président et les gens du
roi vinrent rendre compte au régent de ce qui s'était passé au
parlement, au sortir de chacune des deux premières séances.

Les mêmes assemblées continuèrent les deux jours suivants et le
troisième encore, mais chez le premier président, pour rédiger leurs
remontrances par écrit et leurs demandes. Law, sans y être nommé, y
était fortement attaqué, ainsi que l'administration du régent au fond et
en la forme. Elles ne tendaient pas à moins qu'à se mêler de tout avec
autorité, et à balancer celle du régent de manière à ne lui en laisser
bientôt plus qu'une vaine et légère apparence.

Informé à peu près de ce qui se préparait, il m'en parla avec plus de
feu et de sensibilité qu'il n'en avait d'ordinaire. Je ne répondis rien.
Nous nous promenions tout du long de la galerie de Coypel et du grand
salon qui est au bout sur la rue Saint-Honoré. Il insista, et me pressa
de lui parler. Alors je lui dis froidement qu'il savait bien qu'il y
avait longtemps que je ne lui ouvrais pas la bouche sur le, parlement ni
sur rien qui pût y avoir rapport, et que, lorsqu'il m'en avait
quelquefois ouvert le propos, j'en avais toujours changé et évité
d'entrer en aucun discours là-dessus\,; que, puisqu'il me forçait
aujourd'hui, je lui dirais que rien ne me surprenait dans cette
conduite\,; qu'il se pouvait souvenir que je la lui avais prédite, et
que je lui avais dit, il y avait longtemps, que sa mollesse à l'égard du
parlement le conduirait enfin à n'être plus régent que de nom, ou à la
nécessité d'en reprendre l'autorité et les droits par des tours de force
très hasardeux. Là-dessus il s'arrêta, se tourna à moi, rougit, se
courba tant soit peu, mit ses deux poings sur ses côtés, et me regardant
en vraie et forte colère\,: «\, Mort\ldots\,! me dit-il, cela vous est
bien aisé à dire à vous qui êtes immuable comme Dieu, et qui êtes d'une
suite enragée.\,» Je lui répondis avec un sourire et un froid encore
plus marqué que devant\,: «\,Vous me faites, monsieur, un grand honneur
de me croire tel que vous dites\,; mais si j'ai trop de suite et de
fermeté, je voudrais vous en pouvoir donner mon excédant, cela ferait
bientôt deux hommes parfaits, et vous en auriez bon besoin.\,» Il fut
tué à terre, ne répondit mot et continua sa promenade à plus grands pas,
la tête basse, comme il avait accoutumé quand il était embarrassé et
fâché, et ne proféra pas un mot depuis le salon où cela se passa jusqu'à
l'autre bout de la galerie. Au retour, il me parla d'autre chose, que je
saisis avidement pour rompre la mesure sur le parlement.

Le 26 janvier, le parlement alla, sur les onze heures du matin, faire
ses remontrances au roi en présence de M. le duc d'Orléans. Le premier
président les lut tout haut\,: elles étaient de la dernière force contre
le gouvernement, et en faveur des prétentions du parlement, et par
plusieurs demandes qui étaient autant d'entreprises les plus fortes\,:
Le régent ne dit pas un mot\,; le roi, que son chancelier leur rendrait
sa réponse\,; le chancelier, que, lorsque le roi aurait assemblé son
conseil, il leur enverrait ses ordres auxquels il espérait (terme bien
chétif et bien faible) qu'ils obéiraient sans remise.

Le soir même, M. le duc d'Orléans fit répandre force copies des lettres
patentes enregistrées au parlement le 21 février 1641, Louis XIII
présent, qui réduisent le parlement aux termes de son devoir et de son
institution de simple cour de justice pour juger les procès entre les
sujets du roi, sans pouvoir prétendre à plus, et singulièrement à
entrer, ni se mêler en sorte quelconque du gouvernement de l'État, ni
d'aucune de ses parties\,: cette défense et réduction, appuyée de
citations de pareilles du roi Jean, François Ier, Charles IX, et
plusieurs pareilles ordonnances du même Louis XIII. On aurait pu et dû y
en ajouter de Louis XIV, surtout lorsqu'il alla seoir au parlement en
habit gris, une houssine à la main', dont il le menaça en parlant bien à
lui.

Il a fallu faire tout de suite le récit des premières démarches
publiques du parlement, pour n'en pas interrompre un autre, dont
l'événement éclata le lendemain que le premier président eût rendu
compte au parlement de ses remontrances, c'est-à-dire le 28 janvier,
surlendemain du jour qu'il les avait été lire au roi aux Tuileries.

À mesure que le régent se trouvait plus embarrassé, il se rapprochait de
moi sur les gens et les matières sur lesquelles on l'avait mis en garde.
Il m'avait parlé plus d'une fois du duc de Noailles et du chancelier,
avant la séance de la Raquette, de la jalousie du premier contre Law, de
l'ineptie du second en affaires d'État, de finances, du monde. Il ne
m'avait pas caché son dégoût de tous les deux, et d'une union intime qui
rendait en tout et pour tout le chancelier esclave volontaire du duc de
Noailles. Le langage de celui-ci lui plaisait\,: son désinvolte et des
moeurs toujours à la mode, quelle qu'elle fût, le mettaient à l'aise
avec lui. Son esprit et sa tribu si établie lui donnaient de la crainte.
D'autre part, Law et son système était ce dont il ne se pouvait
déprendre par ce goût naturel des voies détournées, et par ces mines
d'or que Law lui faisait voir tout ouvertes et travaillées par ses
opérations. À bout d'espoir de faire compatir ensemble le duc de
Noailles et Law après tout ce qu'il avait fait pour y parvenir, son
malaise devint extrême quand il vit enfin qu'il fallait choisir entre
les deux. Il m'en parla souvent, et j'étais instruit par Law de tout ce
qui se passait là-dessus.

Quel que fût son système, il y était de la meilleure foi du monde\,; son
intérêt ne le maîtrisait point\,; il était vrai et simple\,; il avait de
la droiture\,; il voulait marcher rondement. Il était donc doublement
outré des obstacles qui lui étaient suscités à chaque pas par le duc de
Noailles, et de la duplicité de sa conduite à son égard\,; il ne l'était
pas moins des lenteurs multipliées du chancelier pour, de concert avec
Noailles, arrêter et faire échouer chaque opération\,; il lui fallait
souvent aller persuader des principaux du parlement, son premier
président et celui de la chambre des comptes que Noailles suscitait, et
dont il faisait peur au régent, et il arrivait que, quand Law les avait
persuadés, les ruses ne manquaient pas à Noailles, et les lenteurs
affectées au chancelier, pour rendre inutiles les opérations qui
semblaient résolues et ne trouver plus de difficulté. Law me venait
conter ses chagrins et ses peines, souvent près de tout quitter, et
s'allait plaindre au régent à qui il faisait toucher au doigt tous ces
manèges. Le régent m'en parlait avec amertume, mais ne tirait de moi que
de le plaindre de ces contrastes, et des aveux de mon ignorance en
finance qui m'empêchait de lui donner aucun conseil.

Dès avant le départ de l'abbé Dubois pour l'Angleterre, pressé par Law
et par son double intérêt, il avait porté de rudes coups à Noailles
auprès du régent et au chancelier par contre-coup. Son intérêt en cela
était double\,; il commençait à tirer gros de Law. Ce qu'il en tirait
demeurait dans les ténèbres\,; il pensait déjà au cardinalat, et au
besoin qu'il aurait de forcer d'argent à Rome. C'est ce qu'il ne pouvait
espérer que de Law, et cela seul l'eût entraîné\,; mais il en avait un
autre\,: il voulait dès lors, comme je l'ai déjà expliqué, se préparer à
gouverner seul son maître. Il fallait pour cela écarter de lui peu à peu
ceux qui, de façon ou d'autre, avaient le plus de part en sa confiance.
La charge des finances l'entraînait nécessairement, et lui était
redoutable dans un homme tel que le duc de Noailles\,: Il saisit donc
l'occasion de l'écarter, persuadé qu'après l'éclat de l'avoir sacrifié à
Law, Noailles ne reprendrait plus de, confiance, et ne serait plus un
homme qu'il pût craindre.

Je savais par Law que les coups de Dubois avaient porté, et c'était ce
qui le désolait de son absence. Il eût bien voulu m'engager à y
suppléer\,; mais je connaissais trop les défiances du régent, pour me
presser\,: il me regardait avec raison comme l'ennemi déclaré et sans
mesure du duc de Noailles, mes discours à son égard, auraient porté à
faux. D'ailleurs je me trouvais hors d'état de me décider moi-même sur
le meilleur parti à prendre pour les finances entre eux, et je ne
voulais pas prendre sur moi, quelque haine que j'eusse contre Noailles,
de jeter l'État et le régent entre les bras de Law, et d'un système
aussi nouveau que le sien. Je laissais donc aller les choses, attentif
cependant à en être bien instruit et à me tenir dans un milieu à l'égard
du régent, à ne le pas refroidir de me parler là-dessus avec confiance,
mais surtout à ne me point avancer et à ne me point commettre. Cette
conduite dura jusqu'à la séance de la Raquette, après laquelle je vis le
parti pris, et qui n'était retardé que par la faiblesse qui s'arrête
toujours au moment d'exécuter.

Alors le maréchal de Villeroy s'ouvrit entièrement à moi, comme à
l'ennemi du duc de Noailles, qu'il ne pouvait souffrir par le dépit de
n'être qu'un vain nom dans les finances, dont Noailles avait tout le
pouvoir et l'administration. Le maréchal m'apprit les bottes qu'il lui
portait depuis qu'il le voyait ébranlé, et m'instruisait des divers
avancements de sa chute. Pour l'entretenir à m'informer, je lui disais
ce que je pouvais lui confier sans crainte de ses indiscrétions, et je
voyais un homme ravi de joie, qui n'oubliait rien pour précipiter la
chute de celui dont l'autorité dans les finances lui était si odieuse.

À la fin, M. le duc d'Orléans s'expliqua tout à fait avec moi, et mit en
délibération à qui il donnerait les finances et les sceaux. Son objet
était de disposer des finances, en sorte que Law ne trouvait plus
d'obstacle en ses opérations. Law et moi avions souvent traité cette
matière. Il avait eu souvent recours à d'Argenson, qui était fort entré
dans ses pensées, et c'était à lui qu'il désirait les finances, parce
qu'il comptait être avec lui en pleine liberté.

Argenson était un homme d'infiniment d'esprit et d'un esprit souple,
qui, pour sa fortune s'accommodait à tout. Il valait mieux, pour la
naissance, que la plupart des gens de son état, et il faisait depuis
longtemps la police et avec elle l'inquisition d'une manière
transcendante. Il était sans frayeur du parlement, qui l'avait souvent
attaqué, et il avait sans cesse obligé les gens de qualité, en cachant
au feu roi et à Pontchartrain des aventures de leurs enfants et parents,
qui n'étaient guère que des jeunesses, mais qui les auraient perdus sans
ressource, s'il ne les eût accommodées d'autorité et subitement tiré le
rideau dessus. Avec une figure effrayante, qui retraçait celle des trois
juges des enfers, il s'égayait de tout avec supériorité d'esprit, et
avait mis un tel ordre dans cette innombrable multitude de Paris, qu'il
n'y avait nul habitant, dont jour par jour il ne sût la conduite et les
habitudes, avec un discernement exquis pour appesantir ou alléger sa
main à chaque affaire qui se présentait, penchant toujours aux partis
les plus doux avec l'art de faire trembler les plus innocents devant
lui. Courageux, hardi, audacieux dans les émeutes, et par là maître du
peuple. Ses moeurs tenaient beaucoup de celles qui avaient sans cesse à
comparaître devant lui, et je ne sais s'il reconnaissait beaucoup
d'autres divinités que celle de la fortune. Au milieu de fonctions
pénibles et en apparence toutes de rigueur, l'humanité trouvait aisément
grâce devant lui, et quand il était en liberté avec des amis obscurs et
d'assez bas étage, auxquels il se fiait plus qu'à des gens plus relevés,
il se livrait à la joie, et il était charmant dans ces compagnies. Il
avait quelques lettres, mais peu ou point de capacité d'ailleurs en
aucun genre, à quoi l'esprit suppléait, et une grande connaissance du
monde, chose très rare en un homme de son état\footnote{Voy. sur
  d'Argenson notes à la fin du volume.}.

Il s'était livré sous le feu roi aux jésuites, mais en faisant tout le
moins de mal qu'il lui était possible, sous un voile de persécution
qu'il se sentait nécessaire pour persécuter moins en effet, et secourir
même les persécutés. Comme la fortune était sa boussole, il ménageait
également le roi, les ministres, les jésuites, le public., Il avait eu
l'art\,; comme on l'a vu en son lieu, de se faire un grand mérite auprès
de M. le duc d'Orléans, alors fort maltraité, de ce cordelier amené
d'Espagne par Chalais, qu'il fut chargé d'interroger à la Bastille, et
M. le duc d'Orléans n'avait pu l'oublier. Depuis, il m'avait courtisé
sans bassesse, sans visites, mais dans toutes les choses où il avait pu
me témoigner toute son attention, et il avait bien voulu se laisser
charger du temporel fort dérangé du monastère de la Visitation de
Chaillot en qualité de commissaire, où M\textsuperscript{me} de
Saint-Simon avait une soeur d'un vrai mérite, que nous aimions fort,
monastère d'ailleurs rétabli par la famille de M\textsuperscript{me} la
maréchale de Lorges.

Law avait trouvé beaucoup d'accès auprès de ce magistrat, qui lui-même
s'en était fait auprès de l'abbé Dubois, et qui n'aimait point du tout
M. de Noailles, sans être pourtant mal avec lui. Le parlement lui en
voulait cruellement, dont on a vu des traits bien forts. Sa charge ne le
rendait pas réconciliable avec cette compagnie, et le régent et lui
avaient eu souvent besoin l'un de l'autre. De sa nature il était royal
et fiscal, il tranchait, il était ennemi des longueurs, des formes
inutiles ou qu'on pouvait sauter, des États neutres et flottants. Mais
comme il cherchait à se concilier tout, il avait, du temps du feu roi,
et cultivé depuis, des liaisons avec ses bâtards, beaucoup plus étroites
que nous ne nous en doutions M. le duc d'Orléans et moi.

Cette ignorance, les raisons tirées de ce qui vient d'être expliqué de
son caractère et de sa conduite, beaucoup aussi l'éloignement extrême
qui était entre le parlement et lui dans un temps où il s'agissait
d'avoir le dessus sur cette compagnie, qui se mettait en état de
dominer, me détermina à lui pour les finances et pour les sceaux, afin
de lui donner plus d'autorité, et au régent un garde des sceaux en sa
main, ferme, hardi, et qui, pour sa propre vade\footnote{Pour sa propre
  cause.}, se trouverait intéressé à ne pas ménager le parlement. Je
m'expliquai donc en sa faveur à Law qui goûta infiniment mes raisons, et
au régent à qui je les détaillai. La chose demeura entre nous trois et
fut bientôt déterminée. Alors je pressai le régent de finir, dans la
crainte de quelque transpiration qui déconcertât la résolution prise, et
le coup à frapper fut fixé au vendredi 28 janvier pour laisser passer
les remontrances du parlement au roi, dont j'ai parlé avant ceci.

Je priai le régent de me permettre d'avertir et de disposer Argenson. Ce
n'était pas que je fusse en peine qu'il n'acceptât une telle décoration,
mais je voulais profiter du moment pour concilier le futur garde des
sceaux avec le cardinal de Noailles, et que ce prélat ne perdît au
chancelier que tout le moins qu'il se pourrait. Je présentai donc au
régent la nécessité de faire entendre à d'Argenson d'avance le parfait
concert, pour ne rien dire de plus, qu'on souhaitait de lui dans les
finances avec Law, et de corriger ce que cela pouvait avoir d'amer par
l'éclat des sceaux. M. le duc d'Orléans le trouva bon, de sorte que je
mandai par un billet à d'Argenson le jeudi matin de se trouver chez moi
le soir même, entre sept et huit heures du soir, pour chose pressée et
importante, où je l'attendrais portes fermées. Rien ne transpirait
encore, et quoiqu'on commençât depuis deux fois vingt-quatre heures tout
au plus à se douter de quelques nuages sur le duc de Noailles et sur le
chancelier, on n'avait pas été plus avant.

Argenson se rendit chez moi à l'heure marquée. Je ne le fis pas languir.
Je trouvai un homme effarouché du poids des finances, mais bien flatté
de la sauce des sceaux, et assez à lui-même, dans cette extrême
surprise, pour me faire bien des difficultés sur les finances, sans
néanmoins risquer les sceaux. Je lui expliquai au long les volontés du
régent par rapport à Law, et je ne m'expliquai pas moins nettement avec
lui par rapport au parlement et à tout ce que le régent comptait trouver
en lui à cet égard. Law et les finances étaient conditions \emph{sine
qua non}, qu'il fallut bien passer. Pour le parlement, il pensait comme
moi et comme M. le duc d'Orléans, et de ce côté-là, il était l'homme
qu'il fallait. Ses lumières, la cabale en mouvement, son personnel, tout
l'y portait. On peut juger de tout ce qu'il me dit de flatteur sur un
honneur tel que celui des sceaux, qu'il crut avec raison me devoir, et
sur lequel je fus modeste, mais toutefois en lui laissant sentir toute
la part que j'y avais.

J'avais pour cela mes desseins, et, la conversation importante à peu
près finie, je saisis un renouvellement de son éternelle reconnaissance
et de son attachement entier pour moi, pour lui demander amitié et
secours pour le cardinal de Noailles, que je lui déclarai très nettement
que je ne distinguais pas de moi-même. Nous entrâmes en matière. Je ne
lui cachai pas que j'étais bien instruit de ses liaisons avec les
jésuites et avec tout le parti de la constitution\,; que je comprenais
parfaitement que sa place le demandait sous le feu roi, mais que je
sentais aussi, qu'il était trop éclairé sur le fond des choses, et
encore plus par tant de détails qui avaient passé par ses mains, pour ne
porter pas un jugement sain de la chose, par rapport à la religion et à
l'État, et de la violence et de la tyrannie des procédés, qui n'avaient
de fondement que les plus insignes faussetés et les plus atroces
friponneries\,: par conséquent, que les temps étant changés et lui monté
à la première place tout à coup d'une fort subalterne, il ne vît, avec
tant d'esprit, d'expérience et de lumière, quel était le bon parti et
celui où la religion, l'État, la vérité, l'honneur le devaient attacher
sans lever d'étendard, ce qui ne convenait pas à la première place de la
magistrature. La discussion là-dessus fut longue, et j'y sentis de sa
part plus de discours et de compliments que de réalité. Je me persuadai
que la palinodie le retenait, sa vieille et `ancienne peau, ses
engagements de plusieurs années, et qu'une conversation avec le cardinal
de Noailles enlèverait ce que je voyais que je n'emportais pas. Je la
lui demandai, et il s'y prêta de bonne grâce\,; mais il me pria que ce
fût chez moi et le soir, pour la dérober à la connaissance du monde, et
il me promit de m'avertir et de me donner le premier soir que la
nouveauté de l'état où il allait entrer lui laisserait la première
liberté. Nous nous séparâmes de la sorte sur les dix heures du soir,
avec de grandes protestations de sa part de n'oublier jamais qu'il me
devait toute son élévation et sa fortune, et dans l'attente certaine du
grand événement du lendemain vendredi 28 janvier.

Ce jour-là La Vrillière, qui avait été mandé au Palais-Royal la veille
au soir, assez tard, alla sur les huit heures du matin redemander les
sceaux au chancelier et lui dire de la part du régent de s'en aller
jusqu'à nouvel ordre en sa maison de Fresnes, sur le chemin de Paris à
Meaux. Le chancelier lui dit qu'il portait un nom bien fatal aux
chanceliers. Il lui demanda avec fermeté et modestie s'il ne pouvait pas
voir le régent, et, sur le refus, de lui écrire\,; La Vrillière lui dit
qu'il se chargerait de la lettre. Le chancelier l'écrivit, la lut à La
Vrillière, la ferma devant lui et la lui donna. De là il écrivit un
billet d'avis au duc de Noailles et alla apprendre sa disgrâce à sa
femme qui était en couche. Il s'en alla le lendemain à Fresnes, n'ayant
laissé sa porte ouverte, à Paris, qu'à sa plus étroite famille ou amis
plus intimes, et sa femme le fut trouver quand sa santé le lui permit.

Noailles, averti de la bombe par le billet du chancelier, ne douta plus
de ce qui allait arriver sur les finances. Il résolut de prévenir le
régent et de se mettre en situation d'en tirer bon parti. Il l'alla
trouver sur-le-champ et eût la fausseté de lui demander ce que
signifieraient les sceaux qu'il voyait sur la table. Le régent eut la
bonté de lui dire qu'il les avait envoyé redemander au chancelier.
Noailles, d'un air le plus dégagé qu'il put, lui demanda à qui il les
donnait, et le régent eut la complaisance de le lui dire. Alors Noailles
répliqua qu'il voyait que la cabale l'emportait et qu'il ne pouvait
mieux faire que de céder et de rendre sa commission des finances. Tout
de suite le régent lui dit\,: «\,Ne demandez-vous rien\,? --- Rien du
tout, répondit Noailles. --- Je vous destine, ajouta le régent, une
place dans le conseil de régence. --- J'en ferai peu d'usage,\,»
répondit-il arrogamment, profitant de la faiblesse du prince\,; et
mentit bien puamment, car il vint au premier conseil de régence et n'en
manqua plus aucun. Il tint sa porte fermée les premiers jours.

Un moment après, d'Argenson arriva mandé par le régent. Il rencontra le
duc de Noailles dans les appartements, qui sortait\,; ils se saluèrent
sans se parler. Il fut un peu de temps seul avec le régent. À sa sortie,
il fut déclaré garde des sceaux et président des finances. Au sortir de
dîner, La Vrillière lui apporta ses commissions, et sur les trois
heures, il prêta son serment entre les mains du roi, en présence du
régent et en public aux Tuileries, et emporta les sceaux, que le roi lui
remit.

J'avais envoyé aux nouvelles au Palais-Royal, parce que j'aime à être
assuré que les choses sont faites. Comme j'étais à dîner chez moi en
grande compagnie, un valet de chambre d'Argenson m'apporta une lettre de
lui. Il imita dans cette lettre, que j'ouvris et montrai à la compagnie,
la modestie du célèbre cardinal d'Ossat, qui devait sa fortune et sa
promotion à M. de Villeroy, et à qui au sortir de chez le pape qui lui
avait donné la barrette, {[}il{]} le manda, et pour la dernière fois lui
écrivit encore \emph{monseigneur}. Argenson me traita de même, et me
manda qu'il venait d'être déclaré\,; en même temps que, prévoyant les
affaires qu'il aurait toute la journée, il avait été dès le matin de
bonne heure à Chaillot, et me rendait compte de ce qu'il y avait fait.
Les remerciements et les marques d'attachement et de reconnaissance
terminaient la lettre, et toujours \emph{monseigneur} dessus et dedans.

Ainsi le chancelier fut la victime du duc de Noailles, et le bouc
émissaire qui expia les péchés de son ami, et qui lui rendit tous les
effets de l'innocence. Noailles se servit de lui comme d'un bouclier, et
lui faisait voir et faire tout ce qui lui convenait sans ménagement
aucun et sans le plus léger voile. Il abusa ainsi sans cesse de
l'amitié, de la reconnaissance, de la confiance entière d'un homme de
bien et d'honneur, qui, dans l'ignorance parfaite des finances et du
monde, et dans les ténèbres de sa nouvelle vie, ne comptait de guide sûr
que celui qui l'avait mis dans cette grande place. Elle lui a été si
fatale que, quoique je me sois étendu ailleurs sur son caractère, je ne
puis me refuser d'en ramentevoir encore ici quelque chose.

Avec un des plus beaux et des plus lumineux esprits de son siècle, et
c'est peu dire, vastement et profondément savant, fait exprès pour être
à la tête de toutes les académies et de toutes les bibliothèques de
l'Europe, et pour se faire admirer à la tête du parlement, jamais rien
de si hermétiquement bouché en fait de finance, d'affaires d'État, de
connaissance du monde, ni de si incapable d'y rien entendre. Le parquet,
où il avait si longtemps brillé en maître, l'avait gâté pour tout le
reste par l'habitude de cet exact et parfait balancement de pour et de
contre de toutes les affaires contentieuses. Sa science et ses lumières
le rendaient fécond en vues\,: sa probité, son équité, la délicatesse de
sa conscience s'y embarrassaient, en sorte que plus il examinait, plus
il voyait, et moins il se déterminait. C'était pour lui un accouchement
que de prendre un parti sur les moindres choses. De là, devenu le père
des difficultés, c'étaient des longueurs infinies. Il était arrêté `tout
court par les moindres vétilles, mais surtout par la forme qui le
maîtrisait plus qu'un procureur qui en vit, en sorte qu'à qui ne
connaissait pas le fonds sincère et solide de sa justice, de sa piété,
de l'honneur, même de la bonté dont il était pétri, et véritablement
vertueux en tout, on aurait pris sa conduite pour un déni de justice,
parce qu'elle en avait tous les dehors et tous les inconvénients. Telle
fut la cause et la source des variations en affaires de toutes les
sortes, qui du faîte de la plus grande réputation, la plus accomplie, la
mieux méritée, l'a précipité dans un état si différent à cet égard, où
il est tombé par degrés, et à ce changement si prodigieux de lui-même,
qui l'a rendu méconnaissable dans des points capitaux sous lesquels il
est demeuré accablé, et dont sa considération et sa réputation ne se
relèveront jamais, quoiqu'il n'ait jamais cessé d'être le même. Une
correction, une perfection trop curieusement recherchée dans tout ce
qu'il veut qui sorte de sa plume, naturellement excellente, décuple son
travail, tombe dans la puérilité, dans la préférence de la justesse de
la diction sur l'exposition nette et claire des choses, dans une
augmentation de longueurs insupportables. Il épuise l'art académique, se
consume en des riens, et l'expédition en souffre toutes sortes de
préjudices.

Un autre défaut, qui vient du préjugé, de l'habitude de cet orgueil
secret que les plus gens de bien ignorent souvent en eux, parce que
l'amour-propre, si inhérent en nous, le leur sait cacher, est une
prévention si étrange en faveur de tout ce qui porte robe, qu'il n'y a
si petit officier de justice la plus subalterne, qui puisse avoir tort à
ses yeux, ni friponnerie si avérée qui, par la forme dont il est
esclave, ne trouve des échappatoires qui méritent toute sa protection.
Est-il enfin à bout de raisons, on le voit qui souffre, que sa
souffrance l'affermit en faveur de cette vile robe, dont l'impalliable
afflige, sa sensible délicatesse, sans le déprendre de la soutenir. Je
dis vile robe, telle qu'un procureur du roi ou un juge royal de justice
très subalterne, dont les friponneries et les excès, demeurés à
découvert et incapables d'excuses, en trouvent dans son coeur et dans
son esprit, et jusque dans sa raison et sa justice, quand elles ont
perdu toutes ressources d'ailleurs. Alors il se jette sur les
exhortations à pardonner les choses les moins pardonnables et les plus
susceptibles de recommencer de nouveau\,: il allègue comme un grand
malheur les conséquences du châtiment qui obscurcit tout un petit
siège\,; sur la nécessité de procéder dans les formes, en attaquant
juridiquement ce petit officier, et quelque cher et long que cela puisse
être, de se rendre partie contre lui. Ces exemples arrivent tous les
jours sur les faits les plus criants, sans qu'aucunes suites qui, pour
ce premier exil et première perte des sceaux lui ont été fatales, ni
aucunes considérations aient jamais pu avoir aucune prise sur lui à cet
égard, d'où naissent des inconvénients sans nombre, par la certitude que
toute robe a sa protection, que rien ne peut affaiblir. Oser se pourvoir
en cassation d'arrêts des parlements, ou contester quoi que ce soit à
ces compagnies en général ou en détail personnel en aucun genre, est une
profanation qui lui est insupportable, quoiqu'il ait été plus d'une fois
et en face bien mal récompensé de cette espèce de culte et en pleine
séance au parlement, sans que rien l'en ait pu détacher. S'il voit que,
malgré ce qu'il a pu tenter pour parer, la cassation passe au conseil,
il interrompt contre la règle, harangue, se rend l'avocat du parlement
et de son arrêt, et cela des autres parlements comme de celui de Paris.
Il reprend les voix, il intimide les maîtres des requêtes, cherche à
embarrasser le rapporteur et les commissaires, il reprend les avis. Tout
le conseil s'en plaint et s'accoutume à lui résister respectueusement
mais fermement, et ne s'en cache pas. S'il sent enfin qu'il ne gagne
rien, et que l'arrêt passe, il ne peut toutefois se résoudre à prononcer
le blasphème de cassation. Il a inventé pour l'éviter une formule
jusqu'à lui inconnue. Il prononce que, «\,sans s'arrêter à l'arrêt du
parlement, etc., qui demeurera comme non avenu, etc.\,;» et les
parlements qui sentent et comptent sur cette vénération si loin poussée
pour eux, n'ont cessé d'en abuser, et tout cela pourtant de la meilleure
foi, et avec l'intégrité la plus parfaite.

On peut juger de là combien d'Aguesseau était peu propre à soutenir
l'autorité royale résidente dans un régent, contre les entreprises du
parlement\,; et je ne craindrai point de le dire, combien, à l'entrée de
ces mouvements, qui annonçaient tant de choses, il était important de
renvoyer ce premier magistrat, d'ailleurs si digne de toute autre place,
mais si peu propre à la première de son état, où le duc de Noailles
l'avait bombardé en un instant, uniquement pour soi, en abusant en cela,
comme en bien d'autres choses, de la facilité du régent, qui, ébloui de
la grande réputation de celui qu'il lui proposa à l'instant de la
vacance, l'en crut sur sa parole, sans connaissance de celui qu'il
mettait si subitement dans une place si importante. Ce prince n'avait
guère tardé à se repentir d'un choix si brusque, dont il s'était enivré
d'abord\,; mais il fut sensible au cri public, à la louange du
chancelier, et à le plaindre.

Toute la robe, vivement intéressée à un chef qui était véritablement
idolâtre d'elle, et tout ce qui cabalait d'ailleurs contre le régent,
aidés des échos qui répètent tout ce qu'ils entendent, élevèrent
d'autant plus d'Aguesseau que le contre-coup naturel portait davantage
en aigre censure contre le régent et contre son gouvernement. Il avait
bien et longtemps combattu, avant de se résoudre à ce tour de force. Il
n'y était venu qu'à la dernière extrémité. Épuisé de l'avoir fait et
abattu, de la manière dont il était reçu du monde, il retomba, dans sa
faiblesse naturelle à l'égard de l'autre parti. L'esprit et la tribu de
Noailles lui fit peur. Non content d'avoir mis le duc de Noailles dans
le conseil de régence, quoique le véritable criminel, tandis qu'il
exilait le chancelier et ne lui ôtait les sceaux que pour avoir été
l'esclave de Noailles, il jeta tout de suite à la tête de ce dernier la
survivance de sa charge et de ses gouvernements pour son fils à la
jaquette, qui n'avait pas encore cinq ans, lui fut obligé d'avoir bien
voulu l'accepter, et ne lui marqua jamais tant de considération et
d'amitié. Si le public s'irrita de la disgrâce du chancelier, il ne se
scandalisa pas moins aigrement des grâces prodiguées au duc de Noailles,
et n'applaudit dans tout cet événement, qu'à lui voir ôter les finances
où il s'était extrêmement fait haïr de tout ce même public et des
particuliers. Mais il tenait le bon bout encore. Les propos le
touchèrent peu, et il a montré par toute la suite de sa vie et par son
propre exemple, le peu de cas qu'on peut et doit faire de sa réputation,
qu'il a sans cesse vendue pour ce qu'il a estimé être plus réel.

Par une suite nécessaire, Rouillé du Coudray, qui avait été son bras
droit et souvent son conducteur dans les finances, n'y put être
conservé. Depuis assez longtemps, il n'y faisait presque plus rien que
continuer à se faire mépriser et détester par ses brutalités et ses
continuelles indécences, abruti par le vin et par toutes sortes de
débauches. Il s'y plongea de plus en plus depuis qu'il n'eut plus
l'occupation des finances, et acheva ainsi une assez longue vie dans les
vices dont il faisait trophée, laissant admirer qu'avec une capacité
très médiocre, une grossièreté et une brutalité extrême, une indécence
continuelle qui n'avait honte de rien, il fût devenu sous le feu roi
directeur des finances et mille livres de pension.

Machault, maître des requêtes, eut la police dont il fit la moindre de
ses occupations, sur le pied plus que scabreux où Argenson l'avait mise.
Aussi n'y satisfit-il ni soi ni le régent, et n'y put demeurer
longtemps. C'était un homme intègre et capable, exact et dur, magistrat
depuis les pieds jusqu'à la tète, fantasque et bourru, qui ne se
radoucissait qu'avec des créatures de mauvaise vie, dont il ne se
laissait jamais manquer.

Châteauneuf, revenant de Hollande où il avait très bien servi, et qui
avait une pension de six mille livres, en eut une pareille en
augmentation, une place de conseiller honoraire au parlement, et
promesse de la seconde place de conseiller d'État qui vaquerait, la
parole de la première étant engagée à Bernage, qui allait intendant en
Languedoc, en la place de Bâville.

Torcy eut cent cinquante mille livres d'augmentation de brevet de
retenue, qui lui en fit un de quatre cent mille livres sur sa charge des
postes, et maria sa seconde fille assez tristement à
Duplessis-Châtillon.

Le duc d'Albret, occupé à se marier à une fille de Barbezieux, malgré
toute sa famille, et à y intéresser le régent, en obtint une
augmentation d'appointements et une de brevet de retenue de cent mille
livres sur son gouvernement d'Auvergne.

\hypertarget{chapitre-xi.}{%
\chapter{CHAPITRE XI.}\label{chapitre-xi.}}

1718

~

{\textsc{M. le duc d'Orléans mène M. le duc de Chartres aux conseils de
régence et de guerre, sans y opiner.}} {\textsc{- Entreprises du
parlement.}} {\textsc{- Mort et dépouille de Simiane et du grand
fauconnier des Marais.}} {\textsc{- Madame assiste scandaleusement à la
thèse de l'abbé de Saint-Albin.}} {\textsc{- Ballet du roi, qui s'en
dégoûte pour toujours.}} {\textsc{- M. {[}le duc{]} et
M\textsuperscript{me} la duchesse de Lorraine à Paris.}} {\textsc{-
Bassesse de courtisan du duc de Lorraine.}} {\textsc{- M. le Duc et
ensuite M\textsuperscript{me} la duchesse de Berry donnent une fête à M.
et à M\textsuperscript{me} de Lorraine.}} {\textsc{- Insolence de Magny
punie\,; quel il était et ce qu'il devint.}} {\textsc{- M. de Lorraine
va voir plaider à la grand'chambre, puis à la Bastille, et dîner chez le
maréchal de Villeroy.}} {\textsc{- Objet et moyens du duc de Lorraine
dans ce voyage.}} {\textsc{- Il est ennemi de la France.}} {\textsc{-
Ses demandes sans droit ni prétexte.}} {\textsc{- Ses lueurs mises au
net par moi au régent.}} {\textsc{- Altesse royale, pourquoi et quand
accordée au duc de Savoie.}} {\textsc{- Le régent entraîné à tout
accorder au duc de Lorraine.}} {\textsc{- Ses mesures pour
l'exécution.}} {\textsc{- Caractère de Saint-Contest, nommé pour faire
le traité avec le duc de Lorraine, qui obtient un grand démembrement en
Champagne en souveraineté, et le traitement d'Altesse Royale.}}
{\textsc{- Misère du conseil de régence.}} {\textsc{- Le régent tâche
inutilement, par Saint-Contest et par lui-même, de vaincre ma résistance
au traité\,; vient enfin à me prier de m'absenter du conseil de régence
le jour que ce traité y sera porté.}} {\textsc{- J'y consens.}}
{\textsc{- Il m'en arriva de même lorsque le régent accorda le
traitement de Majesté au roi de Danemark, et celui de Hautes Puissances
aux États généraux des Provinces-Unies.}} {\textsc{- Le traité passe
sans difficulté au conseil de régence\,; est de même aussitôt après
enregistré au parlement.}} {\textsc{- Départ de M. et de
M\textsuperscript{me} de Lorraine.}} {\textsc{- Audacieuse conduite du
duc de Lorraine, qui ne voit point le roi.}} {\textsc{- Le grand-duc
{[}de Toscane{]} et le duc de Holstein-Gottorp, sur l'exemple du duc de
Lorraine, prétendent aussi l'Altesse Royale, et ne l'obtiennent pas.}}
{\textsc{- Bagatelles entre M. le duc d'Orléans et moi.}} {\textsc{-
M\textsuperscript{me} de Sabran\,; quelle.}} {\textsc{- Son bon mot au
régent.}} {\textsc{- Conduite {[}du régent{]} avec ses maîtresses.}}

~

M. le duc d'Orléans, à l'insu de tout le monde, mena, le 30 janvier, M.
son fils au conseil de régence, auquel il fit un petit compliment, et
dit qu'il n'opinerait point, qu'il venait seulement pour apprendre. Je
n'ai point su qui lui donna ce conseil prématuré, qui n'a pas rendu
grand fruit. Il le mena le lendemain au conseil de guerre. M. le Duc y
faisait une tracasserie au maréchal de Villars sur la liasse de ce
conseil qu'il portait au régent, lequel, par son goût pour les
\emph{mezzo-termine}, régla qu'elle ne lui serait plus portée, et qu'il
irait au conseil de guerre tous les quinze jours où il lui serait rendu
compte de ce qui s'y serait fait pendant la quinzaine.

Il envoya en ce même temps d'Effiat au premier président, donna des
audiences au premier président seul, puis à lui et aux gens du roi
ensemble\,; enfin, une le 7 février aux députés du parlement, qui, par
la bouche du premier président attaquèrent fort les divers conseils,
comme embarrassant et allongeant les affaires, matière fort étrangère au
parlement, où même elle avait passé le jour de la régence. Ils ne
laissèrent pas d'être traités plus que fort honnêtement.

Simiane, l'un des deux premiers gentilshommes de la chambre de M. le duc
d'Orléans, mourut, et sa charge fut donnée à son frère. Il avait eu à la
mort de Grignan, son beau-père, l'unique lieutenance générale de
Provence, de vingt-sept mille livres de rente, et un brevet de retenue
de deux cent mille livres, et ne laissa point d'enfants. Un mois après
elle fut donnée à Brancas, devenu longtemps après grand d'Espagne et
maréchal de France, qui était de mes amis, et pour le fils duquel j'en
obtins la survivance dans la suite. Des Marais, grand fauconnier, mourut
en ce même temps jeune et obscur\,: on a vu en son lieu comment son fils
enfant avait eu sa survivance.

M. le duc d'Orléans avait de la comédienne Florence un bâtard qu'il n'a
jamais reconnu et à qui néanmoins il a fait une grande fortune dans
l'Église. Il le faisait appeler l'abbé de Saint-Albin. Madame, si
ennemie des bâtards et de toute bâtardise, s'était prise d'amitié pour
celui-là avec tant de caprice, qu'à l'occasion d'une thèse qu'il soutint
en Sorbonne, elle y donna le spectacle le plus scandaleux et le plus
nouveau, et en lieu où jamais femme, si grande qu'elle pût être, n'était
entrée ni ne l'avait imaginé. Telle était la suite de cette princesse.
Toute la cour et la ville fut invitée à la thèse et y afflua. Conflans,
premier gentilhomme de la chambre de M. le duc d'Orléans, en fit les
honneurs, et tout s'y passa de ce côté-là comme si M. le duc de Chartres
l'eût soutenue. Madame y alla en pompe, reçue et conduite à sa portière
par le cardinal de Noailles, sa croix portée devant lui. Madame se plaça
sur une estrade qu'on lui avait préparée dans un fauteuil. Les
cardinaux-évêques et tout ce qui y vint de distingué se placèrent sur
des sièges à dos, au lieu de fauteuils. M. {[}le duc{]} et
M\textsuperscript{me} la duchesse d'Orléans furent les seuls qui n'y
allèrent pas, et moi je n'y allai pas non plus. Cette singulière scène
fit un grand bruit dans le monde\,; jamais M. le duc d'Orléans et moi ne
nous en sommes parlé.

Le maréchal de Villeroy, adorateur du feu roi jusque dans les bagatelles
et très attentif à les faire imiter au roi de bonne heure, lui fit
danser un ballet, plaisir qui n'était pas encore de son âge, et lui ôta
pour toute sa vie, par cette précipitation, le goût des bals, des
ballets, des spectacles et des fêtes, quoique ce divertissement eût tout
le succès qu'on s'y pût proposer\,; mais le roi, se trouva excédé de
l'apprendre, d'essayer des habits, encore plus de le danser en public.

Le duc de Lorraine, tout tourné et dévoué qu'il fût à la cour de Vienne,
n'était pas homme à négliger les avantages qu'il pourrait tirer de la
facilité du régent dont il avait l'honneur d'être beau-frère, et
l'amitié tendre de ce prince pour une soeur avec qui il avait été élevé,
de sa faiblesse pour Madame, qui n'avait à l'allemande des yeux que pour
son gendre et pour sa grandeur. Ce qu'il avait éprouvé là-dessus au
voyage qu'il avait fait pour rendre au feu roi son hommage, pour le
duché de Bar, lui devint une raison décisive d'en faire un second à
Paris, sous l'étrange incognito du nom de comte de Blamont pour voiler
tout ce à quoi il ne pouvait atteindre.

Cette petite cour arriva de très grande heure, le vendredi 18 février,
rencontrée au deçà de Bondy par Madame, qui avait dans son carrosse M.
{[}le duc{]} et M\textsuperscript{me} la duchesse d'Orléans, M. le duc
de Chartres et M\textsuperscript{me} de Valois, depuis duchesse de
Modène. Elle y fit monter M. et M\textsuperscript{me} de Lorraine qui,
n'étant point incognito par son rang décidé de petite-fille de France,
et de rang égal à M\textsuperscript{me} la duchesse d'Orléans qui lui
fit les honneurs du carrosse de Madame, se mit au fond avec elle.
M\textsuperscript{me} la duchesse d'Orléans sur le devant avec M. de.
Chartres et M\textsuperscript{lle} de Valois, où M. le duc d'Orléans
n'eût pu tenir en troisième avec elle, qui se mit à une portière et le
duc de Lorraine à l'autre.

Ils arrivèrent et logèrent au Palais-Royal dans l'appartement de la
reine mère, que M. le duc de Chartres leur céda. Un moment après ils
allèrent tous à l'Opéra dans la grande loge de Madame, d'où M. le duc
d'Orléans mena le duc de Lorraine voir un moment M\textsuperscript{me}
la duchesse de Berry dans la sienne, et le ramena dans la loge de
Madame. Au sortir de l'Opéra, M\textsuperscript{me} la duchesse de
Lorraine vit quelques moments du monde dans son appartement, où elle
avait trouvé en arrivant une commode pleine des plus riches galanteries,
qui fut un présent de M\textsuperscript{me} la duchesse de Berry, et
force belles dentelles, qui en fut un de M\textsuperscript{me} la
duchesse d'Orléans. Elle descendit chez elle, où il y eut grand jeu et
grand souper. Avant de se retirer, M\textsuperscript{me} de Lorraine vit
d'une loge le bal de l'Opéra. Le dîner fut toujours chez Madame, et le
souper chez M\textsuperscript{me} la duchesse d'Orléans, où M. le duc
d'Orléans soupa fort rarement et ne dînait point. Il prenait du
chocolat, entre une heure et deux heures après midi, devant tout le
monde\,: c'était l'heure la plus commode de le voir.

C'est ce qui a dérangé l'heure du dîner depuis, et les dérangements une
fois établis ne se réforment plus. Le lendemain de leur arrivée ils
virent la comédie italienne sur le théâtre de l'Opéra, après quoi M. le
duc d'Orléans les mena à Luxembourg voir M\textsuperscript{me} la
duchesse de Berry, où la visite se passa debout.

Le dimanche, Madame mena M\textsuperscript{me} la duchesse de Lorraine
aux Tuileries. Le roi, qui dînait, se leva de table et alla embrasser
M\textsuperscript{me} la duchesse de Lorraine. Il se remit à table, et
elles le virent dîner de dessus leurs tabourets. Lorsque le roi sortit
de table elles s'en allèrent dîner chez Madame, où le duc de Lorraine
les attendait. Ensuite Madame mena M\textsuperscript{me} de Lorraine aux
Carmélites du faubourg Saint-Germain, où M\textsuperscript{me} la
duchesse de Berry se trouva, qui y avait un appartement. Le lundi après
dîner, M\textsuperscript{me} la duchesse de Lorraine alla voir
M\textsuperscript{me} la grande-duchesse, et le lendemain toutes les
princesses du sang, qui toutes l'avaient vue chez elle, se masqua après
souper, et alla en bas au bal de l'Opéra. Il y eut toujours beaucoup de
dames aux soupers avec elle chez M\textsuperscript{me} la duchesse
d'Orléans.

Le jeudi 24 février, le roi fut au Palais-Royal voir
M\textsuperscript{me} la duchesse de Lorraine. M. de Lorraine, qui
n'oubliait rien pour plaire au régent et pour en obtenir ce qu'il se
proposait, lui demanda pour le chevalier d'Orléans la lieutenance
générale de Provence. Cela ne déplut pas au régent, mais il répondit
qu'il avait d'autres vues.

Le samedi 26 février, il y eut un banquet superbe à l'hôtel de Condé
pour M. {[}le duc{]} et M\textsuperscript{me} la duchesse de Lorraine.
M. le Duc y avait invité grand nombre de dames, qui toutes furent
extrêmement parées et M\textsuperscript{me} de Lorraine aussi. Il y eut
beaucoup de tables, toutes magnifiquement servies en gras et en maigre.
Ce fut une nouveauté que ce mélange, qui fit quelque bruit. On se masqua
après souper.

Le lundi 28 février, M\textsuperscript{me} la duchesse de Berry donna le
soir à M. {[}le duc{]} et à M\textsuperscript{me} la duchesse de
Lorraine la plus splendide et la plus complète fête qu'il fût possible
en toute espèce de magnificence et de goût. M\textsuperscript{me} de
Saint-Simon, qui l'ordonna toute et qui en fit les honneurs, eut tout
l'honneur que de telles bagatelles peuvent apporter par le goût, le
choix, l'ordre admirable avec lequel tout fut exécuté. Il y eut une
table de cent vingt-cinq couverts pour les dames conviées, toutes
superbement parées, et pas une en deuil, et une autre de pareil nombre
de couverts pour les hommes invités. Les ambassadeurs, qui le furent
tous, ne s'y voulurent pas trouver, parce qu'ils prétendirent manger à
la table où seraient les princes du sang, lesquels mangèrent avec le duc
de Lorraine, tous sans rang, à la table des dames où était
M\textsuperscript{me} la duchesse de Berry, fille de France, avec qui
les ambassadeurs ne pouvaient pas manger, ni, pour en dire la vérité, M.
de Lorraine non plus sous son incognito, mais qui y mangea pourtant sans
difficulté. Le palais de Luxembourg était admirablement illuminé en
dedans et en dehors.

Le souper fut précédé d'une musique et suivi d'un bal en masque, où il
n'y eut de confusion que lorsque M\textsuperscript{me} la duchesse de
Berry et M\textsuperscript{me} de Lorraine en voulurent, pour s'en
divertir. Tout Paris y entra masqué. M\textsuperscript{lle} de Valois ne
se trouva point au souper, mais au bal seulement\,: je n'en ai point su
ni deviné la raison. Trois ou quatre personnes non invitées et non
faites pour l'être se fourrèrent hardiment à la table des hommes.
Saumery, premier maître d'hôtel de M\textsuperscript{me} la duchesse de
Berry, leur en dit son avis, par son ordre, au sortir de table. Ils ne
répondirent rien et s'écoulèrent, excepté Magny, qui dit tant
d'insolences que Saumery le prit à la cravate pour le conduire à
M\textsuperscript{me} la duchesse de Berry, et l'eût exécuté, si Magny
n'eût trouvé moyen de s'en dépêtrer, et de se sauver hors du Luxembourg
dans la ville, où le lendemain il continua à débiter force sottises.

Il était fils unique de Foucault, conseiller d'État\footnote{Nicolas-Joseph
  Foucault, dont il est ici question, avait été intendant dans les
  généralités de Pau, de Cahors, de Poitiers et de Caen. Il a laissé un
  journal où il retrace son administration de 1668 à 1709. ce journal
  inédit fait partie des manuscrits de la Bibliothèque impériale. Il
  confirme presque toujours ce que Saint-Simon dit de l'administration
  de Louis XIV et surtout de Louvois. On trouvera un extrait de ce
  journal dans les notes à la fin du volume.}, qui s'était élevé par les
intendances, et qui, par un commerce de médailles, s'était fait une
protection du P. de La Chaise. Tous deux s'y connaissaient fort, et en
avaient ramassé de belles et curieuses collections. Foucault eut ainsi
le crédit de faire succéder ce fils à l'intendance de Caen, lorsqu'il la
quitta pour une place de conseiller d'État. Les folies que fit Magny
dans une place si sérieuse et les friponneries dont il fut convaincu
furent si grossières et si fortes, qu'il fut rappelé avec ignominie, et
que, n'osant plus se présenter au conseil ni espérer plus aucune fortune
de ce côté-là, il se défit de sa charge de maître des requêtes, prit une
épée, battit longtemps le pavé, et après la mort du roi essaya de se
raccrocher par une charge d'introducteur des ambassadeurs que le baron
de Breteuil lui vendit.

C'est à ce titre qu'il se fourra à table à cette fête, et que par ses
insolences il se fit mettre deux jours après à la Bastille, après que
M\textsuperscript{me} la duchesse de Berry en eut fait une honnêteté à
Madame, parce que Foucault était chef de son conseil. Magny, au sortir
de la Bastille, eut ordre de se défaire de sa charge, qui avait besoin
d'un homme plus sage auprès des ministres étrangers. La rage qu'il
conçut de ce qu'il méritait et qu'il était allé chercher le jeta parmi
les ennemis du gouvernement, qui faisaient alors recrue de tout, et qui
trouvèrent en lui de l'esprit et beaucoup de hardiesse. Il s'embarqua en
tout, et passa bientôt en Espagne. Il y fut bien reçu et bien traité, et
quoiqu'il n'eût jamais été que de robe, il fut colonel, et tôt après
brigadier. Je m'étends sur lui, parce que je l'y trouvai majordome de la
reine. Il expédiait fort promptement ce qu'il touchait, trouvait fort
mauvais de ne faire pas assez tôt fortune, et l'indigence où il se
jetait lui-même. La mauvaise humeur le rendit fort impertinent, et le
fit honteusement chasser, tellement qu'après la mort du régent, il
repassa les Pyrénées dans l'espérance du changement des temps. Mais
comme les brouillons n'étaient plus nécessaires à ceux qui les avaient
recherchés pendant la vie de ce prince, Magny demeura sur le pavé,
chargé de mépris et de dettes pour le malheur d'une fort honnête femme
et riche, qu'il avait épousée, lorsqu'il était à Caen, et qu'il avait
sucée et abandonnée. Il a depuis traîné une vie obscure et misérable, et
{[}est{]} retourné enfin en Espagne où le même mépris et la même
indigence l'ont suivi.

M. de Lorraine alla courre le cerf à Saint-Germain avec les chiens du
prince Charles. Le duc de Noailles n'eut garde de manquer cette occasion
de faire sa cour au régent. Il donna à M. de Lorraine un grand retour de
chasse au Val. De son côté, M\textsuperscript{me} la duchesse de
Lorraine alla voir deux soeurs du duc d'Elboeuf, religieuses, l'une à
Pantemont, l'autre fille de Sainte-Marie à la rue Saint-Jacques. Le
lundi 7 mars, le duc de Lorraine alla ouïr plaider dans une des
lanternes de la grand'chambre\,; de la voir la Bastille, puis dîner à
l'hôtel de Lesdiguières où le maréchal de Villeroy le traita
magnifiquement, avec beaucoup de dames, et leur donna une grande
musique. Quelques jours après, M. de Lorraine dîna chez l'ambassadeur de
l'empereur\,: il était là plus dans son centre. M\textsuperscript{me} la
duchesse de Lorraine fut voir danser le ballet du roi, et quelques jours
après voir, avec M. de Lorraine, M\textsuperscript{lle} sa nièce à
Chelles, qui y avait pris l'habit, puis avec Madame aux Carmélites, où
M\textsuperscript{me} la duchesse de Berry se trouva.
M\textsuperscript{me} et M. le duc d'Orléans firent chacun un présent
magnifique à M\textsuperscript{me} la duchesse de Lorraine, dont le
séjour à Paris fut à diverses fois prolongé. Le 15 mars,
M\textsuperscript{me} la duchesse de Berry alla de bonne heure se
baigner à Saint-Cloud\,; M. le duc d'Orléans y mena
M\textsuperscript{me} la duchesse de Lorraine l'après-dînée. Ils
soupèrent tous de fort bonne heure dans la petite maison de
M\textsuperscript{me} de Maré, avec elle, leur ancienne gouvernante, et
ce souper fut poussé fort tard. Le duc de Lorraine avait dîné le même
jour chez la comtesse d'Harcourt, dont le mari avait eu la pension de
seize mille livres de notre monnaie, qu'il donnait au feu prince
Camille. M. de Lorraine fut quelques jours après voir Chantilly\,;
après, avec M\textsuperscript{me} la duchesse de Lorraine, voir
M\textsuperscript{me} la princesse de Conti, fille du roi, à Choisy, et
voir encore Mademoiselle à Chelles. M\textsuperscript{me} la duchesse de
Lorraine, étant au Cours, y trouva le roi, et arrêta devant lui comme de
raison. Le roi passa dans son carrosse sans lui rien dire. Le lendemain,
le duc de Lorraine alla voir la reine d'Angleterre à Saint-Germain, et
M\textsuperscript{me} de Lorraine fut à la comédie française, qu'elle
n'avait vue que sur le théâtre de l'Opéra. Le même soir M. le duc
d'Orléans soupa avec le duc de Lorraine à Luxembourg chez
M\textsuperscript{me} la duchesse de Berry. Le 29 mars, M. et
M\textsuperscript{me} de Lorraine allèrent voir Versailles, et le 1er
avril de bonne heure voir Marly, rabattirent à Saint-Cloud, où M. le duc
d'Orléans les promena fort et leur donna à souper dans la petite maison
de M\textsuperscript{me} de Maré, avec elle quelques jours après M. le
duc d'Orléans les mena dîner chez d'Antin.

Tout ce voyage et tous ces divers délais n'avaient d'objet que
l'arrondissement de la Lorraine, dont aucun duc ne gagna jamais tant, si
gros ni à si bon marché que celui-ci, et ne fut pourtant jamais si peu
considérable. M. le duc d'Orléans aimait fort M\textsuperscript{me} sa
soeur, avec laquelle il avait été élevé et {[}avait{]} vécu jusqu'à son
mariage avec le duc de Lorraine. Il avait pour Madame un respect timide,
qui opérait une déférence extrême quand elle n'attaquait ni ses goûts ni
ses plaisirs, et Madame, qui aimait extrêmement M\textsuperscript{me} sa
fille, avait une passion aveuglément allemande pour le duc de Lorraine
son gendre, pour sa famille, pour sa grandeur. Il était parfaitement
bien informé de toutes ces choses\,; il en avait eu de grandes preuves
en son premier voyage, comme on l'a vu alors. Tout autrichien qu'il
était, il avait eu grand soin de cultiver ces dispositions par toutes
les attentions possibles de M\textsuperscript{me} sa femme et de
lui-même, et il en sut tirer le plus grand parti dans cette régence de
M. le duc d'Orléans, dont il ne manqua pas la conjoncture. Ainsi dans le
temps le plus mort pour lui, où sans placés, sans troupes, environné,
enchaîné de toutes parts par la France, il ne pouvait être d'aucun usage
à qui que ce soit en aucun temps, il n'en conçut pas moins le dessein de
s'étendre très considérablement en Champagne, et d'obtenir du roi le
traitement d'Altesse Royale.

Pour le premier il étala de vieilles prétentions usées dans tous les
temps, réprouvées même avec l'appui de l'empereur dans les divers
traités de paix\,; enfin anéanties par les derniers, et singulièrement
par celui en vertu duquel il était rentré dans la possession de la
Lorraine. Il exposa aussi des dédommagements ineptes d'injustices
prétendues du temps du vieux duc Charles IV de Lorraine, dont les
perfidies avaient tout mérité, et le dépouillement par la France, et
bien des années de prison en Espagne, dont il ne sortit qu'à la paix des
Pyrénées, dédommagement dont il ne s'était jamais parlé depuis, et que
M. de Lorraine n'articula que comme une grâce qu'il espérait de l'amitié
et de l'honneur de la proximité. Qui lui aurait proposé à lui-même de
restituer les usurpations sans nombre faites par sa maison aux
Trois-Évêchés, et le dédommagement de tout ce qui a été arraché et
démembré par leurs évêques de la maison de Lorraine et par les ducs de
Lorraine aussi, et incorporé jusqu'à aujourd'hui à leur domaine, il
aurait été bien confondu par les titres qui lui en pouvaient être
représentés en preuves bien solides, et n'aurait pas eu la moindre
défense à opposer au droit ni à apporter à la puissance, si la volonté
de s'en faire justice y eût été jointe, comme elle devait et pouvait
l'être dans la situation présente alors de l'Europe, et avec un prince
qui, pendant les plus grands malheurs, de la dernière guerre du feu roi
pour la succession d'Espagne, avait, à la Guise, ourdi toutes les
perfidies qu'on a vues ici en leur lieu, et les trames les plus funestes
au feu roi et à la France, pour élever sa grandeur sur ses ruines\,;
audace et trahison qui ne se devait jamais oublier, suivant la sage
maxime qui a toujours rendu si redoutable la maison d'Autriche, jusque
dans les temps où elle l'a paru le moins, et qui a été le plus ferme
appui de sa solide grandeur et de cette espèce de dictature qu'elle a si
longtemps et si utilement pour elle exercée en Europe, dont le
démembrement d'Espagne n'a pu encore la déprendre.

À l'égard du traitement, il posait un principe d'exemple dont il sentait
bien tout le faux, mais qu'il entortillait et replâtrait avec souplesse,
parce qu'il n'est rien de si bas que la hauteur, quand elle est grande
mais impuissante, ni bassesse qu'elle ne fasse pour parvenir à ses fins.
Son grand moyen était l'exemple du duc de Savoie, beau-frère comme lui
de M. le duc d'Orléans, et qui n'était pas de si bonne maison que lui,
différence de traitement qu'il ne pouvait regarder que comme très
déshonorante entre deux souverains, égaux d'ailleurs en souveraineté et
en proximité, comme étant maris des deux soeurs qui par elles-mêmes
avaient le traitement d'Altesse Royale, comme petites-filles de France,
qu'il était bien dur que la duchesse de Savoie eût communiqué au duc son
époux, tandis que lui demeurait privé du même avantage.

Il tâchait ainsi de parer à la réponse sur le traitement même qui se
présentait naturellement à lui faire, c'est que Charles II, duc de
Lorraine, gendre de Henri II, ne l'avait jamais eu ni prétendu dans le
temps même de la plus grande puissance de la Ligue et des plus grands
efforts de Catherine de Médicis pour lui préparer la couronne de France
au préjudice de son autre gendre, le véritable héritier, qui a été notre
roi Henri IV. Henri, duc de Lorraine, son fils, qui épousa la soeur de
Henri IV, en janvier 1599, morte sans enfants en février 1604, et qui ne
devint duc de Lorraine que quatre ans après par la mort de son père,
n'eut et ne prétendit jamais ce traitement\,; et Charles-Léopold, père
du duc de Lorraine dont il s'agit ici, reconnu duc de Lorraine par toute
l'Europe (quoiqu'elle lui fût détenue par la France pour en avoir refusé
la restitution à certaines conditions), qui fut un des plus grands
capitaines de l'Europe et qui rendit les plus grands services à
l'empereur Léopold, dans son conseil et à la tête de ses armées\,; qui
de plus avait l'honneur d'avoir épousé sa soeur, reine, veuve de Michel
Wiesnowieski, roi de Pologne, qui en eut le traitement toute sa vie, et
qu'on appelait la reine-duchesse, ce duc son mari, si grandement
considéré à Vienne, n'a jamais eu ni prétendu l'Altesse. Royale à Vienne
ni ailleurs. Il est mort en 1690, et la reine-duchesse en 1697. Le duc
de Lorraine, qui la prétendait maintenant, n'était pas autre que ses
pères, ni plus grandement marié. La réponse était péremptoire, et c'est
ce qu'il voulait parer en se fondant sur l'exemple de M. de Savoie, et
se plaignant tendrement d'une distinction si flétrissante. C'était un
sophisme dont il sentait bien aussi le faux, mais qu'il fournissait
comme prétexte à qui le voulait aveuglément combler. Voici le fait\,:

Aucun duc de Savoie n'avait eu ni prétendu l'Altesse Royale avant le
beau-frère de M. le duc d'Orléans, qui est devenu depuis roi de Sicile,
puis de Sardaigne. Le fameux Charles-Emmanuel, vaincu à Suze par Louis
XIII en personne, ne manquait ni de fierté ni d'audace. Il était gendre
et appuyé de Philippe II, roi d'Espagne\,; jamais il ne l'a eue ni
prétendue, non plus que le beau-frère de Louis XIII'. Longtemps avant
que le duc de Savoie, beau-frère de M. le duc d'Orléans, en ait montré
la première prétention, il avait si bien fait valoir sa chimère de roi
de Chypre, par ce qu'il valait lui-même, et par la situation importante
de ses États, que ses pères et lui avaient peu à peu continuellement
agrandis, qu'il avait enfin obtenu à Rome la salle royale pour ses
ambassadeurs, à Vienne le traitement pour eux d'ambassadeurs de tête
couronnée, et sur ces deux grands exemples, dans toutes les cours de
l'Europe, sans toutefois en avoir aucun traitement pour sa personne, et
tel toujours que ses pères l'avaient eu. Il avait été lors marié
longtemps sans prétendre au traitement d'Altesse Royale, dont la
duchesse son épouse jouissait comme petite-fille de France, et qu'elle
ne lui communiqua point. Mais quand il se vit en possession partout du
traitement de tête couronnée par ses ambassadeurs, il commença à
prétendre un traitement personnel et distingué pour lui-même et par
lui-même, qui fut l'Altesse Royale, n'osant porter ses yeux jusqu'à la
Majesté. Il l'obtint peu à peu partout assez promptement, et dans la
vérité il était difficile de s'en défendre, après avoir accordé à ses
ambassadeurs le traitement de ceux des têtes couronnées. La chimère des
ducs de Lorraine, prétendus rois de Jérusalem, n'avait pas été si
heureuse. Leur faiblesse, ni la situation de leur état n'influait en
rien dans l'Europe, dont aucune cour n'avait besoin d'eux. Le duc de
Savoie, au contraire, pouvait beaucoup à l'égard de l'Italie et de tous
les princes qui y avaient ou y voulaient posséder des États, et qui y
voulaient porter ou en éloigner la guerre\,; c'est ce qui fit toute la
différence entre lés chimères d'ailleurs pareilles de Chypre et de
Jérusalem. Rien donc de semblable entre ces deux souverains, sinon
d'avoir l'un et l'autre épousé deux petites-filles de France, soeurs de
M. le duc d'Orléans, jouissant toutes deux du traitement d'Altesse
Royale, sans que pas une des deux l'ait communiqué à son époux. Tel
était l'état véritable des choses quand le duc de Lorraine crut le temps
favorable, et qu'il en voulut profiter.

M. le duc d'Orléans, attaqué par les soumissions en discours et les
supplications du duc de Lorraine, par les ruses et les ressorts des gens
qui y étaient maîtres en dessous, tels que M. de Vaudemont et ses deux
nièces, par les prières et les amitiés continuelles de
M\textsuperscript{me} la duchesse de Lorraine, qui d'ailleurs se fit
toute à tous, avec une attention infinie, excepté pour
M\textsuperscript{me} du Maine, M. du Maine et le cardinal de Bissy sur
lesquels elle ne se contraignit pas\,; enfin, emporté par l'impétuosité
impérieuse de Madame, qui n'oublia journellement rien pour la grandeur
de son gendre, la faiblesse succomba, mais l'exécution l'embarrassait.

Il sentit bien quelle étrange déprédation il allait faire sur la glèbe
de la couronne et sur sa majesté, qui lui étaient l'une et l'autre
confiées et remises en sa garde pendant la minorité, et sans le moindre
prétexte. Il ne sentait pas moins ce qui s'en pourrait dire un jour. Il
comprit que dans ces commencements de mouvements qu'il ne pouvait se
dissimuler par la cadence de ceux de cette prétendue noblesse, du
parlement et de la Bretagne, il trouverait peut-être une opposition dans
le maréchal d'Huxelles, qui pouvait le faire échouer, mais que, évitant
de le rendre l'artisan du traité, il le pouvait compter plus flexible
quand il ne s'agirait simplement que d'opiner.

Il le cajola donc, et lui fit entendre qu'y ayant beaucoup de petites
choses locales à ajuster avec le duc de Lorraine et des prétentions à
discuter de sa part, il croyait que ces bagatelles, qui voulaient être
épluchées, lui donneraient plus de peine qu'elles ne valaient et lui
feraient perdre un temps mieux employé\,; que, de plus, il fallait
quelqu'un qui fût au fait de toutes ces choses, qui par conséquent
entendrait à demi-mot et qui fût encore rompu dans la connaissance de la
petite cour de Lorraine\,; que ces raisons lui avaient fait jeter les
yeux sur Saint-Contest, qui avait été si longtemps intendant de Metz,
qui savait par coeur le local, les prétentions et la cour de Lorraine,
qui de plus avait été troisième ambassadeur à Bade, où la paix de
l'empereur, qui avait tant porté les intérêts du duc de Lorraine, et
celle de l'empire avaient reçu leur dernière main, et qu'il pensait que
Saint-Contest était celui qu'il pouvait choisir comme le plus instruit
et le plus propre à travailler au traité, comme commissaire du roi, avec
ceux du duc de Lorraine et en rendre compte après au conseil de régence.

L'affaire n'était pas assez friande pour tenter le maréchal d'Huxelles
ni pour lui donner de la jalousie, ravi qu'il fait de tirer son épingle
du jeu pour fronder après tout à son aise avec son ami M. du Maine, qui
ne demandait pas mieux qu'à voir faire au régent des choses qu'on pût
justement lui reprocher, tandis qu'il lui cherchait des crimes dans les
plus innocentes, même dans les plus utiles. Huxelles approuva et mit le
régent fort à l'aise.

Saint-Contest était l'homme qu'il lui fallait pour ne chercher qu'à lui
plaire et ne regarder à rien par delà. Il avait de la capacité et de
l'esprit, infiniment de liant, et sous un extérieur lourd, épais,
grossier et simple, beaucoup de finesse et d'adresse, une oreille qui
entendait à demi-mot, un désir de plaire au-dessus de tout qui ne laissa
rien à souhaiter au régent ni au duc de Lorraine dans tout le cours de
cette affaire, qui ne fut pas long.

Lorsqu'elle fut bien avancée, M. le duc d'Orléans, à qui il en rendait
souvent compte, songea à s'assurer des principaux du conseil de régence.
Les princes du sang, avides pour eux-mêmes, et d'ailleurs n'entendant
rien et ne sachant rien, n'étaient pas pour lui résister\,; les bâtards
pincés de si frais et qui craignaient pis, encore moins, outre la raison
qui vient d'être touchée sur le duc du Maine\,; le garde des sceaux, à
peine en place, ne songeait qu'à s'y conserver\,; le maréchal de
Villeroy, qui aurait eu là de quoi exercer dignement son amertume, était
tenu de court dans cette affaire par son beau-frère le grand écuyer,
devant lequel de sa vie il n'avait osé branler. Tallard, son protégé,
était d'ailleurs tenu aussi de court par les Rohan, soumis à
M\textsuperscript{me} de Remiremont et à M\textsuperscript{me}
d'Espinay. Le duc de Noailles et son ami d'Effiat n'avaient garde de
résister quand il ne s'agissait ni du parlement ni de la robe. Le
matamore Villars était toujours souple comme un gant. Le maréchal
d'Estrées sentait, savait, lâchait quelque demi-mot mais mourait de peur
de déplaire, et se dédommageait, ainsi que le maréchal d'Huxelles, en
blâmant, tout bas ce qui se faisait aux uns et aux autres, à quoi ils
n'avaient pas la force de contredire le régent. La différence était
qu'Estrées était fâché du mal sincèrement et en honnête homme\,;
Huxelles, au contraire, pour s'en donner l'honneur, verser son fiel, et
quand les choses ne touchaient ni à son personnel ni à ses vues, était
ravi des fautes et en riait sous cape, comme il fit en cette occasion,
ainsi que M. du Maine. D'Antin était trop bas courtisan et trop mal en
selle auprès du régent pour oser souffler. Pour la queue du conseil,
elle n'osait donner le moindre signe de vie, sinon Torcy, quelquefois
pressé de lumière et de probité, mais si rarement et avec tant de
circonspection, que cela passait de bien loin la modestie.

M. le duc d'Orléans, qui n'avait pas oublié mon aventure avec lui au
conseil et la convention qui l'avait suivie, que j'ai racontée (t. XIV,
p.~187) et qui se douta que je ne serais pas aisé à persuader sur ce
traité, m'en parla à trois ou quatre diverses fois avec grande
affection. Je lui représentai ce que je viens d'expliquer tant sur le
démembrement des parties considérables de la Champagne, que sur le
traitement d'Altesse Royale. Je le fis souvenir qu'outre que M. de
Lorraine était sans aucun prétexte d'avoir à le ménager pour quoi que ce
fût dans la situation particulière où il était, ni dans celle où
l'Europe se trouvait alors, même où elle pût être dans la suite\,; il
n'y avait pas si longtemps que les traités de paix d'Utrecht et de Bade
avaient passé l'éponge sur toutes ces prétentions et ces dédommagements
tant demandés, si appuyés de l'empereur, et toujours si constamment
refusés\,; qu'il ne pouvait l'avoir oublié, et que je ne comprenais
point comment il osait les faire renaître, les réaliser de sa pure et
personnelle grâce, les faire monter au delà même de toute espérance,
comme lorsque, avant les derniers traités de paix générale, les
prétentions bonnes ou mauvaises subsistaient en leur entier\,; s'exposer
à faire de son chef un présent, et aussi considérable, purement gratuit,
dépouillé de toute cause, raison et prétexte, à un prince son
beau-frère, sans force, sans considération, sans la plus légère
apparence de droit\,; abuser de sa régence aux dépens de l'État qui lui
était confié pendant la minorité d'un roi qui pourrait un jour lui en
demander compte et raison, et qui ne manquerait pas de gens autour de
lui qui l'y exciteraient\,; qu'à l'égard de l'Altesse Royale, dont je
lui démêlai le vrai des fausses apparences dont M. de Lorraine
l'embrouillait à dessein, que je comprenais aussi peu qu'il voulût
avilir la majesté de la couronne, qui ne lui était pas moins confiée que
l'État, et la prostituer sans cause, raison ni prétexte quelconque, que
de sa bonne volonté de gratifier son beau-frère, en la dégradant, et en
même temps la sienne propre, celle de M\textsuperscript{me} sa soeur et
la supériorité des princes du sang sur M. de Lorraine, en lui donnant de
sa pleine et unique grâce un traitement si supérieur à celui des princes
du sang, et traitement, de plus, qui ne pouvait leur être donné. J'allai
jusqu'à lui dire qu'il y avait en lui un aveuglement qui tenait du
prestige de préférer de si loin un petit prince totalement inutile et
sans la moindre apparence de droit, de maison fatale à la sienne tant et
toutes les fois qu'elle l'a pu, et personnellement ennemie, à preuves
signalées, et qui depuis ne respirait toujours que la cour de Vienne, le
préférer, dis-je, et de si loin, à l'État et à la majesté de la
couronne, dont lui était dépositaire, au roi, à soi-même et à sa propre
maison\,; de hasarder les reproches que le roi lui en pourrait faire un
jour, et s'exposer au qu'en-dira-t-on public dans un temps où il voyait
tant de fermentation contre lui et contre son gouvernement. J'ajoutai,
sur l'Altesse Royale, qu'il verrait naître la même prétention, sur cet
exemple, des princes qui n'y avaient pas encore pensé, et qu'il se
trouverait peut-être, par leur position et par les conjonctures,
également embarrassé de satisfaire et de mécontenter.

Ces remontrances, que j'abrège, ne produisirent que de l'embarras et de
la tristesse dans son esprit. S'il ne m'avait pas caché le voyage
jusqu'au moment qu'il fut consenti et prêt à entreprendre, car le secret
en fut généralement observé, et M. de Lorraine en avait bien ses
raisons, j'aurais fait de mon mieux pour le détourner, au moins pour y
faire mettre la condition expresse qu'il ne s'y ferait aucune sorte de
demande, beaucoup moins de traité, et je pense bien aussi que M. le duc
d'Orléans ne se douta d'aucune proposition que lorsque, après l'arrivée,
elles lui furent faites. Il fit quelques tours la tête basse, et rompit
après le silence en me disant qu'il voulait que Saint-Contest vînt chez
moi me rapporter l'affaire, que je la trouverais peut-être autre que je
ne pensais, et que c'était une complaisance que je ne pouvais lui
refuser. Je ne le pus en effet, et tout aussitôt après que j'y eus
consenti il me parla d'autre chose.

Saint-Contest était fort de mes amis\,; son père et son grand-père
maternel, doyen du parlement, avaient toujours été fort, attachés à mon
père. Saint-Contest vint chez moi, rendez-vous pris. Il y passa depuis
la sortie du dîner jusque dans le soir fort tard. Il y déploya tout son
bien-dire en homme qui voulait plaire à M. le duc d'Orléans et lui
valoir ma conquête. Tout fut détaillé, expliqué, discuté, et le plus ou
moins de valeur, et d'autres conséquences de ce qu'on donnait en
Champagne à incorporer pour toujours à la Lorraine en toute
souveraineté. Je n'eus pas peine à reconnaître qu'il avait ordre de ne
rien oublier pour me gagner, et qu'en effet il y mit aussi tous ses
talents. Mais son esprit, son adresse, son accortise, ses ambages et ses
finesses y échouèrent au point qu'après avoir bien tout dit et répété de
part et d'autre, moi avec plus d'étendue et de force que ce que je viens
d'exposer, il ne put me donner aucune sorte de raison du démembrement en
Champagne, ni du traitement d'Altesse Royale, autre que la qualité de
beau-frère de M. le duc d'Orléans, qui se trouvait régent et en état,
par conséquent, de lui faire ces grâces. Il sourit à la fin, et par un
dernier effort, espérant peut-être m'embarrasser, et par là venir à me
réduire, il me demanda franchement ce que, je voulais donc qu'il dît à
M. le duc d'Orléans de notre conférence. «\,Tout ce que je viens de vous
dire, répondis-je, que je ne suis ni si hardi ni si prodigue que lui à
donner pour rien l'honneur du roi et la substance de l'État, qui lui en
demandera compte\,; que c'est à lui à voir ce qu'il répondra lors, et en
attendant comment il soutiendra le cri public et les discours de toute
l'Europe\,; que moi, plus timide et plus Français, plus jaloux de
l'intégrité de l'État et de la majesté royale, il ne me serait pas
reproché d'avoir consenti à un traité qui attaquait l'un et l'autre de
gaieté de coeur, unique par ses fondements en faveur du prince du monde
qui, à toutes sortes de titres, en méritait moins les grâces\,; que je
m'y opposerais de toutes mes forces et de toutes mes raisons, quoique
parfaitement convaincu que ce serait en vain, mais uniquement pour
l'acquit de ma conscience et de mon honneur, que j'y croirais autrement
fortement engagés l'un et l'autre.\,» Saint-Contest, effrayé de ma
fermeté, me demanda si je voulais sérieusement qu'il rapportât
fidèlement au régent tout ce que je venais de lui dire. Je l'assurai
qu'il le pouvait, et que j'avais dit pis encore à M. le duc d'Orléans.

Saint-Contest s'en alla fort consterné et rendit compte à M. le duc
d'Orléans de notre conférence. M. le duc d'Orléans m'envoya chercher, et
fit encore des efforts pour gagner au moins ma complaisance. Voyant
qu'il n'y pouvait réussir, il me pria à la fin de ne me point trouver au
conseil de régence, lorsque Saint-Contest y apporterait ce traité. Je le
lui promis avec grand soulagement, car mon avis ne l'aurait pas empêché
de passer, et aurait fait du bruit et grand'peine à M. le duc d'Orléans.
Pareille chose m'arriva lorsque le régent eut la faiblesse d'accorder le
traitement égal de Majesté au roi de Danemark, et de Hautes Puissances
aux États généraux. Il ne put le gagner, ni moi l'empêcher, et je
m'absentai du conseil de régence le jour que M. le duc d'Orléans y fit
passer cette dégradation de la couronne de France. Il m'avertit deux
jours auparavant. Je me fis excuser par La Vrillière à ce conseil et
même au suivant, comme incommodé, pour, qu'il n'y parût pas
d'affectation, et je mis le régent fort à l'aise. Le traité passa au
conseil, au rapport de Saint-Contest, sans la plus légère contradiction,
quoique sans l'approbation de personne, où mon absence ne laissa pas
d'être doucement remarquée.

Le parlement, devenu si épineux et bientôt après, si fougueux,
l'enregistra tout de suite le 7 avril sans la moindre ombre de
difficulté. Il blessait fort le roi et l'État\,; mais il ne touchait ni
à la bourse, ni aux chimères, ni aux prétentions de ces prétendus
tuteurs de nos rois mineurs, et protecteurs du royaume et de ses
peuples.

M. de Lorraine, ravi d'aise d'avoir obtenu par-dessus même ses
espérances, ne voulut point partir avant l'enregistrement fait au
parlement. Mais l'affaire ainsi entièrement consommée, il ne songea plus
qu'à s'en aller. Sûre de l'enregistrement dès la veille,
M\textsuperscript{me} la duchesse de Lorraine fut aux Tuileries prendre
congé du roi, qui le lendemain vint au Palais-Royal lui souhaiter un bon
voyage. Elle fut ensuite dire adieu à M\textsuperscript{me} la duchesse
de Berry à Luxembourg, qui le même soir vint au Palais-Royal l'embrasser
encore. Le lendemain 8 avril elle partit avec le duc de Lorraine, qui
eut de quoi être bien content et se bien moquer de nous.

Il ne laissa pas d'être bien singulièrement étrange que le duc de
Lorraine, sous le ridicule incognito de comte de Blamont, soit venu à
Paris, y soit demeuré près de deux mois, logé et défrayé de tout au
Palais-Royal, y ait paru aux spectacles, au Cours, dans tous les lieux
publics, ait été voir Versailles et Marly, ait visité la reine
d'Angleterre à Saint-Germain, ait paru publiquement partout, ait reçu
plusieurs fêtes, et que le roi étant dans les Tuileries pendant ces deux
mois\footnote{On a écrit à la marge du manuscrit\,: «\,Le duc de
  Saint-Simon se trompe. Le duc de Lorraine, le lendemain de son
  arrivée, 19 février, vit le roi. Ce fait est peu important\,; mais il
  y a de l'affectation à dire le contraire.\,» Cette note marginale est
  probablement de M. Le Dran, comme celle que nous avons déjà citée, t.
  XIV, p.~339.}, ce beau comte de Blamont ne l'ait pas vu une seule
fois, ni pas un prince, ni une princesse du sang\,; que cette audace ait
été soufferte, dont l'insolence s'est fait d'autant plus remarquer, que
M\textsuperscript{me} la duchesse de Lorraine a rempli et reçu tous les
devoirs de son rang, parce qu'il était tout certain, comme petite-fille
de France\,; il ne le fut pas moins qu'il n'y ait pas été seulement
question de son hommage de Bar au roi, qui de son règne ne l'avait pas
encore reçu. Mais il sembla être arrêté que tout ce voyage serait
uniquement consacré à la honte et au grand dommage du roi et du royaume.

Le concours fut grand au Palais-Royal pendant ce voyage\,; on en crut
faire sa cour au régent. M. de Lorraine voyait le monde debout chez
M\textsuperscript{me} la duchesse de Lorraine. Peu de gens allèrent chez
lui, et encore sur la fin. C'est où je ne mis pas le pied\,: j'allai
seulement deux fois chez M\textsuperscript{me} la duchesse de
Lorraine\,; je crus avec cela avoir rempli tout devoir. J'ai voulu
couler à fond tout ce voyage de suite, pour n'avoir pas à en interrompre
souvent d'autres matières. Je n'y ajouterai que peu de choses
nécessaires avant que de reprendre le fil de celles que ce récit a
interrompues.

M. le duc d'Orléans ne fut pas longtemps à attendre un des effets de ce
qu'il avait accordé, que je lui avais prédits. Le grand-duc {[}de
Toscane{]}, gendre de Gaston, et M\textsuperscript{me} la
grande-duchesse, petite-fille de France, vivante, dont il avait des
enfants, se crut avec raison au même droit que M. de Lorraine. Il était
plus considérable que lui par l'étendue, la richesse, la position de ses
États\,; il avait toujours été attaché à la France\,; il en avait donné
au feu roi dans tous les temps toutes les preuves que sa sagesse et la
politique lui pouvait permettre, et, quoique sa maison ne pût égaler
celle de Lorraine, elle avait eu l'honneur au-dessus d'elle de donner
deux reines à la France, de la dernière desquelles la branche régnante
est issue, et d'avoir les plus proches alliances avec la maison
d'Autriche et la plupart des premiers princes de l'Europe, tandis que la
reine Louise, fille d'un particulier cadet de Lorraine, n'avait été ni
pu être épousée par Henri III que par amour et n'avait jamais eu
d'enfants. Le grand-duc fit donc instance pour obtenir aussi le
traitement d'Altesse Royale, et il n'y eut pas jusqu'au duc de
Holstein-Gottorp qui ne se mît à la prétendre, fondé sur sa proche
alliance avec les trois couronnes du nord. Mais ces princes n'avaient
pas auprès du régent les mêmes accès du duc de Lorraine aussi ne
purent-ils réussir.

Je ne puis, à propos de ce voyage à Paris de M. et de
M\textsuperscript{me} de Lorraine, omettre une bagatelle, parce qu'elle
ne laisse pas de montrer de plus en plus le caractère de M. le duc
d'Orléans. Un jour que M\textsuperscript{me} la duchesse d'Orléans était
allée à Montmartre, qu'elle quitta bientôt après, me promenant seul avec
M. le duc d'Orléans, dans le petit jardin du Palais-Royal, à parler
d'affaires assez longtemps et qui n'étaient point du traité de Lorraine,
il s'interrompit tout à coup, et se tournant à moi\,: «\,Je vais, me
dit-il, vous apprendre une chose qui vous fera plaisir.\,» De là il me
conta qu'il était las de la vie qu'il menait\,; que son âge ni ses
besoins ne la demandaient plus, et force choses de cette sorte\,; qu'il
était résolu de rompre ses soirées, de lés passer honnêtement, et plus
sobrement et convenablement, quelquefois chez lui, souvent chez
M\textsuperscript{me} la duchesse d'Orléans\,; que sa santé y gagnerait,
et lui du temps pour les affaires, mais qu'il ne ferait ce changement
qu'après le départ de M. et de M\textsuperscript{me} de Lorraine qui
serait incessamment, parce qu'il crèverait d'ennui de souper tous les
soirs chez M\textsuperscript{me} la duchesse d'Orléans avec eux et avec
une troupe de femmes\,; mais que, dès qu'ils seraient partis, je pouvais
compter qu'il n'y aurait plus de soupers de roués et de putains, ce
furent ses propres termes, et qu'il allait mener une vie sage,
raisonnable et convenable à son âge et à ce qu'il était.

J'avoue que je me sentis ravi dans mon extrême surprise par le vif
intérêt que je prenais en lui. Je le lui témoignai avec effusion de
coeur en le remerciant de cette confidence. Je lui dis qu'il savait que
depuis bien longtemps je ne lui parlais plus de l'indécence de sa vie ni
du temps qu'il y perdait, parce que j'avais reconnu que j'y perdais le
mien\,; que je désespérais depuis longtemps qu'il pût changer de
conduite\,; que j'en avais une grande douleur\,; qu'il ne pouvait
ignorer à quel point je Pavais toujours désiré par tout ce qui s'était
passé entre lui et moi là-dessus à bien des reprises, et qu'il pouvait
juger de la surprise et de la joie qu'il me donnait. Il m'assura de plus
en plus que sa résolution était bien prise, et là-dessus je pris congé
parce que l'heure de sa soirée arrivait.

Dès le lendemain je sus par gens à qui les roués venaient de le conter,
que M. le duc d'Orléans ne fut pas plutôt à table avec eux qu'il se mit
à rire, à s'applaudir et à leur dire qu'il venait de m'en donner d'une
bonne où j'avais donné tout de mon long. Il leur fit le récit de notre
conversation, dont la joie et l'applaudissement furent merveilleux.
C'est la seule fois qu'il se soit diverti à mes dépens, pour ne pas dire
aux siens, dans une matière où la bourde qu'il me donna, que j'eus la
sottise de gober par une joie subite qui m'ôta la réflexion, me faisait
honneur et ne lui en faisait guère. Je ne voulus pas lui donner le
plaisir de lui dire que je savais sa plaisanterie ni de le faire
souvenir de ce qu'il m'avait dit\,: aussi n'osa-t-il m'en parler.

Je n'ai jamais démêlé quelle fantaisie lui avait pris de me tenir ce
langage pour en aller faire le conte, à moi qui depuis des années ne lui
avais pas ouvert la bouche de la vie qu'il menait, dont aussi il se
gardait bien de me rien dire ni de rien qui y eût trait. Bien est-il
vrai que quelquefois étant seul avec ses valets confidents, il lui est
assez rarement échappé quelque plainte, mais jamais devant d'autres, que
je le malmenais et lui parlais durement, cela en gros, en deux mots,
sans y rien ajouter d'aigre ni que j'eusse tort avec lui. Il disait vrai
aussi\,: quelquefois, quand j'étais poussé à bout sur des déraisons ou
des fautes essentielles, en affaires et en choses importantes, qui
regardaient ou lui ou l'État, et qu'après encore être convenus par
bonnes raisons de quelque chose d'important à éviter ou à faire, lui
très persuadé et résolu, sa faiblesse ou sa facilité me tournaient dans
la main et lui arrachaient tout le contraire, que lui-même sentait comme
moi tel qu'il était, et c'est une des choses qui m'a le plus cruellement
exercé avec lui\,; mais la niche qu'il me faisait volontiers plus tète à
tête que devant des tiers, et dont ma vivacité était toujours la dupe,
c'était d'interrompre tout à coup un raisonnement important par un
\emph{sproposito} de bouffonnerie. Je n'y tenais point, la colère me
prenait quelquefois jusqu'à vouloir m'en aller. Je lui disais que, s'il
voulait plaisanter, je plaisanterais tant qu'il voudrait, mais que de
mêler les choses les plus sérieuses de parties de main, de
bouffonneries, cela était insupportable. Il riait de tout son coeur, et
d'autant plus que cela n'étant pas rare, et moi en devant être en garde,
je n'y étais jamais et que j'avais dépit et de la chose et de m'en
laisser surprendre\,; et puis il reprenait ce que nous traitions. Il
faut bien que les princes se délassent et badinent quelquefois avec ceux
qu'ils veulent bien traiter d'amis. Il me connaissait bien tel aussi, et
quoiqu'il ne fût pas toujours content de ce qu'il appelait en ces
moments dureté en moi, et que sa faiblesse, qui le faisait quelquefois
cacher de moi sur des choses qu'il sentait bien que je combattrais,
l'entraînât trop souvent, il ne laissait pas d'avoir pour moi toute
l'amitié, l'estime, la confiance dont il était capable, qui surnageait
toujours aux nuages qui s'élevaient quelquefois et aux manèges et aux
attaques de ceux de sa plus grande faveur, comme l'abbé Dubois,
Noailles, Canillac et d'autres de ses plus familiers. Ses disparates
avec moi, qui étaient très rares et toujours avec grande considération,
étaient froid, bouderie, silence. Cela était toujours très court. Il n'y
tenait pas lui-même\,; je m'en apercevais dans le moment\,; je lui
demandais librement à qui il en avait et quelle friponnerie on lui avait
dite\,; il m'avouait la chose avec amitié et il en avait honte, et je me
séparais d'avec lui toujours mieux que jamais.

Le hasard m'apprit un jour ce qu'il pensait de moi le plus au naturel.
Je le dirai ici, pour sortir une fois pour toutes de ces bagatelles. M.
le duc d'Orléans, retournant une après-dînée du conseil de régence des
Tuileries au Palais-Royal, avec M. le duc de Chartres et le bailli de
Conflans, lors premier gentilhomme de sa chambre, seul en tiers avec
eux, se mit à parler de moi dès la cour des Tuileries, fit à M. son fils
un éloge de moi tel que je ne l'ose rapporter. Je ne sais plus ce qui
s'était passé au conseil ni ce qui y donna lieu. Ce que je dirai
seulement, c'est qu'il insista sur son bonheur d'avoir un ami en moi
aussi fidèle, aussi constant dans tous les temps, aussi utile que je lui
étais et lui avais été en tous, aussi sûr, aussi vrai, aussi
désintéressé, aussi ferme, tel qu'il ne s'en trouvait point de pareil,
sur qui il avait pu compter dans tous les temps, qui lui avait rendu les
plus grands services, et qui lui parlait vrai, droit et franc sur tout,
et sans intérêt. Cet éloge dura jusqu'à ce qu'ils missent pied à terre
au Palais-Royal, disant à M. son fils qu'il voulait lui apprendre à me
connaître, et le bonheur et l'appui, car tout ce qui est rapporté ici
fut exactement ses termes, qu'il avait toujours trouvés dans mon amitié
et dans mes conseils. Le bailli de Conflans, étonné lui-même de cette
abondance, me la rendit le surlendemain sous le secret, et j'avoue que
je n'ai pu l'oublier. Aussi est-il vrai que, quoi qu'on ait pu faire, et
jusqu'à moi-même, par dégoût et dépit quelquefois de ce que je voyais
mal faire, il est toujours revenu à moi, et presque toujours le premier,
avec honte, amitié, confiance, et ne s'est jamais trouvé en aucun
embarras, qu'il ne m'ait recherché, ouvert son coeur, et consulté de
tout avec moi, sans néanmoins m'en avoir cru toujours, détourné après
par d'autres. Cela n'arrivait pourtant pas bien souvent, et c'est après
où il était honteux et embarrassé avec moi, et où quelquefois je
m'échappais un peu avec lui, quand il se trouvait mal de s'être laissé
aller à des avis postérieurs différents du mien\,: on l'a vu souvent
ici, et la suite le montrera encore.

Il n'était pas pour se contenter d'une maîtresse. Il fallait de la
variété pour piquer son goût. Je n'avais non plus de commerce avec elles
qu'avec ses roués. Jamais il ne m'en parlait, ni moi à lui. J'ignorais
presque toujours leurs aventures. Ces roués et des valets s'empressaient
de lui en présenter, et dans le nombre il se prenait toujours de
quelqu'une. M\textsuperscript{me} de Sabran (Foix-Rabat par elle), et de
qui j'ai parlé, lorsque sa mère eut besoin pour ses affaires de paraître
quelques moments à la cour, s'était échappée d'elle pour épouser un
homme d'un grand nom mais sans biens et sans mérite qui la mît en
liberté. Il n'y avait rien de si beau qu'elle, de plus régulier, de plus
agréable, de plus touchant, de plus grand air et du plus noble, sans
aucune affectation. L'air et les manières simples et naturelles,
laissant penser qu'elle ignorait sa beauté et sa taille, qui était
grande et la plus belle du monde, et quand il lui plaisait, modeste à
tromper. Avec beaucoup d'esprit, elle était insinuante, plaisante,
robine, débauchée, point méchante, charmante surtout à table. En un mot
elle avait tout ce qu'il fallait à M. le duc d'Orléans, dont elle devint
bientôt la maîtresse, sans préjudice des autres.

Comme elle ni son mari n'avaient rien, tout leur fut bon, et si ne
firent-ils pas grande fortune. Montigny, frère de Turmenies, un des
gardes du trésor royal, était un des chambellans de M. le duc d'Orléans,
à six mille livres d'appointements, qui le fit son premier maître
d'hôtel à la mort de Matharel qui l'était. M\textsuperscript{me} de
Sabran trouva que six mille livres de rente étaient toujours bonnes à
prendre pour son mari, dont elle faisait si peu de cas, qu'en parlant de
lui elle ne l'appelait que son mâtin. M. le duc d'Orléans lui donna la
charge qu'il paya à Montigny. C'est elle qui, soupant avec M. le duc
d'Orléans et ses roués, lui dit fort plaisamment que les princes et les
laquais avaient été faits de la même pâte, que Dieu avait dans la
création séparée de celle dont il avait tiré tous les autres hommes.

Toutes ses maîtresses, en même temps, avaient chacune leur tour. Ce
qu'il y avait d'heureux, c'est qu'elles pouvaient fort peu de chose et
n'avaient part en aucun secret d'affaires, mais tiraient de l'argent,
encore assez médiocrement\,; le régent s'en amusait et en faisait le cas
qu'il en devait faire. Retournons maintenant d'où le voyage de M. et de
M\textsuperscript{me} de Lorraine et ces bagatelles nous ont détournés.

\hypertarget{chapitre-xii.}{%
\chapter{CHAPITRE XII.}\label{chapitre-xii.}}

1718

~

{\textsc{Mouvement du parlement à l'occasion d'arrêts du conseil sur les
billets d'État et les monnaies.}} {\textsc{- Lettres de cachet à des
Bretons.}} {\textsc{- Députation et conduite du parlement de Bretagne.}}
{\textsc{- Breteuil intendant de Limoges.}} {\textsc{- Conférence du
cardinal de Noailles avec le garde des sceaux chez moi, dont je suis peu
content.}} {\textsc{- Sommes données par le régent aux abbayes de la
Trappe et de Septfonts.}} {\textsc{- Ma conduite à cet égard avec le duc
de Noailles et avec M. de Septfonts, avec qui je lie une étroite
amitié.}} {\textsc{- Mariage de Maurepas avec la fille de La
Vrillière.}} {\textsc{- Mort de Fagon, premier médecin du feu roi.}}
{\textsc{- Mort et dispositions de l'abbé d'Estrées.}} {\textsc{-
Conversion admirable de la marquise de Créqui.}} {\textsc{- Cambrai
donné au cardinal de La Trémoille, et Bayeux à l'abbé de Lorraine.}}
{\textsc{- Promotion et confusion militaire.}} {\textsc{- J'obtiens un
régiment pour le marquis de Saint-Simon\,; qui meurt trois mois après\,;
puis pour son frère.}} {\textsc{- Broglio l'aîné\,; son caractère.}}
{\textsc{- Il engage le régent à un projet impossible de casernes et de
magasins, et à l'augmentation de la paye des troupes.}} {\textsc{-
Sagesse de l'administration de Louvois.}} {\textsc{- Les chefs des
conseils mis dans celui de régence sans perdre leurs places dans les
leurs.}} {\textsc{- Survivances du gouvernement de Bayonne, etc., et du
régiment des gardes, accordées au fils aîné du duc de Guiche, et autres
grâces faites à Rion, Maupertuis, La Chaise, Heudicourt.}} {\textsc{-
Nouvelles étrangères.}} {\textsc{- Légèreté du cardinal de Polignac, qui
tâche inutilement de se justifier au régent de beaucoup de choses.}}
{\textsc{- Désordre des heures d'Argenson.}} {\textsc{- Law et lui font
seuls toute la finance.}} {\textsc{- Il obtient le tabouret pour sa
femme, à l'instar de la chancelière, premier exemple dont Chauvelin
profita depuis.}} {\textsc{- Mort de Menars, président à mortier.}}
{\textsc{- Maupeou, aujourd'hui premier président, a sa charge.}}
{\textsc{- Querelles domestiques du parlement suspendues par des
considérations plus vastes.}} {\textsc{- Beaufrémont, de concert avec
ceux qui usurpaient le nom collectif de noblesse, insulte impunément les
maréchaux de France, qui en essuient l'entière et publique
mortification.}} {\textsc{- Caractère de Baufremont, qui se moque après
et aussi publiquement de M. le Duc, et aussi impunément.}} {\textsc{-
Catastrophe de Monasterol.}} {\textsc{- Mort de La Hire et de l'abbé
Abeille.}} {\textsc{- Mort de Poirier, premier médecin du roi.}}
{\textsc{- Dodart mis en sa place.}} {\textsc{- Prudente conduite du
régent en cette occasion.}} {\textsc{- Caractère de Dodart et de son
père.}} {\textsc{- Caractère et infamie de Chirac.}}

~

Le samedi 12 février, il fut résolu au conseil de régence de faire
recevoir à la monnaie les vieilles espèces et matières d'or et d'argent,
et d'en prendre un sixième porté en billets d'État, dans l'espérance de
remettre beaucoup d'argent dans le commerce, et de moins de perte sur
les billets en faveur de qui s'en voulait défaire. On publia le
lendemain deux arrêts du conseil sur la monnaie et sur les billets, qui
perdirent moins dès le même jour, et presque aussitôt après, un
troisième pour recevoir les louis d'or à dix-huit livres qui en valaient
vingt-quatre, et au contraire les écus à quatre livres dix sous qui ne
valaient que quatre livres. Ces arrêts donnèrent lieu au parlement de
remuer. Il résolut des remontrances et les fit au roi le 21 février\,:
le premier président ne dit que trois mots\,; il n'en fallait pas
davantage pour commencer. Il y eut une autre assemblée le lendemain, qui
se passa avec assez de chaleur et de bruit. On y fut mal content de la
réponse vague du garde des sceaux, et la résolution y fut prise de se
rassembler le premier vendredi de carême pour arrêter de nouvelles
remontrances. Le premier président et les gens du roi vinrent en rendre
compte au régent. Law fut l'objet de ce premier mouvement. L'assemblée
projetée se tint au jour arrêté\,; on ne put s'y accorder il y eut trois
différents avis. À la fin ils convinrent de nommer quatorze
commissaires, dont sept de la grand'chambre, et un de chacune des cinq
chambres des enquêtes et des deux des requêtes, pour examiner ce qu'il
convenait à la compagnie de dire et de demander sur cette réponse vague
du garde des sceaux aux premières remontrances.

Rochefort, président à mortier du parlement de Bretagne\,; Lambilly,
conseiller du même parlement, et quelques gentilshommes du même pays qui
s'assemblaient souvent et fort hautement chez ce président à Rennes,
reçurent des lettres de cachet pour venir à Paris rendre compte de leur
conduite. Il y arriva une députation du parlement de Bretagne chargée de
remontrances au roi, sur le contenu desquelles ils disputèrent fort avec
le garde des sceaux et envoyèrent un courrier à leur compagnie. Elle
modéra les articles qui avaient causé l'envoi du courrier. Dans tout cet
intervalle les gentilshommes bretons mandés et arrivés à Paris furent
exilés. La conduite du parlement de Bretagne ayant paru plus
respectueuse par la réforme de ses remontrances, le garde des sceaux se
chargea de les porter au régent qui, ravi de trouver occasion de
douceur, permit aux gentilshommes bretons exilés et au président et au
conseiller mandés à Paris, qui y étaient toujours, de retourner chez
eux, et il permit aux députés du parlement de Bretagne de faire la
révérence au roi et de lui présenter les remontrances dont leur
compagnie les avait chargés. Tout cela ne fut pas plutôt exécuté, que le
parlement de Bretagne lit de nouvelles entreprises à propos des quatre
sous pour livre qu'on avait remis sur les entrées, et que le président
de Rochefort et le conseiller Lambilly, renvoyés à Rennes, à condition
d'aller en arrivant voir le maréchal de Montesquiou, qui commandait en
Bretagne, n'y voulurent pas mettre le pied. Après quelque peu de
patience, en espérance de les y réduire, et eux plus fermes que jamais,
ils furent exilés, le président à Auch, le conseiller à Tulle. Cinq
semaines après, Brillac fit aussi des siennes. Il était premier
président du parlement de Bretagne. Sa mauvaise conduite l'avait fait
mander à Paris, où on le tenait exprès depuis quelque temps à se
morfondre. Voyant que cela ne finissait point, il partit un beau jour et
laissa une lettre pour le garde des sceaux, par laquelle il le priait de
recevoir ses excuses et de les vouloir bien aussi porter à M. le duc
d'Orléans de ce qu'il s'en allait à Rennes, où ses affaires domestiques
l'appelaient, sans avoir pris congé. On lui dépêcha sur-le-champ une
lettre de cachet par un courrier qui le rencontra à Dreux, d'où, suivant
cet ordre, il prit le chemin d'une terre qu'il avait en Poitou. On ne
sut ce qui le pressait de retourner en Bretagne, où il était également
mal voulu et méprisé. Sa réputation avec de l'esprit et quelque capacité
était plus qu'équivoque pour en parler modestement. Celle de sa femme ne
l'était pas moins en autre genre. Elle était fort jolie, avait de
l'esprit, beaucoup d'intrigue, et avait aspiré de parvenir à plaire au
régent\,; je crois même qu'il en fut quelque chose, et rien de tout cela
ne déplaisait à Brillac qui savait tirer parti de tout, et qui la laissa
à Paris.

Breteuil, maître des requêtes, fils du conseiller d'État et neveu de
l'introducteur des ambassadeurs, fut en ce temps-ci envoyé intendant de
Limoges, une des moindres de toutes les intendances. Je le remarque ici
parce qu'il y trouva sa fortune, comme on le verra en son lieu.

Le garde des sceaux ne fut pas longtemps sans me tenir parole sur la
conférence que je lui avais demandée avec le cardinal de Noailles. Tous
deux vinrent chez moi un soir à rendez-vous pris. Nous fûmes longtemps
tous trois ensemble. On ne peut mieux dire ni mieux parler que fit le
cardinal. À la politesse près, on ne peut rien dire de plus mal que
furent les propos coupés et embarrassés du garde des sceaux. J'y mis du
mien tout ce que je me crus permis pour réchauffer sa respectueuse
glace\,; mais je vis clairement que le vieux levain prévalait, et qu'il
ne se dépouillerait point de cette vieille peau jésuitique, l'aspect que
la fortune lui avait fait revêtir sous le feu roi, et que ses fonctions
de la police, c'est-à-dire de l'inquisition, avaient de plus en plus
collée et encuirassée en lui. Tout ne se passa qu'honnêtement, et tout
le fruit qui s'en put tirer fut que le cardinal sentit nettement à qui
il avait affaire, et que je compris qu'il y aurait toujours à veiller et
à être en garde contre ce magistrat dans tout ce qui regarderait les
matières de Rome, le cardinal de Noailles et les jésuites et les
croupiers des deux partis.

J'eus lieu d'être plus content de Law. Depuis que le duc de Noailles
n'eut plus les finances, ce fut à Law à qui j'eus affaire pour la Trappe
et pour Septfonts\,; il me facilita tout de la meilleure grâce du monde.
Les payements coulèrent régulièrement. J'avais soin à chacun de faire la
part de Septfonts, et j'eus celui de faire ensuite comprendre cette
abbaye dans un supplément que j'obtins du régent pour la Trappe, qui,
pour le dire tout de suite, eut en tout quarante mille écus, et
Septfonts plus de quatre-vingt mille livres, ce qui sauva ces deux
saintes maisons d'une ruine certaine et imminente, et les rétablit.
Quelque mal et sans mesure que je fusse avec le duc de Noailles, je ne
crus pas devoir oublier qu'il était le premier auteur de cette
excellente oeuvre, et la part qu'il prenait en l'abbaye de Septfonts.
Toutes les fois donc que je recevais un payement de Law, je tirais le
duc de Noailles à part au premier conseil de régence. Je lui disais ce
que je venais de recevoir, et le partage que j'en venais de faire. Il me
remerciait, me faisait des révérences, et je ne lui parlais ni ne le
saluais jusqu'au prochain payement. Ces colloques, quoique courts et
rares, devinrent la surprise des spectateurs et la matière des
spéculations. À la première fois on nous crut raccommodés. Dans la
suite, on ne sut plus que penser. J'en riais et laissais raisonner.
L'abbé de Septfonts se trouvait à Paris\,: c'était à lui que j'envoyais
sa part. Il ne s'était pas douté du supplément de la Trappe. Il l'apprit
par ce que je lui en envoyai\,: à quoi il ne s'attendait pas, et dont il
fut fort touché. Ce commerce nous fit faire connaissance ensemble, qui
bientôt devint une tendre et réciproque amitié. C'était un saint bien
aimable. J'aurais trop de choses à en dire ici\,; elles se trouveront
dans les Pièces à la suite de ce qui regarde M. de La Trappe.

Le chancelier de Pontchartrain fit le mariage de Maurepas, son
petit-fils, avec la fille de La Vrillière, chez qui il logeait, et y
apprenait son métier de secrétaire d'État. Il a bien dépassé son maître
et bien profité des leçons de son grand-père, duquel il tient beaucoup.
Il exerce encore aujourd'hui cette charge avec tout l'esprit l'agrément
et, la capacité possible\footnote{Jean-Frédéric Phélypeaux, comte de
  Maurepas, devint ministre secrétaire d'État de la marine à
  vingt-quatre ans, en 1725. Il fut disgracié et exilé en 1749. Ce
  passage des Mémoires de Saint-Simon prouve que la rédaction de cette
  partie des Mémoires est antérieure à l'année 1749, puisqu'il parle de
  Maurepas comme ministre dans tout l'éclat de sa puissance. Maurepas
  fut rappelé à la mort de Louis XV (1774) et nommé premier ministre. Il
  ne montra pas dans cette haute position les talents qu'on lui avait
  prêtés et dont parle Saint-Simon. Marmontel a caractérisé dans ses
  \emph{Mémoires} cette seconde administration de Maurepas\,: «\,Une
  attention vigilante à conserver son ascendant sur l'esprit du roi, et
  sa prédominance dans les conseils le rendaient jaloux des choix mêmes
  qu'il avait faits\,; et cette inquiétude était la seule passion qui
  dans son âme eût de l'activité. Du reste, aucun ressort, aucune
  vigueur de courage ni pour le bien ni pour le mal\,; de la faiblesse
  sans bonté, de la malice sans noirceur, des ressentiments sans
  colère\,: l'insouciance d'un avenir qui ne devait pas être le sien,
  peut-être assez sincèrement la volonté du bien public, lorsqu'il le
  pouvait procurer sans risque pour lui-même\,; mais cette volonté
  aussitôt refroidie dès qu'il y voyait compromis son crédit ou son
  repos\,; tel fut jusqu'à la fin le vieillard qu'on avait donné pour
  guide ou pour conseil au jeune roi.\,»}. Il est de plus ministre
d'État. La louange pour lui serait bien médiocre, si je disais, qu'il
est de bien loin le meilleur que le roi ait eu dans son conseil depuis
la mort de M. le duc d'Orléans. Il a eu le bonheur de trouver une femme
à souhait pour l'esprit, la conduite et l'union, et d'en faire le leur
l'un et l'autre. Je ne puis plus trouver que ce leur soit un malheur de
n'avoir point d'enfants.

Fagon, perdant sa charge de premier médecin, l'unique qui se perde à la
mort du roi, s'était retiré au faubourg Saint-Victor, à Paris, dans un
bel appartement au Jardin du Roi ou des simples et des plantes rares et
médicinales, dont l'administration lui fut laissée. Il y vécut toujours
très solitaire dans l'amusement continuel des sciences et des
belles-lettres, et des choses de son métier, qu'il avait toujours
beaucoup aimées. Il a été ici parlé de lui si souvent, qu'il n'y a rien
à y ajouter, sinon qu'il mourut dans une grande piété et dans un grand
âge pour une machine aussi contrefaite et aussi cacochyme qu'était la
sienne, que son savoir et son incroyable sobriété avaient su conduire si
loin, toujours dans le travail et dans l'étude. Il fut surprenant qu'à
la liaison intime et l'entière confiance qui avait toujours été entre
M\textsuperscript{me} de Maintenon et lui, qui l'avait fait premier
médecin, et toujours soutenu sa faveur, ils ne se soient jamais vus
depuis la mort du roi.

On a vu, t. IV, p.~209, le caractère de l'abbé d'Estrées, et il a été
parlé de lui et de ses emplois en plusieurs autres endroits. Il
jouissait d'une belle santé dans un âge à profiter longtemps de sa
fortune et de l'archevêché de Cambrai, dont il attendait les bulles,
lorsqu'il fut surpris d'une inflammation d'entrailles pour s'être
opiniâtré à prendre, sans aucun besoin, des remèdes d'un empirique, par
précaution, duquel il s'était entêté. Un mieux marqué le persuada si
bien que son mal n'était rien, qu'il nous donna à plusieurs un grand et
bon dîner\,; mais sur le point de se mettre à table avec nous, les
douleurs le reprirent. Néanmoins il voulut nous voir dîner. Peu de
moments après que le fruit fut servi, l'extrême changement de son visage
nous pressa de le laisser en liberté de penser sérieusement à lui. Une
heure après, le cardinal de Noailles, qui en fut averti, vint l'y
disposer. Il eut peu de temps à se reconnaître, mais il en profita bien.
Il fit son testament de ce dont il n'avait pas encore disposé, reçut ses
sacrements le lendemain, et mourut la nuit suivante. Cette mort
découvrit des dispositions secrètes, qui n'étaient pas nouvelles, dont
son ambition et l'avidité des Noailles furent accusées. Le maréchal
d'Estrées et ses soeurs furent très scandalisés de ces dispositions de
leur frère à leur insu et à leur préjudice. Leur vanité aussi n'en fut
pas moins offensée de sentir qu'il eût cru devoir acheter une
protection, dont leur nom et leur considération ne devait pas avoir
besoin, et dont l'alliance des Noailles, dont le maréchal d'Estrées
avait épousé une, pouvait du moins exclure le payement. Le monde rit un
peu de ce petit démêlé domestique, et les Noailles, qui empochèrent
gros, en rirent encore plus\,; mais, en conservant leur proie, ils
n'oublièrent rien pour apaiser ce bruit, et en assez peu de temps ils y
parvinrent. Outre cent mille écus, dont les Noailles profitèrent, l'abbé
d'Estrées donna quarante-cinq mille écus aux pauvres de ses abbayes,
récompensa très bien ses domestiques, et fit présent de sa belle
bibliothèque aux religieux de l'abbaye de Saint-Germain des Prés, où il
avait logé longtemps avec son oncle, le cardinal d'Estrées, qui en était
abbé.

Cette mort opéra subitement une conversion éclatante, durable, et dont
les bonnes oeuvres et la pénitence augmentèrent toujours avec une
simplicité, une humilité, une aisance dans le peu de commerce qui fut
conservé, une paix et une joie singulière parmi les plus grandes et les
plus répugnantes austérités\,: ce fut {[}celle{]} de la marquise de
Créqui, veuve sans enfants, fille du feu duc d'Aumont et de la sueur de
M. de Louvois et du feu archevêque de Reims, qui l'avait enrichie et
qu'on avait soupçonné de l'avoir aimée autrement qu'en oncle, auquel
l'abbé d'Estrées avait parfaitement succédé. De la plus mondaine de
toutes les femmes, la plus occupée de sa personne, de la parure, de
toute espèce de commodités et de magnificence et passionnée du plus gros
jeu, elle devint la plus retirée, la plus modeste, la plus prodigue aux
pauvres et la plus avare pour elle-même\,; sans cesse en prières chez
elle ou à l'église\,; assidue aux prisons, aux cachots, aux hôpitaux,
dans les plus horribles fonctions à la nature, et y a heureusement
persévéré jusqu'à sa mort, qui lui a laissé bien des années de
pénitence.

Je fus fâché de la mort de l'abbé d'Estrées qui était de mes amis et
qui, avec quelque ridicule et un peu de fatuité, avait de bonnes choses,
de l'honneur, de la sûreté, de la droiture. M. le duc d'Orléans y perdit
un vrai serviteur et me témoigna d'abord son embarras sur Cambrai. Je
lui conseillai de trancher court pour se délivrer des demandeurs d'une
si belle place, qui par sa situation ne se devait donner qu'avec
beaucoup de choix. Je lui proposai tout de suite le cardinal de La
Trémoille, sans que j'eusse la moindre connaissance avec lui. Je dis au
régent qu'étant chargé des affaires du roi à Rome, sans biens par
lui-même et panier percé de plus, il avait besoin de beaucoup de secours
en pensions ou en bénéfices\,; que la richesse de celui-là suppléerait
aux grâces qui coûteraient au roi\,; que son personnel était sans
crainte et sans soupçon quand il résiderait à Cambrai, où il était
apparent qu'il n'irait jamais, ainsi qu'il est arrivé. Le régent m'en
crut et sur-le-champ le lui donna. Ce présent fit vaquer Bayeux qu'il
avait. L'abbé de Lorraine avait depuis longtemps fort changé de vie. Il
s'était fort attaché au cardinal de Noailles que M. le Grand aimait et
respectait fort sans s'en être jamais contraint dans les derniers temps
du feu roi. Le cardinal de Noailles désira qu'il eût Bayeux. M. et Aime
de Lorraine en pressèrent M. le duc d'Orléans. Il le lui donna.

Le régent, qui faisait litière de ce qui ne lui coûtait rien et trop
souvent encore de ce qui coûtait beaucoup, fit, en ce temps de paix, et
au commencement de mars, une promotion de vingt-six lieutenants généraux
et de trente-six maréchaux de camp. La confusion était déjà montée à tel
point qu'il y eut quatre-vingts personnes qui se crurent à portée de
demander l'agrément des régiments que la promotion des maréchaux de camp
fit vaquer. J'eus celui de Sourches pour le marquis de Saint-Simon, que
je tirai des gardes françaises, qui était déjà attaqué de la poitrine et
qui mourut trois mois après, dont ce fut grand dommage, car il était
plein d'honneur, de valeur, de volonté et d'application, avec une figure
fort agréable, et il promettait beaucoup. J'eus à toute peine le
régiment pour son frère, parce que c'était un enfant encore sous le
fouet au collège.

M. le duc d'Orléans se laissa aller en même temps à deux projets pour
les troupes dont il eut tout lieu de se repentir. L'aîné Broglio, gendre
du feu chancelier Voysin, était un homme déshonoré sur la valeur,
quoique devenu lieutenant-général et directeur d'infanterie par son
beau-père, et déshonoré encore sur toutes sortes de chapitres. Méchant,
impudent, parlant mal de tout le monde, quoique souvent cruellement
corrigé, fort menteur, audacieux à merveilles, sans que les affronts
qu'il avait essuyés eussent pu abaisser son air et son ton avantageux\,;
avec cela beaucoup d'esprit et orné, grande opinion de soi et mépris des
autres, avare au dernier excès, horriblement débauché et impie\,; se
piquait de n'avoir point de religion\,; en faisait des leçons. Il
parlait bien et le langage qu'il voulait tenir suivant ceux à qui il
parlait et quand il lui plaisait\,; ne manquait pas d'agrément dans la
conversation et de politesse. Son intrigue et ses moeurs
l'introduisirent parmi les roués, où il s'insinua si bien par la
hardiesse de ses discours qu'il devint bientôt de tous les soupers et
des plus familiers. On a vu que ce nom était celui que M. le duc
d'Orléans donnait aux débauchés de ses soirées. Il prit si bien dans le
monde que personne ne les nommait plus autrement. Quand celui-ci se
trouva assez bien ancré auprès du régent et de M\textsuperscript{me} la
duchesse de Berry, qui soupait très souvent avec eux, pour oser aspirer
plus haut, il imagina de se tourner vers l'importance et de s'ouvrir un
chemin dans le cabinet du régent et dans les affaires.

Il conçut pour cela un dessein de remédier aux friponneries des routes,
des étapes et des magasins des troupes, par un projet qui ressemblait
tout à fait à celui de la comédie des \emph{Fâcheux} de Molière et à
l'avis qu'un de ces fâcheux y donne de mettre toutes les côtes en ports
de mer. Broglio proposa par un mémoire d'obliger toutes les villes et
autres communautés qui sont sur les passages ordinaires des troupes, de
construire à leurs dépens des casernes pour les loger et des magasins
fournis pour leur usage, moyennant quoi plus de routes, d'étapiers ni de
magasiniers, et leurs friponneries, insignes en effet, coupées par la
racine, ce qui donnerait, disait-il, un soulagement infini aux peuples,
aux finances, aux troupes. Il sentit bien qu'il avait besoin de
quelqu'un de poids pour faire passer un projet si absurde. La merveille
fut qu'il sut si accortement courtiser et arraisonner Puységur qu'il
l'infatua de son projet.

Puységur, pétri d'honneur, abhorrait toutes ces friponneries, qu'il
avait vues sans cesse de ses yeux. Il a été parlé souvent de lui dans
ces Mémoires. Il était extrêmement estimé pour sa vertu, sa valeur, sa
capacité\,; très considéré de M. le duc d'Orléans qui, comme on l'a vu,
l'avait mis comme un homme principal dans le conseil de guerre, et il
est enfin, longtemps après, devenu maréchal de France avec l'acclamation
publique.

Broglio, assuré d'un tel appui, proposa au régent son projet avec
confiance et travailla plusieurs fois seul avec lui, et après avec
Puységur en tiers. Il eut encore l'adresse de profiter de la défiance
naturelle du régent, pour le détourner d'en parler au conseil de guerre,
pour faire précipiter les ordres aux intendants des provinces pour une
prompte exécution, et pour l'armer contre les représentations qu'il
s'attendait bien qui lui viendraient de toutes parts, dès que ce projet
serait connu. Il en coûta beaucoup en bâtiments aux villes et aux
communautés, avant que les personnes employées dans les finances et dans
le conseil de guerre, les plus accrédités intendants et beaucoup
d'autres gens eussent pu dessiller les yeux au régent et fait abandonner
une folie si ruineuse, qui tomba enfin après avoir bien fait du mal.

L'autre projet, pour lequel Broglio crut n'avoir pas besoin de second,
ce fut l'augmentation de la paye des troupes telle qu'elle est
aujourd'hui. Il en persuada la nécessité au régent par la grande
augmentation du prix des choses les plus communes et les plus
indispensables à leur subsistance, et qu'il s'en ferait adorer par une
grâce si touchante, dont le bien-être le rendrait maître des coeurs de
tous les soldats. Il se gardait bien de lui dire qu'on n'avait cessé de
les maltraiter et de rogner sur elles depuis la mort du roi, comme sur
la partie faible et indéfendue, quoique la force et la ressource de
l'État, et qui était la source de l'autorité du roi et de la sûreté de
toutes les autres parties de l'État. Il se garda bien aussi de
représenter la sagesse de la manutention de Louvois, transmise par son
exemple à ses successeurs jusqu'à Voysin exclusivement, qui avait fait
sa cour et sa bourse d'une conduite qui avait été suivie depuis, et même
de plus en plus appesantie.

Louvois dès lors sentait l'exiguïté de la paye des troupes et de celle
des officiers. Il comprenait en même temps de quelles sommes la plus
légère augmentation chargerait les finances. Pour éviter un si pesant
inconvénient, et subvenir néanmoins raisonnablement à la nécessité des
troupes, il les distribuait avec grande connaissance, suivant leurs
besoins, en des lieux où le soldat gagnait sa vie et le cavalier se
raccommodait, et, comme il en avait le dessein, il fermait les yeux à
tout ce qui n'allait ni à pillage, désordre, ou manque de discipline, et
les remettait ainsi pour du temps, de laisser à d'autres ces mêmes
secours très effectifs quoique peu perceptibles. Il avait la même
attention et les mêmes ménagements pour les officiers, qu'il
rétablissait de même par les avantages des postes ou des quartiers
d'hiver. C'est ce qu'il réglait lui-même et sans y paraître le moins du
monde que par des ordres secrets aux intendants, etc. Il avait l'oeil
attentif à une exécution précise\,: c'est à quoi ses bureaux dressés par
lui-même suppléèrent après lui sous son fils et sous Chamillart ensuite,
quoique peut-être avec moins d'équité et de désintéressement. C'est ce
qui prit fin par l'ignorance, la rudesse, la dureté, l'avarice de
Voysin, et la parade qu'il fit au feu roi, dans de si malheureux temps,
de retrancher ce qu'il traita d'abus au profit de ses finances. C'était
donc à cette sage et savante pratique de Louvois qu'il fallait revenir,
au lieu de tirer et de grappiller incessamment sur les troupes dans le
faux objet de soulager les finances à leurs dépens.

Personne n'eut loisir d'aviser le régent\,; il s'enivra du projet de
Broglio, il n'en voulut partager l'honneur avec personne. La déclaration
en parut subitement\,; elle surprit tout le monde. Les plaintes des non
consultés du conseil de guerre et de ceux des finances, du terrible
poids ordinaire dont cette augmentation les surchargeait, ne purent se
faire entendre qu'après le coup porté de manière à ne pouvoir s'en
dédire. Le régent alors sentit toute sa faute, et n'en recueillit pas la
plus légère reconnaissance des troupes, qui regardèrent ce bienfait
comme dû et de nécessité.

Quand il y aurait eu de bonnes raisons pour cette pesante augmentation
de dépense, si M. le duc d'Orléans m'en avait parlé, comme il ne fit
point, auparavant ni après, je crois par embarras, ni moi à lui, je lui
aurais représenté que ce n'était pas à un régent à charger ainsi les
finances fortement et pour toujours, mais à en représenter les raisons
au roi, devenu non seulement majeur, mais en âge d'entendre et de se
résoudre plus que ne le comporte l'âge précis de la majorité des rois,
qui est encore assez longtemps mineure. Il sentit si bien l'inconvénient
où il s'était laissé entraîner, que Broglio retomba tout à coup dans le
néant dont il avait voulu s'élancer, et fut trop heureux de trouver, par
la table et l'effronterie, à se raccrocher à l'état des roués qu'il
avait voulu tâcher de laisser loin derrière lui, sans toutefois l'avoir
quitté, et n'approcha plus du cabinet de M. le duc d'Orléans ni d'aucun
particulier avec lui.

Ce prince mit incontinent après le maréchal de Villars dans le conseil
de régence, sans quitter celui de guerre, pour le faire taire. Il était
de mauvaise humeur de l'affaire de la liasse dont il a été parlé plus
haut, et de quelques autres tracasseries qu'il avait essuyées dans le
conseil de guerre. Il était piqué des deux résolutions prises sur les
troupes, suggérées par Broglio, sans en avoir ouï parler. Il était
secrètement d'avec ceux qui voulaient attaquer le régent d'une manière
solide. Il ne contraignit donc pas ses propos sur la folie du projet des
casernes et dés magasins, et sur le poids accablant pour les finances de
l'augmentation de la paye. Tout en craignant de déplaire et n'osant
résister à rien, la gourmette se lâchait aussi, et il parlait avec
éloquence, force et une sorte d'autorité qui imposait au gros, et que le
régent craignait. À peu de jours de là cet exemple obtint la même grâce,
successivement, d'exemple en exemple, aux maréchaux d'Huxelles, puis
d'Estrées, enfin à d'Antin aussi, sans perdre leurs places dans leurs
conseils. Il ne put refuser à M\textsuperscript{me} la duchesse de Berry
de payer à Rion le régiment de Berry-cavalerie, puis de le lui changer
pour les dragons Dauphin. Il donna dix mille livres de pension à
Maupertuis, qui avait été capitaine des mousquetaires gris, quoiqu'il
eût le gouvernement de Saint-Quentin et la grand'croix de Saint-Louis.
Il permit à Heudicourt de céder, par un très vilain marché, sa charge de
grand louvetier à son fils. Il accorda à La Chaise la survivance de sa
charge de capitaine de la porte pour son fils, qui ne vécut pas, dont le
P. de La Chaise lui avait procuré trois cent mille livres de brevet de
retenue, et quelques jours après au duc de Guiche les survivances pour
son fils aîné du régiment des gardes et de ses gouvernements, au grand
déplaisir de la duchesse de Guiche, qui n'en sut rien qu'après, et qui
désirait la charge pour son second fils, qui était sa prédilection.

Ce fut ici le temps de l'arrivée de Londres à Paris de Chavigny, envoyé
par l'abbé Dubois\,; du départ de Nancré pour Madrid\,; de la naissance,
le derniers mars, à Madrid de l'infante M. A. Victoire, qui vint depuis
à Paris comme future épouse du roi, qui fut le sujet de mon ambassade
extraordinaire en Espagne, et qui a depuis épousé le prince du Brésil,
avec qui elle vit aujourd'hui à Lisbonne, avec postérité, attendant la
couronne de Portugal. C'est aussi le temps où arriva l'horrible
catastrophe du czarowitz, si connue de tout le monde, toutes choses qui
trouveront mieux qu'ici leur place parmi les affaires étrangères.

Le cardinal de Polignac, qui avait autrefois recommencé jusqu'à trois
licences, sans en avoir pu achever aucune, et si ce n'était pas manque
de science ni d'esprit, résolut enfin de passer de l'ordre de
sous-diacre, où il était demeuré jusqu'alors, dans celui de prêtrise. Je
ne sais s'il imagina que cette résolution, qu'il ne tint pas secrète,
donnerait du poids à ses protestations, mais il demanda en même temps
une audience au régent pour se justifier de beaucoup de choses dont il
était plus que soupçonné et, dont à force d'esprit et de grâces, il
espéra se bien tirer avec un prince aussi facile que l'était M. le duc
d'Orléans. Ce cardinal était depuis longues années dans la plus étroite
confiance de M\textsuperscript{me} la duchesse du Maine, et de M. du
Maine par conséquent. Leurs cabinets lui étaient de tout ce temps-là
ouverts à toute heure\,: il était sur le pied avec eux qu'ils ne
faisaient rien sans son conseil. Son frère, qui était un imbécile, qu'il
gouvernait, venait de sortir de prison pour cette requête en faveur des
bâtards, que lui sixième avait présentée au parlement, et qui n'avait
pas été faite sans M. et M\textsuperscript{me} du Maine et sans le
cardinal. On peut juger quelle put être sa justification à tout ce qui
se brassait, et qu'on n'apercevait pourtant que fort imparfaitement
encore, mais assez pour qu'avec le passé le régent sût à quoi s'en tenir
avec M. et M\textsuperscript{me} du Maine, et par conséquent avec lui,
qui, depuis, ne cessa de s'enfoncer de plus en plus en leurs criminelles
et pernicieuses menées.

Argenson, avec les finances et les sceaux, ne se contraignit point sur
ses heures. La place de la police, devenue entre ses mains une véritable
inquisition universelle, l'avait accoutumé à travailler sans règle à
toutes sortes d'heures du jour et de la nuit, où il était fort souvent
réveillé\,; il ne tint point de table ni d'audiences, ce qui embarrassa
fort tout ce qui eut affaire à lui. Les magistrats des finances, les
financiers et ses commis ne le furent pas moins. Il leur donnait le plus
souvent les heures de la nuit une, deux, trois heures du matin étaient
celles qu'il leur donnait le plus souvent\,; j'en ai vu Fagon désolé
bien des fois. M. de La Rochefoucauld, qu'il se piquait de considérer
par l'ancien respect de la province, il lui donna une audience à deux
heures après minuit. Il prit la coutume, qu'il garda toujours, de dîner
dans son carrosse, allant de chez lui, près les Grands-Jésuites, au
conseil aux Tuileries, ou travailler l'après-midi au Palais-Royal. Il
était depuis longtemps ami intime de M\textsuperscript{me} de Veni,
prieure perpétuelle de la Madeleine de Traisnel, au faubourg
Saint-Antoine. Il y avait un appartement au dehors\,; il avait valu
beaucoup à cette maison. Il y couchait souvent étant lieutenant de
police. En changeant de place, il ne changea point de coutume à cet
égard\,; dès qu'il avait quelques moments, il y courait, il y couchait
tant qu'il pouvait\,: il lui est arrivé plus d'une fois d'y oublier les
sceaux, et d'être obligé de les y aller chercher. Cela lui faisait
perdre beaucoup de temps\,; ce qui, joint à la difficulté de le voir et
de lui parler, causa de grands murmures. Si j'avais pu deviner cette
conduite avant qu'il eût changé de place, je lui en aurais bien dit mon
avis d'avance\,; mais devenu ce qu'il était, il n'était plus temps. Lui
et Law faisaient seuls les finances.

Ils travaillaient souvent avec le régent, presque jamais tous deux
ensemble avec lui et d'ordinaire tête à tête, d'où les résolutions et
les expéditions suivaient sans autre forme ni consultation. Le duc de La
Force, à qui le vain nom de président du conseil des finances et de
celui du commerce avait été donné lorsque le duc de Noailles le quitta,
n'eut plus de département. Le conseil des finances n'avait plus guère
d'occupation, et le conseil de régence du samedi après dîner, l'un des
deux qui étaient destinés aux affaires de finances, cessa de
s'assembler, faute de matières.

Dans cette première nouveauté de faveur, Argenson en voulut profiter
pour obtenir pour sa femme, soeur de Caumartin, le tabouret, à l'instar
de la chancelière. On a vu comment M\textsuperscript{me} Séguier
l'obtint, à quelles conditions et qu'elles sont toujours les mêmes.
Depuis cet événement il n'y avait eu qu'un seul garde des sceaux marié.

C'était le second chancelier Aligre, qui les eut deux ans, à la mort du
chancelier Séguier, pendant lesquels il n'y eut point de chancelier, et
au bout desquels il le devint lui-même\footnote{Voy. dans les notes à la
  fin du t. X, p. 447, la liste des chanceliers et gardes des sceaux et
  un extrait du \emph{Journal d'Olivier d'Ormesson} pour la tenue du
  sceau après la mort du chancelier Séguier (1672). --- Voy. aussi les
  notes à la fin du présent volume.}. Dans cet intervalle ni trace ni
vestige quelconque que sa femme ait eu le tabouret, dont les preuves ne
manqueraient pas dans la mémoire de main en main ni par écrit sur les
registres, si elle l'avait eu. Aligre apparemment n'osa tenter une
extension si nouvelle. Il songeait fort à être chancelier. Il avait le
pied à l'étrier pour l'être. Il aima mieux apparemment attendre qu'il le
frit que de s'exposer à un refus de prétention nouvelle, ou même de
mettre un nuage à ses vues si apparentes et si prochaines, par un
empressement mal à propos pour ce que l'office de chancelier ferait de
soi-même.

Argenson, qui se voyait sur la tête un chancelier bien qu'exilé, plus
jeune que lui de beaucoup, n'avait pas la même espérance, et n'eut pas
aussi le ménagement d'Aligre. Il voulut profiter de la facilité du
régent et de son agréable et importante situation auprès de lui, dans
une primeur encore toute radieuse. Il lui représenta l'entière
similitude extérieure du chancelier et du garde des sceaux\,; qu'il
suivait de là qu'elle devait être pareille entre leurs femmes, et obtint
ainsi le tabouret pour sa femme, qui en prit deux jours après possession
aux mêmes conditions que la chancelière.

C'est le premier exemple de cette nouveauté, qui a servi de règle pour
donner de même le tabouret longtemps depuis à la femme du garde des
sceaux Chauvelin, qui en a joui, même en présence de la chancelière,
depuis que d'Aguesseau fut rappelé la seconde fois de Fresne, et qu'il
fit les fonctions de chancelier en même temps que Chauvelin faisait
celles de garde des sceaux. Armenonville, qui les eut après Argenson et
avant Chauvelin, était déjà veuf, et ils furent rendus au chancelier
d'Aguesseau, à la chute de Chauvelin.

Maupeou, je le remarque parce qu'il est longtemps depuis devenu premier
président, fut président à mortier à la place de Menars, frère de
M\textsuperscript{me} Colbert, qui avait fait sa fortune, mort en ce
temps-ci en ce beau lieu de Menars-sur-Loire, près de Blois. C'était une
très belle figure d'homme, et un fort bon homme aussi, peu capable, mais
plein d'honneur, de probité, d'équité et modeste, prodige dans un
président à mortier. Le cardinal de Rohan acheta sa précieuse
bibliothèque, qui était celle du célèbre M. de Thou, qui fut pour tous
les deux un meuble de fort grande montre, mais de très peu d'usage.

Les enregistrements faits par la grand'chambre seule du rétablissement
des quatre sous pour livre et du traité de Lorraine, causèrent une
grande rumeur dans les enquêtes et requêtes, qui prétendent être
appelées aux enregistrements et qui s'en prirent avec chaleur au premier
président. Ces chambres arrêtèrent entre elles que tous les conseillers
des enquêtes et requêtes s'abstiendraient d'aller chez lui sans des cas
indispensables qui n'arrivent presque jamais. Elles s'assemblèrent
plusieurs fois entre elles, et elles entrèrent en la grand'chambre où le
président Lamoignon se trouva présider, firent leurs protestations, et
les laissèrent par écrit sur le bureau du greffier, à qui il fut défendu
après de les mettre dans les registres, tant il est commode d'être juge
et partie. Après bien du vacarme domestique, des souplesses du premier
président et divers manèges, de plus vastes vues imposèrent à la fin la
suspension ordinaire de cette querelle qui se renouvelle assez souvent.

La grand'chambre les laisse crier à moins que quelque intérêt plus
grand, comme il arriva alors, ne l'oblige à les ménager. La
grand'chambre a des prétentions, les autres chambres s'en offensent et
ne prétendent pas être moins que la grand'chambre, parties intégrantes
du parlement, sans l'avis desquelles rien ne doit être censé enregistré
par leur commune compagnie à toutes qui est le parlement. La
grand'chambre répond que c'est à elle qu'il appartient de les faire,
puisque c'est chez elle qu'ils se font. Celles-ci répliquent que le
local ne donne à la grand'chambre aucun droit privatif aux autres
chambres, puisque l'adresse de tout ce {[}qui{]} s'envoie pour être
enregistré est faite à tout le parlement\,; qu'elles sont du corps du
parlement tout comme en est la grand'chambre, laquelle n'a sur les
autres chambres que la primauté de rang\,; enfin que, lorsque le roi y
va seoir, elles y sont toujours mandées. Le point est que la cour, qui
est plus aisément maîtresse d'un petit nombre que d'un grand, et des
têtes mûres et expérimentées de la grand'chambre que de la jeunesse et
de la foule des autres sept chambres, favorise toujours à cet égard la
prétention de la grand'chambre, et que le premier président, qui connaît
mieux la grand'chambre, où il préside, que les autres sept chambres où
il ne va jamais, et où il ne peut rien, tandis que c'est à lui à
distribuer les procès aux conseillers de la grand'chambre, dont quantité
sont avides du sac, il les manie plus aisément que tout le parlement
assemblé, et par cette raison favorise pour soi-même cette même
prétention de la grand'chambre contre les sept autres chambres. C'est ce
qui a toujours fini cette dispute à l'avantage de la grand'chambre
toutes les fois qu'elle s'est élevée, ce qui prouve continûment que ce
n'est pas le tout d'avoir raison pour gagner son procès.

Une autre querelle domestique leur fait encore bien du mal, sans que
l'orgueil d'aucun des prétendants en ait rien voulu rabattre, quoique
chacun en sente l'extrême inconvénient, et que tous de bonne foi en
gémissent. Lorsque la ruse ou le hasard fait que tous les présidents à
mortier sont absents ou se retirent, c'est sans difficulté au doyen du
parlement, ou, s'il n'y est pas, au plus ancien conseiller de la
grand'chambre à présider, mais de sa place sans en changer\,; mais,
lorsque ce cas arrive, lorsque toutes les chambres se trouvent
assemblées, triple prétention, triple querelle. Le plus ancien des
présidents des enquêtes veut présider. Le premier des présidents de la
première chambre des enquêtes le lui dispute comme droit de charge et
non d'âge ni d'ancienneté, et le doyen du parlement, ou, s'il n'y est
pas, le plus ancien des conseillers de la grand'chambre présents,
prétend les exclure l'un et l'autre, fondé sur ce que les présidents des
chambres des enquêtes et requêtes ne sont que conseillers comme eux,
quoiqu'ils aient, mais en cette qualité de conseillers, une commission
pour présider en telle ou telle chambre des enquêtes ou des requêtes, ce
qui ne change pas même à leur propre égard leur état inhérent, réel,
fondamental et personnel de conseillers, beaucoup moins à l'égard des
conseillers de la grand'chambre, où lorsque les chambres sont
assemblées, ces présidents des enquêtes et requêtes ne les précèdent
pas, et ne sont admis avec leurs chambres qu'en qualité de conseillers,
d'où il résulte qu'ils ne peuvent jamais présider au préjudice d'aucun
des conseillers de la grand'chambre.

Ce sont ces querelles domestiques qui ont toujours affaibli le parlement
contre la cour\,; par exemples fréquents, cette dernière {[}en a
profité{]}. Toutes les fois qu'on n'a pu empêcher le parlement de
s'assembler sur des affaires où la cour voulait s'intéresser en faveur
de matières de Rome, de jésuites, de choses ayant trait à la
constitution, et que les présidents à mortier voyaient qu'ils n'en
seraient pas les maîtres, ils sortaient tous en même temps, ou pas un ne
venait à l'assemblée des chambres. Ils livraient ainsi la séance à la
division et à la querelle pour la présidence, et la forçaient à se lever
et s'en aller sans rien faire faute de présidence, que pas un des
prétendants n'a jamais voulu céder.

Les maréchaux de France qui, par leur âge et leur union, s'étaient
jusqu'à ce temps-ci assez bien soutenus, sentirent à leur tour
l'humiliation du désordre dans lequel le régent se persuadait trouver sa
sûreté et sa grandeur. Les maréchaux de France qui n'étaient pas ducs
s'étaient doucement unis avec ce qui avait usurpé le nom collectif de la
noblesse\,; celle-ci pour protection et pour se parer du contraste,
ceux-là pour tâcher d'en profiter. Mais cette noblesse, devenue fière de
son ralliement et de la faiblesse que le régent lui avait montrée, ne
tarda pas à faire sentir aux maréchaux ses amis qu'elle ne voulait rien
au-dessus d'elle, tant qu'elle pourrait rapprocher le niveau. Le marquis
de Beaufremont se chargea de le leur apprendre. Avec de l'esprit et de
la valeur et un des premiers noms de Bourgogne, il serait difficile
d'être plus hardi, plus entreprenant, plus hasardeux, plus audacieux,
plus fou, qu'il l'a été toute sa vie.

Le maréchal de Villars, comme chef du conseil de guerre, écrivait aux
colonels la plupart des lettres que sous le feu roi le secrétaire d'État
de la guerre avait accoutumé de leur écrire, et on a vu (t. XII, p.~401)
sur quel énorme pied Louvois avait su mettre à son avantage l'inégalité
extrême du style qui a duré sans exception autant que la vie du feu roi.
Personne jusqu'à ce temps-ci ne s'était avisé de se plaindre des lettres
du maréchal de Villars. Cette noblesse se mit tout à coup à s'en
offenser, et Beaufremont, qui se trouva en avoir reçu une, lui fit une
réponse si étrange qu'il en fut mis à la Bastille. Il y coucha à peine
deux ou trois nuits, et en sortit se moquant de plus belle des maréchaux
de France qui étaient assemblés en ce moment sur cette affaire et ne
savaient pas un mot de sa sortie. Ils demandèrent au moins que
Beaufremont fît des excuses au maréchal de Villars de la réponse qu'il
lui avait faite, sans rien pouvoir tirer du régent. Cette poursuite dura
huit jours. Je ne sais sur quel demi-mot qu'il articula mal, je crois,
pour se moquer d'eux, ils se persuadèrent que Beaufremont recevrait
l'ordre qu'ils demandaient, tellement que le maréchal de Villars, prêt à
partir pour Villars, l'attendit chez lui, à Paris, toute la journée, et
y coucha, ayant dû s'en aller dès le matin, sans qu'il entendît parler
de Beaufremont, qui courait les lieux publics, disant qu'il n'avait nul
ordre, et se répandant sans mesure en dérisions. Les maréchaux de France
demeurèrent étrangement déconcertés, au point qu'ils n'osèrent plus se
plaindre ni rien dire, tandis que Beaufremont les accablait de brocards.
Outre la maxime favorite du régent \emph{divide et regna}, et de tout
révolter les uns contre les autres, je crus toujours qu'il y avait du
personnel de Villars, et du peu de mesure de ses propos sur les casernes
et l'augmentation de la paye.

Quand le régent se fut bien diverti six bonnes semaines de ce scandale
public, il fit trouver Beaufremont au Palais-Royal un matin que le
maréchal de Villars y travaillait avec lui, le fit entrer, et sans autre
façon dit au maréchal que M. de Beaufremont n'avait jamais prétendu lui
manquer, qu'il en était caution pour lui, et qu'il fallait oublier de
part et d'autre toutes ces petites tracasseries, et tout de suite
renvoya Beaufremont, qui sortit riant comme un fou, sans que le maréchal
ni lui eussent proféré une seule parole. On peut juger du dépit du
maréchal et de MM. ses confrères. Je crois pourtant que Beaufremont eut
ordre de se taire et de ne pas pousser les choses plus loin, car il ne
parla plus. Il pouvait être content de tout ce qu'il avait débité, et
d'en sortir de cette étrange façon.

Les ducs ne prirent aucune part en cette querelle. Quelques-uns en
rirent. Il était raisonnable aussi que les maréchaux de France eussent
aussi leur tour.

Ce n'est pas à moi à paraphraser cette conduite de M. le duc d'Orléans à
l'égard d'un office de la couronne, dont le caractère distinctif est de
juger l'honneur de la noblesse, et d'officiers qui ne le peuvent devenir
que par leur sang, leurs services et leur mérite, et qui ne peuvent être
que des personnages dans l'État. Comme il était grand maître en
\emph{mezzo-termine}, et qu'il voulut toujours favoriser des gens sans
mesure, dont le rameutement ne tendait qu'à le culbuter, comme il y
parut bientôt, il régla que toutes les lettres désormais seraient en
style de mémoire, contenant les ordres à donner, les réponses et les
choses à faire, qui seraient signées Villars, et avec lui Biron pour
l'infanterie, Lévi pour la cavalerie, et Coigny pour les dragons.

Beaufremont, victorieux des maréchaux de France, le voulut être bientôt
après des princes du sang. On vit, moins de deux mois après, les preuves
de ses menées en Bourgogne contre le service du roi, et le rang, le
crédit et l'autorité de M. le Duc, gouverneur de cette province, qui en
était allé tenir les états. Il en rapporta quantité de lettres que
Beaufremont y avait écrites dans cet esprit, sans aucun détour, partie
surprises, partie livrées par ceux qui les avaient reçues. M. le Duc ne
les cacha pas à son retour, ni les plaintes qu'il en porta à M. le duc
d'Orléans, mais dont il ne fut autre chose. Les maréchaux de France
rirent tout bas à leur tour de se trouver en, si bonne compagnie.

Il a été parlé ici plus d'une fois de Monasterol, envoyé de l'électeur
de Bavière, qui a été bien des années avec toute sa confiance à Paris,
qu'il quittait fort rarement pour faire quelques courts voyages vers son
maître. On a parlé aussi de la belle femme qu'il avait épousée, veuve de
La Chétardie, frère du curé de Saint-Sulpice, si bien avec
M\textsuperscript{me} de Maintenon, qui n'influait pas sur la conduite
de cette belle-soeur, dont le fils a longtemps fait tant de bruit en
Russie, où il fut de la part du roi par deux fois. Monasterol était un
Piémontais dont la famille, assez médiocre, s'était transplantée en
Bavière comme quelques autres italiennes. C'était un homme fort
agréable, toujours bien mis, souvent paré, d'un esprit très médiocre,
mais doux, liant, poli, cherchant à plaire, fort galant, qui, en fêtes,
en chère, en meubles, en équipages et en bijoux, vivait dans le plus
surprenant luxe, et jouait le plus gros jeu du monde. Sa femme, encore
plus splendide, augmenta encore sa dépense, et mêla un peu sa compagnie
qui auparavant n'était que du meilleur de la cour et de la ville. On ne
pouvait comprendre comment un homme de soi si peu avantagé de biens, et
ministre d'un prince si longtemps sans États, pouvait soutenir, et tant
d'années, un état si généralement magnifique. Il payait tout avec
exactitude, et passait pour un fort honnête homme. Outre les affaires
dont il était chargé, il l'était encore des pécuniaires de l'électeur,
en subsides, pensions, etc., qui allaient tous les ans à de grandes
sommes, que son prince tirait de la France. Peu à peu ses comptes
languirent. Ceux que l'électeur employa dans ses finances, depuis qu'il
fut rétabli, songèrent sérieusement à en réparer les ruines, et
voulurent voir clair à la longue administration de celles qui avaient
passé et qui continuaient à passer par Monasterol. Il tira de longue
tant qu'il put, aidé même de la protection et de la pleine confiance de
son maître\,; mais à la fin, ce prince fut si pressé par ses ministres,
qu'il envoya des ordres positifs à Monasterol de venir rendre compte à
Munich de toute sa gestion. Alors il n'y eut plus moyen de reculer
davantage. Monasterol, d'un air serein, publia que son voyage serait
court, laissa sa femme et presque toute sa maison à Paris, et partit.
Arrivé à Munich, il fallut compter autres délais. Le soupçon qu'ils
donnèrent fit presser davantage\,; à bout et acculé, il se tira
d'affaires un matin par un coup de pistolet qu'il se donna dans la tête
dans sa chambre. Il laissa des dettes sans nombre, rien pour les payer,
et des comptes en désordre qui firent voir à quel excès il avait abusé
et trompé la confiance et la facilité de l'électeur. Ce prince, qui
l'avait toujours aimé, voulut encore étouffer la catastrophe, et fit
courir le bruit que Monasterol était mort subitement. Sa veuve se trouva
bien étonnée, promptement abandonnée et réduite au plus petit pied d'une
vie qu'elle a depuis menée fort obscure.

La Hire, connu par toute l'Europe pour un des plus grands astronomes
qu'il y ait eu depuis longtemps, mourut à l'Observatoire à près de
quatre-vingts ans, jusque alors dans une continuelle et parfaite santé
de corps et d'esprit\,; l'abbé Abeille, presque en même temps, assez
âgé\,: c'était un homme d'esprit et de beaucoup de lettres, qui
l'avaient mis dans l'Académie française, qui avait des moeurs, de la
religion, de la probité, de la franchise, beaucoup de douceur, de liant,
de modestie, et un grand désintéressement, avec une naïveté et une
liberté charmante. Il s'était attaché de bonne heure au maréchal de
Luxembourg, qu'il suivit en toutes ses campagnes, qui l'avait mis dans
le grand monde et dans les meilleures compagnies, où il se fit toujours
désirer et dont il ne se laissa point gâter. M. le prince de Conti
l'aimait fort. M. de Luxembourg lui avait fait donner des bénéfices.
Après sa mort, il demeura avec la même confiance chez M. de Luxembourg,
son fils, où il est mort regretté de beaucoup de gens considérables et
de tout ce qui le connaissait. C'était en effet un des meilleurs hommes
du monde\,; pour qui j'avais pris de l'amitié, et lui pour moi, pendant
la campagne de 1694, que ma séparation éclatante d'avec M. de
Luxembourg, sur notre procès de préséance, n'avait pu interrompre.

Poirier, premier médecin du roi, mourut presque subitement. M. le duc
d'Orléans déclara aussitôt au duc du Maine et au maréchal de Villeroy
qu'ils pouvaient lui choisir un successeur\,; qu'il ne voulait s'en
mêler en aucune façon\,; qu'il approuverait leur choix quel qu'il fût\,;
qu'il donnait seulement l'exclusion à deux hommes, à Chirac pour l'un, à
Boudin pour l'autre, qui avait été premier médecin de Monseigneur, puis
de M\textsuperscript{me} la Dauphine, et duquel j'ai parlé ici
quelquefois. J'avais fort exhorté M. le duc d'Orléans à toute cette
conduite. Il était d'une part trop inutile à ses intérêts, de l'autre
trop délicat pour lui de se mêler du choix d'un premier médecin dans la
position où il était et à toutes les infamies qu'on avait répandues
contre lui à la mort de nos princes, et qu'on ne cessait de renouveler
de temps en temps. Cette même raison fut la cause des deux exclusions
qu'il donna à Chirac, son médecin de confiance, qu'il avait toujours
gardé auprès de lui depuis qu'il l'avait pris en Languedoc, allant
commander l'armée d'Italie. À l'égard de Boudin, je fis souvenir M. le
duc d'Orléans des propos énormes et sans mesure qu'il avait eu l'audace
de répandre partout, tête levée, lors des pertes dont la France ne se
relèvera jamais, et qui lui tournèrent la tête pour son intérêt
particulier, auquel il était sordidement attaché\,; et qu'il était de
tout temps, comme il l'était encore, vendu à tous ceux qui lui étaient
le plus opposés, et en faisait gloire, outre que c'était un grand
intrigant, de beaucoup d'esprit, fort gâté et très audacieux. Ces
exclusions firent tomber le choix sur Dodart, qui avait été médecin des
enfants de France, et qui avait eu auparavant d'autres emplois de
médecin à la cour.

C'était un fort honnête homme, de moeurs bonnes et douces, éloigné de
manèges et d'intrigues, d'esprit et de capacité fort médiocre, et
modeste. Il était fils d'un très savant et fort saint homme, qui avait
été médecin du prince et de la princesse de Conti-Martinozzi, et qui
l'était demeuré jusqu'à sa mort de la princesse de Conti, fille du roi,
qui avait toujours grande envie de le chasser de la cour pour son grand
attachement à Port-Royal, sans avoir jamais pu trouver prise sur la
sagesse de sa conduite. M\textsuperscript{me} la princesse de Conti, qui
avait en lui toute confiance, indépendamment de celle de sa santé, et
qui ne faisait presque que de le perdre, porta fort son fils à la place
de premier médecin.

Poirier n'avait pas eu le temps, depuis la mort de Fagon, de prendre la
direction du jardin des simples. Je fus surpris que Chirac vînt un matin
chez moi, car je ne crois pas qu'alors je lui eusse jamais parlé ni
presque rencontré. Ce fut pour me prier de lui faire donner cette
direction. Il me dit qu'avec le bien qu'il avait, et en effet il était
extrêmement riche, ce n'était pas pour augmenter son revenu, mais au
contraire pour y mettre du sien. Il me peignit si bien l'extrême abandon
de l'entretien de tant de plantes curieuses et rares et de tant de
choses utiles à la médecine, qu'on devait avoir soin d'y démontrer et
d'y composer, qu'un premier médecin, tout occupé de la cour, ne pouvait
maintenir dans la règle, encore moins les réparer au point où tout y
était tombé, qu'il me persuada que l'utilité publique demandait qu'un
autre en fût chargé. Il ajouta que, par devoir et par goût, il prendrait
tout le soin nécessaire au rétablissement, à l'entretien et au bon ordre
d'un lieu qui, tenu comme il le devait être, honorait la capitale et
instruisait médecins, savants et curieux\,; qu'il serait plus à portée
que nul autre d'y faire venir de toutes parts et élever les plantes les
plus intéressantes et les plus rares, par les ordres de M. le duc
d'Orléans, tant de choses, enfin, que je lui demandai seulement
pourquoi, ayant la confiance de son maître, il ne s'adressait pas
directement à lui. Il me satisfit là-dessus, car il avait beaucoup de
langage, d'éloquence, de tour, d'art et de finesse. C'était le plus
savant médecin de son temps, en théorie et en pratique, et, de l'aveu de
tous ses confrères et de ceux de la première réputation, leur maître à
tous, devant qui ils étaient tous en respect comme des écoliers, et lui
avec eux en pleine autorité comme un autre Esculape. C'est ce que
personne n'ignorait\,; mais ce que je ne sus que depuis et ce que
l'expérience m'apprit aussi dans la suite, c'est que l'avarice le
rongeait en nageant dans les biens\,; que l'honneur, la probité,
peut-être la religion lui étaient inconnus et que son audace était à
l'épreuve de tout. Il sentait que son maître le connaissait, et il
voulait s'appuyer auprès de lui de qui ne le connaissait pas pour
emporter ce qu'il désirait et ce qu'il n'osait espérer de soi-même. J'en
parlai deux jours après à M. le duc d'Orléans, qui l'accorda après
quelque résistance. Oncques depuis n'ai-je ouï parler de Chirac\,; mais,
ce qu'il fit de pis, c'est qu'il ne mit rien au jardin des simples, n'y
entretint quoi que ce soit, en tira pour lui la quintessence, le
dévasta, et en mourant le laissa en friche, en sorte qu'il fallut le
refaire et le rétablir comme me en entier. J'aurai lieu ailleurs de
parler encore de, lui.

\hypertarget{chapitre-xiii.}{%
\chapter{CHAPITRE XIII.}\label{chapitre-xiii.}}

1718

~

{\textsc{Mort de la duchesse de Vendôme.}} {\textsc{- Adresses et ruses
pour l'obscure garde de son corps, sur même exemple de
M\textsuperscript{lle} de Condé\,; ce qui n'a pas été tenté depuis.}}
{\textsc{- Le grand prieur sert à la cène le jeudi saint pour la
dernière fois, et s'absente, le lendemain, de l'adoration de la croix.}}
{\textsc{- Cardinal de Polignac prétend présenter au roi l'évangile à
baiser, de préférence au premier aumônier\,; est condamné.}} {\textsc{-
Le roi visite M\textsuperscript{me} la Princesse et
M\textsuperscript{me}s ses deux filles sur la mort de
M\textsuperscript{me} de Vendôme.}} {\textsc{- Douglas obscur,
misérable, fugitif.}} {\textsc{- M\textsuperscript{me} la duchesse de
Berry parle fort mal à propos au maréchal de Villars\,; se hasarde de
faire sortir M\textsuperscript{me} de Clermont de l'Opéra, etc.\,; se
raccommode bientôt après avec elle et avec M\textsuperscript{me} de
Beauvau.}} {\textsc{- Abbé de Saint-Pierre publie un livre qui fait
grand bruit, et qui le fait exclure de l'Académie française dont il
était.}} {\textsc{- Incendie au Petit-Pont à Paris.}} {\textsc{- Mort et
caractère de M\textsuperscript{me} de Castries.}} {\textsc{-
M\textsuperscript{me} d'Épinai dame d'atours de M\textsuperscript{me} la
duchesse d'Orléans en sa place.}} {\textsc{- Mort de la reine
d'Angleterre à Saint-Germain.}} {\textsc{- Mort, extraction et famille
du duc de Giovenazzo.}} {\textsc{- Bureau de cinq commissaires du
conseil de régence pour examiner les moyens de se passer de bulles.}}
{\textsc{- La peur en prend à Rome qui les accorde toutes, et sans
condition, aussitôt.}} {\textsc{- Mort du comte d'Albemarle.}}
{\textsc{- Sa fortune fatale à celle de Portland.}} {\textsc{- Mort,
caractère, faveur de M. le Grand.}} {\textsc{- Mort de
M\textsuperscript{me} de Chalmazel et de la duchesse de Montfort.}}
{\textsc{- Mariage du duc d'Albret avec une fille de Barbezieux, et du
fils du prince de Guéméné avec une fille du prince de Rohan.}}
{\textsc{- Origine des fiançailles dans le cabinet du roi de ceux qui
ont rang de prince étranger.}} {\textsc{- Mariage du comte d'Agenois et
de M\textsuperscript{lle} de Florensac.}} {\textsc{- Prince et princesse
de Carignan à Paris, où ils se fixent incognito.}} {\textsc{- Triste
éclat de l'évêque de Beauvais.}} {\textsc{- Yolet, ayant quitté le
service depuis treize ou quatorze ans, étant mestre de camp, fait
maréchal de camp.}} {\textsc{- Bruit des mestres de camp de cavalerie
sur le style des lettres que le comte d'Évreux leur écrivait, qui finit
par un \emph{mezzo-termine}.}} {\textsc{- Augmentation de pension à la
duchesse de Portsmouth.}} {\textsc{- Grandes grâces pécuniaires à M. le
prince de Conti.}} {\textsc{- Origine de ce débordement de finances du
roi aux princes et princesses du sang.}} {\textsc{- D'Antin obtient pour
ses deux petits-fils les survivances de ses gouvernements, et Silly une
place dans le conseil du dedans du royaume.}} {\textsc{- Grande sédition
à Bruxelles.}} {\textsc{- Affaires étrangères.}}

~

M\textsuperscript{me} de Vendôme mourut à Paris le 11 avril de cette
année, sans testament ni sacrements, de s'être blasée sur tout de
liqueurs fortes dont elle avait son cabinet rempli. Elle était dans sa
quarante-unième année. Tout ce qu'on en peut dire, c'est que ce fut une
princesse du sang de moins. Elle était fort riche, parce que M. de
Vendôme lui avait donné tous ses biens par son contrat de mariage. On a
vu ici, en son lieu, de quelle manière il se fit, lui par orgueil, elle
pour s'affranchir, M. du Maine pour relever d'autant la bâtardise. En
deux ans de mariage on peut compter au plus par jours ce qu'ils ont été
ensemble, et comme il n'y eut point d'enfants et que le grand prieur,
son beau-frère, ne pouvait hériter de rien, toute cette grande
succession tomba à M\textsuperscript{me} la Princesse, dont elle était
la dernière fille, et à ses autres enfants.

Cette mort donna lieu à une continuation adroite et hardie des princes
du sang de faire garder son corps. Jamais autres que reines, dauphines
et filles de France n'avaient été gardées jusqu'à Mademoiselle, fille de
Gaston, frère de Louis XIII, et de sa première femme, héritière de
Montpensier, comme petite-fille de France, morte en 1693, et celle en
faveur de qui ce nouveau rang de petit-fils de France fut formé comme on
l'a vu, t. VII, p.~167, lequel tient plus du fils de France que du
prince du sang. M\textsuperscript{lle} de Condé étant morte le 23
octobre 1700, M. le Prince, bien plus attentif à usurper qu'aucun autre
prince du sang, même que le grand prince de Condé, son père, fit
doucement en sorte que quelques dames de médiocre étage gardassent le
corps de M\textsuperscript{lle} sa fille, et à leur exemple quelque peu
d'autres d'un peu de meilleur nom, mais hors de tout et de savoir ce
qu'on leur faisait faire. Cette nouveauté, bien que si délicatement
conduite, ne laissa pas de faire du bruit, quoique M. le Prince n'eût
fait inviter les dames que de sa part, n'ayant osé le hasarder de celle
du roi, et ce bruit, qui ouvrit les yeux, causa le refus des dernières
invitées. Cela fit enrayer tout court. M. le Prince se hâta de faire
enterrer M\textsuperscript{lle} de Condé, pour couper court à l'occasion
de la garder. Il profita de l'absence de Blainville, grand maître des
cérémonies, qui était sur la frontière des Pays-Bas, où tout se
regardait déjà, sur l'extrémité du roi d'Espagne qui mourut le 1er
novembre suivant. Desgranges, un des premiers commis de Pontchartrain,
était maître des cérémonies, et peu bastant pour faire à M. le Prince la
plus légère résistance, qui fit glisser dans son registre ce qu'il
voulut.

Sur ce fondement, les princes du sang voulurent continuer
l'entreprise\,; mais ils craignirent M\textsuperscript{me} la Princesse
qui, toute glorieuse qu'elle fût, n'était pas si hardie qu'eux, ni si
confiante en leurs forces et en la sottise du public\,; elle savait
comme eux et mieux qu'eux, pour en avoir été témoin, que l'exemple de
M\textsuperscript{lle} de Condé avait été une tentative hardie, adroite,
ténébreuse et peu heureuse\,; ils se doutèrent qu'elle ne voudrait pas
se commettre à une seconde. Ils s'avisèrent de la faire tonneler par
Dreux, duquel j'ai eu occasion de parler assez pour n'avoir rien à
ajouter, et qui n'était pas homme à manquer de faire sa cour à qui il
craignait, et à ne pas courir au-devant de tout ce qui leur pouvait
plaire. Ils comprirent que la timidité de M\textsuperscript{me} la
Princesse céderait à l'autorité d'un grand maître des cérémonies, sur le
témoignage duquel elle aurait toujours, en tout cas, de quoi s'excuser
ou à le faire valoir. L'expédient réussit comme ils l'avaient espéré.
Néanmoins ils prirent bien garde au choix de dames qui ne pussent
connaître ce qu'on leur proposait, ni qui sussent se sentir, bien plus
encore de s'adresser à pas une femme titrée ou même simple maréchale de
France, ou encore d'un certain air dans le monde, ni qui sussent ce
qu'elles étaient par leur qualité. Contents d'une récidive aussi adroite
et aussi délicate, qui confirmait la première entreprise, au premier
petit bruit qu'ils en entendirent, et qui ne tarda pas, ils imitèrent la
prudence de M. le Prince, et en firent cesser l'occasion tout court en
se hâtant de faire enterrer le corps de, M\textsuperscript{me} de
Vendôme.

Il fut porté, le 16 avril, aux Carmélites du faubourg Saint-Jacques,
conduit par M\textsuperscript{lle} de Clermont, accompagnée des
duchesses de Louvigny et d'Olonne, priées par M\textsuperscript{me} la
Princesse et par M. le Duc, et point du tout de la part du roi. La
cérémonie se passa comme celle de M\textsuperscript{lle} de Condé, où
étaient ma mère et la duchesse de Châtillon, priées par M. le Prince,
comme on l'a vu t. II, p. 443, et Dreux mit sur ses registres ce qu'il
plut aux princes du sang, très peu scrupuleux d'ailleurs sur ce qu'il y
écrivait ou omettait. Il est mort depuis bien des princesses du sang,
sans qu'il ait plus été parlé de la garde de pas une. Les intéressés ont
jugé apparemment qu'il n'était pas à propos de la tenter davantage.

Continuons le récit des entreprises. Le jeudi saint de cette année le
grand prieur servit hardiment à la cène comme les princes du sang. Cette
récidive de l'inouïe nouveauté de l'année passée, contre la parole
expresse du régent, fut l'effet de la même politique qui l'avait permise
la première fois. Elle piquait, elle excitait ce qu'il y avait de plus
grand les uns contre les autres, qui était son manège favori. Cette
année fut pourtant la dernière que cette entreprise eut lieu, quelque
respect, comme on l'a expliqué ailleurs, que le régent eût pour le grand
prieur, qui ne se présenta pas même le lendemain matin chez le roi, à
l'office pour l'adoration de la croix. À la grand'messe de ce même jeudi
saint, le cardinal de Polignac, qui eût mieux fait d'être en son
archevêché d'Auch, où il n'a mis le pied de sa vie, prétendit présenter
le livre des évangiles à baiser au roi, de préférence à l'évêque de
Metz, premier aumônier, parce que le grand aumônier cardinal n'y était
pas. Cette dispute toute nouvelle empêcha le roi de baiser l'évangile.
Deux jours après le régent décida en faveur du premier aumônier, à qui
les cardinaux ne l'ont plus disputé depuis. Il est vrai aussi que depuis
que je suis chevalier de l'ordre, je me suis trouvé à une fête de
l'ordre où il n'y eut ni grand ni premier aumônier, où les cardinaux de
Polignac et de Bissy étaient en leurs places de commandeurs, et où le
cardinal de Polignac présenta au roi l'évangile à baiser, de préférence
aux deux aumôniers de quartier présents en leurs places, qui ne le
disputèrent pas. Ce même jeudi saint, après ténèbres, le roi alla voir
M\textsuperscript{me} la Princesse et M\textsuperscript{me}s ses deux
filles, de Conti et du Maine, sur la mort de M\textsuperscript{me} de
Vendôme.

On a vu, t. XIII, p.~291, l'affreuse aventure du Prétendant, échappé à
Nonancourt par le courage et la sagacité de la maîtresse de la poste, à
Douglas et aux autres assassins, dépêchés sous lui par Stairs après ce
prince, et leur impudence après leur coup manqué. Ce Douglas était
depuis tombé dans la dernière obscurité, par l'horreur de tous les
honnêtes gens\,; mais il était souffert à Paris sous la protection de
Stairs, à qui le régent ne pouvait rien refuser. Douglas, fort
misérable, avait fait des dettes de nature à pouvoir être arrêté chez
lui. On le tenta, il se sauva par les derrières, et Stairs s'interposa
en sa faveur. Mais le répit accordé fut court, et ne servit qu'à lui
donner moyen de sortir de Paris et de se cacher ailleurs. On n'en a plus
ouï parler depuis, quoiqu'il ait traîné encore du temps en France son
infâme et obscure vie, qu'il aurait dû perdre entre quatre chevaux en
revenant de Nonancourt. Il avait épousé à Metz une demoiselle qui avait
du bien et qu'il a laissée veuve sans enfants il y a bien des années, et
presque à la mendicité.

M\textsuperscript{me} la duchesse de Berry fit presque de suite deux
traits qui furent très contradictoires, et qui montrèrent également
l'excès de son orgueil et de son peu de jugement. Entraînée par les
roués de M. le duc d'Orléans, avec qui, toute fille de France qu'elle
était, elle soupait souvent, et dont plusieurs étaient pour se recrépir
d'avec cette prétendue noblesse à qui tout était bon, {[}elle{]} se
hasarda de parler chez elle, publiquement et fort mal à propos, au
maréchal de Villars sur ses lettres aux colonels, dont cette prétendue
noblesse s'avisait de se plaindre. On fut surpris de la sagesse et de la
modération du maréchal, qui n'était pas fait pour recevoir, non pas même
du régent, une réprimande publique\,; cette princesse, transportée
d'orgueil, qui se croyait droit de tout, et qui n'avait pourtant pas
celui de reprendre personne sur ce qui ne lui manquait pas de respect,
et si encore, avec la mesure convenable aux personnes, ne comprit pas
qu'elle était en cela l'instrument et le jouet d'un ramas de gens de
toutes les sortes, excités adroitement par M. et M\textsuperscript{me}
du Maine et les plus dangereux ennemis de M. le duc d'Orléans, pour le
culbuter, et qui, en attendant que leurs conducteurs vissent le moment
de les faire frapper au véritable but, se laissaient éblouir du beau
dessein de mettre tout dans une égalité qui, en défigurant l'État, le
rendant dissemblable à ce qu'il est depuis sa fondation, et à tous les
autres États du monde, anéantissait les avantages de la grande, ancienne
et véritable noblesse, ôtait les gradations, supprimait les récompenses,
détruisait radicalement toute ambition, attaquait l'autorité, le droit
et la majesté du trône, à réduisait tout au même niveau, et par une
suite nécessaire, dans la dernière confusion, jetait tout dans
l'oisiveté, dans la paresse, dans le néant, vidait la cour, désertait
les armées, les ambassades, etc., et ne laissait de distinctions et
d'avantages qu'aux richesses, par conséquent à la bassesse, à l'avarice,
à la cupidité d'en acquérir et de les conserver par toutes sortes de
moyens. En même temps elle {[}ne{]} vit pas combien par cette folle
action elle manquait de respect au roi, eu usurpant, bien que sa
sujette, une autorité inséparable de sa couronne, et au régent son père,
unique dépositaire, comme régent, de l'autorité du roi mineur, et le
seul en France qui eût caractère pour l'exercer en son nom.

Incontinent après s'être si étrangement montrée protectrice de cette
écume de noblesse, elle se porta à insulter en public toute la véritable
et la haute noblesse, qu'elle offensa toute en la personne de deux dames
de cette qualité. On a vu, ci-dessus, p.~64, comment et pourquoi
M\textsuperscript{me}s de Beauvau et de Clermont-Gallerande avaient
quitté les places qu'elles avaient auprès d'elle. Elle le leur
pardonnait d'autant moins qu'elles en étaient fort approuvées et
qu'elles et leurs maris n'en avaient pas été moins bien traités depuis
par Madame, et par M. et M\textsuperscript{me} la duchesse d'Orléans.
Étant à l'Opéra, dans sa petite loge, elle se trouva si piquée de voir
M\textsuperscript{me} de Clermont vis-à-vis d'elle dans la petite loge
de M. le comte de Toulouse qui n'y était pas, qu'elle envoya
sur-le-champ lui défendre par Brassac, exempt de ses gardes, de se
trouver jamais dans les lieux où elle serait. C'était bien en dire
autant à M\textsuperscript{me} de Beauvau si elle s'y fût trouvée.
Aussitôt M\textsuperscript{me} de Clermont sortit fort sagement de la
loge et s'en alla avec la jeune M\textsuperscript{me} d'Estampes, qui
s'y trouva seule avec elle. Cette action fit un grand bruit dans le
monde, et fut en effet un acte de vraie souveraineté, tel qu'il
n'appartient qu'au roi, qui seul a le pouvoir d'exiler et de bannir
partout de sa présence. C'était attenter aussi à la liberté publique, et
se mettre au-dessus de toute mesure, de toute règle, de toute loi. Les
propos ne se continrent pas, mais ce fut presque tout. La princesse
était fille du régent, on connaissait sa violence et toute, la faiblesse
de son père. Madame et lui ne laissèrent pas de lui en dire leur avis.

Après quelques jours de furie contre le scandale du public, elle ne put
se dissimuler qu'elle n'en fût embarrassée. C'était dans ses embarras
qu'elle s'ouvrait à M\textsuperscript{me} de Saint-Simon\,; qui n'était
point à cet opéra avec elle, et toutes deux jusqu'alors ne s'étaient pas
ouvert la bouche l'une à l'autre de toute cette belle aventure. Elle
connaissait la sagesse de ses conseils, quoiqu'elle les prît rarement.
Elle savait combien elle était aimée et honorée dans sa maison\,; elle
n'ignorait pas les sentiments de ces deux dames pour elle, qui, avant et
depuis leur retraite, ne s'étaient pas cachées, que la seule
considération de M\textsuperscript{me} de Saint-Simon les avait arrêtées
longtemps. M\textsuperscript{me} de Saint-Simon profita de ce trouble de
M\textsuperscript{me} la duchesse de Berry pour lui faire sentir toute
sa faute, et lui persuader de finir honnêtement et convenablement des
procédés qui étaient insoutenables. Enfin elle la fit consentir à voir
les deux dames et les deux maris, avec des manières, des honnêtetés et
des propos qui pussent réparer tout ce qui s'était passé. Ce ne fut pas
sans peine qu'elle l'amena à ce point\,; la manière en fut une autre.
Cette espèce d'avance en public pesait trop à son orgueil. Elle voulut,
pour cette première fois, éviter Luxembourg. Il fut donc convenu entre
elles deux que M\textsuperscript{me} la duchesse de Berry irait deux
jours après aux Carmélites du faubourg Saint-Germain où elle avait un
appartement\,; que M\textsuperscript{me} de Saint-Simon avertirait M. et
M\textsuperscript{me} de Beauvau et M. et M\textsuperscript{me} de
Clermont, et qu'elle-même les mènerait aux Carmélites, où elle serait
témoin de la réception.

Cela fut exécuté le 4 juin, six semaines après l'affaire de l'Opéra,
arrivé le 25 avril. Ils entrèrent tous dans le monastère, et allèrent
droit à l'appartement de M\textsuperscript{me} la duchesse de Berry qui
les y attendait. Chacun de son côté se posséda assez pour que l'accueil
fût également obligeant et bien reçu. Les deux hommes demeurèrent peu
dans le couvent, parce qu'il est très rare que les hommes y entrent.
M\textsuperscript{me} de Beauvau y fut retenue, et M\textsuperscript{me}
la duchesse de Berry lui fit des merveilles. M\textsuperscript{me} de
Clermont se trouva lors près de Fontainebleau, chez M. le comte de
Toulouse, à la Rivière, et n'en put revenir à temps. Dès qu'elle fut
revenue, elle alla chez M\textsuperscript{me} la duchesse de Berry, où
tout se passa très bien de part et d'autre\,; et depuis elles ont toutes
deux été, et leurs maris, chez M\textsuperscript{me} la duchesse de
Berry de temps en temps.

Une forte plate chose fit alors un furieux bruit. J'ai parlé quelquefois
ici des Saint-Pierre, dont l'un était premier écuyer de
M\textsuperscript{me} la duchesse d'Orléans\,; l'autre, son frère,
premier aumônier de Madame. Celui-ci avait de l'esprit, des lettres et
des chimères. Il était de l'Académie française depuis fort longtemps et
fort rempli de lui-même, bon homme et honnête homme pourtant, grand
faiseur de livres, de projets et de réformations dans la politique et
dans le gouvernement en faveur du bien public. Il se crut en liberté par
le changement du gouvernement et de donner l'essor à son imagination en
faveur du bien public. Il fit donc un livre qu'il intitula \emph{la
Polysynodie}\footnote{Ce mot, qui signifie pluralité des conseils, fut
  inventé par l'abbé de Saint-Pierre. L'ouvrage qui porte ce titre parut
  en 1718. Voy. à la fin du t. XII, une note sur les conseils tirée des
  \emph{Mémoires du marquis d'Argenson}.}**, dans lequel il peignit au
naturel le pouvoir despotique et souvent tyrannique que les secrétaires
d'État et le contrôleur général des finances exerçaient sous le dernier
règne, qu'il appela des vizirs, et leurs départements des vizirats, et
s'espaça là-dessus avec plus de vérité que de prudence.

Dès qu'il parut, il causa un soulèvement général de tout l'ancien
gouvernement et de tous ceux encore qui se flattaient d'y revenir après
la régence. Les anciens courtisans du feu roi se piquèrent aux dépens
d'autrui d'une reconnaissance qui ne leur coûtait rien. Le maréchal de
Villeroy se signala par un vacarme épouvantable, et de gré ou de force
ameuta toute la vieille cour. Hors ceux-là personne ne se scandalisait
d'un ouvrage qui pouvait manquer de prudence, mais qui ne manquait en
rien à la personne du feu roi, et qui n'exposait que des vérités, dont
tout ce qui vivait alors avait été témoin, et dont personne ne pouvait
contester l'évidence. Les académies, les autres gens de lettres, le
reste du monde, s'indigna même et le montra, que ces messieurs de la
vieille cour ne pussent encore souffrir la vérité et la liberté, tant
ils s'étaient accoutumés à la servitude. Mais le maréchal de Villeroy
fit tant de manèges, de déclamations, de tintamarre, entraîna par ses
violences tant de gens à n'oser ne pas crier en écho que M. le duc
d'Orléans, qui de longue main n'aimait pas les Saint-Pierre, et à qui le
maréchal de Villeroy imposait, ne voulut pas pour eux résister à ce
tumulte. L'abbé de Saint-Pierre fut donc chassé de l'Académie française
malgré l'Académie, qui n'osa résister jusqu'au bout\,; mais de peu de
maisons, dont à la vérité il en fréquentait peu de considérables. Le
livre fut supprimé\,; mais l'Académie, profitant du goût du régent, pour
les \emph{mezzo-termine}, obtint qu'il ne se ferait point d'élection et
que la place de l'abbé de Saint-Pierre ne serait point remplie\,; ce qui
a été exécuté malgré les cris de ses persécuteurs jusqu'à sa mort.

Le feu prit, le 27 avril, au Petit-Pont. Un imprudent, cherchant quelque
chose avec une chandelle dans des recoins d'un bateau de foin,
l'embrasa. La frayeur qu'il ne communiquât le feu, à plusieurs autres,
au milieu desquels il était, le fit pousser à vau-l'eau avec
précipitation. Il vint donner contre un pilier des arches de ce
Petit-Pont. La flamme, qui s'élevait de dessus, prit à une des maisons
du pont, et causa un assez grand incendie. Le duc de Tresmes, gouverneur
de Paris, les magistrats de police et beaucoup de gens y coururent. Le
cardinal de Noailles y passa une partie de la nuit à faire porter chez
lui quantité de malades de l'Hôtel-Dieu, dont les salles étaient en
danger, et à les faire secourir chez lui en vrai pasteur et père.
L'archevêché en fut tout rempli, et ses appartements ne furent point
ménagés. On vit le moment que l'Hôtel-Dieu entier allait être brûlé\,;
mais, par le bon et prompt ordre, il n'y eut que très peu de chose de
cet hôpital et une trentaine de maisons brûlées ou abattues. Les
capucins s'y signalèrent très utilement. Les cordeliers y servirent
aussi fort bien. Le duc de Guiche y fit venir le régiment des gardes,
qui rendit de grands devoirs, et le duc de Chaulnes fit garder les
meubles et les effets par ses chevau-légers à cheval. On s'y moqua un
peu du maréchal de Villars, qui y fit venir du canon pour abattre des
maisons, remède qui n'eût pas été moins fâcheux que le mal sur des
maisons toutes de bois et si entassées. Le maître des pompes n'y acquit
pas d'honneur.

M\textsuperscript{me} de Castries, dame d'atours de
M\textsuperscript{me} la duchesse d'Orléans, fut trouvée le matin dans
son lit sans connaissance, qui, malgré tous les remèdes, ne revint point
jusqu'à huit heures du soir, qu'elle mourut sans laisser d'enfants\,:
elle se portait très bien, et M\textsuperscript{me} de Saint-Simon avait
passé une partie du soir de la veille chez elle. Ce qui surprit
davantage, c'est que ce n'était qu'esprit et âme sans presque de corps.
Le sien était petit et si mince, qu'un souffle l'eût renversée. Ce fut
grand dommage\,: j'ai parlé ailleurs d'elle et de son mari, qui, avec
raison, ne s'en est jamais consolé. C'était une petite poupée manquée,
foncièrement savante en tout, sans qu'il y parût jamais, mais pétillante
d'esprit, souvent aussi de malice, avec toutes les façons, les grâces,
et ce tour et cette sorte d'esprit et d'expressions charmantes et
uniques, si vantées et si singulièrement propres aux Mortemart. Deux
jours après, M\textsuperscript{me} d'Épinai fut choisie pour lui
succéder. Un laquais de M\textsuperscript{me} de Castries, l'apprenant
dans la cour du Palais-Royal\,: «\,Ah\,! ma pauvre maîtresse,
s'écria-t-il, dans quel étonnement serait-elle si elle savait qui lui
succède\,! » M\textsuperscript{me} la duchesse d'Orléans la voulut
absolument parce qu'elle était fille de M. d'O. On a souvent parlé
ailleurs de toute cette cordelle de bâtardise, M\textsuperscript{me} la
duchesse d'Orléans voulut persuader le monde que ce choix était de M. le
duc d'Orléans, qui le nia et lui renvoya la balle, et fut le premier à
se moquer du choix. La pauvre femme y fit pourtant fort bien, et s'y fit
aimer de tout le monde.

La reine d'Angleterre mourut le 7 mai à Saint-Germain, après dix ou
douze jours de maladie. Sa vie, depuis qu'elle fut en France, à la fin
de 1688, n'a été qu'une suite de malheurs qu'elle a héroïquement portés
jusqu'à la fin, dans l'oblation à Dieu, le détachement, la pénitence, la
prière et les bonnes oeuvres continuelles, et toutes les vertus qui
consomment les saints. Parmi la plus grande sensibilité naturelle,
beaucoup d'esprit et de hauteur naturelle, qu'elle sut captiver
étroitement et humilier constamment, avec le plus grand air du monde, le
plus majestueux, le plus imposant, avec cela doux et modeste. Sa mort
fut aussi sainte qu'avait été sa vie. Sur les six cent mille livres que
le roi lui donnait par an, elle s'épargnait tout pour faire subsister
les pauvres Anglais, dont Saint-Germain était rempli. Son corps fut
porté le surlendemain aux Filles de Sainte-Marie de Chaillot, où il est
demeuré en dépôt, et où elle se retirait souvent. La cour ne prit aucun
soin ni part en ses obsèques. Le duc de Noailles alla à Saint-Germain
comme gouverneur du lieu et comme capitaine des gardes, pour ordonner
seulement que tout y fût décent. Le deuil ne fut que de trois semaines.

Cellamare, ambassadeur d'Espagne à Paris, perdit en même temps son père
à Madrid, qui s'appelait le duc de Giovenazzo, duquel le grand-père
était médecin à Gênes, où il s'enrichit par le commerce. Son fils se
transplanta à Naples, y fit de grandes acquisitions, continua le
commerce, mais faisant l'homme de qualité, et augmenta beaucoup ses
richesses. Ses deux fils se trouvèrent avoir beaucoup d'esprit, surtout
l'aîné, qui s'intrigua si bien à la cour d'Espagne, qu'il s'y poussa à
tous les emplois, et que Charles II le fit grand de troisième classe, et
pour trois races, c'est-à-dire son fils et son petit-fils. Sa capacité
très reconnue le fit mettre dans le conseil d'État, qui était lors le
dernier comble de fortune. Philippe V le trouva ainsi revêtu, et eut
pour lui beaucoup de considération, et il est vrai qu'il était fort
compté à Madrid. Il mourut extrêmement vieux, et s'était toujours très
bien conduit. Son frère ne s'était pas moins poussé à Rome. Son argent
l'éleva de charge en charge, et enfin à la pourpre romaine. C'est le
cardinal del Giudice, dont il est parlé ici en tant d'endroits. Il vécut
aussi fort vieux, mais pas assez pour voir son neveu cardinal, qui prit
aussi le nom de cardinal del Giudice. Celui-ci était frère de Cellamare,
et passa sa vie à Rome dans les charges de prélature, puis de la maison
du pape, et enfin dans le cardinalat. Pour Cellamare, il donnera ample
occasion de parler de lui.

Il y avait longtemps que le pape, persécuté par son nonce Bentivoglio,
par les cardinaux de Rohan, surtout de Bissy, et par les plus emportés
de ce parti, s'était rendu à eux malgré lui, à refuser des bulles. Grand
nombre d'églises étaient sans évêque, quoique nommés la plupart. Il en
était de même des abbayes, et le cardinal Fabroni tenait le pape de
court avec ses emportements ordinaires, pour empêcher que le pied lui
glissât là-dessus. Dans les commencements de cette résolution, ils
n'auraient pas été fâchés d'accorder des bulles à des conditions
honteuses pour la France et pour des évêques, utiles à la domination
romaine, qui est le but où toutes choses tendent en cette cour\,: des
lettres soumises des nommés au pape, des signatures chez le nonce telles
qu'il les aurait présentées, exclusion, indépendante de qui ils auraient
voulu. Le régent, quelquefois ébranlé, serait assez volontiers entré en
composition sur la qualité des conditions\,; mais le maréchal
d'Huxelles, qui avait quelquefois de bons intervalles sur ces matières
de Rome, lui en remontra si bien la honte présente, et les conséquences
pernicieuses pour l'avenir, qu'il le raffermit contre les manèges de
toutes les sortes que la cabale employait auprès de lui. À la fin,
pressé par ceux qui avaient plus de sang français dans les veines, il
prit un parti dont Rome et les siens ne le jugeaient pas capable, et
qui, toutes les fois qu'on en prendra un semblable suivant la nature des
affaires, amènera toujours cette cour à raison.

Le régent déclara au conseil de régence qu'il fallait pourvoir à la
dureté de la cour de Rome\,; que, puisqu'elle s'opiniâtrait depuis si
longtemps à refuser des bulles contre la loi réciproque du concordat, il
fallait chercher et trouver le moyen de se passer d'elle là-dessus\,;
qu'il était d'avis d'établir un bureau de personnes capables défaire les
recherches nécessaires à cet effet, d'en rendre compte au conseil de
régence le plus tôt qu'il serait possible, et aussitôt après se servir
de la voie qui aurait été reconnue la meilleure pour faire sacrer tous
les évêques nommés. Le conseil applaudit d'une voix, au grand regret de
M. de Troyes, qui n'osa se commettre à se montrer d'avis différent, et
qui se contenta de consentir d'une inclination de tête, en faisant la
grimace en dessous. Tout de suite le régent proposa le choix qu'il
faisait de cinq commissaires pour composer ce bureau, et nomma le
maréchal de Villeroy, d'Antin, le maréchal d'Huxelles, Torcy, et moi
pour chef de ce bureau qui se tiendrait chez moi, comme l'ancien pair de
ce bureau et de tout le conseil de régence, et le choix en fut approuvé.
C'était à moi à donner les jours de bureau, et pour cela à en préparer
les matières\,; à moi encore, quand le travail y serait achevé, de le
rapporter au conseil de régence.

La matière m'était tout à fait nouvelle, je voulus m'en instruire à
fond. Je pris donc soin de m'informer de ceux qui seraient les plus
capables de me bien endoctriner. Je les vis au nombre de sept ou huit
qui passaient pour l'être le plus en cette matière. J'eus quelques
conversations et des mémoires de quelques-uns. Celui de tous qui me
satisfit le plus par sa profonde science, sa mémoire sur les faits, son
sens et son jugement pour l'application et le raisonnement, et ce que je
trouvai assez rare parmi ces doctes, par la politesse et la science du
monde, fut un abbé Hennequin, retiré dans une maison d'une des cours de
l'abbaye de Sainte-Geneviève. M. Petitpied, qui avait été des années en
Hollande, exilé après au loin, puis rapproché près de Paris, me satisfit
fort aussi, et un M. Le Gros, qui demeurait en Sorbonne. Je demandai à
M. le duc d'Orléans de permettre à M. Petitpied de revenir à Paris,
parce que je ne pouvais pas aller souvent le chercher à Asnières. Il me
l'accorda, et cela finit son exil.

Je n'eus pas le temps de me rendre bien habile ni de tenir un seul
bureau. Rome en prit une telle frayeur que, sans balancer, le pape manda
le cardinal de La Trémoille, à qui le régent avait défendu de prendre
les bulles de Cambrai, sans que les autres nommés eussent les leurs en
même temps. Le pape, sans lui faire de plaintes du parti que le régent
prenait, qui avait répandu l'alarme dans Rome, lui déclara qu'il
accordait toutes les bulles, et le pria de ne pas différer de dépêcher
un courrier à Paris pour y porter cette nouvelle. Elle fit grand plaisir
et aurait dû servir d'une grande leçon à l'avenir pour se conduire avec
Rome. Les bulles furent expédiées incontinent après, et on n'entendit
plus parler à Paris que de sacres d'évêques. Oncques depuis, Rome ne
s'est jouée à un pareil refus, ni à faire faire aucune proposition à pas
un nommé pour en obtenir. Ainsi finit ce bureau avant de s'être pu
assembler, dont nous fûmes tous fort aises, et je pense que l'opinion
que de longue main Bentivoglio et les principaux boutefeux avaient
donnée à Rome de la plupart des commissaires, sur les matières qui
regardent cette cour, et la constitution en particulier, n'y fit guère
moins d'impression que la chose même, et que cette cour comprit par là
qu'on voulait sérieusement conduire à fin. Il y avait trois archevêchés
et douze ou treize évêchés.

On apprit la mort du comte d'Albemarle, gouverneur de Bois-le-Duc, et
général des troupes hollandaises. Je le remarque, parce que ce fut lui
dont la faveur naissante auprès du roi Guillaume prévalut sur celle de
Portland, pendant sa brillante ambassade ici, aussitôt après la paix de
Ryswick, et que cette jalousie lui fit abréger le plus qu'il put. La
faveur de Portland {[}fut{]} la plus ancienne, la plus entière, la plus
durable, et qui avait eu la confiance de tous les manèges de ce prince
en Hollande, pour s'y rendre peu à peu le maître, comme il le devint, de
toutes ses pratiques dans toutes les cours de l'Europe, pour allumer et
entretenir la guerre contre la France, enfin de toute l'affaire
d'Angleterre, où devenu roi, il le fit comte de Portland, chevalier de
la Jarretière, et lui donna des charges et des emplois. Portland,
jusqu'à ce qu'il fût pair d'Angleterre, portait le nom de Benting, qui
était celui de sa famille. Il était Hollandais, et sa faveur avait
commencé dès le temps qu'il était paie de ce même prince d'Orange, et
toujours augmenté depuis. Keppel, Hollandais comme lui, le désarçonna
pendant sa courte ambassade de France, quoique sa faveur fût nouvelle.
Il fut fait comte d'Albemarle. Elle augmenta sans cesse, et dura jusqu'à
la mort de Guillaume, auprès duquel Portland n'eut plus que la
considération, qu'après une si longue et si entière confiance, son
maître ne lui put refuser. Belle leçon pour les courtisans et les
favoris. Si un aussi grand homme que Guillaume III a été capable d'une
telle légèreté, sans autre cause qu'une légèreté dont il avait paru si
incapable, lui si solide et si suivi en tout, et encore à son âge, quel
fonds faire sur les autres princes\,! Portland pensa plusieurs fois à se
retirer en Hollande\,; lui et son émule Albemarle s'y retirèrent tout à
fait après la mort de Guillaume.

M. le Grand mourut en même temps à Royaumont, abbaye depuis longtemps
dans sa famille, dont son père et lui avaient fait leur maison de
plaisance et où il était allé prendre l'air, à près de soixante-dix-sept
ans, à même âge et même maladie que le feu roi. Il fut un des exemples,
également long et sensible, du mauvais goût de ce prince en favoris,
dont il n'eut aucun qui ait joui, d'une si constante et parfaite
{[}faveur{]}, jointe à la considération et à la distinction la plus
haute, la plus marquée, la plus invariable. Une très noble et très belle
figure\,; toute la galanterie, la danse, les exercices\,; les modes de
son temps\,; une assiduité infatigable\,; la plus basse, la plus puante,
la plus continuelle flatterie\,; toutes les manières et la plus
splendide magnificence du plus grand seigneur, avec un air de grandeur
naturel qu'il ne déposait jamais avec personne, le roi seul excepté,
devant lequel il savait ramper comme par accablement de ses rayons,
furent les grâces qui charmèrent ce monarque et qui acquirent, quarante
ans durant, à ce favori toutes les distinctions et les privantes, toutes
les usurpations qu'il lui plut de tenter, toutes les grâces, pour soi et
pour les siens, qu'il prit la peine de désirer, qui réduisirent tous les
ministres, je dis les plus, audacieux, les Seignelay, les Louvois et
tous leurs successeurs, à se faire un mérite d'aller chez lui et
au-devant de tout ce qui lui pouvait plaire, et qu'il recevait avec les
façons de supériorité polie comme ce qui lui était dû. Il avait su
ployer les princes du sang même, bien plus, jusqu'aux bâtards et
bâtardes du roi, à la même considération pour lui et à une sorte
d'égalité de maintien avec eux chez lui-même. La goutte, qui lui fut
d'abord un prétexte puis une nécessité de ne point sortir de chez lui,
une grande et excellente table, soir et matin, et le plus gros jeu du
monde, toute la journée, où abondait une grande partie de la cour, lui
furent d'un grand secours pour maintenir un air de supériorité si
marquée. Il ne sortait que rarement pour se faire porter chez le roi ou
pour aller à Marly jouer dans le salon.

Jamais homme si court d'esprit ni si ignorant, autre raison d'avoir mis
le roi à son aise avec lui, instruit pourtant de ce qui intéressait sa
maison et des choses de la Ligue, dont, avec plus d'esprit, il aurait eu
l'âme fort digne. L'usage continuel du plus grand monde et de la cour
suppléait à ce peu d'esprit, pour le langage, l'art et la conduite, avec
la plus grande politesse, mais la plus choisie, la plus mesurée, la
moins prodiguée et l'entregent de captiver quoique avec un mélange de
bassesse et de hauteur, tout l'intérieur des principaux valets du roi.
D'ailleurs brutal, sans contrainte avec hommes et femmes, surtout au
jeu, où il était très fâcheux et lâchait tout plein d'ordures, sur le
rare pied que personne ne se fâchait de ses sorties, et que les dames,
je dis les princesses du sang, baissaient les yeux et les hommes riaient
de ses ordures. Jamais homme encore si gourmand, qui était une autre
occasion fréquente de tomber sur hommes et femmes sans ménagements, si
le hasard leur faisait prendre un morceau dont il eût envie, ou s'il
était prié à manger quelque part ou que lui-même eût demandé un repas et
qu'il ne se trouvât pas à sa fantaisie. C'était, de plus, un homme
tellement personnel qu'il ne se soucia jamais de pas un de sa famille, à
la grandeur près, et qu'à la mort de sa femme et de ses enfants il ne
garda aucune bienséance ni sur le deuil, ni sur le jeu, ni sur le grand
monde. Au fond, il était bon homme, avait de l'honneur, aimait à servir
et avait en affaires d'intérêts les plus nobles et les plus grands
procédés qu'il fût possible. Avec tout cela il ne fut regretté de
personne. J'ai rapporté en leur temps ici quelques traits de lui
singuliers, en bien et en mal. Il n'avait presque servi qu'à la suite du
roi dans les armées. Il vécut toujours au milieu du plus grand monde
sans amis particuliers, et ne se mêla jamais de rien à la cour que de ce
qui regardait le rang de sa maison, dont il fut toujours très
sensiblement occupé, sans aucun soin de ses affaires particulières, que
M\textsuperscript{me} d'Armagnac savait très bien gouverner et qu'il
laissa conduire à ses gens après elle. Il ne découchait presque jamais
des lieux où le roi était, et c'était auprès de lui un autre grand
mérite.

M\textsuperscript{me} de Chalmazel mourut\,; je le remarque par la
singularité d'être soeur de père du maréchal d'Harcourt et de mère de la
maréchale sa femme.

Le comte de Grammont, de Franche-Comté, qui y commandait, mourut à
Besançon. J'obtins ce commandement pour M. de Lévi, en conservant sa
place et son emploi au conseil de guerre, que je me doutais déjà qui ne
durerait pas longtemps, non plus que les autres conseils. Ce fut un état
assuré, et vingt mille livres d'appointements.

La duchesse de Montfort, fille unique de Dangeau de son premier mariage,
mourut au couvent de la Conception, où elle s'était retirée à la mort de
son mari, malgré père et beau-père et belle-mère, qui la voulaient
garder à l'hôtel de Luynes. C'était une bonne et aimable femme, qui
avait de l'esprit, mais à qui des infirmités presque continuelles
avaient donné des fantaisies qui avaient un peu altéré ses biens.

Ces morts furent bientôt suivies de trois mariages. Il y avait longtemps
que le duc d'Albret voulait épouser M\textsuperscript{lle} de Culant,
qui était fort riche, fille de Barbezieux et de M\textsuperscript{lle}
d'Alègre, sa seconde femme. Toute la famille de M. de Louvois ne le
voulait point, et d'Alègre, grand-père, était d'accord avec le duc
d'Albret. La fille n'avait ni père ni mère. Les procédés tournés en
procès furent arrêtés par les menées de M. le prince de Conti, qui en
fit son affaire pour M. d'Albret, et par l'autorité de M. le duc
d'Orléans, qui n'y avait que faire, mais qui s'y laissa peu à peu
engager, dont M. de La Rochefoucauld et le duc de Villeroy, qui lui
parlèrent vivement, furent fort piqués. Enfin, après bien du bruit, du
temps et des difficultés, le curé de Saint-Sulpice publia deux bans. Dès
que les Louvois le surent, ils s'y opposèrent, et se plaignirent
amèrement du curé, qui les étonna fort en leur montrant un ordre du
régent. Le troisième ban suivit et la nuit même la célébration du
mariage à Saint-Sulpice. L'abbé de Louvois y accourut avec une
opposition en forme. On s'en doutait. M. le prince de Conti s'y trouva
exprès, alla au-devant de lui et l'arrêta par un ordre qu'il lui fit
voir de M. le duc d'Orléans. Peu de gens approuvèrent la chose et la
manière.

Le fils aîné du prince de Guéméné épousa la troisième fille du prince de
Rohan avec de grandes substitutions. Le mariage se fit dans l'église de
Jouars, dont une fille du prince de Rohan était abbesse, et où ils
allèrent tous pour éviter des fiançailles publiques.
M\textsuperscript{me} la duchesse de Berry s'était fort choquée d'en
voir faire dans le cabinet du roi pour les maisons de Lorraine, Rohan et
Bouillon quand le marié et la mariée sont de même rang, ce que la faveur
de l'un des deux a étendu quelquefois, comme aux fiançailles de
M\textsuperscript{me} de Tallard, et de cette similitude avec celles des
princes et des princesses du sang. Elle s'en était laissé entendre, et
les prudents Rohan évitèrent de s'y commettre. Ces fiançailles et même
les mariages en présence du roi et de la reine étaient communs à tous
les grands seigneurs, même aux gens de faveur. La restriction peu à peu
aux princes étrangers fut un des fruits de la Ligue, auquel MM. de
Bouillon d'aujourd'hui et de Rohan ont participé, quand l'intérêt du
cardinal Mazarin pour les premiers, et la beauté de
M\textsuperscript{me} de Soubise pour les seconds, les a faits princes.

Le comte d'Agenois, fils du marquis de Richelieu, épousa
M\textsuperscript{lle} de Florensac, presque aussi belle que sa mère,
qui était Saint-Nectaire. Son père était frère du duc d'Uzès, gendre du
duc de Montausier. Elle n'avait plus ni l'un ni l'autre. Ces mariés ont
fait depuis du bruit dans le monde lui par ses charmes, dont les
intrigues de M\textsuperscript{me} la princesse de Conti, soeur de M. le
Duc, ont récompensé les longs services et très publics, de l'usurpation
juridique de la dignité de duc et pair d'Aiguillon, sans cour ni service
de guerre\,; elle, par l'art de gagner force procès, de faire une riche
maison et de dominer avec empire sur les savants et les ouvrages
d'esprit, qu'elle a accoutumés à ne pouvoir se passer de son attache, et
les compagnies les plus recherchées à l'admirer, quoique assez souvent
sans la comprendre.

Le prince de Carignan arriva ici. Il était fils unique de ce fameux
muet, qui l'était du prince Thomas et de la dernière princesse du sang
de la branche de Soissons. Ce prince de Carignan n'avait rien entre les
enfants de M. de Savoie et lui, qui était lors roi de Sicile, et il en
était regardé comme l'héritier très possible. Ce prince en prit soin
comme d'un de ses fils, et ne s'opposa point à l'amour qu'il conçut pour
la bâtarde qu'il avait de M\textsuperscript{me} de Verue, qui le
conduisit à l'épouser. Le roi de Sicile, qui aimait tendrement cette
fille, en fut ravi, et redoubla pour eux de soins et de grâces. Les
moeurs, la conduite et les folles dépenses du prince de Carignan y
répondit si mal qu'il se brouilla avec le roi de Sicile, de la cour et
des États duquel il s'échappa. Il n'osa, par cette raison, être ici
qu'incognito sous le nom de comte del Bosco. On l'y laissa, pour que
cette contrainte l'engageât à s'en retourner, comme le roi de Sicile le
voulait. Au lieu de cela, M\textsuperscript{me} de Carignan se sauva de
Turin, ou en fit le semblant, pour venir trouver son mari. Celui-ci
{[}y{]} est demeuré toute sa vie, c'est-à-dire plus de vingt ans,
M\textsuperscript{me} de Carignan y est encore. M\textsuperscript{me} de
Verue sut la dresser, et trouva au delà de ses espérances. Les
personnages qu'ils y ont joués, les millions qu'ils y ont pris à toutes
mains, ne se peuvent ni expliquer ni nombrer. Tout le monde l'a vu et
senti\,; on n'y a que trop reconnu les louveteaux du cardinal d'Ossat,
même les plus grands et les plus affamés. L'incognito a toujours duré et
a masqué les prétentions.

Le dérangement éclatant de l'évêque de Beauvais fit un étrange bruit, et
ne put être arrêté ni étouffé par tous les soins de la duchesse de
Beauvilliers, ni toute la charité du cardinal de Noailles, qui y firent
tous deux des prodiges dont je fus témoin de bien près. Ce scandale, qui
ne dura que trop longtemps, se termina enfin par la démission de son
évêché, qui fut donné à un fils du duc de Tresmes, et le démis fut mis
en retraite avec une grosse abbaye et des gens sûrs auprès de lui pour
en prendre soin. M\textsuperscript{me} de Beauvilliers, qui l'avait
toujours aimé, et dont la surprise fut aussi grande que celle de tout le
monde, en pensa mourir de douleur.

J'aurais dû placer à la suite de la promotion militaire dont j'ai parlé,
il n'y a pas longtemps, une grâce que j'obtins de M. le duc d'Orléans,
qui fit du bruit, mais qui me fit un plaisir très sensible. Yolet,
mestre de camp du régiment de Berry, connu en Auvergne pour être de très
bonne et ancienne noblesse, et dans les troupes pour avoir toujours
servi avec valeur et application, avait quitté le service il y avait
treize ou quatorze ans, piqué de n'avoir pas été fait brigadier, en
l'ancienneté de l'être, dans la promotion où le lieutenant-colonel du
régiment dont il était mestre de camp l'avait été. Il vendit ce régiment
au marquis de Sandricourt, c'est-à-dire à moi pour lui, qui en faisais
comme de mon fils, et le marché se fit d'une manière si noble et si
aisée de sa part que j'en fus singulièrement content, à propos des
hoquets qu'il fallut essuyer du père de Sandricourt. Je suppliai le
régent, avec instance, de remettre Yolet dans le service, en lui rendant
son ancienneté, et de le faire maréchal de camp. Je l'obtins avec une
joie extrême. Yolet était venu faire un tour à Paris pour ses affaires,
bien éloigné de plus penser à rien sur le service, depuis qu'il avait
quitté. Je le sus à Paris, parce qu'il passa chez moi sans me trouver,
depuis son affaire faite, comme j'allais lui écrire. Je le fis chercher,
je lui dis qu'il était maréchal de camp, je le présentai à M. le duc
d'Orléans. Je ne vis jamais homme si surpris ni si aise. On cria fort de
cet avancement, parce qu'il faut toujours crier de tout\,; mais tant
d'autres qui avaient quitté sont rentrés avec conservation de leur
ancienneté, Fervaques par exemple, et le beau cordon bleu dont cette
grâce a été depuis le prétexte, que je ne troublai pas ma joie de
l'envie des jaloux. Le pauvre Yolet n'en eut que le plaisir, j'avais
parole qu'il servirait quand il y aurait guerre\,; je le lui avais dit,
il en pétillait, et sûrement il s'y serait fort avancé. Il mourut avant
d'avoir vu la première campagne.

Le comte d'Évreux, qui n'avait de commun avec son grand-oncle, M. de
Turenne, que d'être l'homme du monde le moins simple en affectant de le
paraître le plus, et qui, avec un esprit au-dessous du médiocre, avait
le plus d'art, de manèges sous terre et d'application vers ses buts,
comme M. de Turenne aussi, le plus attentif au rang qu'ils avaient
conquis, et le plus touché d'usurper de plus en plus, était ravi de voir
l'étrange fermentation contre les dignités du royaume et les officiers
de la couronne, de ce qui s'appelait si faussement la noblesse par le
dépit de n'être pas ce qu'ils pouvaient devenir comme ceux qui y étaient
parvenus, tandis que cet aveuglement ne leur permettait pas de
s'indisposer contre des nouveautés infiniment offensantes, puisque le
rang de prince étranger ne porte que sur la différence de la naissance,
et que ces messieurs ne trouvaient point mauvais parce qu'ils n'étaient
pas nés de maisons souveraines, et ce qui est encore plus rare, parce
qu'ils ne pouvaient espérer les mêmes conjonctures, qui avaient fait
princes étrangers des gentilshommes comme eux, tels que, depuis si peu
d'années, les Bouillon et les Rohan. Le comte d'Évreux, sans cesse
appliqué à accroître ses avantages, essaya de profiter de la
conjoncture\,; il exerçait quelques parties de sa charge de colonel
général de la cavalerie, et avait par là occasion d'écrire aux mestres
de camp. Il hasarda un style qui leur déplut, et qui lui attira des
réponses toutes pareilles, avec des propos publics qui firent grand
bruit. Il ne fut pas à se repentir de sa tentative\,; il couvrit le
prétendu prince du colonel général, et prétendit que la supériorité de
sa charge lui donnait le droit de la conserver dans sa manière d'écrire
aux mestres de camp. M. le duc d'Orléans qui craignait bien moins ce qui
n'avait point de fondement, et ce qui se pouvait détruire comme ces
rangs de princes étrangers, encore moins ceux qui n'en avaient que le
rang sans en avoir la naissance comme les Bouillon, les Rohan, que les
dignités de l'État et les offices de la couronne, dont les racines
sortent de celles de la monarchie même, et qui sont de sa même
antiquité, eut recours à ses chers \emph{mezzo-termine}, où il trouva
moyen que le comte d'Évreux ne perdît pas tout ce qu'il aurait dû
laisser du sien dans cette belle entreprise.

Le régent accorda à la duchesse de Portsmouth huit mille livres
d'augmentation de pension à douze mille livres qu'elle en avait déjà\,:
elle était fort vieille, très convertie et pénitente, très mal dans ses
affaires, réduite à vivre dans sa campagne. Il était juste et de bon
exemple de se souvenir des services importants et continuels qu'elle
avait rendus de très bonne grâce à la France, du temps qu'elle était en
Angleterre, maîtresse très puissante de Charles II.

M. le duc d'Orléans fit une autre grâce, et fort grande, à M. le prince
de Conti, qui n'eut pas les mêmes raisons. Il augmenta ses pensions de
trente mille livres pour qu'il en eût une de cent mille livres comme M.
le Duc, et peu de jours après au même prince de Conti, quarante-cinq
mille livres d'augmentation d'appointements du gouvernement de Poitou,
qui lui en valait trente-six mille, qui firent en tout quatre-vingt-un
mille livres, et cent quatre-vingt-un mille livres avec la pension\,; en
sorte que ce fut en quinze jours un présent de soixante-quinze mille
livres de rente. Ces débordements furent encore un fruit des bâtards. Le
premier prince du sang, comme tel, n'a jamais eu plus de soixante mille
livres de pension. Celles des autres princes et princesses du sang,
quand ils en ont eu, n'en ont jamais approché. Les bâtards et bâtardes,
gorgés de tout, laissèrent longtemps les princes du sang à sec.~M. le
Prince avec M\textsuperscript{me} la Princesse avaient un million huit
cent mille livres de rente, en comptant son gouvernement de Bourgogne et
sa charge de grand maître de France. M. son fils avait eu les deux
survivances en épousant M\textsuperscript{me} la Duchesse, et des
pensions, lui et elle en bâtards, dont elle lui communiqua la profusion
et à leurs enfants peu à peu. Il n'y avait que M. le prince de Conti de
prince du sang, qui n'eût que sa naissance, son mérite, sa réputation,
l'amour, l'estime, et la plainte de tout le monde. Quelque dépit que le
roi en eût, qui ne lui avait jamais pardonné le voyage de Hongrie, et
peut-être moins sa réputation et l'attachement public, par jalousie pour
le duc du Maine qui n'eut jamais rien moins, ce contraste à la fin ne
put se soutenir, et il fallut lui donner des pensions et à son fils\,:
de là, titre envers le régent, qui leur laissa tout aller, et qui n'eut
pas la force de défendre les finances de leurs infatigables assauts.

D'Antin, qui avait perdu son fils aîné, comme on l'a vu, dans le temps
de la mort de M. le Dauphin et de M\textsuperscript{me} la Dauphine, qui
avait laissé deux fils, obtint enfin pour l'aîné la survivance de son
gouvernement d'Orléanais, etc., et pour le second celle de sa
lieutenance générale d'Alsace. Il avait déjà depuis quelque temps celle
des bâtiments pour Bellegarde, son second fils, qui l'exerçait sous lui.

Silly, dont j'aurai lieu de parler dans la suite plus à propos qu'ici,
obtint d'être mis dans le conseil des affaires du dedans du royaume.

Le marquis de Prié, commandant général des Pays-Bas, excita une grande
sédition à Bruxelles qui dura plusieurs mois et à violentes reprises. La
cour de Vienne avait fait mettre un impôt extraordinaire sur les corps
des métiers par le conseil de finances de Bruxelles. Cet impôt fut
refusé avec grande rumeur. On persista à Vienne à ne vouloir point
écouter les représentations qui y furent envoyées par les taxés. Ils
continuèrent, ce nonobstant, à refuser de payer. Prié leur parla fort
hautement, puis les menaça, et s'attira par sa hauteur des réponses qui
l'engagèrent à des procédés militaires, qui excitèrent la sédition. Elle
ne fut enfin apaisée que parce que Prié n'aurait pu venir à bout d'eux
que par des remèdes pires que le mal, et que la cour de Vienne, tout
impérieuse et inflexible qu'elle soit, n'osa les pousser à bout. La taxe
fut abandonnée, et personne ne fut châtié. C'était le même Prié qu'on a
vu ici en son temps ambassadeur de l'empereur à Rome, lorsque le
maréchal de Tessé y était de la part du roi, et qu'il en fit partir peu
décemment, parce qu'il força le pape, par les exécutions militaires des
troupes impériales dans l'État ecclésiastique, de reconnaître l'archiduc
comme roi d'Espagne.

Il est temps de passer aux affaires étrangères, et de remonter pour cela
au commencement de cette année\,; mais il est à propos d'avertir, avant
cette transition, que beaucoup de petites choses, qui viennent d'être
racontées, sont un peu postérieures à d'autres plus importantes, dont la
nature et la chaîne demandent de n'être pas séparées des événements qui
les ont suivies. C'est ce qui les a fait laisser en arrière pour les
exposer sans interruption des moindres choses qui viennent d'être
narrées, et qui les fait remettre après le récit de ce qui s'est passé
sur les affaires étrangères dans les premiers six mois de cette année.

\hypertarget{chapitre-xiv.}{%
\chapter{CHAPITRE XIV.}\label{chapitre-xiv.}}

1718

~

{\textsc{État de la négociation à Londres pour traiter la paix entre
l'empereur et le roi d'Espagne.}} {\textsc{- Deux difficultés
principales.}} {\textsc{- Staremberg le plus opposé à la cession future
de la Toscane.}} {\textsc{- Propositions des Impériaux pleines de
jalousie et de haine.}} {\textsc{- Plaintes artificieuses des Impériaux
du régent.}} {\textsc{- Point de la tranquillité de l'Italie pendant la
négociation.}} {\textsc{- Partialité ouverte des Anglais pour
l'empereur.}} {\textsc{- Leurs hauteurs et leurs menaces au régent.}}
{\textsc{- Le roi d'Angleterre, inquiet sur le nord, s'assure du czar\,;
méprise le roi de Prusse.}} {\textsc{- La czarine veut s'assurer de la
Suède pour la transmission de la succession de Russie à son fils.}}
{\textsc{- Agitations et reproches du czar sur cette affaire.}}
{\textsc{- Le régent pressé par l'Angleterre.}} {\textsc{- L'Espagne ne
pense qu'à se préparer à la guerre\,; déclare à l'Angleterre qu'elle
regardera comme infraction tout envoi d'escadre anglaise dans la
Méditerranée.}} {\textsc{- Albéroni ennemi de la paix.}} {\textsc{- Ses
efforts\,; ses manèges\,; sa politique.}} {\textsc{- Il veut gagner le
régent et le roi de Sicile.}} {\textsc{- Forte conversation d'Albéroni
avec le ministre d'Angleterre.}} {\textsc{- Plaintes et chimères
d'Albéroni.}} {\textsc{- Il écrit au régent avec hardiesse.}} {\textsc{-
Inquiétude sur Nancré.}} {\textsc{- Albéroni espère du régent, pressé
par Cellamare et Provane, d'augmenter l'infanterie et d'envoyer un
ministre à Vienne.}} {\textsc{- Le régent élude enfin leurs demandes.}}
{\textsc{- Reproches de Cellamare à la France\,; sort peu content d'une
audience du régent.}} {\textsc{- Cellamare, pour vouloir trop pénétrer
et approfondir, se trompe grossièrement sur les causes de la conduite du
régent.}}

~

La paix à faire entre l'empereur et le roi d'Espagne était toujours sur
lé tapis et l'objet de l'attention de toute l'Europe. Penterrieder pour
l'empereur, et l'abbé Dubois pour la France, la négociaient à Londres
avec les ministres du roi d'Angleterre. La Hollande paraissait s'en
rapporter à ce monarque, sans charger de rien à cet égard le ministre
que la république tenait à Londres. Le Pensionnaire, dévoué en toute
dépendance à ce prince, apprenait de lui-même ses volontés, lorsqu'il
voulait faire entrer cette république dans les engagements qu'il voulait
prendre de concert avec elle. Monteléon, ambassadeur d'Espagne à
Londres, très habile et fort expérimenté, aurait été plus capable que
personne de servir utilement son maître, si ce prince eût voulu traiter
sur le plan qui lui était proposé. Monteléon croyait que la paix
convenait à l'Espagne, mais il craignait de dire franchement son avis,
persuadé qu'Albéroni ne pensait pas comme lui, et que ce serait se
perdre inutilement que de combattre son sentiment et peut-être son
intérêt. Il se contenta donc pendant quelque temps de combattre
l'espérance que ce tout puissant ministre avait prise de voir bientôt
des troubles en Angleterre, en lui démontrant que la désunion du roi
d'Angleterre et du prince de Galles ne causerait aucun mouvement dans le
royaume, qu'il n'y avait aucun fondement à faire sur les mesures et
l'impuissance des mécontents du gouvernement, et que le roi d'Angleterre
trouverait dans la suite des séances de son parlement la même soumission
à ses volontés qu'il avait éprouvée à leur ouverture. Cet ambassadeur ne
se rebuta point d'assurer le roi d'Espagne que les intentions du régent
à son égard étaient bonnes, que l'abbé Dubois lui avait répété plusieurs
fois que les instructions qu'il attendait formeraient une union et une
intelligence parfaites entre Sa Majesté Catholique et Son Altesse
Royale\,; et il représenta, sous le nom de cet abbé, que, si le roi
d'Espagne différait à s'expliquer, le ministre de l'empereur gagnerait
du terrain à Londres\,; et il était vrai que les ministres les plus
confidents du roi d'Angleterre étaient tous à l'empereur, et traitaient
de prétentions injustes les propositions que le régent faisait et
appuyait en faveur de l'Espagne.

Les principales difficultés roulèrent sur deux points, tous deux
essentiels, que le régent demandait\,: le premier une renonciation
absolue et perpétuelle de la part de l'empereur à tous les États de la
monarchie d'Espagne actuellement possédés par Philippe V\,; le second
que, les maisons de Médicis et Farnèse venant à s'éteindre, la
succession aux États de Toscane et de Parme fût assurée au fils aîné de
la reine d'Espagne, et successivement à ses enfants mâles, cette
princesse étant héritière légitime des deux maisons.

Les Impériaux se plaignirent de ce que le régent était plus attentif à
procurer les avantages du roi d'Espagne que ce prince n'était à les
demander. Ils dirent qu'il était injuste d'exiger une renonciation
absolue de l'empereur à ses droits sur la monarchie d'Espagne, pendant
qu'on ne lui en offrait pas une pareille du roi d'Espagne aux États
d'Italie et des Pays-Bas possédés par Sa Majesté Impériale, regardant
comme une sorte de violence de faire subsister les droits d'une partie
pendant qu'on éteignait avec tant de soin ceux de l'autre partie.

Ils s'écrièrent encore plus sur les successions de Toscane et de Parme,
comme s'il s'agissait de porter la guerre en Italie, et de la faire
perdre à l'empereur, par la facilité de débarquer les troupes d'Espagne
à Livourne, d'entrer sans peine en Lombardie, tandis que les Impériaux
arrêtés par les Apennins ne pourraient pénétrer en Toscane, pour
empêcher les Espagnols de s'y fortifier et de s'y faciliter les secours
d'Espagne. Ils cédèrent néanmoins, sur l'article de Parme et de
Plaisance, parce que ses États éloignés de la mer ne pourraient recevoir
de secours étrangers, et dépendraient toujours de l'empereur, enclavés
comme ils sont dans les terres, si le prince qui les posséderait tentait
de s'agrandir. Mais la Toscane, surtout Livourne, entre les mains d'un
prince de la maison de France, leur paraissait d'un péril continuel et
inévitable à chasser l'empereur d'Italie toutes les fois que la France
et l'Espagne le voudraient.

Le comte de Staremberg, qui avait acquis la plus grande confiance de
l'empereur, pour avoir été son conseil et le général sous lui en
Espagne, était le plus touché de cette crainte de tous les ministres de
la cour de Vienne. Il dit qu'il se croyait en droit plus que personne
d'insister fortement au refus de l'article de la Toscane, parce qu'il
avait appuyé plus fortement que personne le projet de prendre de justes
mesures pour assurer le repos de l'Europe, et qu'il s'était souvent
exposé à déplaire à l'empereur en combattant les visions dont on
entretenait sa passion de recouvrer la monarchie d'Espagne\,; que cet
article de Toscane, au lieu d'établir une paix solide, entretiendrait
une cause de guerre perpétuelle, et ferait perdre l'Italie à
l'empereur\,; qu'il lui conseillerait plutôt que d'y consentir, de faire
la paix avec les Turcs aux dépens même de toutes ses conquêtes sur eux,
et de regarder comme sa plus capitale affaire d'empêcher l'établissement
en Italie d'une branche de la maison de France, et qu'elle y prît des
racines assez solides pour donner la loi à la maison d'Autriche\,; et il
n'estimait pas que l'acquisition de la Sicile pût balancer la crainte
d'un pareil établissement. Il convenait aussi que l'Europe aurait raison
de s'alarmer si l'empereur prétendait s'emparer quelque jour de ces
successions\,; qu'aussi son intention était d'en assurer l'expectative
au duc de Lorraine (que Vienne voulait faire regarder comme un prince
neutre, quoique de tout temps et lors plus que jamais seule et même
chose avec elle) et dont l'agrandissement ne devait donner d'ombrage à
aucune puissance. L'empereur, voulait bien qu'il achetât ce bel
établissement par la cession du Barrois, mouvant à la France\footnote{Terme
  féodal qui signifie \emph{relevant de la France}.}. Néanmoins, les
ministres de l'empereur, n'espérant pas qu'on pût se relâcher sur la
Toscane en faveur d'un fils de la reine d'Espagne, imaginèrent de la
partager avec lui en faisant céder l'État de Pise au duc de Lorraine.
Leur grand objet était que le prince d'Espagne n'eût point de ports de
mer, et ils prétendaient y intéresser les Anglais par la jalousie du
commerce du Levant. Ils renouvelèrent aussi les instances qu'ils avaient
inutilement faites aux traités de Rastadt et de Bade, pour la
restitution des privilèges de l'Aragon et de la Catalogne, et celle des
biens confisqués sur les Espagnols qui avaient suivi le parti de
l'empereur. Outre l'honneur de ce prince, ils étaient persuadés que la
suppression des privilèges de ces deux provinces augmentait de quatre ou
cinq millions le revenu du roi d'Espagne, à qui ils les voulaient faire
perdre par ce rétablissement. À l'égard des biens confisqués, l'empereur
s'ennuyait de payer libéralement ces rebelles sur ses revenus d'Italie.
Ses ministres, qui les haïssaient, se plaignaient aigrement sur cet
article des instances trop opiniâtres, disaient-ils, du régent, pour les
avantages du roi d'Espagne.

La cour de Vienne, accoutumée à reprocher à ceux avec qui elle traite,
le peu de bonne foi dont elle-même ne sait que trop s'aider, la
reprochait à ce prince dans cette négociation de Londres. Elle
prétendait que Bonnac avait tâché par ses démarches et ses discours
d'engager les principaux officiers ottomans de continuer la guerre
contre l'empereur\,; que le régent avait envoyé Ragotzi en Turquie\,;
que Son Altesse Royale n'avait rien oublié pour engager le roi de Prusse
à faire un traité avec la France, et en conséquence la guerre à
l'empereur, quoique ce traité fût très innocent. Ils accusaient le
régent d'avoir communiqué à l'Espagne le plan du traité dressé avec le
roi d'Angleterre à Hanovre, et d'être, sinon le promoteur, au moins la
cause indirecte de l'entreprise de Sardaigne. Ces mêmes ministres de
l'empereur lui faisaient un crime de fortifier de garnisons les places
du royaume frontières de l'empire, tandis qu'en amusant Kœnigseck de
belles paroles il s'était fait l'agent du roi d'Espagne, mais bien plus
habile que lui pour en soutenir les intérêts. Leur conclusion était que
l'acquisition de la Sicile ne les mettait pas suffisamment en sûreté\,;
qu'ils n'en pouvaient avoir qu'en maintenant un assez gros corps de
troupes en Italie, pour empêcher la maison de France d'y mettre jamais
le pied, encore moins de s'y établir en aucune des parties maritimes.

Comme un des points principaux de la négociation était d'assurer, au
moins pendant sa durée, le repos de l'Italie, le roi d'Espagne avait
demandé que l'empereur promit de n'y point commettre d'hostilité, de n'y
lever aucunes contributions, et de n'y point faire passer de troupes
pendant le cours de la négociation. L'empereur parut assez disposé aux
deux premières demandes\,; pour la troisième, il prétendit que ce serait
abandonner l'Italie à un ennemi qui l'avait attaqué, tandis qu'il était
occupé contre les Turcs en Hongrie, qui lui avait enlevé la Sardaigne\,;
qu'il en demandait la restitution si l'Espagne voulait un engagement
formel de sa part de n'envoyer point de troupes en Italie. Ses
ministres, persuadés que le régent traitait secrètement, et ne songeait
qu'à s'unir avec l'Espagne, déclarèrent que leur maître ferait la paix
avec le Turc à quelques conditions que ce pût être.

La cour de Londres pressait la négociation. Elle représentait au régent
qu'elle était dans sa crise\,; qu'il ne tenait qu'à lui de la finir par
une bonne résolution qui le mettrait pour toujours en sûreté, et le
délivrerait de la tutelle insupportable d'une cabale espagnole très
puissante en France, et totalement occupée à sa ruine. Les ministres
hanovriens soutenaient comme excellent le projet de donner l'État de
Pise avec Livourne et Portolongone au duc de Lorraine, en cédant par lui
à la France le Barrois mouvant. Ils ne se rebutèrent point du refus.
Voyant enfin qu'ils ne réussiraient pas, ils firent un dernier effort
sans espérance, mais pour se justifier auprès de l'empereur et le
persuader qu'il n'avait pas tenu à leurs soins d'emporter un point qui
lui était si capital, qui était le moins, ajoutèrent-ils, qu'ils pussent
faire pour Sa Majesté Impériale. Avec une telle partialité on ne devait
pas se flatter que l'Angleterre acceptât la proposition que le régent
lui fit alors de s'unir à lui et à l'Espagne, pour forcer les
oppositions de l'empereur, et d'accepter enfin le projet du traité tel
qu'il était proposé. Aussi les ministres hanovriens dirent-ils nettement
que, si la proposition était sérieuse, il ne restait que de rompre toute
négociation\,; et se défiant toujours des intentions secrètes du régent,
ils déclarèrent que le roi leur maître faisait dresser un plan du traité
tel qu'il prétendait qu'il fût signé\,; que l'article de la renonciation
de l'empereur et celui de la succession de la Toscane y seraient compris
de la manière que Son Altesse Royale le désirait\,; qu'on y comprendrait
aussi les engagements qu'elle devait prendre pour assurer la Sicile à
l'empereur\,; qu'on la prierait de signer ce plan, qu'il serait ensuite
envoyé à Vienne pour le faire signer à l'empereur\,; qu'enfin, si le
régent refusait sa signature, le roi d'Angleterre saurait à quoi s'en
tenir, et prendrait d'autres mesures. Ces menaces furent faites à l'abbé
Dubois à Londres, en même temps que Stairs eut ordre d'expliquer à
Paris, en même sens, les intentions du roi d'Angleterre.

Ce prince avait eu de grandes inquiétudes des négociations du czar avec
la Suède, de ses attentions pour le roi de Prusse, de ses préparatifs
par mer et par terre qu'on croyait destinés contre les Turcs\,; et il
craignait que, très mal satisfait de lui depuis longtemps, il ne méditât
quelque vengeance. Il fut enfin rassuré par la promesse qu'il en tira de
fermer tout accès auprès de lui aux Anglais rebelles, et d'interdire
l'entrée de Pétersbourg au duc d'Ormont, s'il s'y voulait réfugier.
Georges crut savoir avec certitude que les négociations avec la Suède
n'étaient fondées que sur les instances de la czarine, pour engager le
czar d'écouter le baron de Gœrtz, par sa passion dominante d'assurer la
succession au trône de Russie à son fils, au préjudice de son frère aîné
du premier mariage. Elle avait pris des mesures auprès du roi de Suède,
et engagé le czar à lui restituer une partie de ses conquêtes, moyennant
quoi le roi de Suède devait garantir ce nouvel ordre de succession.

Le czar, naturellement opposé à restituer, parut sentir les remords du
renversement de l'ordre naturel et légal de la succession, surtout quand
il vit la joie de ses peuples au retour d'Italie du czarowitz, qui lui
fit craindre même une révolution s'il poussait ce projet en faveur de
son jeune fils. Il était tombé dans un chagrin extrême. Il reprochait à
la czarine les embarras où le jetait son ambition pour son fils, et les
peines que lui coûtait cette malheureuse affaire. Il se plaignait de ses
sollicitations de faire sa paix particulière avec la Suède\,; il
craignait la puissance et la vengeance de ses alliés dans cette guerre
s'il les abandonnait. Il traitait de scélérat Menzicoff jusqu'alors son
favori, avec qui la czarine était fort liée. Il en disait autant de
Goertz qui avait traité avec lui de la part de la Suède, et le tenait
capable de tromper et lui et son propre maître. Le roi d'Angleterre,
informé de ces agitations du czar, ne le croyait pas en état de prendre
des liaisons avec la Suède au préjudice de la ligue du nord, à laquelle
l'impuissance plus que la volonté l'obligerait de demeurer fidèle\,; la
bonne foi du roi de Prusse lui était également suspecte\,; mais ses
ministres le regardaient comme un zéro (c'était leur expression),
capable de rien sans l'appui du czar, ni d'oser déplaire à l'empereur
sans des sûretés bien réelles. Ils espéraient tout de la témérité du roi
de Suède à la veille de périr dans chacune de ses entreprises. Son
entrée en Norvège, à la fin de janvier, leur parut aussi folle qu'elle
l'avait semblé à ses ministres et à ses généraux qui s'y étaient tous
inutilement opposés, et Gœrtz plus qu'aucun, dans la vue d'intérêt
particulier qu'il avait de porter le roi de Suède vers le Holstein, pour
rétablir son neveu dans cet État usurpé par le roi de Danemark. Le
ministère anglais, uni à celui de Hanovre, se fondait sur ces
dispositions des affaires du nord, pour montrer au régent qu'il se
flatterait en vain d'y former une ligue capable de tenir tête à
l'empereur\,; qu'il n'y avait d'alliance assurée pour Son Altesse Royale
que celle dont il s'agissait actuellement\,; qu'elle devait donc en
aplanir les difficultés\,; et que l'article de la Toscane n'en était pas
une assez importante pour retarder une conclusion si essentielle à la
France, et si nécessaire à l'Europe.

Le roi d'Espagne, loin de souscrire an projet dont il s'agissait pour,
la paix, ne songeait qu'à se préparer à la guerre. Il déclarait qu'il
voulait conserver la bonne intelligence avec l'Angleterre\,; mais il lui
fit en même temps déclarer par son ambassadeur que, si elle envoyait
quelque escadre dans la Méditerranée, il regarderait cette expédition
comme faite contre ses intérêts, et non pour se venger du pape d'avoir
fait arrêter le comte de Peterborough. Enfin, Sa Majesté Catholique
exigeait du roi d'Angleterre une déclaration générale à l'égard de toute
escadre anglaise qui pourrait être employée dans la Méditerranée. Il
semblait qu'Albéroni, en faisant demander toutes ces sûretés, cherchait
un prétexte de déclarer la guerre. Il faisait, avec empressement, tous
les préparatifs nécessaires pour la commencer, cherchait chez l'étranger
ce que l'Espagne ne lui pouvait fournir pour se défendre et pour
attaquer, et regardait tout autre soin comme inutile. Néanmoins, malgré
les assurances de Beretti, il ne put tirer aucuns vaisseaux des
Hollandais. Il menaçait en même temps les Anglais et les Hollandais de
la ruine de leur commerce, s'ils donnaient le moindre sujet de plainte à
l'Espagne par leurs liaisons avec l'empereur. Il était si persuadé de
l'effet de ces menaces qu'il regardait la négociation de Londres comme
un vain amusement, et que, lorsqu'il apprit l'envoi de Nancré, il dit
qu'il y serait le bienvenu, mais qu'il s'ennuierait bientôt à Madrid, et
souhaiterait retourner promptement à Paris, comme il était arrivé à
Monti. À l'égard du public, à qui il fallait un leurre, il fondait
l'éloignement du roi d'Espagne pour la négociation commencée sur la
connaissance qu'il avait des mauvais desseins et de la mauvaise foi des
Allemands par la conduite tyrannique qu'ils avaient en Italie, qu'il
détaillait, et parce qu'ils bloquaient actuellement les États de Parme
et de Plaisance. En même temps, il exhortait le duc de Parme de souffrir
ces vexations, de ne point augmenter la garnison de Parme, quoique
l'Espagne en voulût bien faire la dépense\,; qu'il ne convenait point à
un petit prince d'irriter l'empereur, main d'attendre que l'oppression
de tous les princes d'Italie les obligeât d'implorer unanimement le
secours du roi d'Espagne pour les affranchir de la tyrannie de
l'empereur. Albéroni, sans nommer personne, espérait gagner incessamment
le roi de Sicile. Il fit dire au régent que, s'il voulait s'unir au roi
d'Espagne\,; le roi de Sicile entrerait sur-le-champ dans la même
union\,; qu'elle suffirait pour forcer les Allemands à sortir
d'Italie\,; que les Hollandais verraient cet événement avec plaisir et
tranquillité, mais qu'ils auraient souhaité, à ce qu'il prétendait
savoir, qu'immédiatement après la conquête de la Sardaigne, le roi
d'Espagne eût fait marcher ses troupes à celle du royaume de Naples.

Ce cardinal n'oublia rien pour piquer les médiateurs du point d'honneur.
Il leur disait que la conduite de l'empereur était pour eux le dernier
mépris, puisque leur seule considération y avait suspendu le progrès des
armes d'Espagne, qui sans cela auraient été en état de s'opposer avec
plus de vigueur à son ambition\,; que la reconnaissance qu'il en
témoignait à la France et à l'Angleterre était la continuation des mêmes
violences, sans nul égard aux offices et à l'honneur de ces deux
couronnes\,; qu'il était étonné que, malgré ce peu d'égards de
l'empereur, le ministre d'Angleterre à Madrid lui avait fait des
propositions, encore nouvellement, en faveur de l'empereur, et lui avait
dit depuis deux jours que, si la médiation du roi, son maître était
acceptée, il ferait en sorte d'engager l'empereur à renoncer à l'Espagne
aussi bien qu'à la succession de Toscane. Sur quoi il avait répondu
qu'un médiateur serait inutile lorsqu'il ne s'agirait que de telles
conditions\,; que le roi d'Espagne ne craignait point d'être attaqué
dans le continent de son royaume\,; que, quant à la succession de
Toscane, il la regardait comme un futur contingent, persuadé que,
suivant les conjonctures, toute garantie pou voit devenir inutile, dont
il citait pour exemple l'effet des garanties promises pour la Catalogne
et pour Majorque. L'Anglais défendit son maître par ses engagements pris
avec l'empereur. Le cardinal répondit qu'il était malheureux qu'il se
souvînt si bien de ses engagements avec l'empereur, et qu'il eût sitôt
et si aisément oublié tant de services essentiels et de preuves d'amitié
qu'il avait reçues du roi d'Espagne, dont il avait promis une
reconnaissance éternelle. Il ajouta que la nation anglaise trouverait
peut-être quelque peine à soutenir des engagements pris contre un prince
dont elle recevait continuellement tant d'avantages considérables pour
son commerce, et pris en faveur d'un autre dont elle ne pouvait que
recevoir beaucoup de préjudices. Alors le ministre anglais, oubliant un
peu ses ordres et son caractère, répondit, suivant le génie de sa
nation, que tout bon Anglais connaissait assez la force des engagements
pris avec l'empereur, qui au fond étaient considérés comme s'ils
n'existaient pas. Son but néanmoins fut toujours de persuader que rien
n'était plus capable d'assurer le repos public que de traiter suivant le
plan proposé, et de conclure une paix dont l'exécution serait garantie
par les principales puissances de l'Europe. Albéroni protestait des
désirs sincères du roi d'Espagne pour une solide paix\,; qu'il ne
faisait point la guerre pour agrandir ses États, mais pour se venger des
insultes des Allemands, et pour affranchir le monde, particulièrement
l'Italie, de leurs violences\,; que d'en chasser les Allemands, et de
rendre leurs usurpations à la couronne d'Espagne, aurait à la vérité été
le moyen d'assurer le repos de l'Italie et l'équilibre de l'Europe\,;
mais que Sa Majesté Catholique, occupée seulement du bien public, était
prête d'acquiescer à tout autre expédient qu'on trouverait utile et
conduisant également au but qu'elle se proposait.

Albéroni s'élevait souvent contre la léthargie des puissances de
l'Europe. Il condamnait l'ignorance crasse, disait-il, de ceux qui
croyaient une guerre universelle nécessaire pour mettre l'empereur à la
raison. Il formait un projet facile selon lui pour parvenir à ce but. Il
demandait seulement que la France fournît quarante mille hommes, et
s'unît aux rois d'Espagne et de Sicile pour s'opposer de concert aux
entreprises des Allemands. Il assurait que, cette union faite, aucune
autre puissance n'aiderait l'empereur\,; que les Hollandais
demeureraient spectateurs\,; que les Anglais, retenus par l'intérêt du
commerce, n'oseraient, pour complaire à leur roi, fournir à l'empereur
les secours qu'il lui avait promis. Dans cette confiance, il protestait
que rien ne l'empêcherait de suivre son chemin. Il avouait qu'il se
flatterait d'un succès certain si la France entrait dans les projets
qu'il méditait. Il écrivait au régent qu'il ne pouvait trouver d'intérêt
ni de bonheur solide que dans une union avec le roi d'Espagne, la seule
que l'honneur et la probité lui indiquaient\,; que tout autre engagement
serait au contraire accompagné de déshonneur et d'opprobre. Il soutenait
que l'un et l'autre se trouvaient dans ce qui se proposait à Londres\,;
que les garanties des successions de Parme et de Toscane, dont les
souverains et un successeur de chacun étaient pleins de vie, étaient des
sûretés imaginaires\,; qu'il serait nécessaire, avant d'entrer en
négociation, de proposer des moyens plus solides d'empêcher ces États de
tomber entre les mains de l'empereur lorsque ces successions viendraient
à s'ouvrir.

Le bruit du prochain envoi de Nancré à Madrid s'y étant répandu, les
ministres étrangers qui y résidaient en prirent de l'inquiétude, et
interrogèrent Albéroni sur les dispositions qu'ils crurent voir à
quelque nouveau traité. Il répondit qu'il était vrai que Cellamare
l'avait averti du voyage que Nancré se disposait à faire, mais que le
motif en était inconnu à l'ambassadeur et à lui-même, que le temps
l'éclaircirait, et qu'il protestait cependant non comme ministre, mais
comme homme d'honneur, qu'il n'en avait pas la moindre connaissance.
L'empressement des dispositions qu'il faisait pour la guerre, et qui
coûtaient beaucoup, répondait à son éloignement de la paix. On y
remarqua néanmoins un ralentissement, qui fut attribué aux scrupules du
roi d'Espagne et aux représentations de son confesseur. Mais Aubenton,
dont Albéroni était bien sûr, n'aurait osé proposer au roi d'Espagne
d'autres points de conscience que ceux qui convenaient aux intérêts du
cardinal. Lui-même attendait peut-être quelques changements aux projets
dont il était question. Cellamare et le comte de Provane, envoyé du roi
de Sicile à Paris, ne cessaient de détourner le régent des mesures qu'il
voulait prendre avec l'empereur et l'Angleterre, et de le presser d'en
prendre d'autres, qu'ils représentaient comme plus honorables et plus
sûres pour s'opposer aux desseins de l'empereur. Ils prétendirent que le
régent, acquiesçant à leurs raisons, leur avait promis deux choses\,:
l'une d'augmenter incessamment l'infanterie française, l'autre d'envoyer
à Vienne de la part du roi\,; mais ils n'eurent pas longtemps cette
espérance, qui les avait fort flattés, du peu d'effet qu'aurait la
négociation d'Angleterre. Il ne fut pas question de l'augmentation de
l'infanterie. Cellamare crut avoir pénétré que les ministres des
finances et même le maréchal de Villars avaient représenté la facilité
de la faire du jour au lendemain, dès que cela serait nécessaire, et
l'inconvénient de charger de ce surcroît les finances si chargées de
dettes avant la nécessité. Sur ce fondement, il fut répondu à Cellamare
que les forces impériales qui étaient en Italie n'étaient pas à
craindre, et qu'elles ne passaient pas vingt mille hommes, suivant les
traités. Sur l'envoi à Vienne on lui dit qu'il s'y était trouvé deux
difficultés\,: la répugnance invincible de Biron qui avait été choisi,
dont l'ambassadeur fut bien aise, parce que Biron était beau-père de
Bonneval, et qu'on supposait que les ministres du roi ne jugeaient pas
convenable d'envoyer à Vienne, sans charger celui qui y irait de
propositions préliminaires pour procurer un accommodement raisonnable
entre l'empereur et l'Espagne.

Cellamare se plaignait, comme d'un reproche injuste, {[}de{]} celui que
la France faisait à l'Espagne de renouveler les hostilités et les
troubles de l'Europe. Il reprochait lui-même aux Français de se laisser
tellement frapper de la crainte de la puissance des Allemands, qu'il
semblait que ceux qui avaient part aux affaires eussent toujours devant
les yeux le fantôme formidable de la dernière ligue, qui rendait
inutiles les meilleures raisons, en sorte que la terreur des forces
ennemies persuadait bien plus que l'intérêt de l'État. Il disait que le
régent, seul capable de calmer ces frayeurs, était poussé par une force
secrète, dont la source était dans son intérêt particulier différent de
celui de l'État. Persuadé que le moyen de l'en détourner était de
l'engager à l'exécution des deux points dont on vient de parler, il en
obtint, le 13 janvier, une audience particulière, dans laquelle il
insista sur ces deux points qu'il prétendit qu'on lui avait promis, et
au plus tôt. Sur le premier le régent répondit qu'il donnerait toute son
attention à choisir un sujet capable de se bien acquitter de l'emploi de
Vienne\,; que cependant, avant de le nommer, il voulait avoir encore des
réponses de l'abbé Dubois, et savoir les intentions du roi d'Angleterre
plus précisément qu'il n'en était instruit. Sur le second, il dit à
Cellamare, mais comme en confidence intime, que, suivant l'avis de ceux
qu'il avait chargés des affaires de la guerre, même de plusieurs
officiers généraux, il avait abandonné sa première idée d'augmenter de
dix hommes chaque compagnie d'infanterie\,; que, prenant un expédient
plus conforme à l'épuisement des finances, son dessein était de former
un corps de soldats de milices de soixante mille hommes commandés par
les officiers réformés que le roi entretenait, avec quoi il comptait
pouvoir mettre aisément en campagne les cent quatre-vingts bataillons
que le roi avait à sa solde. Cellamare combattit ce projet, puis voyant
ses objections inutiles, il représenta qu'il ne suffirait pas de prendre
des précautions pour la sûreté de l'Italie, si Son Altesse Royale ne les
faisait savoir au roi de Sicile à temps, parce que, se croyant
abandonné, il était vraisemblable qu'il ferait quelque démarche, où on
ne pourrait plus remédier quand une fois l'engagement serait pris.
L'ambassadeur obtint du régent promesse d'en parler à Provane\,; mais,
peu content de son audience, il voulut remonter à la source du
changement qu'il trouvait. Il crut avoir pénétré que le maréchal de
Villars et Broglio avaient proposé l'expédient des milices dans la vue
d'empêcher une nouvelle guerre, la France n'ayant rien à craindre du
trouble que l'empereur pouvait apporter au repos de l'Italie, ni de ses
entreprises contre le roi de Sicile. Cette opinion, frondée par
Cellamare, était, disait-il, celle d'un petit nombre de gens peu
éclairés, et mal instruits des véritables intérêts de l'Europe, dont le
maréchal d'Huxelles et la partie la plus judicieuse du ministère
raisonnaient selon lui avec plus de justesse, et trouvaient que le roi
avait grand intérêt de s'opposer aux ambitieux desseins des Allemands,
quoiqu'il ne dût recourir à la force qu'après avoir tenté tous les
moyens possibles de parvenir à un accommodement raisonnable.

Je me suis toujours étonné qu'un homme d'autant d'esprit, de
perspicacité, d'application que Cellamare, et qui n'était pas
nouvellement arrivé, assez mêlé de plus dans la bonne compagnie, et qui
savait en profiter, se trompât si lourdement dans ses conjectures et
dans ce qu'il croyait avoir pénétré. Le mystère toutefois n'était pas
difficile. L'intérêt particulier ne dominait point le régent qui voulait
et allait sincèrement au bien de l'État\,; mais il l'était par l'abbé
Dubois, qui l'avait infatué de bonne heure de l'Angleterre, aidé du duc
de Noailles et de Canillac dans les commencements, qui tous trois
avaient stylé Stairs à lui parler d'un ton à lui imposer, lequel en
avait su si bien profiter qu'il en abusa sans cesse, et réduisit en
assez peu de temps le régent à le craindre, et à n'oser, pour ainsi
dire, branler devant lui, appuyé de plus en plus, et conduit par l'abbé
Dubois à mesure qu'il croissait lui-même. Dubois, qui ne se souciait ni
de l'État ni de son maître que pour sa fortune, et qui de grand matin,
comme on l'a vu, ne l'avait espérée que par l'Angleterre, la voyait par
là en grand train, et nulle espérance par ailleurs. Il avait ainsi
repris son ancien ascendant sur M. le duc d'Orléans\,; cet ascendant se
fortifiait sans cesse par le commerce d'affaires qu'il tirait tout à
soi, mais qu'il ne pouvait embler que relativement à celles
d'Angleterre. L'esprit, les raisons, le bon sens emportaient quelquefois
le régent d'un autre côté, mais pour des moments. Un propos de Stairs,
qui se faisait jour chez lui avec audace, et qui était informé à point
de l'intérieur par les valets affidés à Dubois, une dépêche de cet abbé
renversaient à l'instant les idées que le régent avaient prises, et
l'attachaient de nouveau à l'Angleterre. C'était l'unique cause du
changement que Cellamare cherchait à démêler. Le maréchal de Villars ne
fut jamais Anglais, mais toujours Espagnol. D'ailleurs, c'était l'homme
du monde que le régent consultait le moins, et qui, pour en dire le
vrai, méritait moins de l'être, par son incapacité en affaires et la
légèreté de son sens. Broglio n'était plus de rien depuis ses deux
projets dont j'ai parlé, et dont M. le duc d'Orléans se repentit
toujours. Broglio, retombé au bas étage des roués, fut encore trop
heureux d'y être souffert, et n'en remonta plus. Cette remarque suffit
pour éclaircir bien des choses sur les affaires étrangères, dont il faut
reprendre le cours.

\hypertarget{chapitre-xv.}{%
\chapter{CHAPITRE XV.}\label{chapitre-xv.}}

1718

~

{\textsc{Sage avis de Cellamare au roi d'Espagne.}} {\textsc{- Est
inquiet du prétendu mariage du prince de Piémont avec une fille du
régent, dont le régent et le roi de Sicile sont aussi éloignés l'un que
l'autre.}} {\textsc{- Erreur aveugle de Beretti.}} {\textsc{-
Proposition des Anglais sur la Toscane.}} {\textsc{- Inquiétudes
mutuelles.}} {\textsc{- Division dans la famille du roi d'Angleterre,
qui retranche quarante mille livres sterling de rente au prince de
Galles, et fait payer cent trente mille livres sterling à l'empereur,
qui est fort recherché.}} {\textsc{- Visions d'Albéroni.}} {\textsc{-
Préliminaires demandés par l'Espagne à l'empereur.}} {\textsc{- Folle
conduite d'Albéroni.}} {\textsc{- Il fait faire une déclaration
menaçante aux Hollandais pour en acheter des vaisseaux.}} {\textsc{-
Riperda rappelé\,; résolu depuis longtemps de revenir s'établir en
Espagne.}} {\textsc{- Mauvais état de la personne du roi d'Espagne.}}
{\textsc{- Pouvoir sans bornes d'Albéroni.}} {\textsc{- Aubenton et
Aldovrandi excitent l'Espagne en faveur de la constitution.}} {\textsc{-
Fortes démarches et menaces terribles de l'empereur au pape.}}
{\textsc{- Consternation de Rome.}} {\textsc{- Ses soumises et basses
résolutions.}} {\textsc{- Politique et ruse odieuse de Vienne.}}
{\textsc{- Le pape, dans sa frayeur de l'empereur, tombe pour l'apaiser
sur l'Espagne et sur Aldovrandi.}} {\textsc{- Brefs ne sont point reçus
par l'empereur ni par les rois de France et d'Espagne, sans que leurs
copies n'aient été vues par leurs ministres, qui les admettent ou les
rejettent.}} {\textsc{- Opinion générale prise du pape à l'égard de
l'Espagne.}} {\textsc{- Les Impériaux veulent qu'Aldovrandi soit rappelé
et châtié.}} {\textsc{- Faibles manèges du pape à cet égard\,; jugement
qu'ils en font porter.}}

~

Stairs et Provane dirent tous deux à Cellamare que l'empereur offrait de
s'engager à ne point inquiéter les princes d'Italie, de se contenter des
domaines qu'il y possédait, de ne pas s'opposer aux droits de la reine
d'Espagne sur les États de Parme et de Plaisance, de s'accorder avec les
médiateurs pour régler la succession de la Toscane en faveur d'un prince
qui ne fût ni de la maison d'Autriche ni de la maison de France, parce
que Naples et Milan seraient trop exposés si un des fils de la reine
d'Espagne avait la Toscane avec Parme et Plaisance. Quoique ces
dispositions ne fussent pas telles qu'il était nécessaire pour conclure,
et que Cellamare fût persuadé que l'empereur ne cherchait qu'à suspendre
les entreprises du roi d'Espagne, gagner temps et faire sa paix avec le
Turc, amuser et cependant se mettre en état d'envahir les princes
d'Italie, montrer en attendant que les difficultés ne venaient pas de sa
part, et que, si les médiateurs devaient tourner leurs armes contre
celui qui rejetterait les propositions d'un accommodement raisonnable,
ce n'était pas contre lui qu'elles se devaient employer\,; cet
ambassadeur conseillait au roi son maître de se comporter comme s'il
écoutait les propositions de la cour de Vienne, de peur qu'en les
rejetant, il lui laissât l'avantage de persuader le monde que les
Impériaux étaient véritablement dociles, et que les refus et
l'opiniâtreté venait des Espagnols. Cette maxime, bien suivie, lui
paraissait une base solide pour établir sur elle à l'avenir des
prétentions et des demandes plus essentielles. Il ajoutait que cette
conduite ne pouvait engager le roi d'Espagne au delà de ce qu'il
voudrait, parce qu'il serait toujours le maître d'éloigner la conclusion
tant qu'il voudrait, en demandant des sûretés que vraisemblablement ses
ennemis ne lui accorderaient pas\,; que, par ce refus, il ferait
retomber sur eux la haine de voir échouer une négociation regardée comme
nécessaire pour assurer la tranquillité générale\,; que si, contre son
opinion, ses ennemis consentaient aux sûretés qu'il leur demanderait, il
profiterait par là des avantages qui lui seraient accordés.

Cellamare, inquiet des bruits du mariage du prince de Piémont avec une
fille du régent, en parla à Provane, qui lui dit franchement n'en avoir
pas fait la moindre insinuation\,; que les intérêts d'État, non les
liens du sang, formaient les chemins qui unissent les princes\,; et que
les mariages se faisaient à la fin non au commencement des comédies et
des poèmes. On a vu en son lieu qui avait le premier imaginé ce mariage,
comment il fut traité quelque temps entre Plénœuf, retiré à Turin, et
moi\,; combien peu le régent y prit, et je crois aussi peu le roi de
Sicile\,; combien aussi je fus pressé de prier le régent que j'en
remisse la négociation à l'abbé Dubois, à son premier retour
d'Angleterre, et qu'il n'en fût plus question depuis. Tout ce qui
pouvait éloigner le régent des vues de l'Angleterre était odieux à
l'abbé Dubois. L'empereur était buté à ravoir la Sicile, qui était la
chose que le roi de Sicile craignait le plus. Le roi d'Angleterre,
servilement attaché à l'empereur, par rapport à ses États d'Allemagne et
à l'affermissement de son usurpation des duchés de Brême et de Verden,
aurait été au désespoir de trouver la France trop opposée à ce désir de
l'empereur, qu'il favorisait de tout son pouvoir, par conséquent d'un
mariage qui, dans son commencement surtout, eût lié le régent au roi de
Sicile par intérêt et par honneur, et qui le pouvait jeter dans une
alliance avec l'Espagne et les princes d'Italie, qui aurait renversé
toute la négociation qui se faisait à Londres. L'abbé Dubois y était un
des principaux acteurs\,; il la regardait comme la base de sa plus haute
fortune\,; il n'avait donc garde de la laisser troubler par le mariage
du prince de Piémont avec une fille de M. le duc d'Orléans.

Cellamare et Provane, de concert, ne cessaient de presser le régent de
se préparer à la guerre pour arrêter les violences des Impériaux et
leurs desseins en Italie. L'ambassadeur d'Espagne en Hollande protestait
que, si les Anglais voulaient agir en faveur de l'empereur, ils
n'auraient pour eux ni la France ni la Hollande, et que la nation
anglaise, trop intéressée pour son commerce, résisterait, en ce cas, à
Georges et à ses ministres. Saint-Saphorin, que le roi d'Angleterre
faisait négocier à Vienne, était totalement impérial. Il exagérait les
difficultés sur la Toscane comme insurmontables\,; il y était fortement
appuyé par les ministres hanovriens.

Ceux-ci firent ordonner à Stairs de presser le régent sur cet article.
Il lui proposa même de convenir que la république de Pise serait
rétablie, que Livourne lui appartiendrait, et que le fils de la reine
d'Espagne se contenterait de Florence et de la partie de la Toscane qui
avait autrefois été de la dépendance de cette ville. Ces ministres
hanovriens trouvèrent l'abbé Dubois trop aheurté sur cet article. Ils
n'oublièrent rien pour persuader le régent, tantôt par les espérances,
tantôt par les alarmes des troupes, que l'empereur enverrait
incessamment en Italie, et d'une négociation secrète entre ce prince et
le roi de Sicile. Le ministre piémontais à Londres se définit de l'abbé
Dubois, qui ne lui communiquait rien de la négociation, quoique son
maître lui eût positivement écrit que le régent voulait qu'il en fût
instruit. Monteléon, qui se loua quelque peu de temps de la conduite de
l'abbé Dubois avec lui, de ses assurances de la parfaite intelligence
qui allait régner entre le roi d'Espagne et le régent, de ses desseins
et de ses promesses de procurer dans la négociation toutes sortes
d'avantages à Sa Majesté Catholique, ne trouva bientôt plus que réserve
et mystère en ses discours. Il ne recevait aucune instruction
d'Espagne\,; ses ordres se bornaient depuis longtemps à faire connaître
à la cour d'Angleterre que le roi son maître regarderait comme une
infraction tout envoi d'une escadre anglaise dans la Méditerranée.
Stanhope l'assurait toujours que {[}les Anglais{]} ne donneraient jamais
aucune occasion de plainte ni de soupçon à l'Espagne, mais aussi que le
roi et la nation anglaise seraient obligés pour leur honneur de tirer
satisfaction de l'enlèvement du comte de Peterborough, si le pape ne la
leur donnait lui-même de cet affront qu'il leur avait fait. C'était le
voile dont ils couvraient l'armement destiné pour la Méditerranée. Ce
voile était bien clair\,; il y avait longtemps que Peterborough avait
été relâché après une détention fort courte, et que le pape épouvanté
avait fait toutes les excuses possibles.

Pendant que le roi d'Angleterre se préparait à des guerres étrangères,
la division continuait à régner dans sa famille. Nulle négociation
n'avait pu lui réconcilier le prince de Galles\,; il crut donc devoir
employer d'autres moyens pour le soumettre. Il lui fit déclarer par
Copper, chancelier d'Angleterre, le duc de Kingston et le comte de
Stanhope, que, sur les cent mille livres sterling qui lui étaient
assignées pour la dépense de sa maison, il lui en retranchait quarante
{[}mille{]}, sous prétexte de la dépense que le roi s'obligeait de faire
pour la subsistance des enfants du prince. En même temps Georges fit
passer en parleraient qu'on payerait à l'empereur cent trente mille
livres sterling pour reste des subsides de la dernière guerre, moyennant
une quittance générale de toutes ses prétentions. Ainsi la cour de
Vienne profitait de tout. Elle était sûre des ministres confidents de
Georges, hanovriens et anglais, et recherchée par le roi de Sicile qui
ne songeait qu'à apaiser sa colère, et ne croyait d'alliance solide
qu'avec elle. Il agissait en même temps à Paris et à Londres comme ne
voulant se conduire que par les médiateurs. Il se plaignait de temps en
temps du mystère qu'ils lui faisaient de l'état de la négociation.
Provane s'en plaignait encore davantage, et protestait que son maître
n'écouterait jamais aucune proposition d'échange du royaume de Sicile.
Il voulut se figurer que le régent ne serait jamais favorable à son
maître, parce que Son Altesse Royale avait lieu de croire que, le cas
arrivant, le roi de Sicile aiderait le roi d'Espagne à monter sur le
trône de France, espérant lui-même monter sur celui d'Espagne\,; et
prétendit avoir appris par la comtesse de Verue que le régent traitait
le mariage de M. son fils avec l'infante de Portugal, où on s'alarmait
des préparatifs de l'Espagne, et où l'envoyé d'Angleterre ne parlait que
de guerre et offrait des secours, si l'Espagne l'attaquait. Albéroni
calma bientôt cette inquiétude par les assurances positives qu'il y
donna, et qu'il en reçut, du désir réciproque de demeurer en bonne
intelligence. Il retira même les troupes des frontières de Portugal,
dont l'ambassadeur à Madrid offrit de la part de son maître, de réduire
à trois cent mille écus les six cent mille écus qu'il demandait depuis
longtemps à l'Espagne, si on voulait terminer les différends entre les
deux cours. Albéroni jugea à propos de faire connaître les sentiments
pacifiques de ces deux cours l'une pour l'autre en France, en
Angleterre, en Hollande\,; en prit occasion d'y faire connaître les
intentions du roi d'Espagne, et de publier la chimère qu'on a déjà vue
de ses raisonnements sur l'union de la France et de l'Espagne pour
abaisser l'empereur, la tranquille joie qu'en aurait la Hollande, et
l'inutilité des secours que Georges, démenti par l'intérêt de commerce
de la nation anglaise, voudrait donner aux Allemands, flatté de plus que
ceux du roi de Sicile, si directement opposés à l'envahissement de
l'Italie, le mettraient de son côté.

Persuadé que l'empereur était résolu de sacrifier tout à la paix avec le
Turc, pour avoir la liberté de pousser ses projets en Italie, il ordonna
à Monteléon de déclarer aux Anglais que les conditions que le roi
d'Espagne demandait comme préliminaires avant d'examiner celles de la
paix, étaient un engagement formel de la part de l'empereur sur les
articles suivants\,: 1° qu'il n'enverrait plus de troupes en Italie\,;
2° qu'il n'exigerait aucune contribution, sous quelque prétexte que ce
pût être\,; 3° qu'il promettrait de concourir de bonne foi aux mesures
qu'on jugerait nécessaires pour assurer l'équilibre de l'Italie et le
repos général de l'Europe. À ces conditions, le roi d'Espagne permit à
Monteléon d'écouter les propositions qui lui seraient faites, se
réservant à lui donner de nouveaux ordres, si par quelques changements
nouveaux Sa Majesté Catholique se croyait obligée de changer aussi de
maximes. Le cardinal ne le croyait pas. Son plan était fait\,; il le
voulait suivre, persuadé qu'il était impossible de préserver l'Italie de
sa perte totale, tant que les Allemands y conserveraient un pouce de
terre, que la conjoncture était la plus favorable, et de ses chimères
déjà expliquées sur la France, la Hollande, la nation anglaise et le roi
de Sardaigne. Il affectait une grande fermeté à suivre son projet sans
s'écarter de son point de vue, disant que le pis qu'il en pût arriver à
l'Espagne serait d'avoir à défendre son continent, qui avait des forces
suffisantes pour le défendre, et que tout l'enfer ne pouvait attaquer.
Dans cette complaisance d'avoir mis l'Espagne en si bon état, ce qu'il
regardait comme son ouvrage, il traitait de visions les conditions
offertes par les médiateurs, et s'espaçait en dérisions de toute leur
négociation. Il redoubla de chaleur pour les préparatifs\,; et,
s'apercevant enfin du peu de volonté des Hollandais de l'accommoder de
vaisseaux, il ordonna à Beretti de déclarer aux États généraux que,
s'ils y formaient quelque opposition, le roi d'Espagne la regarderait
comme une offense publique faite à sa personne, et qu'il pourrait même
en venir aux dernières extrémités. Castagneta, chef d'escadre envoyé en
Hollande avec tout l'argent comptant nécessaire pour faire ces achats,
reçut ordre en même temps de revenir diligemment à Madrid, la chose
faite ou manquée, son retour étant un point essentiel d'où dépendaient
toutes les autres négociations.

Riperda continuait de flatter le cardinal sur les bonnes dispositions de
ses maîtres en tout ce qui regardait l'Espagne\,; mais il voulait le
flatter. Les États venaient de rappeler cet ambassadeur. Il avait pris
depuis longtemps la résolution de retourner s'établir en Espagne, après
qu'il aurait rendu compte aux États de son ambassade. Il y avait même
acquis déjà quelques terres, et une maison appartenant autrefois à
l'amirante de Castille et depuis tombée dans la confiscation de ses
biens. Quoique le public doutât encore à la fin de janvier si l'Espagne,
sans la France et sans aucun allié, oserait et pourrait seule
entreprendre la guerre, le dessein d'Albéroni était d'entrer de bonne
heure en campagne. Le duc de Parme l'en pressait sans cesse comme de
chose nécessaire pour le salut de l'Italie. Mais une raison secrète
jetait l'incertitude dans ses résolutions, et le retardement à
l'exécution de ses projets. Le roi d'Espagne, bien plus malade d'esprit
que de corps, se croyait sur le point de mourir à chaque instant, et
persuadé que ses forces l'abandonnaient, il mangeait pour les réparer
avec tant d'excès que tout en était à craindre. Il se confessait tous
les soirs après son souper, et il retenait son confesseur auprès de son
lit jusqu'à ce qu'il se fût endormi. Il n'était pas permis à la reine de
le quitter un seul instant. Ce prince étant donc hors d'état d'entendre
parler d'aucune affaire, le pouvoir d'Albéroni était plus souverain que
jamais. Il réglait tout et disposait de tout au nom du roi\,; qui que ce
soit n'osait lé contredire, et il avait déclaré plusieurs fois aux
secrétaires d'État que, si quelqu'un d'eux manquait à son devoir pour
l'exécution de ses ordres, il lui en coûterait la vie.

On répandait néanmoins dans le public que la santé du roi était
parfaitement rétablie. Le P. Daubenton disait à ses amis que ce prince
avait trop de scrupules. Tout occupé qu'il était auprès de lui, il ne
laissait pas d'apporter tous ses soins à trouver en Espagne des
défenseurs à la constitution. Il y servait d'agent non seulement au
pape, mais au cardinal de Bissy. Il avait eu soin de faire tenir ses
lettres au patriarche de Lisbonne, aussi bien que de solliciter les
évêques et les chapitres d'Espagne d'écrire en faveur de la
constitution. Il aurait voulu modérer leur zèle sur l'infaillibilité du
pape, et sur la supériorité qu'ils lui attribuaient sur les conciles.
Mais cette maxime étant le principe et le fondement de leur soumission
sans réserve à la bulle, le jésuite qui l'avait faite avec Fabroni,
comme on l'a vu en son lieu, aurait en vain essayé de les empêcher,
comme il disait, de fourrer dans leurs écrits des maximes très
déplaisantes à la France. Le nonce Aldovrandi pressait de son côté les
évêques d'Espagne de faire au plus tôt une acceptation universelle,
publique et positive de la constitution. Quoique, par les raisons de
domination suprême qu'on a vues ci-devant, Rome n'eût pas approuvé les
premières instances qu'il avait faites pour la procurer, il crut qu'il
devait les continuer, même les redoubler. Elles lui parurent absolument
nécessaires pour remédier au mal qui se répandait dans l'Espagne. Le
frein du saint-office retenait encore les malintentionnés, et les
obligeait à se cacher\,; mais on avertissait le nonce qu'il n'en fallait
pas moins prendre garde aux progrès qu'ils pourraient faire. Aldovrandi,
continuellement occupé de sa fortune, n'était pas fâché de faire voir à
la cour de Rome que c'était injustement qu'elle lui avait reproché la
démarche qu'il avait faite pour exciter le zèle des évêques d'Espagne,
et que cette cour n'avait pas lieu d'être aussi sûre qu'elle le croyait
des sentiments de la nation espagnole. Je n'insère ce mot sur la
constitution que parce qu'il est nécessaire par rapport à ce nonce sur
les autres affaires. Il avait à se justifier sur d'autres articles plus
considérables, dont ses ennemis se servaient plus utilement pour le
détruire dans l'esprit du pape.

Les Allemands faisaient un crime à Sa Sainteté de l'intelligence que,
par le moyen de son nonce, ils lui supposaient avec le roi d'Espagne
pour l'entreprise de Sardaigne. Comme leurs reproches étaient
ordinairement suivis des effets, le pape les sentait tous par avance, et
gémissait de cette horrible calomnie, qui le présentait à l'empereur
comme complice du funeste manquement de parole du roi d'Espagne envers
Sa Sainteté comme envers toute la chrétienté. Toute frivole et dénuée de
tout fondement que le pape la disait, elle venait de lui attirer des
réponses de Vienne dont Rome était consternée. L'empereur premièrement
avait refusé de recevoir le bref que le pape lui avait écrit. Il avait
dit que, le roi d'Espagne ayant refusé celui que le pape lui avait écrit
sur l'entreprise de Sardaigne, il voulait tenir la même conduite. Le
nonce à Vienne avait inutilement représenté que le bref avait été remis
au roi d'Espagne. Les ministres impériaux pour le démentir montrèrent
une lettre de l'abbé del Maro, portant en termes formels que, par la
collusion d'Aldovrandi avec Albéroni, jamais le bref n'avait été
présenté au roi d'Espagne\,; que le contenu lui en avait été rapporté
seulement, preuve, dirent-ils, de l'intelligence du pape avec le roi
d'Espagne, et cause, par conséquent, du mauvais état où l'empereur avait
laissé la Sardaigne. Ils ajoutèrent des protestations de la plus
terrible vengeance. Ils déclarèrent qu'ils feraient la paix avec les
Turcs, à quelque prix que ce fût\,; que la France leur laissait la
liberté de faire tout ce qu'ils voudraient, déclarant qu'elle n'y
prendrait pas le moindre intérêt. Ainsi l'empereur, ne craignant plus
d'obstacle à ses desseins, fit dire au pape qu'il avait donné ordre à
ses ministres en Angleterre de cesser toute négociation de paix avec
l'Espagne. Il prétendait avoir déjà fait une ligue avec le roi de
Sicile, et laissait entendre que l'Italie en était l'objet. Enfin
l'empereur, affectant une défiance, qu'il traitait de juste, des
intentions du pape, lui demanda pour sûreté de ses protestations et de
sa conduite, la ville de Ferrare pour en faire sa place d'armes. Il
demanda de plus le logement dans l'État ecclésiastique pour douze mille
hommes. Il y joignit plusieurs autres circonstances exigées toutes comme
des satisfactions, dont la cour de Rome eut horreur. Tout commerce avec
la cour fut en même temps, interdit au nonce\,; les ministres impériaux
lui signifièrent qu'il était libre de se retirer de Vienne ou d'y
demeurer, mais que, s'il prenait ce dernier parti, son séjour et sa
présence seraient totalement inutiles. L'empereur déclara en même temps
que c'était de son pur mouvement, et sans consulter aucun de ses
ministres, qu'il avait fait chasser le nonce de Naples\,; que cet ordre
avait été envoyé au comte de Gallas, son ambassadeur à Rome, pour le
faire exécuter, si le pape refusait de lui accorder les satisfactions
qu'il lui avait demandées.

Ces nouvelles causèrent une étrange consternation dans le palais. Le
pape, tremblant, ne connaissait d'autres voies, pour apaiser la colère
de l'empereur, que la soumission, même la bassesse, et de lui accorder
toutes les satisfactions qu'il imposait. Ses neveux, encore plus
consternés, étaient aussi plus empressés que leur oncle, parce qu'il
s'agissait pour eux de perdre les revenus dont l'empereur les faisait
jouir dans le royaume de Naples, qui était le plus bel article de leurs
finances. On ne doutait donc pas des conseils qu'ils donneraient au pape
et qu'il ne les suivît\,; et que, voyant les Impériaux à ses portes,
maîtres d'entrer dans l'État ecclésiastique toutes les fois qu'ils le
voudraient, et nulles forces d'Espagne encore en Italie, jugeant que la
France, dans la crainte de s'engager dans une guerre étrangère,
refuserait de se joindre à l'Espagne, tant de raisons pressantes ne
l'entraînassent à céder à son penchant naturel de timidité et de
faiblesse, indépendamment même de l'intérêt de ses neveux. On ne
laissait pas de lui rendre justice sur le prétexte odieux et supposé que
les Allemands prenaient de lui faire querelle. Il n'y avait personne qui
pût croire que Sa Sainteté eût eu connaissance de l'entreprise sur la
Sardaigne, ni que ce secret eût été conservé si la confidence lui en eût
été faite.

Comme le pape n'osait se plaindre à Vienne de la conduite des Allemands,
il porta ses plaintes à Madrid\,; et, comme il croyait cette cour plus
faible que l'autre, il y joignit les menaces, et fit entendre qu'il
serait obligé de recourir aux remèdes extrêmes pour effacer de l'esprit
des hommes les soupçons indignes et les calomnies répandues contre le
vicaire de Jésus-Christ. Il en représenta les effets pernicieux,
l'interdiction du nonce à Vienne, celui de Naples chassé, et l'autorité
apostolique totalement abolie dans ce royaume\,; enfin, les autres
menaces encore plus fâcheuses, si par des faits il ne démentait
promptement l'imposture. De là, il passait aux supplications, et
demandait instamment à la piété du roi d'Espagne de restituer la
Sardaigne à l'empereur, comme le seul moyen de persuader ce prince qu'il
n'avait jamais concouru à cette invasion. Il demandait pressamment la
réponse au bref du 25 août, se plaignait amèrement qu'au lieu de cette
réponse, attendue depuis si longtemps, on ne songeait en Espagne qu'à se
préparer à la guerre. Aldovrandi reçut en même temps beaucoup de
reproches de sa conduite. Le pape l'accusait d'être la cause indirecte
de tous ces malheurs, fruits des calomnies répandues contre Sa Sainteté,
pour n'avoir pas présenté au roi d'Espagne son bref du 25 août. Il était
également tancé d'avoir délivré les brefs pour la levée des subsides
ecclésiastiques, et de ce qu'ils avaient eu leur exécution. Pour y
remédier, le pape voulut que son nonce pressât le roi d'Espagne de
répondre à ce bref du 25 août, parce que son silence le privait d'un
moyen très nécessaire et très puissant pour confondre ses calomniateurs.
Il lui ordonna de plus très expressément de retirer les brefs contenant
les concessions qu'il avait faites au roi d'Espagne, et disait qu'il ne
comprenait pas la difficulté à les rendre, puisqu'ils ne pouvaient avoir
d'exécution, et n'en devenaient pas plus efficaces pour demeurer entre
les mains des ministres de Sa Majesté Catholique. Il déclara en même
temps que, si le roi d'Espagne prétendait en faire quelque usage, il ne
pourrait s'empêcher de les révoquer expressément pour satisfaire à sa
conscience. Il reprocha vivement à Aldovrandi d'avoir négligé de
l'informer de l'usage que le P. Daubenton avait fait du pouvoir qu'il
lui avait conféré, d'absoudre le roi d'Espagne de ce qu'il avait fait
contre l'autorité du saint-siège pendant les différends entre les deux
cours\,; et se plaignit de plus d'être si mal instruit par son nonce,
qu'il était obligé de recourir aux lettres particulières, même aux
gazettes, pour apprendre ce qui se passait en Espagne\,; en un mot, il
voulait, à quelque prix que ce fût, trouver des sujets de se plaindre,
soit de son nonce, soit de l'Espagne. Il croyait que c'était la seule
voie d'apaiser les Allemands et de les désabuser de l'opinion qu'ils
avaient prise\,; mais les simples paroles n'y suffisaient pas, et le
pape n'avait point d'autre ressource. Plus le péril lui paraissait
grand, plus il cherchait les moyens de s'en tirer. J'ajouterai qu'ils
étaient d'autant plus difficiles que la colère était factice, politique,
utile aux Impériaux de paraître persuadés de ce dont ils ne l'étaient
point, pour avoir prétexte de tirer du pape tout ce qu'ils pourraient en
places et en subsistances de troupes, et pour l'appesantir sur
l'Espagne, au point de causer à cette couronne tous les embarras
possibles au dedans et au dehors. Revenons.

Le pape tint devant lui une congrégation formée à dessein de délibérer
sur les partis à prendre. On y examina\,: 1° si le pape devait recevoir
Gallas à son audience. Toutes les voix furent pour l'y admettre toutes
les fois qu'il la demanderait. Mais loin qu'il en fît instance, pressé
quelques jours auparavant de voir le pape par le cardinal Albane, cet
ambassadeur déclara avec hauteur qu'il n'irait plus au palais. 2° On
agita si le pape devait excommunier les ministres impériaux qui avaient
mis les mains sur les revenus ecclésiastiques séquestrés par ordre de
l'empereur dans le royaume de Naples, et {[}il fut{]} unanimement résolu
de temporiser\,: maxime favorite de tout ce pontificat, surtout quand il
s'agissait des Allemands. 3° On délibéra sur les démarches qu'il
convenait de faire pour apaiser l'empereur. Il fut conclu qu'il fallait
envoyer à Vienne un cardinal, avec des facultés très amples d'accorder à
ce prince toutes les grâces qu'il demanderait, et que le chef de
l'Église avait le pouvoir de lui accorder. Quant à celles qui ne
dépendaient pas de Sa Sainteté, le soin du légat devait être de faire
connaître à l'empereur que, si elle ne les accordait pas, c'était
uniquement parce qu'elles étaient hors de son pouvoir. Il fut après
question du choix. Le cardinal Piazza fut proposé\,; mais l'opinion
publique fut qu'il ne l'accepterait pas. Le pape désirait son neveu, le
cardinal Albane, mais il ne voulait pas le témoigner\,; il voulait
paraître forcé à le nommer sur le refus d'un autre. On délibéra ensuite
sur la conduite à tenir avec le roi d'Espagne. Il fut résolu que le pape
lui écrirait un bref plus doux que celui du 25 août, que ce prince avait
refusé de recevoir, et qu'il serait ordonné au nonce Aldovrandi de
prendre si bien ses mesures que ce bref parvînt entre les mains de Sa
Majesté Catholique.

Albéroni, bien averti de toutes ces délibérations, était maître
d'empêcher Aldovrandi de présenter aucun bref sans en avoir auparavant
communiqué la copie, ainsi qu'on en usait en France et à Vienne. Le
ministre d'Espagne pouvait rejeter le bref ou bien y faire une réponse
peu satisfaisante pour Sa Sainteté, mais ce dernier parti n'aurait pas
été le plus désagréable pour le pape, parce que, recevant une réponse
dure, il en aurait fait usage pour se justifier auprès de l'empereur de
la partialité qu'il lui reprochait\,; et véritablement les Allemands
n'étaient pas les seuls qui, raisonnant sur le véritable intérêt du
saint-siège et de l'État ecclésiastique, croyaient que le pape
regarderait intérieurement comme son salut d'être aidé par l'Espagne\,;
qu'il avait voulu seulement que le public trompé pût croire que les
secours qu'il recevrait lui seraient donnés contre sa volonté, et que la
source de ce ménagement était la crainte que, les Espagnols ne
réussissant pas, toute la fureur allemande ne retombât sur lui. Ils
demandaient pressement qu'Aldovrandi fût châtié, le regardant comme le
promoteur et le confident de l'intelligence secrète qu'ils supposaient
entre le pape et le roi d'Espagne. Sa Sainteté, toujours occupée de
ménager les deux partis autant que la crainte du plus fort le lui
pouvait permettre, voulait par cette raison complaire aux Impériaux par
quelque mortification légère à son nonce, sans toutefois le rappeler par
considération pour la cour d'Espagne, comme le voulait celle de Vienne.
Le pape crut avoir trouvé ce tempérament en changeant là disposition
qu'il avait faite du neveu d'Aldovrandi tout nouvellement arrivé de
Madrid à Rome, d'y retourner sur-le-champ porter à Albéroni la barrette.
Il ordonna donc à ce neveu de partir dans l'instant non pour Madrid,
mais pour Bologne sa patrie, et d'y demeurer malgré toutes les instances
du cardinal Acquaviva. Ce neveu fut même accusé d'avoir reçu du roi
d'Espagne une pension sur l'évêché de Malaga. Pendant que le cardinal
Paulucci était chargé de porter ces refus à Acquaviva, le pape, par des
voies souterraines, faisait passer à ce dernier ses gémissements et ses
larmes sur l'état et la conduite d'Aldovrandi\,; et par ce double manège
autorisait les discours de ceux qui ne se contraignaient pas de publier
que tout n'était que fiction dans Sa Sainteté, excepté la frayeur des
Impériaux, et le désir extrême de les apaiser. De là on prévoyait qu'il
ne s'accommoderait ni avec la France ni avec le roi de Sicile, parce que
cela déplairait à la cour de Vienne, et l'obligerait à changer de
langage. Le pape en effet éludait de répondre sur les affaires de
Sicile. Pressé par le cardinal de La Trémoille de déclarer ses
intentions, il prit pour prétexte de se taire qu'il n'avait point encore
de réponse du roi de Sicile\,; qu'il désirait savoir si La Trémoille
pourrait engager ce prince à s'expliquer\,; et qu'il verrait ensuite
s'il ferait quelque proposition qui se pût accepter.

\hypertarget{chapitre-xvi.}{%
\chapter{CHAPITRE XVI.}\label{chapitre-xvi.}}

1718

~

{\textsc{Négroni, odieux à la France, nommé vice-légat d'Avignon sans
participation de la France, contre la coutume établie.}} {\textsc{-
Ottobon veut lier avec Albéroni.}} {\textsc{- Nouvelles scélératesses de
Bentivoglio.}} {\textsc{- Le pape refuse au cardinal Albéroni les bulles
de l'archevêché de Séville.}} {\textsc{- Audace, plan, propos d'Albéroni
uni d'attachement et de sentiment au duc de Parme.}} {\textsc{- Manèges
réciproques entre le régent et Cellamare, qui le veut entraîner dans la
guerre avec l'Espagne contre l'empereur.}} {\textsc{- Concert entre
Cellamare et Provane.}} {\textsc{- Ils découvrent le mariage proposé de
M. le duc de Chartres avec une soeur du roi de Portugal sans succès par
les difficultés du rang.}} {\textsc{- Objets des ministres d'Espagne.}}
{\textsc{- Corsini envoyé du grand-duc à Paris\,; quel\,; passe à
Londres pour y faire des représentations inutiles.}} {\textsc{- Le
régent s'ouvre à Provane de l'état de la négociation de Londres.}}
{\textsc{- Sentiment de Cellamare là-dessus.}} {\textsc{- Plaintes de la
cour de Vienne de la France, et ses propositions sur la Toscane appuyées
des Anglais.}} {\textsc{- Quel était Schaub.}} {\textsc{- L'empereur
répond par de fortes demandes aux demandes préliminaires de l'Espagne,
et y est appuyé par l'Angleterre.}} {\textsc{- Manèges et souplesses de
Stanhope.}} {\textsc{- Langage de l'abbé Dubois à Monteléon.}}
{\textsc{- Il lui envoie avec précaution le modèle d'un billet à
Albéroni en faveur de Nancré et de sa négociation, qu'Albéroni méprise,
averti par Monteléon.}} {\textsc{- Conversation de Monteléon avec
Stanhope qui le veut tromper, puis éblouir sur la destination de
l'escadre anglaise.}} {\textsc{- Monteléon tâche à prendre d'autres
mesures pour arrêter l'effet de cet armement.}} {\textsc{- Sagacité de
Monteléon.}} {\textsc{- Fermes réponses des ministres de Sicile à Paris
et à Londres à l'égard de la conservation de cette île à leur maître.}}
{\textsc{- Plaintes et mouvements de Cellamare.}} {\textsc{- Monti peu
satisfait du régent.}} {\textsc{- Monteléon, sur des ordres réitérés,
fait à Londres les plus fortes déclarations sur la destination de
l'escadre.}} {\textsc{- Efforts d'Albéroni en Hollande.}} {\textsc{- Ses
sentiments sur les traités d'Utrecht.}} {\textsc{- Ses vanteries.}}
{\textsc{- Cache bien où il veut attaquer.}} {\textsc{- Sagacité de
l'abbé del Maro.}} {\textsc{- Beretti trompé ou trompeur sur la
Hollande.}} {\textsc{- Sage avis de Cellamare à Albéroni sur la
France.}} {\textsc{- Propos publics de Cellamare\,; retient sagement
Provane\,; dit à Nancré qu'il ne réussira pas.}}

~

Une affaire de peu de conséquence donna lieu à augmenter les
brouilleries que la constitution causait depuis trop longtemps entre
Rome et la France. La vice-légation d'Avignon vaquait. Avant d'y nommer,
les papes faisaient toujours donner au ministre du roi à Rome les noms
de ceux entre lesquels il voulait choisir pour n'y pas envoyer un légat
désagréable, prévenir le roi sur le nouveau vice-légat, et lui concilier
une protection dont il avait besoin dans un État aussi peu étendu,
enclavé de toutes parts dans ceux du roi. Malgré cet usage le pape crut
devoir profiter d'un temps de faiblesse et de minorité, plus encore d'un
temps où on se croyait tout permis à Rome contre la France, pour secouer
ce qu'il voulut trouver être servitude. Ainsi il nomma le prélat Négroni
sans en avoir rien fait dire au cardinal de La Trémoille. Tout le mérite
du nouveau vice-légat était d'être neveu du cardinal Négroni, si noté
par l'extravagance de ses emportements contre la France. Apparemment que
le pape crut aussi que plus ce vice-légat serait reconnu partial contre
la France, plus le public serait persuadé qu'elle n'avait point de part
à sa nomination. Quelque attention qu'eût le cardinal de La Trémoille à
plaire à Rome et à prévenir les moindres sujets de plaintes, il ne
laissa pas de s'apercevoir de l'impossibilité de dissimuler cette
innovation. Quelque peu disposé qu'il fût à se plaindre du pape, il osa
néanmoins le faire. On se plaignit aussi à Rome de cette prétention,
quoique si bien fondée et si établie par l'usage. On ajouta que depuis
quelques années les vice-légats d'Avignon étaient au moins soupçonnés en
France de favoriser les fabrications de fausse monnaie dans le royaume,
et de leur donner asile dans le comtat\,; que Négroni était rigide,
attentif, prudent, fort instruit des matières criminelles, et très
propre à écarter les faux-monnayeurs. On comptait à Rome pouvoir
impunément entreprendre tout contre la France\,; ceux même qui devaient
être le plus attachés à la couronne par les bienfaits qu'ils en avaient
reçus cherchaient des protections étrangères.

Le cardinal Ottobon, qui en était si comblé, écrivit au cardinal
Albéroni, sous prétexte de zèle pour le bien de l'Italie, pour lui
proposer d'établir et d'entretenir un commerce de lettres avec lui.
D'ailleurs aucun des cardinaux regardés comme Français ne s'employait à
pacifier les troubles que les véritables ennemis de la France
cherchaient à susciter dans le royaume, sous ombre de maintenir la bonne
doctrine en soutenant la constitution. Bentivoglio, le plus enragé de
tous, ne se contentait pas d'interpréter faussement, à son escient même,
les intentions du régent sur les affaires de Rome. Fâché d'avoir eu
ordre de le remercier de ses offices en Angleterre sur le ressentiment
et les menaces de vengeance de la détention de Peterborough, il
prétendit que ce prince n'avait agi que parce qu'il savait parfaitement
que le roi d'Angleterre ne songeait nullement à se venger du pape\,; que
si les bruits d'un armement de mer étaient évanouis, on ne le devait
attribuer qu'aux menaces de Monteléon, et à la juste crainte des Anglais
de voir leur commerce interrompu. Ce nonce ajoutait qu'il fallait faire
connaître le juste prix des services que le régent rendait au pape\,; et
sur cette supposition, il se croyait en droit, même obligé de donner de
fausses couleurs à toutes les démarches de Son Altesse Royale dont le
pape aurait dû lui savoir le plus de gré.

Bentivoglio ramassait tous les discours que le public mal instruit
tenait sur les affaires d'Angleterre, et les donnait comme des vérités.
Il avançait hardiment que, sous prétexte de concilier et de terminer les
différends entre l'empereur et le roi d'Espagne, le régent songeait
uniquement à s'unir et à faire des ligues avec les puissances
principales de l'Europe, pour être secouru d'elles en cas d'ouverture à
la succession à la couronne\,; qu'il voulait sur toutes choses prévenir
une alliance entre l'empereur, le roi d'Espagne et le roi de Sicile,
empêcher que ces princes ne convinssent entre eux pour leurs intérêts
communs de faire monter le roi d'Espagne sur le trône de France, et
celui de Sicile sur le trône d'Espagne, suivant la disposition des
traités d'Utrecht. On ne démêlait point encore la vérité de celui qui se
négociait à Londres. Toutefois on en savait assez pour donner au nonce
lieu de dire qu'on offrait à l'empereur la Sicile, avec promesse de le
laisser agir en Italie comme il le jugerait à propos pour ses intérêts
sans y former le moindre obstacle\,; qu'on promettait au roi de Sicile
des récompenses dans le Milanais avec le titre de roi de Lombardie\,; et
qu'on espérait endormir le roi d'Espagne, en le flattant d'établir en
faveur de ses enfants du second lit des apanages considérables en
Italie, tels que les États de Toscane, de Parme et de Plaisance.
Bentivoglio, ajoutant ses réflexions à ce qu'il croyait savoir du traité
d'alliance, concluait que, si des projets si légèrement formés, si
difficiles à exécuter, étaient cependant accomplis, la France en serait
la victime, parce qu'elle aurait elle-même contribué à rendre ses
ennemis trop puissants\,; qu'en cet état ils feraient ce qu'ils
croiraient le plus avantageux pour eux, non ce qu'ils auraient promis,
et ce qu'ils seraient engagés de faire en vertu de l'alliance. Ces
affaires, étrangères à celles de la constitution, étaient comme des
épisodes que le nonce employait pour animer la cour de Rome contre la
conduite du régent, et pour faire comprendre au pape que le nombre de
ses partisans augmenterait en France, à mesure que celui des ennemis de
Son Altesse Royale grossissait par les négociations qu'elle faisait avec
les étrangers. Sur ce fondement, il ne cessait d'empoisonner tout ce qui
se passait en France, et de porter le pape à tout ce qu'il pouvait de
plus violent sur les affaires de la constitution.

Le pape, continuant de penser qu'il ne pouvait apaiser l'empereur qu'en
se montrant irrité contre l'Espagne, voulut le paraître extrêmement
contre lés ministres du roi d'Espagne, qui se portaient, disait-il,
contre l'autorité ecclésiastique et contre celle du saint-siège. Le roi
d'Espagne ayant nommé le cardinal Albéroni à l'archevêché de Séville, Sa
Sainteté se porta à un plus grand éclat. Elle lui en refusa les bulles,
et lui fit dire qu'elle les lui aurait accordées, si, dans le temps
qu'elle était sur le point de les proposer au consistoire, elle n'eût
appris que l'évêque de Vich et un autre avaient été chassés violemment
de leurs diocèses par ordre du roi d'Espagne. Ce frivole prétexte ne
trompa personne\,; tout le monde pénétra aisément le vrai motif du
refus. Il n'y eut que les Impériaux qui ne voulurent pas en convenir\,;
mais les plaintes du pape firent peu d'effet à Madrid. Albéroni insista
sur les raisons que le roi d'Espagne avait eues de ne pas répondre au
bref du 25 août, parce qu'il n'aurait pu le faire qu'en termes amers, et
à peu près dans le sens que le public s'était expliqué sur cette pièce
quand il l'avait vue dans les gazettes. Ce cardinal prétendait même
avoir rendu un grand service au pape d'avoir gardé ce bref entre ses
mains, parce qu'il ne pouvait produire qu'un effet pernicieux. Il
s'applaudissait par avance de l'obligation que Rome lui avait de ne
s'être pas laissé endormir par les pièges des Impériaux, et de ce que le
roi d'Espagne serait incessamment maître de l'Italie\,; mais il
exhortait en vain le pape et les princes d'Italie à profiter, par
l'union, la force et le courage, des desseins trop déclaras de
l'empereur par ses dernières réponses au nonce de Vienne.

Le duc de Parme, le plus faible et le plus menacé de tous, et qui
s'était attiré la colère de l'empereur par le mariage de la reine
d'Espagne et par les offices qu'il avait rendus pour la promotion
d'Albéroni à Rome, désirait d'être secouru d'argent, pour mettre au
moins Plaisance hors d'insulte. Son ministre était maître absolu en
Espagne\,; il lui devait les commencements de cette fortune, et beaucoup
encore sur son cardinalat. Il paraissait avoir en vue les intérêts de
son premier maître\,; il suivait ses maximes, et pensait comme lui qu'il
était impossible que l'Italie fût tranquille tant que les Allemands y
conserveraient une seule place. Sur ce fondement, il traitait de
verbiages et d'illusoire le plan proposé à Londres. Il disait qu'il
n'était pas étonné de voir le roi d'Angleterre agir sous main en faveur
de l'empereur, parce que depuis longtemps les engagements publics et
secrets de l'électeur de Hanovre avec la maison d'Autriche étaient
parfaitement connus\,; mais qu'il était difficile de comprendre que le
régent, sensible à l'honneur, aimant la gloire et connaissant ses
véritables intérêts, prît des partis si opposés à des considérations si
puissantes, qu'il choisît des routes si dangereuses pour lui, et que, se
laissant aller à des conseils de gens qui ne songeaient qu'à leurs
propres intérêts, il fermât les yeux à ses propres lumières pour se
laisser conduire dans le précipice. Le cardinal assurait que, loin de
réussir par de telles routes, le régent verrait la guerre civile allumée
dans le sein de la France. Ce présage alors ne paraissait fondé que sur
le génie des Français, portés à se faire la guerre entre eux quand ils
ne sont pas occupés par des guerres étrangères\,; et comme la crainte
d'engager le royaume dans une guerre nouvelle avec les étrangers était
l'unique motif qui avait obligé Son Altesse Royale à travailler aux
moyens de ménager la paix entre l'empereur et l'Espagne, Albéroni, loin
d'approuver cette crainte juste mais peu conforme à ses idées, la
traitait de terreur panique et s'épuisait en raisonnements. Il croyait
intimider le roi d'Angleterre par la fermentation qui régnait chez lui,
et se savait gré d'avoir menacé Bubb, à Madrid, de donner de puissants
secours au Prétendant. Il voulait engager le régent à parler sur le même
ton à Georges. Il disait que, s'unissant au roi d'Espagne, il lui ferait
dépenser bien des millions en Italie, qu'il garderait certainement pour
des occasions plus éloignées, si Son Altesse Royale s'amusait encore à
des négociations frivoles, comme il paraissait par le départ prochain de
Nancré pour se rendre à Madrid. En même temps, il tâchait de faire
répandre que, sur l'article des négociations pour la paix, il n'était
pas maître de l'esprit du roi d'Espagne\,; que non seulement là-dessus,
mais en beaucoup de choses qui ne regardaient que des affaires
particulières, il avait fort à le ménager et à compter avec lui.

Ces discours modestes d'Albéroni ne firent nulle impression à Paris ni à
Londres\,; on était très persuadé, parce que lui-même l'avait dit
plusieurs fois, qu'en grandes comme en petites choses il disposait
absolument de la volonté du roi d'Espagne. L'opinion en était confirmée
par les ordres que recevaient les ministres d'Espagne et par la manière
dont ils expliquaient les intentions du roi leur maître. Cellamare ne
parlait que de tirer la France de sa léthargie. Il employait auprès du
régent Monti, nouvellement arrivé d'Espagne, qu'on croyait fort avant
dans la confidence d'Albéroni. Il ne s'agissait point de négocier sur
aucun plan de paix, de changer ou de modérer les conditions d'un traité.
Les vues, et tous les discours de Cellamare au régent n'allaient qu'à le
convaincre de la nécessité d'une union inaltérable entre la France et
l'Espagne, et de ne pas compter que les insinuations ni les offices des
médiateurs détournassent les Allemands des projets qu'ils pourraient
faire pour troubler le repos de l'Italie. Le régent convenait de tous
les avantages de l'union des deux branches de la maison royale. Il
ajoutait même que, si les offices étaient inutiles, la France
emploierait ses forces pour empêcher un mal que la persuasion n'aurait
pu détourner. Cellamare ne se reposait pas sur de pareilles assurances.
Il les trouvait contredites par la conduite de l'abbé Dubois, qui
agissait seul à Londres sans aucun concert avec Monteléon, en sorte que
le roi d'Espagne ne recevait ni de Paris ni de Londres aucune
communication de ce qui se passait à Londres par rapport à ses intérêts.
Cellamare faisait les mêmes plaintes pour lui-même, et jugeait de ce
silence que les réponses que l'empereur avait faites ne pouvaient être
acceptées en Espagne, et que le voyage de Nancré, qu'on pressait de
partir pour Madrid, serait inutile. Le régent l'assura cependant qu'il
ordonnerait à l'abbé Dubois de confier à Monteléon le plan et l'état de
la négociation. Mais Son Altesse Royale ne voulut point s'ouvrir sur les
nouvelles qu'elle venait de recevoir de Vienne parle secrétaire de
Stanhope, qui tenaient Cellamare dans une grande curiosité. Il en reçut
encore une assurance positive que Nancré ne partirait pas de Paris sans
porter avec lui un plan de paix dont le roi d'Espagne eût lieu d'être
satisfait. L'ambassadeur prétendit que Nancré lui avait dit de plus
qu'on obligerait la cour de Vienne de recevoir ce plan de gré ou de
force\,; mais il demeurait persuadé que le régent aurait grand'peine à
s'y résoudre, qu'il serait mal secondé par la cour de Londres, dont il
était souvent obligé de combattre les idées et les propositions. Le
régent lui fit même valoir la fermeté de l'abbé Dubois, et dit que
c'était pour s'en plaindre que Stanhope avait envoyé son secrétaire,
espérant le trouver plus facile que son ministre. Cellamare ne le
croyait pas. Fortifié de Monti, ses représentations ne tendaient point à
modifier les conditions du traité, mais à faire voir la nécessité de
prendre les armes, et de prévenir la conclusion de la paix entre
l'empereur et les Turcs. Elle était encore éloignée. Paris, plein de
raisonnements politiques, croyait avec Cellamare qu'elle était aisée à
détourner, en employant le crédit et les talents de Ragotzi et la force
de ses partisans en Hongrie, et de leur animosité contre la maison
d'Autriche. Cellamare disait que c'était par des motifs de passion
particulière que des Alleurs, nouvellement revenu de Constantinople,
décriait le prince Ragotzi, et que le maréchal de Tessé était au
contraire le seul qui jugeât sainement de l'utilité d'une diversion
qu'on pourrait exciter en Hongrie par le moyen des mécontents. Il
flattait ainsi les idées d'Albéroni, qui semblait compter sur la
continuation de la guerre de Hongrie, et sur le secours dont elle lui
serait pour l'exécution de ses desseins.

Comme il paraissait encore alors que les intérêts du roi d'Espagne et
ceux du roi de Sicile étaient parfaitement unis, la même union régnait
aussi entre leurs ministres à Paris. Provane disait à Cellamare que son
maître s'exposerait aux plus grands dangers plutôt que de consentir à
l'échange de la Sicile. Cellamare faisait agir Provane, soit auprès du
régent pour le disposer plus favorablement pour l'Espagne, soit auprès
des ministres étrangers résidents lors à Paris, qu'il croyait à propos
de ménager. Il sut par là que l'ambassadeur de Portugal avait dit que le
régent avait fait proposer le mariage de M. le duc de Chartres avec
l'infante, soeur du roi de Portugal, et qu'il s'y trouvait des
difficultés sur le rang de M. le duc de Chartres. Cette affaire n'était
qu'un incident. Toute l'attention des ministres d'Espagne se portait sur
la négociation de Londres. Ils regardaient Georges comme un ennemi, et
livré à l'empereur pour ses intérêts d'Allemagne. Ils y voulaient
opposer ceux de la nation anglaise pour leur commerce, et persuader les
membres du parlement de s'opposer au départ des vaisseaux destinés pour
la Méditerranée, comme à une résolution capable de causer une rupture et
d'entraîner la ruine totale du commerce. Ils pénétraient, mais ils ne
savaient encore qu'imparfaitement les points et les difficultés de la
négociation. Cellamare et Provane commençaient à découvrir par les
bruits publics qu'il s'agissait d'échanger la Sicile avec la Sardaigne,
et se plaignaient tous deux de la liberté que se donnaient les
médiateurs de disposer d'États dont ils n'étaient pas les maîtres. Les
princes d'Italie, quoique fort alarmés, faisaient peu de mouvements.
Enfin, le grand duc envoya ordre à son envoyé à Paris de passer à
Londres, et d'y représenter l'injustice de disposer de ses États contre
son gré. Ceux qui connaissaient le négociateur jugèrent peu
favorablement de son succès. D'ailleurs, les choses étaient trop
avancées pour attendre quelque changement. Cet envoyé du grand-duc était
Corsini, qui est devenu cardinal et premier ministre à Rome, sous le
pontificat de son oncle Clément XII, douze ans après. M. le duc
d'Orléans expliqua lors à Provane de quoi il était question, mais
verbalement. Provane aurait souhaité le plan du traité par écrit. Il se
plaignit à Stairs de l'appui que le roi d'Angleterre donnait à l'échange
de la Sicile. La réponse fut simplement en termes fort généraux.
Cellamare, instruit par Provane, dit à Nancré que, s'il ne portait à
Madrid des propositions plus avantageuses que celles dont on le disait
chargé, il ne devait pas être étonné de ne pas réussir. Il se vanta même
d'avoir convaincu Nancré, qui néanmoins partit.

La cour de Vienne prétendait que le plan sur lequel on négociait à
Londres était absolument différent de celui que l'abbé Dubois avait
proposé, et {[}dont il{]} était convenu à Hanovre. Elle se plaignait
aussi d'entendre dire de tous côtés que, si l'empereur ne consentait pas
aux demandes de la France, cette couronne se joindrait à l'Espagne pour
lui faire la guerre. Cette espèce de menace blessait sa hauteur. Elle
menaçait de son côté de se rendre plus difficile, si elle parvenait à
faire la paix avec la Porte avant la conclusion du traité qui se
négociait à Londres. Les ministres de Georges semblaient appuyer les
menaces des Impériaux. Non seulement Saint-Saphorin les trouvait bien
fondées, et tâchait d'alarmer le régent mais Stairs, secondé d'un
Suisse, grand fripon, nommé Schaub, qui avait servi de secrétaire à
Stanhope et qu'on renvoyait de Londres à Vienne, parlait haut dans les
conférences qu'ils eurent tous deux avec le régent. Quelque avantageuse
que fût à l'empereur la médiation d'un roi d'Angleterre, électeur de
Hanovre, si partial en sa faveur par tant de raisons générales et
personnelles, l'empereur n'en paraissait que plus difficile\,; et
retardait l'utilité qu'il devait se promettre de la conclusion du
traité, par ses demandes. Il prétendait qu'avant toutes choses le roi
d'Espagne retirât ses troupes de la Sardaigne, et qu'il la remît en
dépôt entre les mains d'un prince neutre, pour la garder en dépôt
jusqu'à ce que toutes les conditions de la paix fussent réglées. Le roi
d'Angleterre était le prince que l'empereur indiquait, parce qu'il n'en
pouvait choisir un dont il fût plus sûr, et d'ailleurs cet honneur,
disait-il, était dû à ce prince par la manière dont il se portait pour
le succès de la négociation. Outre ce dépôt, l'empereur demandait que,
le grand-duc venant à mourir, ses États fussent démembrés, ne pouvant
consentir qu'un prince de la maison de France possédât toute la Toscane
telle qu'elle était possédée par la maison de Médicis. Il voulait donc
faire revivre l'ancienne république de Pise. Il voulait de plus que
Livourne fût érigée en ville libre sous la protection de l'empire. Il
comptait par ces propositions engager encore plus en sa faveur les
puissances intéressées au commerce du Levant\,; et véritablement les
plus confidents ministres du roi d'Angleterre les appuyaient, jusqu'au
point de représenter au régent qu'il s'exposerait à faire échouer la
négociation s'il s'opiniâtrait à la totalité de l'expectative des États
du grand-duc pour un des fils de la reine d'Espagne, et disaient que
souvent on n'obtenait rien pour trop demander. Saint-Saphorin y joignait
les menaces, en faisant revenir au régent par l'Angleterre que les
conférences pour là paix entre l'empereur et le Grand Seigneur
s'allaient ouvrir\,; que les conditions de part et d'autre en seraient
bientôt réglées, les deux parties désirant également la fin de la
guerre\,; que, si ce n'était pas une paix définitive, ce serait une
trêve de quatre ou cinq ans, chacun demeurant dans la possession où il
se trouvait\,; que la cour de Vienne, débarrassée de la guerre de
Hongrie, deviendrait encore plus difficile avec l'Espagne.

Le roi d'Espagne avait demandé deux conditions préliminaires\,: l'une
que l'empereur promît de ne plus envoyer de troupes en Italie, l'autre
de n'y plus exiger de contributions des princes. Les Impériaux
répondaient à la première qu'il était étonnant que ce prince prétendît
imposer à l'empereur la nécessité de ne point envoyer de troupes en
Italie, quand elles y étaient le plus nécessaires pour la conservation
de ses États, que l'Espagne avait attaqués au préjudice de la
neutralité\,; qu'elle continuait d'armer, et que, si elle voulait
empêcher l'empereur d'envoyer des troupes en Italie, il fallait qu'elle
discontinuât auparavant ses armements par mer et par terre, qu'elle
promît elle-même de demeurer en repos, et que, pour sûreté de sa parole,
elle remît la Sardaigne en dépôt au roi d'Angleterre. Quant aux
contributions, il y fut répondu que l'empereur ne les avait demandées
qu'en vertu d'un résultat de la diète de l'empire, fondé sur la
nécessité de soutenir la guerre contre l'ennemi commun de la
chrétienté\,; qu'il était juste que toute puissance dépendante de
l'empire, comme étaient les princes d'Italie, concourussent aux besoins
et aux succès de cette guerre\,; et que ce n'était point agir contre la
neutralité que d'exiger d'eux des contributions pour cet effet\,;
qu'enfin, si l'Espagne réparait les infractions qu'elle avait faites à
la neutralité, et qu'elle cessât d'en commettre de nouvelles, l'empereur
cesserait aussi d'exiger aucunes sommes des princes d'Italie, n'étant
pas juste que, pendant que l'empereur se lierait les mains, le roi
d'Espagne se crût le maître d'agir librement comme il croirait convenir
à ses intérêts. Ces réponses de l'empereur furent non seulement goûtées
à Londres, mais particulièrement appuyées du roi d'Angleterre et de ses
ministres.

Stanhope n'oublia rien pour intimider Monteléon, et par lui le roi
d'Espagne, en lui représentant les suites funestes de la guerre que ce
prince voulait allumer en Italie, qui, en deux ans, deviendrait
générale, ferait revivre les droits de l'empereur sur l'Espagne, ceux de
Philippe sur la France, et qu'il se trouverait peut-être des princes qui
prétendraient aussi régler la succession d'Angleterre\,; et que le seul
moyen d'éviter tant de maux était de terminer les différends entre
l'empereur et l'Espagne de manière que le roi d'Espagne pût être
satisfait, et que la négociation entreprise à Londres eût un heureux
succès. Il employait les espérances et les menaces. Quelquefois il
promettait que, si l'empereur se rendait trop difficile, le roi
d'Angleterre se croirait dégagé de toute garantie\,; il disait la même
chose si les refus venaient de la part du roi d'Espagne. Stanhope
cependant avait l'adresse de faire voir un penchant particulier pour
l'Espagne\,; ou bien Monteléon voulait le faire croire à Madrid, soit
pour se faire un mérite d'avoir su gagner un des principaux ministres de
Georges, soit pour donner plus de poids aux insinuations qu'il faisait
de temps en temps au cardinal Albéroni, mais toujours en tremblant pour
le porter à la paix\,: II était persuadé que ce cardinal ne la désirait
pas, dont la preuve était le silence qu'il gardait à son égard, à lui
qui était le seul ministre du roi d'Espagne à portée de veiller à la
négociation\,; et de ménager les intérêts du roi son maître. Il fallait
pour y réussir qu'il fût instruit de ses intentions, et il les ignorait
absolument\,; en sorte que Stanhope le pressant pour savoir enfin ce que
Sa Majesté Catholique demandait, il était obligé de répondre en termes
généraux, et de se servir de son esprit pour cacher le peu de confiance
que sa cour avait en lui. Il était instruit néanmoins de ce qui se
passait, mais par Stanhope et par Dubois. Cet abbé l'assurait que le
régent communiquerait tout au roi d'Espagne\,; que c'était le principal
objet de la, mission de Nancré\,; qu'il agirait à Madrid d'un parfait
concert avec Albéroni\,; et que, jusqu'à ce qu'il sût par lui les
intentions du roi d'Espagne, le régent différerait de consentir au
projet qui lui était proposé par les Anglais. Voulant donner à Monteléon
une preuve de la confiance qu'il prenait en lui, il lui dit qu'il
reconnaissait en tout la partialité des ministres hanovriens et des
Anglais de leur parti pour la cour de Vienne\,; qu'il remarquait qu'ils
oubliaient souvent leurs intérêts pour favoriser celui de l'empereur. Il
excitait Monteléon à redoubler ses assiduités auprès de Stanhope, pour
animer davantage son penchant pour l'Espagne. Désirant disposer Albéroni
favorablement pour Nancré, il pria l'ambassadeur d'en écrire à ce
premier ministre en termes qui le disposassent favorablement pour la
négociation, et le prévinssent en faveur du négociateur. Il parut même
qu'il craignit de s'en rapporter à lui, car il lui envoya par Chavigny
le modèle du billet qu'il le pria d'écrire à Madrid, et pour plus de
sûreté, de lui en renvoyer la minute. Ce billet était conçu dans les
termes suivants\,:

«\,L'abbé Dubois, que je sais de bonne part s'intéresser à votre gloire
particulière, conjure V. E. de bien peser ce que le sieur de Nancré lui
dira, et de ne perdre pas cette occasion de réunir la France,
l'Angleterre et la Hollande avec l'Espagne, contre l'empereur, ce qui
arrivera infailliblement si elle donne les mains à ce que ces trois
puissances lui proposeront, soit qu'ensuite l'empereur l'accepte ou
qu'il le refuse.\,»

Malgré ces précautions prudentes, Albéroni sut que le billet n'était pas
du style de Monteléon, que l'abbé Dubois l'avait dicté, et cependant
n'en fit pas grand cas. Peut-être Monteléon lui-même eut-il quelque part
au peu d'impression que firent les protestations de l'abbé Dubois\,; car
il est certain que cet ambassadeur prétendit avoir découvert (on dit
{[}du{]} moins qu'il l'écrivit à Madrid) que la France et l'Angleterre
s'étaient promis réciproquement de demeurer unies pour soutenir le
projet du traité, et d'employer leurs forces pour obliger l'Espagne à
l'accepter si elle y résistait.

Quoi qu'il en soit, le roi d'Angleterre continuait d'armer par mer. On
disait sans mystère que l'escadre, qui serait de onze navires de guerre,
était destinée pour la Méditerranée, où elle se joindrait à sept autres
navires que l'Angleterre avait déjà dans cette mer. Le roi d'Espagne fit
demander à quel usage l'Angleterre destinait cette escadre\,; et comme
jusqu'alors les ministres anglais s'étaient contentés d'assurer en
général que l'intention du roi leur maître était d'entretenir la paix et
la bonne intelligence avec Sa Majesté Catholique, Monteléon eut ordre de
les engager à lui donner quelque parole plus précise. Il pressa donc
Stanhope de lui déclarer par écrit, au nom du roi d'Angleterre, que
l'escadre qu'il faisait armer, non seulement ne serait pas employée
contre les intérêts du roi d'Espagne, mais même qu'elle ne passerait pas
dans la Méditerranée. Comme Stanhope répugnait à donner une pareille
déclaration, Monteléon lui proposa, pour tout expédient, d'ordonner au
colonel Stanhope, alors envoyé d'Angleterre à Madrid, de la faire, ou
tout au moins de s'expliquer clairement au cardinal Albéroni sur la
destination de l'escadre. L'une et l'autre de ces propositions fut
également rejetée. Stanhope voulut faire croire à Monteléon que le seul
objet du roi d'Angleterre était d'obtenir du pape la satisfaction qu'il
lui avait demandée pour l'enlèvement de Peterborough\,; qu'il ne doutait
pas qu'elle ne lui fût accordée\,; mais qu'il fallait presser les
délibérations de la cour de Rome, et faire paraître aux côtes d'Italie
des forces suffisantes pour obliger le pape, par la crainte, à ce qu'il
ne voudrait pas de bonne grâce accorder là-dessus aux instances de
l'ambassadeur de l'empereur. Stanhope ajouta qu'il ne croyait pas même
qu'il fût nécessaire d'envoyer des vaisseaux dans la Méditerranée pour
mettre le pape à la raison\,; qu'on avait donc travaillé très lentement
à l'armement de cette escadre, et que, si depuis quelques jours il y
paraissait plus de diligence, la Méditerranée n'en était pas l'objet,
mais la mer Baltique, où le roi d'Angleterre prétendait faire passer
vingt navires de guerre et dix bâtiments de suite. Monteléon aurait
souhaité que Stanhope, lui confiant, disait-il, les véritables
intentions du roi d'Angleterre, lui eût promis formellement ce qu'il ne
lui disait que comme simple confidence. Il essayait de faire voir à ce
ministre qu'il ne devait avoir aucune peine à promettre, pour le bien de
la paix, que le roi d'Angleterre n'enverrait point de vaisseaux dans la
Méditerranée, puisqu'il n'en avait pas l'intention\,; mais ces instances
furent inutiles. Stanhope lui dit que le roi d'Angleterre ne pouvait
donner une telle parole sans manquer formellement aux engagements du
traité qu'il avait signé avec l'empereur, dont une des principales
conditions était de lui garantir la possession des États dont il
jouissait actuellement en Italie. Stanhope déclara nettement que
l'intention de son maître était d'y satisfaire ponctuellement, en sorte
que personne ne pouvait dire positivement jusqu'à quelle extrémité les
choses seraient peut-être portées\,; qu'il pouvait seulement protester
qu'à moins d'un grand malheur, l'Angleterre ne prendrait aucun nouvel
engagement capable d'altérer la bonne correspondance qu'elle prétendait
entretenir avec l'Espagne. Monteléon répliqua que le moyen de la
conserver entre les puissances amies était de s'expliquer franchement\,;
que les réponses ambiguës n'entretenaient point l'amitié\,; qu'à son
égard, il se croyait obligé de dire nettement que, si l'Angleterre
envoyait une escadre dans la Méditerranée, le roi d'Espagne ne pourrait
s'empêcher de prendre des mesures contraires au commerce des deux
nations. Stanhope convint de tous les avantages que ce commerce
apportait à l'Angleterre, et comme il affectait en toutes occasions de
paraître disposé favorablement pour l'Espagne, il dit à Monteléon, qu'il
consentirait de tout son coeur à la proposition qu'il lui avait faite
d'ordonner au colonel Stanhope de confier au roi d'Espagne les
intentions secrètes du roi d'Angleterre\,; mais qu'il n'avait que sa
voix dans le conseil, composé d'ailleurs de différentes nations, en
sorte qu'il ne pouvait répondre ni des délibérations ni de la
résolution. Il offrit ce qui était en lui, c'est-à-dire de rendre compte
au roi d'Angleterre et à son conseil des propositions de Monteléon.

Cet ambassadeur était trop éclairé et connaissait trop le caractère des
Anglais pour se laisser éblouir par des réponses si vagues. Il jugeait
donc que si l'intention du roi d'Angleterre et de ses ministres était de
se réserver la liberté d'accorder ou de refuser absolument la
déclaration sollicitée, suivant le tour que prendraient les affaires
générales, une telle incertitude ne pouvant convenir aux intérêts du roi
d'Espagne, Monteléon résolut d'agir par d'autres voies\,: celle qu'il
crut la plus sûre fut d'intéresser la nation. Rien ne lui était plus
sensible que l'interruption de son commerce avec l'Espagne. Il n'oublia
rien pour alarmer les membres du parlement, faisant envisager
secrètement à quelques-uns des principaux le péril prochain dont ce
commerce serait menacé, si le roi d'Angleterre faisait passer, comme on
le disait, une escadre dans la Méditerranée. Il leur insinua, comme un
moyen d'éviter ce danger, de presser le roi leur maître de communiquer
au parlement tous les traités qu'il avait faits, en sorte que la nation
assemblée pût aviser aux moyens de ne pas rompre avec l'Espagne.
L'orateur de la chambre basse, frappé de cette crainte, vit secrètement
Monteléon\,; il reçut de lui des instructions, et protesta que la plus
grande partie de la nation s'opposerait à toute résolution de la cour,
qui tendrait à rompre avec l'Espagne.

Quelques jours après, dans une séance du parlement, on tint quelques
discours sur l'escadre que le roi d'Angleterre devait envoyer dans la
Méditerranée. Deux députés des communes représentèrent que ce serait
ruiner l'Angleterre que de donner occasion à l'Espagne d'interrompre le
commerce si avantageusement établi entre les deux nations. Le premier
effet des diligences de Monteléon ne l'éblouit pas. Comme il connaissait
le caractère et le génie de la nation anglaise, et les passions des
particuliers qui avaient le plus de crédit sur l'esprit du roi
d'Angleterre, il comprit qu'il ne devait pas compter sur les
dispositions apparentes de quelques membres du parlement, parce que la
cour saurait bien les gagner si leurs suffrages étaient de quelque
poids, sinon que leurs contradictions ne traverseraient pas, ses
résolutions. Quant aux ministres, il était persuadé que ce serait
inutilement qu'il entreprendrait de faire combattre la raison contre le
désir qu'ils avaient de plaire aux Allemands, comme l'unique moyen de
parvenir à l'avancement que chacun d'eux se proposait. Ainsi, voyant les
choses de près, il n'espérait rien de bon de l'Angleterre pour le roi
son maître. Il ne se promettait pas un succès plus heureux de la
négociation que la France voulait entamer à Madrid. Toutefois il croyait
que, si on pouvait envisager un moyen de sortir d'affaires avec quelque
avantage, c'était celui de savoir plier aux conjonctures présentes, et
de convenir, s'il était possible, de quelque proposition capable de
concilier les intérêts de l'Espagne avec l'empressement que la France et
l'Angleterre témoignaient à l'envi de ménager et de conclure la paix
entre l'empereur et le roi d'Espagne.

Raisonnant sur le caractère des ministres de l'empereur, il pensait que
la cour de Vienne, inflexible et déraisonnable, disait-il, n'admettrait
aucun expédient quand il s'agirait de réduire ses vastes prétentions, et
qu'elle découvrirait elle-même son ambition de manière que ses amis même
comprendraient les raisons et la nécessité de s'unir pour contraindre
les Allemands à sortir de l'Italie. Cette cour, en effet, ne voulait
alors entendre à rien sur le point d'assurer l'expectative de la Toscane
à un fils de la reine d'Espagne. Le plan du traité lui plaisait en ce
qui regardait ses avantages\,; mais l'empereur considérant ce qui lui
était offert comme une restitution d'un bien qui lui appartenait
légitimement, croyait que les demandes faites en faveur du roi d'Espagne
étaient autant de démembrements que les médiateurs voulaient arracher
aux droits légitimes de la maison d'Autriche.

On était à la fin de février\,; jusqu'alors le détail de la négociation
n'avait pas encore passé les cours de Vienne, de France et d'Angleterre.
Le roi de Sicile était inquiet d'un traité dont il devait fournir la
matière principale, puisque la Sicile était le prix que les négociateurs
proposaient à l'empereur pour l'engager à se désister pour toujours de
toute prétention sur la monarchie d'Espagne. Il paraissait juste d'avoir
le consentement de ce prince, qui possédait actuellement la Sicile en
vertu des traités faits seulement depuis cinq ans à Utrecht, dont la
France et l'Angleterre étaient également garantes. Toutefois on ne
parlait encore clairement au roi de Sicile ni de la disposition de cette
île, ni du dédommagement qu'on lui offrirait pour obtenir son
consentement. Le comte de Sunderland dit seulement à son envoyé que le
roi d'Angleterre songeait aux intérêts du roi de Sicile\,; qu'il lui en
dirait davantage dès le moment qu'il pourrait s'expliquer plus
clairement. Bernsdorff, le principal des ministres hanovriens, dit à ce
même envoyé qu'il jugeât lui-même s'il était possible au roi
d'Angleterre de rien communiquer au roi de Sicile avant de savoir si
l'empereur et le roi d'Espagne consentiraient à s'accommoder ensemble\,;
il ajouta qu'un projet n'était pas un traité, qu'avant d'en venir à la
conclusion, il y avait toujours beaucoup de choses à changer dans un
premier plan\,; que, lorsqu'elles en seraient à un certain point, le roi
de Sicile en aurait une entière communication. L'envoyé fit en cette
occasion les protestations que tout ministre croit être du goût de son
maître en pareille conjoncture. Il dit que jamais ce prince ne plierait
pour quelque raison que ce pût être quand il s'agirait de son honneur,
de son avantage, de celui de sa maison\,; que, plutôt que d'y souffrir
volontairement le moindre préjudice, il s'exposerait à toute sorte de
péril\,; que, s'il y succombait, la honte de sa perte tomberait
entièrement sur les garants des derniers traités. Provane employait
moins de paroles, mais il parlait plus fortement à Paris que La Pérouse
ne parlait à Londres\,; car il laissait entendre que, si son maître
manquait de forces ou de volonté, et ne défendait pas pied à pied la
Sicile, et s'il n'employait pas pour la conserver tous les moyens que
suggère un cas désespéré, il pourrait bien songer à des échanges très
douloureux pour la France. Un tel discours n'avait pas besoin
d'explications, car il était aisé d'entendre que l'échange qu'il voulait
faire craindre était celui des États de Piémont et de Montferrat, que le
roi de Sicile céderait à l'empereur pour avoir de lui le royaume de
Naples à joindre à la Sicile. Cellamare appuyait les menaces indirectes
de Provane. Il se plaignait qu'il ne trouvait que léthargie dans le
gouvernement. Il réitérait souvent et vivement ses sollicitations, mais
il trouvait que tout le monde criait à la paix, et que personne
n'appuyait alors les propositions de l'Espagne.

Peterborough, nouvellement sorti des prisons du pape, vint à Paris dans
ces circonstances. Cellamare ne manqua pas de le voir, et crut ne
pouvoir mieux employer son éloquence qu'à le persuader que l'Angleterre
devait éviter avec soin de rompre avec l'Espagne. Peterborough convint
de tout ce que lui dit Cellamare, il lui promit même de soutenir
fortement les intérêts de l'Espagne quand il serait en Angleterre. Il ne
se contraignit point sur les sujets qu'il avait de se plaindre de la
cour de Vienne\,; mais Cellamare s'aperçut cependant qu'il battait la
campagne, et qu'il y avait aussi peu de fondement à faire sur ses
raisonnements que sur ses promesses. Comme il perdait peu à peu
l'espérance d'interrompre le cours et d'empêcher le succès de la
négociation de Londres, il crut devoir faire de nouveaux efforts en
France pour détourner le régent de la suivre. Il représenta que le
voyage de Nancré était inutile, que ses propositions seraient mal
reçues. Il confiait à ses amis que l'air que la cour de Madrid respirait
n'était que de guerre. Monti, qui en arrivait nouvellement, parla en
même sens au régent. Il lui répondit qu'il avait nouvellement combattu
pour procurer au roi d'Espagne les conditions meilleures et les plus
avantageuses, et qu'il ne fallait pas exposer au hasard d'une guerre ce
qu'on pouvait obtenir par un traité.

Albéroni raisonnait différemment. Le duc de Parme lui représentait
souvent qu'il ne fallait pas se laisser endormir par les Impériaux, et
le persuadait aisément que, si l'Espagne leur donnait le temps de
s'établir en Italie, ils le feraient de manière que bientôt ils se
trouveraient maîtres d'exécuter toutes les résolutions violentes qu'il
leur plairait de prendre. Ce raisonnement était depuis longtemps celui
d'Albéroni, et, pour engager la France à s'y conformer, il disait
qu'elle suivait une politique non seulement fausse, mais pernicieuse,
même mortelle, en regardant comme un acte de prudence et d'habileté
d'éviter de prendre les armes hors les cas de nécessité forcée. Il
s'étendait en raisonnements fondés sur ses désirs, tout au plus sur ses
espérances, qu'il prétendait appuyées sur des secrets dont lui seul
avait la connaissance. Ces secrets étaient ses anciennes chimères de
l'éloignement de la paix des Turcs, de celui de la nation anglaise de
perdre son commerce qui ne permettrait pas au roi d'Angleterre de rompre
avec l'Espagne, de la jalousie secrète des Hollandais qui verraient sans
se remuer, même avec joie, attaquer et humilier l'empereur. C'était avec
quoi il ne se rebutait point de vouloir persuader au régent de prendre
les armes et de s'unir à l'Espagne et au roi de Sicile avec lequel
pourtant il n'était rien moins que d'accord. Il voulait cependant faire
en sorte, par la France, pour que la haine du refus des propositions de
paix ne tombât pas sur l'Espagne, mais sur les Impériaux. Il ne trouvait
aucune sûreté pour les garnisons espagnoles à mettre dans les États de
Toscane et de Parme contre l'enlèvement que les troupes de l'empereur en
pourraient faire d'un moment à l'autre. Il s'écriait contre la violence
qu'on voulait exercer contre des princes vivants et possédant justement
leurs États, tels que le grand-duc qui avait un fils, le duc de Parme
surtout, beau-père et oncle de la reine d'Espagne, lequel avait un frère
qui pouvait avoir des enfants, et qu'on voulait amuser et repaître de
visions éloignées, et laisser cependant les Allemands si bien prendre
leurs mesures qu'ils feraient échouer d'autres projets plus raisonnables
et plus capables de maintenir l'équilibre de l'Europe. Tous ces langages
furent tenus au régent par Cellamare, qui eut ordre de lui faire voir la
lettre d'Albéroni, et par Monti son ami de confiance, chargés tous deux
de n'oublier rien pour arracher le régent à la négociation de Londres et
l'unir à l'Espagne et au roi de Sicile, duquel ils prétendirent être
sûrs.

Albéroni, persuadé qu'il fallait marquer beaucoup de fermeté et de
confiance en ses forces pour intimider, envoya ordre à Monteléon de
s'expliquer beaucoup plus clairement qu'il ri avait fait sur la
destination de l'escadre anglaise.

Ainsi cet ambassadeur déclara que, si elle pas soit dans la
Méditerranée, il partirait sur-le-champ et retournerait en Espagne,
parce que le roi son maître regarderait cette démarche comme un premier
acte d'hostilité de la part du roi d'Angleterre. Monteléon eut ordre
d'instruire les membres du parlement, particulièrement les intéressés en
la compagnie de l'Asiento, des ordres qu'il avait reçus, et de leur dire
nettement qu'après tout ce que le roi d'Espagne avait fait pour le roi
Georges et pour la nation anglaise en des temps critiques, il avait lieu
d'attendre plus de reconnaissance de leur part\,; qu'il aurait au moins
dû compter sur leur indifférence\,; qu'il voulait enfin connaître ceux
qui seraient ses amis ou ses ennemis, et pour mettre l'épée à la main
s'il était nécessaire. Enfin, comme s'il y eût eu lieu de douter de
l'exactitude de Monteléon et de le soupçonner de timidité et d'intérêt
capable de le retenir ou de le ralentir, il reçut de nouveaux ordres
très positifs de parler sans crainte et sans incertitude, et d'autant
plus clairement que le roi d'Espagne savait qu'on faisait à Naples et à
Lisbonne de grands préparatifs pour l'escadre anglaise qui devait passer
dans la Méditerranée.

Beretti, ambassadeur d'Espagne en Hollande, eut ordre, de son côté, de
déclarer que le roi son maître ne se laisserait pas amuser par de
prétendus médiateurs ni par des propos de paix dont on répandait les
conditions dans le monde sans toutefois que Sa Majesté Catholique en eût
encore la moindre connaissance\,; mais que certainement ce serait se
tromper que de croire une pareille démence, comme la république de
Hollande se tromperait elle-même si elle laissait à la maison d'Autriche
la supériorité que les traités d'Utrecht lui avaient procurée. Albéroni
s'abandonnait à ses vanteries sur le bon état où il avait déjà mis
l'Espagne, qui ne craindrait plus personne dans deux ans. Ses discours
annonçaient bien plus la guerre que la paix. Ses préparatifs se
poussaient avec la plus grande diligence et le plus impénétrable secret.
Il détestait la paix d'Utrecht, il soutenait que le feu roi n'avait
point de pouvoir légitime pour faire tomber comme il avait fait tout le
poids du traité sur le roi, son petit-fils, et que le consentement qu'y
avait donné ce prince n'avait point été libre, mais forcé par une juste
crainte pour le roi son grand-père\,; respect si imprimé dans son coeur
qu'il lui aurait donné sa femme et ses enfants, s'il les lui eût
demandés, avec la même docilité qu'il avait cédé la Sicile. Il ajoutait
que les souverains étaient toujours mineurs, maîtres par conséquent de
se délivrer des violences qu'ils avaient souffertes quand la Providence
en faisait naître les occasions. La cession de la Sicile, citée par
Albéroni comme un exemple de la complaisance du roi d'Espagne pour le
roi son grand-père, ne fut pas regardée si simplement par l'abbé del
Maro, ambassadeur de Sicile à Madrid. Il soupçonnait depuis longtemps la
cour d'Espagne de former des desseins sur ce royaume, et il persista
toujours dans sa pensée, quoique l'opinion publique fût que la
destination de la flotte fût pour Naples. On disait même que le dessein
était d'attaquer cette capitale, sans s'amuser à Gaëte ni à Capoue. On
prévoyait cependant que la France et l'Angleterre ne le souffriraient
pas tranquillement, et que, s'il était impossible de porter l'Espagne à
un accommodement, ces deux puissances prendraient si bien leurs mesures
par mer et par terre, qu'elles feraient échouer les projets de
l'Espagne. Albéroni aurait bien voulu détruire cette opinion du public
en lui laissant croire qu'il y avait entre la France et l'Espagne une
intelligence secrète\,; mais il ne put le tromper. Il réussit mieux à
lui cacher son véritable projet\,; en sorte que bien des gens crurent
qu'il pourrait tourner ses armes contre le Portugal, autant que les
porter en Italie. Albéroni cependant vantait la puissance de l'Espagne,
qui avait sur pied quatre-vingt mille hommes, une bonne marine, ses
finances en bon état, et continuait ses déclamations et ses péroraisons
contre les propositions des médiateurs, et pour persuader la nécessité,
la facilité et les grands fruits de l'union armée de la France, avec
l'Espagne.

Le voyagé prochain de Nancré à Madrid paraissait moins une disposition
pour rétablir la bonne intelligence entre les deux cours qu'un moyen que
celle de France voulait tenter pour déclarer au roi d'Espagne que, s'il
n'acceptait le projet concerté avec l'Angleterre, son refus produirait
une rupture ouverte entre la France et lui. Mais Albéroni, persuadé
qu'il devait en cette conjoncture tenir et montrer bonne contenance,
disait que nonobstant tout ce qui pourrait arriver, le roi d'Espagne
suivrait son projet\,; que, s'il ne réussissait pas, il en serait quitte
pour se retirer sur son fumier où il attendrait des conjonctures plus
favorables. Enfin la résolution était prise de ne faire aucun
accommodement avec l'empereur. Monti eut ordre d'Albéroni de le dire au
régent et de l'assurer qu'avec un peu de temps il verrait des
changements dans les mesures qu'il avait prises avec le roi Georges, que
le temps ferait aussi que l'amitié du roi d'Espagne serait recherchée,
et d'autres pareilles vanteries. Albéroni comptait sur la neutralité au
moins de la Hollande. Beretti, pressé de plaire et de se faire valoir,
l'en assurait. Il lui mandait l'assurance qu'il en avait eue de Santen,
nouveau bourgmestre d'Amsterdam, que cette ville n'admettrait rien
contre le service du roi d'Espagne, et qu'il en avait averti Buys et le
Pensionnaire pour les contenir, parce qu'il les savait tous deux très
attachés à l'Angleterre et à la maison d'Autriche. La faiblesse où se
trouvait cette république, la difficulté de fournir à un armement très
nécessaire pour la mer Baltique par les dettes immenses qu'elle avait
contractées pendant la guerre terminée par la paix d'Utrecht, lui
rendaient les levées de troupes impossibles, à ce que prétendait
Beretti. Ces mêmes raisons lui ôtaient aussi toute espérance de porter
les États à attaquer l'empereur, et c'est ce qui redoublait le désir
d'Albéroni que la France leur en donnât l'exemple. Cellamare ne le
laissa pas dans l'abus de cette espérance\,: il lui manda que, quelques
bonnes dispositions que le régent eût fait paraître en différentes
occasions pour l'Espagne, son but n'avait jamais varié sur la
conservation de la paix, à quelque prix que ce pût être\,; que ce
n'était que pour gagner du temps qu'il avait quelquefois flatté le roi
d'Espagne d'espérances agréables\,; que le moyen d'éviter ces pièges
était d'obliger Nancré de s'expliquer tout en arrivant et clairement, et
de ne pas remettre à son retour à Paris la décision des affaires.
Cellamare crut qu'il était du service du roi son maître, d'en parler
comme de chose déjà décidée. Il publia que le roi d'Espagne se vengerait
enfin des outrages qu'il avait reçus, et qu'il soutiendrait ses droits
quand même il serait abandonné de ceux dont il devait naturellement et
raisonnablement attendre du secours. Provane, qui le secondait alors,
alla plus loin. Il voulait que le roi d'Espagne demandât passage par la
France pour cinquante mille hommes qu'il enverrait défendre l'Italie\,;
mais Cellamare y trouva trop de rodomontade, et crut qu'il fallait ne
dire que ce qu'on était à peu près en état de faire. Le bruit se
répandit néanmoins que ce passage était demandé pour vingt-cinq mille
hommes. Cellamare, sans appuyer ni démentir ce bruit, dit à Nancré avant
son départ qu'il ne pouvait faire que de mauvais augures de la
négociation dont il était chargé.

\hypertarget{chapitre-xvii.}{%
\chapter{CHAPITRE XVII.}\label{chapitre-xvii.}}

1718

~

{\textsc{Albéroni continue à poursuivre Giudice\,; lui fait redoubler
les ordres d'ôter les armes d'Espagne de dessus la porte de son
palais.}} {\textsc{- Malice et toute-puissance de ce premier ministre.}}
{\textsc{- État personnel du roi d'Espagne.}} {\textsc{- Manèges du pape
et d'Albéroni sur les bulles de Séville et sur le neveu d'Aldovrandi.}}
{\textsc{- Avidité et prodigalité du cardinal Ottobon.}} {\textsc{-
Avidité et dérèglement des neveux du pape.}} {\textsc{- Tracasseries à
cette occasion, où Giudice se barbouille.}} {\textsc{- Propos, mémoires,
menaces, protestation, forte lutte par écrit entre Acquaviva et le pape
sur le refus des bulles de Séville.}} {\textsc{- Querelle d'Acquaviva
avec le gouverneur de Rome.}} {\textsc{- Hauteur et faiblesse du roi
d'Espagne à l'égard de Rome.}} {\textsc{- Adresse d'Aldovrandi à servir
Albéroni.}} {\textsc{- Le pape embarrassé sur deux ordres venus
d'Espagne.}} {\textsc{- Giudice se déchaîne contre Albéroni, et Giudice
et Acquaviva l'un contre l'autre.}} {\textsc{- Albéroni se méfie de tous
les deux.}} {\textsc{- Del Maro seul va droit au but du dessein
militaire d'Albéroni.}} {\textsc{- Manèges d'Albéroni, résolu à la
guerre, à Londres et à Paris\,; s'ouvre à Cellamare.}} {\textsc{-
Remises et avis d'Albéroni au duc de Parme\,; se plaint à l'abbé Dubois,
par Monteléon, de l'ignorance où on le tient des conditions du traité,
et fait des reproches.}} {\textsc{- Plaintes amères contre le régent des
agents anglais entièrement impériaux.}} {\textsc{- Leur audace et leur
imposture.}} {\textsc{- Sage adresse de Monteléon pour oser donner de
bons conseils à Albéroni.}} {\textsc{- Singulières ouvertures de l'abbé
Dubois à Monteléon.}} {\textsc{- L'empereur veut les successions de
Parme et de Toscane pour le duc de Lorraine\,; on leurre le duc de
Modène.}} {\textsc{- Penterrieder déclare à Londres, à l'envoyé de
Sicile, que l'empereur veut la Sicile absolument.}} {\textsc{- Il
indispose tant qu'il peut cet envoyé et son maître contre le régent.}}
{\textsc{- Caractère de Monteléon.}} {\textsc{- Le grand-duc et le duc
de Parme envoient à Londres faire des représentations inutiles.}}
{\textsc{- Désirs des Florentins de retourner en république, et non sans
quelque espérance.}} {\textsc{- Monteléon reçoit des ordres réitérés de
faire des menaces sur l'escadre\,; les communique à Stanhope.}}
{\textsc{- Adresse de celui-ci pour l'amuser.}} {\textsc{- Adresse de
l'autre pour amener l'Espagne au traité.}} {\textsc{- Points sensibles à
Vienne sur le traité.}} {\textsc{- Monteléon, persuadé du danger de
rompre pour l'Espagne, n'oublie rien pour l'en dissuader.}} {\textsc{-
Bruits d'une révolution prochaine en Angleterre, où le ministère est
changé.}} {\textsc{- Ruse inutile d'Albéroni pour opposer la nation
anglaise à son roi.}} {\textsc{- Mécompte de Monteléon.}} {\textsc{-
Cellamare plus au fait.}} {\textsc{- Stairs s'explique nettement sur
l'escadre.}} {\textsc{- Mouvements contraires dans le parlement
d'Angleterre.}} {\textsc{- Nuages sur la fermeté de la cour de Vienne
tournés à Londres avec adresse.}} {\textsc{- Demandes bien mesurées du
grand-duc.}} {\textsc{- Effort d'Albéroni auprès du régent.}} {\textsc{-
Conduite publique et sourdes cabales de Cellamare.}} {\textsc{- Il
cherche d'ailleurs a remuer le nord contre l'empereur.}}

~

Stairs sortit de son naturel insolent autant qu'il put pour tâcher, par
les exhortations et les représentations les plus douces, de persuader
Cellamare, puis par les menaces en ne se contraignant plus. Ce manège
fut inutile. Cellamare savait trop bien que ce serait se perdre auprès
d'Albéroni que montrer la moindre inclination à la paix\,; il n'avait
songé qu'à lui plaire dès le commencement de la fortune de ce premier
ministre, il n'avait garde de ne pas continuer. Il y était d'autant plus
circonspect qu'il craignait toujours de voir retomber sur lui la haine
implacable d'Albéroni contre son oncle le cardinal del Giudice, à qui il
ne cessait de chercher des raisons et des prétextes de lui faire sentir
des marques publiques de l'indignation qu'il inspirait pour lui au roi
d'Espagne. Il accusait Giudice d'entretenir à Madrid des correspondances
séditieuses et criminelles. On avait même emprisonné quelques
particuliers sous ce prétexte. Albéroni se plaignit à Cellamare que son
oncle était incorrigible, et lui manda d'un ton d'amitié qu'il avait
fallu, du temps que Giudice était à Madrid, les bons offices de
quelqu'un qu'il ne voulait pas nommer, et la bonté des maîtres pour les
empêcher de prendre contre lui des résolutions violentes. Leurs Majestés
Catholiques, continuait-il, étaient irritées de son opiniâtreté à
différer d'obéir à leurs {[}ordres{]} d'ôter à Rome les armes d'Espagne
de dessus la porte de son palais\,; il en fit craindre les suites à
Cellamare, et lui conseilla d'avertir son oncle de ne pas s'exposer plus
longtemps à l'insulte de les voir arracher avec violence\,; il n'en,
fallait pas tant pour intimider Cellamare.

Le courroux d'Albéroni était d'autant plus à craindre que tout le monde
le regardait comme le maître absolu et unique de l'Espagne. Il laissait
au roi le seul extérieur de sa dignité royale, et sous son nom et sans
lui disposait absolument des affaires. Soir et matin le cardinal lui
présentait tous les jours une liasse de papiers qui demandaient sa
signature. Quelquefois il disait en peu de mots la substance de quelques
affaires principales, mais jamais il n'entrait dans le détail, et jamais
il n'en faisait de lecture. Après un tel compte si superficiellement
rendu, la stampille était apposée sur les expéditions. La maladie du roi
était le prétexte de lui donner si peu de connaissance des affaires. Sur
ce même prétexte, l'entrée de son appartement était interdite à tous
ceux dont on voulait juger que la présence lui donnerait la moindre
contrainte. Il était donc réduit à passer ses jours entouré de médecins
et d'apothicaires, et bannissant toute autre cour, et se crevant
toujours de manger. Il s'amusait les soirs à les voir jouer, ou de jouer
avec eux. Ces sortes de gens ne faisaient point d'ombrage au cardinal,
et ne pouvaient attaquer son pouvoir despotique. Tout autre personnage
plus élevé lui était suspect. Il parut même qu'il commençait à se défier
du duc de Popoli, quoique le plus soumis et le plus rampant de ceux qui
voulaient être considérés comme dépendants de lui. C'est qu'il ménageait
trop les Espagnols. Il fut même accusé d'avoir des liaisons secrètes
avec quelques-uns des principaux de la nation. On alla jusqu'à dire
qu'il inspirait des sentiments peu favorables au prince des Asturies,
dont il était gouverneur, pour le cardinal. Il y eut cependant lieu de
croire dans les suites qu'ils s'étaient raccommodés.

Malgré le grand pouvoir d'Albéroni, malgré le respect que la cour de
Rome a toujours témoigné pour les ministres en faveur, en quelque cour
que ce soit, on peut encore ajouter malgré la déclaration publique de ce
cardinal pour la constitution et contre les maximes de France, le pape
continuait à lui refuser les bulles de Séville. Ce refus était fondé en
apparence sur les raisons de se plaindre du gouvernement d'Espagne, en
effet sur la crainte de déplaire aux Allemands. Albéroni même n'eut pas
lieu d'en douter, car le pape lui offrit\,; secrètement de lui faire
toucher les revenus de Séville s'il voulait bien faire suspendre les
instances du roi d'Espagne pour les bulles, et différer pendant quelque
temps, sa translation à cet archevêché. Cette complaisance pour les
Allemands, qu'Albéroni traitait de bassesse, n'était pas la seule qu'il
reprochait à Sa Sainteté. Retenue par la crainte de l'empereur, elle
n'osait tenir, la parole qu'elle avait donnée à la cour d'Espagne d'y
envoyer le neveu d'Aldovrandi porter le bonnet à Albéroni. C'était un
nouveau sujet de plainte qu'Albéroni mettait sur le compte de Leurs
Majestés Catholiques, en faisant au pape les compliments les plus soumis
et les plus dévoués sur le sien. Mais le roi et la reine d'Espagne
étaient inflexibles, et avaient, disait-il, déclaré pie nul autre que ce
neveu d'Aldovrandi ne serait reçu en Espagne pour apporter ce bonnet, et
lé cardinal Acquaviva eut ordre de faire entendre au pape qu'on pourrait
se porter à faire sortir son nonce de l'Espagne. Albéroni citait le P.
Daubenton comme premier témoin du peu qu'il s'en était fallu que cette
résolution ne fut prise, et plaignait le sort d'Aldovrandi. Le cardinal
disait que, si jamais le bref dont il était question arrivait à Madrid,
il donnerait le dernier coup pour achever la ruine de ce pauvre prélat
qui avait servi le pape avec tant d'honneur et de probité, et tant
d'utilité pour le saint-siège. Il lui rendait témoignage de la
préférence qu'il donnait à son attachement pour le pape à toute
satisfaction personnelle, par les instances que ce nonce avait faites à
Leurs Majestés Catholiques de lui permettre de supplier Sa Sainteté de
nommer tout autre que son neveu pour apporter cette barette, mais
qu'elles avaient répondu que cette affaire n'était plus la sienne, mais
la leur, et que toutes ses instances seraient inutiles. Albéroni ne
voulant pas se prendre directement au pape de tous les mécontentements
qu'il en avait, attribuait sa partialité pour les impériaux aux conseils
du cardinal Albane. Il l'accusait de penser trop au présent, de
s'aveugler sur l'avenir, de ternir la gloire du pontificat de son oncle
au lieu de profiter des exemples passés qu'il avait devant les yeux, qui
suffisaient pour corriger les neveux des papes et les rendre sages. En
même temps il cherchait à gagner, mais par de simples compliments et des
assurances de services, le cardinal Ottobon, neveu du feu pape Alexandre
VIII, protecteur des affaires de France à Rome et vice-chancelier de
l'Église.

Ottobon s'était attiré ces compliments par les avances qu'il avait
faites dans l'espérance de grossir, par le secours de l'Espagne, les
grands revenus qu'il tirait de France, soit en pensions ou en bénéfices
qui, sans compter ses charges à Rome et ses bénéfices en Italie, ne
suffisaient pas encore à ses dépenses. Les neveux du pape n'étaient pas
moins avides que ceux qui les avaient précédés, ni moins sujets aux
autres défauts que Rome avait souvent reprochés à ceux que la fortune
d'un oncle avait élevés dans les premiers postes de l'État, et donnés
comme en spectacle aux yeux du public. Le pape, plein de bonnes
intentions, principal auteur de la bulle contre le népotisme, faite par
son prédécesseur, se flattait que ses neveux, qu'il n'avait pas voulu
reconnaître, se feraient une loi inviolable d'imiter sa modération\,;
mais ils ne pensaient pas comme lui. Les passions de toute espèce et le
désir de profiter du temps présent, dérangeaient les conseils de leur
oncle, et pour lui épargner des chagrins inutiles, on lui cachait avec
soin leur dérèglement. Mais il était difficile que ces sortes de secrets
fussent fidèlement gardés. On dit qu'une âme simple découvrit au pape le
désordre de ses neveux\,; que le cardinal Albane fut fort chargé\,; que
don Alexandre, le troisième des frères, fut dépeint avec des couleurs
encore plus noires. Ils essayèrent de découvrir leur accusateur, et le
soupçon répandu sur plusieurs, tomba principalement sur le cardinal del
Giudice. Quoique dans un âge avancé, il se permettait un attachement de
jeune homme pour la princesse de Carbognano, et lui seul ne remarquait
pas le ridicule que le reste du monde voyait évidemment dans ses
empressements pour elle. Don Alexandre Albane aimait la connétable
Colonne\,; une querelle particulière entre ces deux dames porta le
cardinal del Giudice à venger la princesse de Carbognano, en avertissant
indirectement le pape des empressements de don Alexandre pour la
connétable Colonne. Ce fut peut-être faussement qu'on accusa Giudice de
cet indigne personnage, car il avait beaucoup d'ennemis\,; et depuis
qu'il était sorti d'Espagne, ceux qui voulaient plaire au cardinal
Albéroni ne l'épargnaient pas.

Acquaviva, traitant de frivoles les causes alléguées du refus des bulles
de Séville, entreprit de les détruire\,; il prétendit que le roi
d'Espagne avait été obligé de tenir la conduite qu'il avait tenue pour
arrêter les pratiques de ses sujets rebelles, et empêcher les troubles
qu'ils voulaient exciter dans sou royaume sous ombre de la juridiction
et des immunités ecclésiastiques, et que, quand même son ministre
Albéroni lui aurait donné de mauvais conseils là-dessus, cette raison
n'en était pas une de lui refuser des bulles, puisqu'elles ne le
pouvaient être dans les règles que pour mauvaises moeurs ou mauvaise
doctrine. Il ajouta que, si le pape tenait consistoire sans y proposer
l'archevêque de Séville, il protesterait publiquement, et qu'il
appellerait en cause tous les princes qui ont droit de nommer aux
bénéfices de leurs États, que cette affaire ne regardait pas moins que
le roi d'Espagne. Ce mémoire, qu'Acquaviva fit remettre au pape, fut
accompagné de menaces de rupture et de protestations dont il fut fort
irrité. Il refusa le délai du consistoire, parce qu'il y fallait
proposer l'évêché de Nankin, en expédier les bulles, les envoyer
diligemment à Lisbonne où les vaisseaux destinés pour les Indes étaient
prêts à faire voile. Il dit qu'il proposerait Séville quand le roi
d'Espagne lui aurait donné satisfaction sur ses sujets de plainte\,; et
comme il craignit qu'Acquaviva ne rendît pas un compte assez fidèle de
ce qu'il lui avait fait dire, il chargea particulièrement son nonce à
Madrid de bien expliquer ses intentions à Albéroni\,; que ce n'était pas
un refus, mais un délai pour lui donner le temps d'agir auprès du roi
d'Espagne pour lui procurer, de Sa Majesté Catholique, les justes
satisfactions qu'il attendait de sa piété\,: en même temps de bien faire
entendre qu'il ne consultait en cela que sa conscience, et nullement la
satisfaction des Allemands, en faisant de la peine au roi d'Espagne,
comme Acquaviva le lui avait fait reprocher.

Ce dernier cardinal, également insensible aux plaintes et aux
justifications du pape, fondé sur quelques exemples de protestations en
pareil cas, et récemment en 1710, à l'occasion d'une translation de
l'archevêché de Saragosse à l'archevêché de Séville, fit remettre l'acte
de sa protestation entre les mains de l'auditeur du pape, par Herrera,
auditeur de rote pour la Castille. Le pape, qui avait auparavant dit à
Acquaviva qu'il pouvait protester, ne laissa pas d'être fort irrité. Il
prétendit qu'il y avait plusieurs propositions fausses dans ce que ce
cardinal avançait dans sa protestation, et déclara qu'il avait résolu de
disposer des revenus de Séville si utilement, que personne ne pourrait
dire qu'il en engraissât la chambre apostolique, ni {[}qu'il en eût{]}
fait un usage contraire aux saints canons. Il fit remettre à Acquaviva
une réponse par écrit à sa protestation, dont le point principal allait
à faire voir que les papes ne sont pas obligés d'admettre les
nominations des princes dans un consistoire plutôt que dans un autre.
Acquaviva répondit à cet article qu'il était vrai que le pape n'était
pas obligé à tenir un consistoire le jour même qu'une nomination lui
était présentée\,; mais que, lorsqu'il tenait consistoire, il ne
pouvait, sans donner de grands sujets de plaintes légitimes, différer
l'effet de la nomination, à moins qu'il n'y eût des empêchements
canoniques\,; autrement, qu'il ne tiendrait qu'à lui d'éluder les grâces
que les princes faisaient à leurs sujets, et par conséquent il
disposerait indirectement des bénéfices dans les royaumes et dans les
pays étrangers. Ce cardinal se plaignit de plus que le pape lui avait
manqué de parole. La conduite de Sa Sainteté envers l'Espagne lui sembla
pleine d'ingratitude, car il paraissait, en ce même temps, un grand
empressement de plaire à Rome de la part de quelques évêques d'Espagne,
et celui de Badajos s'était signalé\,; ce qui n'empêchait pas sa
partialité pour les Impériaux marquée dans les plus grandes comme dans
les plus petites affaires.

Falconieri, gouverneur de Rome, fort impérial, voulant montrer de
l'égalité, fit passer des sbires aux environs du palais de l'empereur,
puis autour du palais d'Espagne. Cette dernière marche produisit une
querelle. En des soldats qu'Acquaviva y entretenait pour se garantir des
violences des Allemands fut arrêté et mis en prison par les sbires.
Acquaviva en demanda satisfaction. Il eut pour réponse qu'elle était
faite par la délivrance du prisonnier. Piqué contre le pape, et
connaissant son caractère timide et faible, il crut devoir repousser la
force contre la force, et se venger sur les auteurs de l'emprisonnement
de son soldat, si la satisfaction qu'il en avait demandée ne lui était
accordée de bonne grâce. Il en demanda la permission en Espagne, et en
l'attendant il résolut d'augmenter les gardes du palais d'Espagne, et de
le mettre en état de défense s'il était attaqué. Il crut aussi qu'il
était bon pour le service du roi d'Espagne, d'entretenir cette querelle,
les princes ayant toujours besoin, de prétextes pour rompre quand il
leur convient d'en venir à cette extrémité. La France avait fait insérer
les droits de la maison Farnèse dans le traité de Pise, conclu pour
satisfaire à l'insulte faite par les Corses de la garde du pape au duc
de Créqui, ambassadeur de France. On pouvait peut-être tirer de grands
avantages de la faiblesse de cette cour toujours éloignée d'accorder des
satisfactions, mais souple et disposée à souffrir patiemment toutes les
impertinences que les étrangers lui veulent faire supporter. C'était
ainsi qu'Acquaviva s'en expliquait, et il en donnait pour exemple
l'issue de l'emprisonnement du comte de Peterborough. Quoique Albéroni
pensât aussi de même, la conduite du roi d'Espagne n'était pas uniforme
à l'égard de Rome. En même temps qu'il soutenait ses droits avec
fermeté, et qu'il était sur le point de rompre avec Rome, plutôt que
d'en souffrir quelque atteinte à la prérogative de sa couronne, ce
prince avait reçu l'absolution, qu'il avait eu la faiblesse de faire
demander secrètement au pape, des censures que Sa Sainteté prétendait
qu'il avait encourues pour avoir violé par ses décrets l'immunité
ecclésiastique.

En même temps le conseil de Castille prenait feu sur les affaires de
Rome. Les amis et les protecteurs de Macañas autrefois procureur
général, qu'ils disent fiscal de ce conseil, faisaient de grands
mouvements pour qu'il lui fût permis de retourner à Madrid, d'où il
avait été chassé pour avoir signalé son zèle et sa capacité à soutenir
les droits du roi d'Espagne contre les prétentions de Rome, par des
écrits d'autant plus désagréables à cette cour qu'ils étaient pleins de
raisons et de preuves solides pour maintenir la cause qu'il défendait.
Le grand nombre et la considération de ses amis alarma Aldovrandi. Il
craignait les suites de leur union et de leurs représentations. Il
paraissait déjà quelques écrits capables d'altérer là soumission sans
bornes que les Espagnols avaient pour la cour de Rome. Ces questions
étaient mauvaises à traiter dans un pays où on avait toujours regardé
comme un crime de former des doutes, encore plus des disputes sur la
plénitude de puissance et sur l'infaillibilité du pape. Aldovrandi, dont
la politique avait toujours été. de s'appuyer pour avancer sa propre
fortune du crédit du premier ministre, eut recours à lui pour arrêter le
cours du mal qu'il prévoyait, et représenta au pape le besoin qu'il
avait de ménager un homme aussi puissant, qui avait toujours été zélé
pour le saint-siège, dont l'autorité seule pouvait faire cesser des maux
naissants qu'on aurait peine à arrêter dans la suite, lequel pouvait
enfin se dégoûter par les traitements personnels qu'il recevait de Sa
Sainteté, et grossir, aisément au roi d'Espagne les sujets des plaintes
qu'il croyait avoir d'elle.

Acquaviva venait de recevoir deux ordres d'Espagne qui embarrassaient le
pape\,: l'un de lui déclarer que, s'il accordait au marquis de
Sainte-Croix les honneurs de grand d'Espagne dont l'empereur lui avait
nouvellement conféré le titre, Sa Majesté Catholique regarderait cette
complaisance comme un nouveau sujet de dégoût et de plainte l'autre
regardait l'ordre que le roi d'Espagne avait donné au cardinal del
Giudice d'ôter de dessus la porte de son palais les armes d'Espagne
qu'il y avait, comme étant de la faction d'Espagne. Le pape avait montré
de la pente à favoriser ce cardinal. Il entrait dans les plaintes qu'il
faisait de la malice d'Albéroni et d'Acquaviva, et les accusait de
s'être liés ensemble pour attaquer son honneur et sa fidélité, et disait
qu'après avoir fait ses efforts de se procurer le repos, il tâcherait
enfin de se faire entendre, si ses ennemis prétendaient le pousser à
bout. Pour se venger d'Albéroni, il se déchaînait contre la chimère de
ses projets qui embraseraient l'Italie sans fruit pour le roi d'Espagne,
parce que, la France qui, à quelque prix que ce fût, voulait conserver
la paix, n'entrerait pas dans ses desseins. Tandis que d'intelligence
avec le régent, il vendait son maître pour l'obliger à confirmer ses
renonciations à la couronne de France, Acquaviva, non moins ardent de
son côté, accusait Giudice de s'entendre, avec la France par le cardinal
de La Trémoille qui avait été longtemps son plus intime ami. Il sut en
effet par cette voie que Giudice avait écrit au régent qu'il l'avait
supplié d'envoyer et d'appuyer auprès du roi d'Espagne la lettre qu'il
écrivait à ce monarque pour lui rendre compte de sa conduite et se
justifier des accusations faites contre lui. Le sentiment d'Acquaviva
était de lui renvoyer sa lettre sans l'ouvrir et passer en même temps un
décret dans les conseils d'Espagne pour le déclarer coupable de
désobéissance, et l'arrêter si jamais il était trouvé en pays de
l'obéissance du roi d'Espagne. Comme la haine d'un Italien ne se borne
pas aisément, Acquaviva voulait que toute la famille de Giudice se
ressentît de sa faute. Il proposa de procéder directement contre
Cellamare, protestant cependant par bienséance qu'il ne pouvait lé
croire capable de manquer de fidélité, quoique son oncle fût dans la
disgrâce, et qu'il attendît tout son bien de la part de la France. Après
les avoir attaqués l'un et l'autre sur l'honneur, la fidélité, les
qualités les plus essentielles, il continua d'attaquer encore Giudice
sur des sujets moins importants. Il prétendit qu'ayant passé quelques
jours à la campagne avec don Alexandre Albane, il l'avait trouvé
persuadé que Giudice était l'auteur des mauvais offices qu'on lui avait
rendus auprès du pape, à l'occasion de quelques galanteries avec la
connétable Colonne. La guerre était devenue plus vive entre elle et la
princesse de Carbognano, et l'extravagance de ces deux femmes préparait
Acquaviva au plaisir de voir entre elles des scènes dont Giudice et son
neveu le prélat seraient les victimes, parce que le pape, suivant sa
coutume, après avoir été mécontent de ses, neveux se raccommodait
facilement avec eux.

Giudice, de son côté, tâchait d'inspirer à la cour d'Espagne des
soupçons sur la fidélité d'Acquaviva. Un de ses neveux dans la prélature
parut à un bal que donnait l'ambassadeur de l'empereur\,; cela donna
lieu à Giudice de publier qu'il y avait bien des réflexions à faire sur
l'inclination que de tout temps Acquaviva avait témoignée pour le parti
impérial, et sur les sentiments qu'il conservait, quoique les instances
qu'il avait faites par le prince d'Avellino pour se réconcilier avec la
cour de Vienne n'eussent pas été admises. Albéroni se défiait presque
également de ces deux cardinaux. Le caractère de son esprit et de son
pays ne lui permettant pas d'avoir en qui que ce soit une confiance
absolue, toute la différence qu'il mettait entre l'un et l'autre était
que, Acquaviva servant actuellement le roi d'Espagne et voulant obtenir
des grâces pour sa famille, ménageait le premier ministre\,; qu'il ne
devait, au contraire, attendre nul ménagement de Giudice déclaré son
ennemi capital. Mais il s'agissait alors d'affaires plus importantes
pour l'Espagne que celles des querelles et des passions particulières de
ces cardinaux. On était au commencement de mars, le printemps
s'approchait\,: Albéroni redoublait ses soins et son application pour
hâter les préparatifs de guerre que le roi d'Espagne faisait par terre
et par nier.

Il n'était plus douteux qu'il ne voulût tenter le sort des armes\,; il
ne l'était pas aussi que l'Italie n'en fût l'objet, mais il était
incertain quelle partie d'Italie ce projet pouvait regarder. On
commençait à croire que c'était le royaume de Naples. Le soin que la
cour eut d'en appuyer sourdement le bruit confirma del Maro dans ses
premiers soupçons que c'était la Sicile qu'Albéroni voulait envahir.
D'autres parlaient de Livourne et du duc de Berwick, pour en commander
l'expédition, si la France en était d'accord ou voulait bien seulement
fermer les yeux. Parmi ces divers bruits, Albéroni laissait en suspens
toutes les affaires que l'Angleterre poursuivait en Espagne. Il ne
s'expliquait point sur le traité que le roi d'Angleterre proposait, et
comme il prévoyait des dispositions de la cour d'Angleterre qu'il aurait
bientôt lieu de se plaindre d'elle, il suspendait toutes les affaires
particulières qui regardaient le commerce de cette nation. Comme il ne
voulait pas encore faire paraître qu'il fût directement opposé au
traité, il chargea Monteléon de dire à l'abbé Dubois, lors à Londres,
qu'il prendrait une entière confiance en Nancré quand il serait à
Madrid\,; qu'il souhaitait aussi que l'abbé Dubois sortît avec honneur
et gloire de la négociation qu'il avait entreprise\,; mais ce qu'il
ferait serait inutile s'il n'assurait un parfait équilibre à l'Europe.
Monti, ami particulier d'Albéroni, eut en même temps ordre d'assurer le
régent que Nancré, venant de sa part en Espagne, y serait le bienvenu,
et qu'on écouterait ses commissions. C'étaient des compliments. Albéroni
avertit Cellamare que les réponses qu'il avait faites de la part du roi
d'Espagne, seraient les mêmes que Nancré recevrait à Madrid, en sorte
qu'il y trouverait, pour ainsi dire, le double de Cellamare\,; que
l'Angleterre avait pris une mauvaise habitude aux conférences d'Utrecht,
et que séduite par là douceur qu'elle avait trouvée à régler le sort de
l'Europe, elle se croyait en droit de dépouiller et de revêtir à sa
fantaisie les princes de différents États\,; car il jugeait que tout
accommodement entre l'empereur et le roi d'Espagne ne serait que plâtré,
et qu'il n'était proposé que par ceux qui croyaient que cette apparence
de pacification convenait à leurs fins particulières. Il prétendait même
que la cour de Vienne était peu satisfaite du projet du roi Georges\,;
qu'elle reprochait à ce prince de proposer de vains accommodements au
lieu de satisfaire aux engagements qu'il avait contractés de secourir
l'empereur quand ses États d'Italie seraient attaqués. Albéroni comptait
beaucoup sur la nation anglaise, intéressée à maintenir, l'union et le
commerce avec l'Espagne, et nullement à contribuer par des ligues à
l'agrandissement de l'empereur.

Comme il fallait l'empêcher de surprendre des places qui pouvaient le
plus étendre et affermir sa puissance en Italie, il fit remettre à Gênes
vingt-cinq mille pistoles à la disposition du duc de Parme pour mettre
Parme et Plaisance hors d'insulte et d'entreprise, exhortant le duc de
Parme dont il regardait chèrement les intérêts de travailler à ses
places avec tant de sagesse qu'il ne donnât aucune prise aux Impériaux
de lui faire querelle sur ses justes précautions. Il accompagna cela des
discours les plus pacifiques. Monteléon eut ordre de dire à l'abbé
Dubois qu'apparemment le conseil qu'il avait donné au régent n'avait pas
été suivi, puisqu'il n'avait communiqué au roi d'Espagne aucune des
conditions du traité que la France et l'Angleterre avaient remis à
l'empereur pour l'examiner\,; que néanmoins Son Altesse Royale aurait dû
se souvenir de la déférence que Sa Majesté Catholique avait eue pour
elle en suspendant au mois d'août dernier l'exécution infaillible de ses
projets (on a déjà remarqué ailleurs que l'embarquement ne s'étant pu
faire à temps à Barcelone par faute d'une infinité de chose, Albéroni en
avait couvert l'impuissance d'une complaisance, après laquelle il
courut, et qu'il se serait bien gardé d'avoir s'il avait pu exécuter ce
qu'il avait projeté)\,; que le roi d'Espagne avait eu la complaisance de
laisser à la France et à l'Angleterre le temps de lui procurer une juste
satisfaction, et d'assurer l'équilibre\,; que sept mois passés sans la
moindre probabilité de parvenir à cette fin avertissaient suffisamment
l'abbé Dubois de procéder dans sa négociation avec plus de précaution
qu'il n'avait fait jusqu'alors, puisqu'il était évident que l'unique
objet de l'empereur était de tirer les choses en longueur jusqu'à ce
qu'il vît quel pli prendrait la négociation de la paix avec le Turc.
Albéroni ajoutait force raisonnements historiques et politiques à mettre
dans la bouche de Monteléon pour l'abbé Dubois, afin de lui inspirer
toute la crainte possible de la grandeur de l'empereur, et tout le désir
de joindre la France à l'Espagne ne pour s'y opposer.

Pendant que le premier ministre d'Espagne déclamait ainsi contre la
conduite et la politique du régent, les ministres d'Angleterre se
plaignaient, de leur côté, de l'opiniâtreté de ce prince à demander des
conditions trop avantageuses pour le roi d'Espagne, et surtout de la
manière dont il insistait sur la succession de la Toscane. Cet article
était celui qui déplaisait le plus à Vienne, à qui les agents que le roi
d'Angleterre employait dans cette négociation étaient entièrement
dévoués et livrés\,; l'un était Saint-Saphorin, Suisse, dont il a déjà
été parlé plusieurs fois, qui résidait à Vienne avec commission de Sa
Majesté Britannique\,; le second était Schaub, Suisse aussi, et du
canton de Bâle, qui avait été secrétaire du comte de Stanhope. Outre ces
deux personnages, Robetton, réfugié français, en qui le roi d'Angleterre
témoignait beaucoup de confiance, avait une part intime dans la
négociation. On croyait que Schaub et Saint-Saphorin recevaient pension
de l'empereur\,; mais soit que ce bruit fût vrai ou non, il est certain
que ces trois hommes blâmaient également le régent de n'être pas assez
complaisant pour les prétentions et les demandes de la cour de Vienne,
et qu'ils répétaient souvent qu'il ne devait pas espérer de conclure,
si, persistant à soutenir l'Espagne, il laissait le temps à l'empereur
de signer la paix avec les Turcs. Ils disaient que les Allemands se
défiaient de la fermeté du régent\,; que le prince Eugène,
particulièrement plus éclairé qu'un autre, relevait tous les pas qu'il
faisait en faveur de l'Espagne\,; que Bonac, ambassadeur de France à la
Porte, cabalait pour empêcher les Turcs de faire la paix\,; que ses
démarches étaient si publiques que le comte de Koenigseck aurait ordre
de s'en plaindre au nom de l'empereur, et même d'en demander
satisfaction. Ils ajoutaient que le régent, non content de faire agir
l'ambassadeur de France à Constantinople, avait de plus donné au roi
d'Espagne un officier français pour le faire passer en Turquie, et pour
y seconder, de la part de l'Espagne, les manèges de Bonac\,; qu'il
fallait donc conclure de ce procédé peu sincère que les branches de la
maison de France seraient toujours unies entre elles, et constamment
liées contre les puissances qui pourraient leur faire ombrage. Ils
blâmaient la mauvaise foi de la cour de France, et vantaient la candeur
et la droiture de celle de Vienne, et reprochaient au régent les choses
où il n'avait point de part\,; par exemple, qu'un officier grison, nommé
Salouste, autrefois dans le service du roi, était alors dans son pays,
qu'il y avait été envoyé par le duc du Maine\,; et que sous son nom cet
officier travaillait à renouveler en faveur du roi d'Espagne le
capitulat de Milan, même à lever un régiment grison pour le service de
Sa Majesté Catholique. Non seulement la cour de Vienne se plaignait de
ces envois, où certainement le régent n'avait nulle part, mais elle
prétendait encore que l'abbé Dubois, pendant le séjour qu'il avait fait
à Paris, s'était laissé gagner ou intimider par la faction espagnole.
Saint-Saphorin avertit la cour d'Angleterre que l'abbé Dubois n'aurait
plus à son retour à Londres le même empressement de conclure\,; que,
s'il pouvait même, il ferait naître des incidents au traité. Quoique ces
soupçons fussent contraires {[}non seulement{]} à la vérité, mais même à
la vraisemblance, il arriva cependant que, l'abbé Dubois étant de retour
à Londres, Monteléon et lui parurent contents l'un de l'autre et agir de
concert.

Monteléon désirait en effet que le roi son maître prît de nouveaux
engagements avec l'Angleterre plutôt que de rompre avec cette couronne.
Il le souhaitait, et pour l'intérêt du roi d'Espagne et pour le sien
propre\,; mais il n'osait déclarer ses sentiments trop ouvertement au
cardinal Albéroni dont les sentiments opposés au traité lui étaient
parfaitement connus. Il tâchait donc de le ramener avec adresse, et pour
y réussir, il lui dépeignait l'abbé Dubois comme plein de bonnes
intentions pour les intérêts du roi d'Espagne. Monteléon comptait sur
les assurances qu'il en avait reçues que le régent n'approuverait ni ne
déclarerait les conditions du projet de traité avant de savoir les
intentions de Sa Majesté Catholique, voulant prendre avec elle les
mesures les plus convenables pour en assurer le succès\,; que c'était
dans ce dessein que Nancré était envoyé en Espagne. L'abbé Dubois
supposait qu'une ou deux conversations entre Albéroni et Nancré
suffiraient pour établir entre eux une confiance telle, qu'on pourrait
prendre un point fixe sur les conditions d'un accommodement raisonnable,
et convenir des moyens d'employer la force des armes si la cour de
Vienne ne voulait pas entendre à la négociation. Il regrettait cependant
le temps qu'il laissait échapper, se plaignant de perdre chaque jour du
terrain auprès des ministres anglais, et des moments d'autant plus
précieux qu'il est plus nécessaire {[}là{]} que partout ailleurs de
profiter de l'occasion à cause de l'inconstance de la nation très
conforme à son gouvernement. L'abbé Dubois se plaignait encore à
Monteléon du trop d'égard que les ministres de Hanovre avaient pour la
cour de Vienne, de la faiblesse et de la variété de sentiment des
ministres anglais toujours prêts à changer suivant leurs intérêts
particuliers. Il lui confia que Stanhope était le seul qui osât
présentement soutenir ouvertement les raisons de l'Espagne, et dire que
l'Angleterre ne lui devait jamais donner de justes soupçons ni sujet de
mécontentement à cause des inconvénients qui pouvaient en résulter pour
le commerce qui était l'idole de la nation.

Monteléon faisait bon usage de ces confidences, car en les rapportant,
il insinuait sous le nom d'un autre l'avantage que le roi d'Espagne
trouverait à concilier ses intérêts avec les idées des médiateurs. Il
représentait que, si Sa Majesté Catholique pouvait convenir d'un projet
avec Nancré, assurer dans sa branche les successions de Parme et de
Toscane, elle mettrait l'empereur dans son tort, parce que jamais les
ministres de ce prince n'accepteraient rien de raisonnable\,; qu'en ce
cas l'Espagne, unie avec la France et le roi de Sicile, aurait non
seulement toute la justice de son côté, mais que de plus elle
emploierait librement les armes pour forcer les Allemands à sortir
d'Italie, et que l'Angleterre, perdant tout prétexte de se mêler de la
querelle, serait obligée de demeurer neutre et indifférente. Monteléon
ajoutait que, si l'Espagne voulait faire la guerre en Italie, il serait
de la dernière importance de la commencer avant que celle de Hongrie fût
achevée. Il lui conseillait encore d'apaiser les plaintes des marchands
anglais sur le commerce d'Espagne, afin d'engager la nation à s'opposer
plus fortement dans les séances du parlement aux résolutions qu'on
pourrait y proposer à prendre au préjudice de l'Espagne. Il soutint
assez longtemps sans se rebuter les reproches d'Albéroni, et
l'impatience que lui causaient des conseils si directement opposés à ses
vues. Monteléon, quoique sûr de ne pas plaire, osa représenter que
l'abbé Dubois lui avait répété les mêmes choses qu'il lui avait déjà
dites sur les intérêts du roi d'Espagne, qu'il continuait à prier le
cardinal Albéroni, pour le bien du service de Sa Majesté Catholique, de
traiter confidemment avec Nancré comme sûr de la sincérité de ses
intentions. L'abbé Dubois assurait en même temps que Nancré avait les
instructions nécessaires pour satisfaire Sa Majesté Catholique, et pour
concerter avec elle les moyens d'employer la force, si Vienne rejetait
les conditions qu'on avait jugé à propos de lui proposer. Monteléon
tâcha de faire voir que la conjoncture était d'autant plus favorable et
d'autant plus précieuse à ménager qu'il venait d'apprendre de l'abbé
Dubois que depuis peu de jours les ministres d'Angleterre commençaient
enfin à comprendre qu'ils ne devaient espérer de la part de l'empereur
aucun accommodement raisonnable. Il laissait donc envisager l'avantage
que l'Espagne retirerait de la complaisance qu'elle aurait témoignée à
la France et à l'Angleterre, si le roi d'Angleterre, justement irrité
des tours et des refus de la cour de Vienne, laissait agir le roi
d'Espagne et ses alliés.

Le duc de Lorraine, si anciennement, si particulièrement, si totalement
attaché à la maison d'Autriche, était le prince qu'on ne pouvait douter
qu'elle eût en vue de préférer pour la succession de Parme et de
Toscane, quoiqu'elle ne laissât pas de leurrer le duc de Modène de cette
expectative. Penterrieder, à Londres, parlait plus franchement à
l'envoyé de Sicile, à qui il dit que son maître ne devait compter sur
l'empereur qu'autant qu'il lui restituerait le bien qu'il lui détenait,
la Sicile, qui était un royaume uni à celui de Naples, qui, pour leur
sûreté réciproque, devaient être possédés par le même maître. Qu'il
fallait donc de deux choses l'une, que son maître tâchât d'acquérir
Naples, ou l'empereur la Sicile. Que l'Angleterre se repentait de
l'avoir procurée à son maître, et qu'elle y remédierait si ce prince, si
habile, ne savait pas se faire un mérite d'une chose qu'il ne pouvait
empêcher, qui d'ailleurs était juste, mais dont l'empereur voulait bien
cependant lui avoir encore obligation, avantage qu'il devait d'autant
moins négliger, qu'il ne serait plus temps d'offrir le sacrifice de la
Sicile, quand la France et l'Espagne se seraient unies ensemble, comme
elles étaient peut-être sur le point de faire pour la lui enlever. Ainsi
parlait le ministre de l'empereur, employé à Londres pour la négociation
de la paix et pour la conclusion du traité qui devait assurer la
parfaite tranquillité de l'Europe. Il y ajoutait de temps en temps des
discours capables d'inspirer au roi de Sicile, naturellement défiant, de
grands soupçons de la bonne foi du régent. Il disait, entre autres, que
pendant son séjour en France il avait souvent remarqué par lui-même que
les dispositions du régent pour le roi de Sicile n'étaient rien moins
que favorables. Que depuis qu'il était à Londres, il savait certainement
que le roi de Sicile ne devait nullement compter sur ce prince. Si La
Pérouse était assez frappé de ce discours pour inspirer à son maître la
défiance du régent, il ne se reposait pas davantage sur les dispositions
de l'Angleterre, croyant remarquer dans la nation anglaise un tel
éloignement pour la guerre, que jamais elle ne s'y déterminerait en
faveur de l'empereur, encore moins contre l'Espagne. Comme il paraissait
cependant que tout tendait à une rupture entre l'Angleterre et
l'Espagne, l'opinion publique était que le ministère de Georges
attendrait la séparation du parlement avant d'engager ce prince à cette
résolution, pour éviter toute contradiction dans un pays obéré de
dettes, plein de divisions intérieures, et d'ailleurs fort attaché au
commerce.

Le bruit public annonçait aussi la destination de l'escadre pour agir
dans la Méditerranée en faveur de l'empereur. Monteléon en était
persuadé\,; mais il croyait que cela dépendrait du succès de la
négociation de Nancré, et que le ministère d'Angleterre souhaitait
qu'elle réussît pour éviter cette dépense et une rupture opposée au goût
général de la nation. Il essayait de faire comprendre à Albéroni que la
paix était entre ses mains\,; que l'Angleterre n'avait nulle mauvaise
intention contre le roi d'Espagne\,; qu'il était le maître d'assurer le
repos de l'Europe et de former pour l'avenir une alliance étroite avec
l'Angleterre\,; mais ces insinuations furent inutiles. Cellamare, au
contraire, bien assuré des pensées d'Albéroni, n'avait nulle opinion du
voyage de Nancré, et les ministres étrangers, attentifs à découvrir le
caractère de ceux qu'ils pratiquent, avait observé qu'il ne fallait pas
toujours compter sur ce que disait Monteléon\,; que souvent il se
servait de son esprit pour faire prendre aux autres de fausses idées\,;
qu'on ne pouvait compter de savoir la vérité de lui qu'autant qu'elle
lui échappait malgré lui-même par la vivacité de la conversation ou de
la dispute, ou bien à force d'encens qu'il recevait avec plaisir, ou par
les louanges qu'il cherchait souvent à se donner.

Quelques princes d'Italie, alarmés du projet de traité dont les
conditions n'étaient pas encore publiques, crurent devoir s'en informer
à Londres, et y représenter leurs droits et leurs intérêts. Corsini y
était déjà passé de la part du grand-duc {[}de Toscane{]}, et le duc de
Parme y envoya le même Claudio Ré, ce secrétaire qu'il y avait
auparavant employé aux conférences d'Utrecht. Corsini représenta qu'il
serait contraire à l'honneur, aux droits, à la souveraineté de son
maître des démarches anticipées sur sa succession. Le penchant de cet
envoyé, ainsi que de toute la noblesse de Florence, était que leur
patrie reprît son ancienne forme de république, si la ligne du grand-duc
venait à s'éteindre. Ils espéraient même y être aidés par la maison
d'Autriche qui éluderait par là les droits de la maison Farnèse, par
conséquent les prétentions des enfants de la reine d'Espagne.

Monteléon eut ordre de renouveler les déclarations qu'il avait déjà
faites de sortir d'Angleterre si l'escadre anglaise passait dans la
Méditerranée, ce que le roi d'Espagne regarderait comme rupture\,; ce
qu'il ne pouvait plus traiter comme bruits sans fondement par les
préparatifs qui se faisaient à Naples et à Lisbonne pour lui fournir des
vivres. Avant que d'exécuter ces ordres, l'ambassadeur en fit la
confidence à Stanhope qui lui dit que cette déclaration lui paraissait
trop forte, d'ailleurs hors de saison, parce que la nouvelle des
préparatifs de Naples et de Lisbonne était tout à fait fausse, et que,
si le roi d'Angleterre envoyait une escadre dans la Méditerranée, cela
ne signifiait pas qu'il voulût agir contre le roi d'Espagne, parce que
l'Angleterre pouvait avoir aussi ses intérêts particuliers et que
personne n'était en droit ni en pouvoir de lui ôter la faculté et la
liberté d'envoyer ses escadres où bon lui semblait\,; que le départ et
la route de cette escadre dépendait de l'issue de la négociation
présente\,; que, si le roi d'Espagne examinait bien ses intérêts, il
trouverait des avantages réels et solides dans le projet du traité qui
lui avait été communiqué, et qu'en ce cas une escadre anglaise dans la
Méditerranée, loin de lui faire ombrage, lui serait utile et deviendrait
peut-être à craindre pour ses ennemis. Stanhope ajouta comme un
avertissement qu'il donnait en ami à Monteléon, que, s'il exécutait
aveuglément les ordres qu'il avait reçus, ils produiraient peut-être un
effet tout contraire à ses intentions\,; que la déclaration positive
qu'il prétendait faire serait regardée comme une menace et comme marque
d'inconsidération pour l'Angleterre\,; qu'il pourrait arriver que la
réponse serait peu agréable\,; qu'elle engagerait deux puissances amies
à se défier l'une de l'autre\,; enfin, à rompre sans sujet et sans
nécessité. Monteléon lui répondit que ses ordres ne lui laissaient de
liberté que sur la manière de les exécuter\,; qu'il le ferait par écrit,
qu'il s'expliquerait en forme de plainte, tendre d'un ami à son ami,
sans toutefois altérer la force des raisons qu'il devait employer et des
protestations qu'il avait ordre de faire, surtout celle de se retirer si
l'escadre avait ordre de passer dans la Méditerranée.

Malgré sa résistance conforme aux intentions et aux ordres qu'il
recevait d'Albéroni, il était intérieurement persuadé que les conseils
de Stanhope étaient bons, mais il n'osait ni l'avouer ni laisser croire
en Espagne que ce fût son sentiment. Il biaisait pour ne pas déplaire,
et sa ressource était de représenter dans toute sa forcé, même d'ajouter
à ce que Stanhope pouvait lui dire, pour faire comprendre que le roi
d'Espagne prendrait un mauvais parti s'il rompait avec le roi
d'Angleterre et s'il refusait de souscrire au traité. Stanhope assura
que l'empereur ne l'accepterait pas\,; il dit même qu'il pourrait
arriver que ses ministres s'expliqueraient en termes durs et
désagréables\,; que le refus de la cour de Vienne précéderait peut-être
la réponse du roi d'Espagne. Monteléon ne perdit pas cette occasion de
représenter à Albéroni que, si le roi d'Espagne suspendait au moins sa
réponse jusqu'à ce qu'on sût en Angleterre le refus de l'empereur, il
pourrait profiter de la dureté de la cour de Vienne pour engager la
France et l'Angleterre à se joindre à l'Espagne et prendre de concert
les mesures nécessaires pour assurer la tranquillité de l'Europe.

L'abbé Dubois comptait d'avoir fait beaucoup, et, comme disait
Monteléon, d'avoir surmonté les mers et les montagnes en réduisant
l'Angleterre à consentir à la disposition des successions de Parme et de
Toscane en faveur des descendants de la reine d'Espagne. En effet, cette
disposition était la seule du projet dont l'empereur pût être blessé.
L'idée d'ériger la Toscane en république, si désirée des Florentins,
n'aurait pas été contredite à la cour de Vienne, mais le projet dont
l'empereur était le plus flatté était celui d'assurer la Toscane au duc
de Lorraine pour l'indemniser du Montferrat donné par les alliés au duc
de Savoie pendant la dernière guerre, dont l'empereur avait promis un
dédommagement au duc de Lorraine, reconnaissant comme valables les
droits de ce prince sur cet État. Ainsi Monteléon laissait entrevoir au
cardinal ce que le roi d'Espagne pouvait espérer de l'alliance qu'on lui
proposait et ce qu'il avait à craindre du refus de l'accepter. Il ajouta
même à ces représentations indirectes qu'il avait découvert par les
discours de l'abbé Dubois que les ombrages du régent sur les
renonciations n'étaient pas dissipés. Il conclut de cette découverte que
le cardinal aurait le champ libre pour satisfaire Son Altesse Royale sur
cet article et pour l'engager à s'intéresser encore plus en faveur du
roi d'Espagne. Monteléon, persuadé qu'il était de l'intérêt de son
maître de demeurer uni avec l'Angleterre, n'eut garde d'appuyer les
bruits des mouvements où bien des gens s'attendaient dans ce royaume,
répandu par les jacobites, d'une entreprise concertée pour le Prétendant
avant la fin de mai. Ceux même qui étaient le plus dans le sein de la
cour, aussi bien que les ennemis du gouvernement, appuyaient l'opinion
d'un projet concerté contre l'Angleterre entre le czar et le roi de
Suède. Enfin, il n'y avait sorte de propos positifs qu'on ne tînt sur
une révolution prochaine. Comme Stanhope reprit alors sa charge de
secrétaire d'État et remit les finances, on dit avec raison que son
objet était de suivre Georges en Allemagne, où l'un et l'autre aimaient
mieux être pendant la révolution, et de demeurer auprès de lui dans un
temps où il aurait autant de besoin d'avoir des ministres fidèles.
Sunderland, qui lui céda sa charge de secrétaire d'État, fut fait
président du conseil et premier commissaire de la trésorerie. L'autre
charge de secrétaire d'État fut ôtée à Addison et donnée à Craggs. Ainsi
les ministres changèrent dans un temps où la fidélité devenait douteuse,
dans une conjoncture où l'intérêt du commerce soulevait l'esprit général
de la nation contre la rupture avec l'Espagne.

Albéroni, pour augmenter l'alarme, ordonna au chevalier Éon, directeur
de \emph{l'Asiento}, de faire à la compagnie du Sud la même déclaration
que Monteléon avait faite aux ministres de Georges, et d'informer en
même temps cette compagnie de deux avantages nouveaux que le roi
d'Espagne voulait bien lui accorder pour le commerce. Mais les promesses
non plus que les menaces ne furent pas capables d'apporter le moindre
changement à la résolution prise sur l'escadre\,; le nombre des
vaisseaux en fut même augmenté et la diligence à l'armer. Toutefois
Monteléon, malgré les ordres qu'il recevait, espérait du voyage de
Nancré, persuadé que la France voulait la paix et que c'était, en vain
qu'Albéroni l'assurait, même de sa main, que la négociation de Nancré
serait infructueuse. Monteléon ne pouvait croire que l'Espagne fît la
guerre quand elle serait seule et que la France s'opposerait à ses
desseins. Il concluait donc que lorsqu'Albéroni et Nancré se parleraient
et qu'ils s'ouvriraient l'un à l'autre avec franchise, ils se
concilieraient, et que la paix en serait le fruit.

Cellamare, parfaitement persuadé de tout le contraire, avouait que la
difficulté venait moins de la chose que de la disposition de la cour
d'Espagne qui voulait absolument la guerre pour ne pas laisser l'Italie
dans les fers des Allemands, et multipliait ses plaintes de ce que la
France, buttée à vouloir demeurer en paix, manquait une conjoncture si
favorable d'abaisser la maison d'Autriche, et s'épuisait en éloquence
là-dessus. Stairs disait à Paris que l'escadre passerait dans la
Méditerranée parce que l'Angleterre, étant garante des traités d'Utrecht
et de la neutralité de l'Italie, ne pouvait se dispenser d'agir quand
ils étaient enfreints par le roi d'Espagne. Cellamare trouvait que ce
raisonnement était absolument contredit par la question alors agitée
dans le parlement d'Angleterre, savoir si la garantie de la neutralité
d'Italie de la part des Anglais subsistait, ou si elle était absolument
cessée\,; même si la nation devait avoir égard au traité d'alliance que
le roi d'Angleterre avait signé en dernier lieu avec l'empereur. Les
discours et la conduite de Cellamare entièrement conformes à l'esprit et
au goût d'Albéroni à qui il cherchait à plaire, lui en attiraient des
louanges. Cet ambassadeur se mit à décrier toutes les conditions du
traité qui selon lui n'offraient à l'Espagne que des avantages limités,
douteux, éloignés, exposés à des inconvénients sans nombre, pleins de
périls et fort chimériques. Non content de s'expliquer publiquement de
la sorte à Paris, il écrivit en même sens à Monteléon, et lui conseilla
de confier à Corsini ou à quelque autre ministre étranger à Londres,
avec un air de mystère, que le roi d'Espagne était bien résolu de
rejeter constamment le projet du traité. La résolution de l'empereur
était plus douteuse\,; Schaub, secrétaire du comte Stanhope, y avait été
dépêché pour demander et en rapporter une réponse précise. Les ministres
d'Angleterre laissaient entendre qu'elle serait négative et que jamais
l'empereur ne consentirait à la proposition d'assurer les successions de
Parme et de Plaisance à un des fils d'Espagne\,; mais ils disaient en
même temps que, s'il était possible de vaincre l'opiniâtreté de la cour
de Vienne, il fallait en ce cas lui savoir gré de sa complaisance, et
que toute la raison se trouvant de son côté, l'Angleterre ne ferait
nulle difficulté de rompre avec l'Espagne et de lui faire la guerre de
concert avec le régent si le roi d'Espagne refusait de signer un traité
qui devait être la tranquillité générale de l'Europe. On ajoutait que le
caractère de poltron était de faire des bravades, et que celles
d'Albéroni découvraient son caractère. Plusieurs étrangers fort peu au
fait trouvaient ces expectatives de successions si avantageuses à
l'Espagne qu'ils croyaient un manège caché de propositions bien
avantageuses que le roi d'Espagne avait faites au régent pour l'engager
à insister si fort sur ce point.

Le grand-duc, voyant ses plaintes inutiles, et se trouvant sans forces
pour les appuyer, se borna à demander au moins que la succession de son
État fût après lui et après son fils conservée à l'électrice palatine sa
fille, et qu'on réglât par avance de concert avec lui et avec le sénat
de Florence le choix du prince, pour succéder à la maison de Médicis.
Cette proposition du grand-duc était nette\,; mais le voeu commun des
Florentins était en ce cas pour le rétablissement de l'état républicain.
Albéroni écrivit à Monti avec ordre de faire voir sa lettre au régent.
Elle contenait des offres positives et réelles du roi d'Espagne de
prendre de nouveau les engagements les plus favorables et les plus
conformes aux intérêts personnels de Son Altesse Royale, si elle voulait
rompre ceux qu'elle avait pris avec l'Angleterre, et en prendre de plus
convenables au repos de l'Europe, puisqu'ils tendaient à mettre des
bornes à la puissance excessive de la maison d'Autriche. Cellamare
appuya la commission de Monti\,; mais cet ambassadeur ne s'en tenait pas
à de simples représentations, non plus qu'aux plaintes de la maxime du
conseil de France d'éviter la guerre à quelque prix que ce fût. Il
exécutait d'autres ordres plus réservés, et laissait croire au public
qu'il bornait ses pratiques aux seuls ministres des princes d'Italie. Il
excitait de plus la vigilance de Provane\,; il lui disait que la France
commençait à soupçonner le roi de Sicile, qu'elle le croyait
actuellement en négociation avec l'empereur\,; qu'il y avait même
actuellement un ministre autrichien à Turin. Enfin ne voulant laisser
rien d'intenté, il fit une liaison étroite avec le baron de Schelnitz
envoyé du czar à Paris, et avec quelques Suédois, croyant pouvoir tirer
de grands avantages du mécontentement que le roi de Suède et le czar,
quoique ennemis, témoignaient de la conduite de l'empereur à leur égard,
et qu'il ne serait pas impraticable de faire, par le moyen des
puissances du nord, une diversion en Allemagne utile à l'Espagne.

\hypertarget{chapitre-xviii.}{%
\chapter{CHAPITRE XVIII.}\label{chapitre-xviii.}}

1718

~

{\textsc{Affaires du nord.}} {\textsc{- La France paraît vouloir lier
étroitement avec la Prusse.}} {\textsc{- Hollandais, fort en brassière
entre l'Espagne et les autres puissances, veulent conserver la paix.}}
{\textsc{- Adresse de Monteléon dans ses représentations à Albéroni,
sous le nom de l'abbé Dubois, en faveur de la paix.}} {\textsc{- Menaces
de l'Espagne méprisées en Angleterre, dont le parlement accorde au roi
tout ce qu'il demande pour les dépenses de mer.}} {\textsc{- Insolence
de Penterrieder.}} {\textsc{- Ses manèges et ses propositions à l'envoyé
de Sicile très dangereuses pour la France.}} {\textsc{- Vanteries et
bévues de Beretti.}} {\textsc{- Le roi de Sicile soupçonné de traiter
secrètement avec l'empereur.}} {\textsc{- Raisonnements d'Albéroni sur
ce prince, sur les Impériaux et sur la France.}} {\textsc{- Fortes
protestations et déclarations de l'Espagne à Paris et à Londres.}}
{\textsc{- Efforts et préparatifs d'Albéroni.}} {\textsc{- Ses
plaintes.}} {\textsc{- Albéroni imagine de susciter la Suède contre
l'empereur.}} {\textsc{- Nancré échoue à Madrid.}} {\textsc{- Albéroni
le veut retenir jusqu'à la réponse de Vienne.}} {\textsc{- Concert entre
Nancré et le colonel Stanhope.}} {\textsc{- Adresse de ce dernier
repoussée par Albéroni.}} {\textsc{- Grands préparatifs hâtés en
Espagne.}} {\textsc{- Le marquis de Lede et Patiño mandés à Madrid.}}

~

Depuis le mois de février on commençait à voir quelque apparence de
réconciliation entre le czar et le roi de Suède. Le comte de
Gyllembourg, auparavant employé en Angleterre, avait fait quelques
propositions de paix de la part du roi de Suède, et le czar avait envoyé
deux hommes à Abo, pour écouter et discuter les offres qu'il voudrait
faire. Le czar avait eu grand soin auparavant d'assurer le roi de Prusse
qu'il ne serait question que de préliminaires, que d'ailleurs il ne
traiterait que de concert avec ce prince, et qu'il ne déciderait rien
sans savoir auparavant ses sentiments. Les flatteries et les apparences
réussissaient à la cour de Berlin, et le roi de Prusse était infiniment
plus touché des attentions du czar que de tout ce qu'il pouvait attendre
de la part de la France et de l'Angleterre, qui véritablement ne
marquaient pas pour lui les mêmes égards. Le régent avait cependant
employé les offices du roi et les siens auprès du roi de Suède, pour
procurer au roi de Prusse la paix aux conditions qu'il désirait. Mais de
simples instances sans effets ne suffisaient pas pour contenter la cour
de Berlin. Elle croyait que rien ne se ferait en France que par la
direction de l'Angleterre, et que les confidences faites à Son Altesse
Royale étaient des confidences faites aux Anglais.

Le roi de Prusse, se croyant donc sûr du czar, et persuadé qu'il ne
ferait point de paix séparée, perdit la pensée qu'il avait eue d'envoyer
un ministre à Stockholm\,; mais avant de l'abandonner, les ministres
apparemment l'avaient laissé pénétrer, car il eut peine à dissiper les
bruits qui se répandirent de la destination du baron de Kniphausen pour
cette commission. Il n'oublia rien pour effacer les soupçons que le
czar, qu'il voulait ménager, pouvait concevoir de cet envoi. Il fit à
peu près les mêmes diligences auprès du régent pour le détromper de
cette opinion\,; il aurait bien voulu l'engager à prendre avec lui des
mesures sur les affaires de Pologne. Il craignait l'effet des desseins
que le roi Auguste avait formés de rendre cette couronne héréditaire
dans sa maison\,; et comme l'assistance de la France lui paraissait
nécessaire pour les traverser, il représenta fortement l'intérêt que le
roi avait d'empêcher que l'empereur ne devînt encore plus puissant dans
l'empire comme il y serait certainement le maître lorsqu'il aurait
absolument lié les maisons de Bavière et de Saxe par le mariage des
archiduchesses. Il prétendait avoir pressenti les principaux seigneurs
de Pologne, et les avoir trouvés très disposés à traverser les manèges
que le roi Auguste pourrait faire pour assurer la couronne
héréditairement à son fils. Le roi de Prusse, pour cultiver de si bonnes
dispositions, fit demander au régent d'ordonner au baron de Bezenval,
envoyé du roi en Pologne, de s'entendre secrètement pour cette affaire
avec les ministres de Berlin. Quoique le roi de Prusse, gendre du roi
d'Angleterre, dût être lié avec lui, les intérêts différents des deux
maisons, ceux de leurs ministres entretenaient entre ces princes la
jalousie et la défiance réciproque, et d'autant plus vivement de la part
du roi de Prusse, qu'il était le plus faible, et que souvent il avait
lieu de croire que son beau-père le méprisait. Il était persuadé que les
ministres anglais et hanovriens s'accordaient dans le désir de faire la
paix avec la Suède. Il croyait qu'ils cherchaient les moyens de traiter
avec elle séparément\,; que, s'il était possible d'y parvenir, le roi
d'Angleterre sacrifierait sans peine les intérêts de son gendre aussi
bien que ceux de ses autres alliés. Ainsi le roi de Prusse, qui
certainement ne portait pas trop loin sa défiance en cette occasion, se
voyait à la veille de perdre tout le fruit de ses peines et des dépenses
qu'il avait faites pour usurper, comme ses voisins, la portion qui lui
convenait des États de Suède, et profiter comme eux du malheur où elle
était réduite.

Rien ne tenait plus au coeur de ce prince que de conserver Stettin et
l'étendue de pays qu'il avait fixée comme le district de cette place. La
France lui en avait promis la garantie par son dernier traité avec
elle\,; mais il craignait le sort ordinaire des garanties, et
l'exécution de celle-ci était d'autant plus difficile, par conséquent
d'autant plus douteuse, que l'éloignement des pays était grand\,; qu'il
n'était guère vraisemblable que la France voulût, pour le roi de Prusse,
faire la guerre dans les extrémités septentrionales de l'Allemagne, ou
l'assister longtemps de subsides suffisants pour le mettre en état de
défendre ses conquêtes. Le plus sûr pour lui était donc d'être compris
dans la paix que, suivant leurs engagements mutuels, les alliés du nord
devaient faire avec la Suède ensemble et de concert. Pour cet effet,
n'osant se reposer sur la foi douteuse de son beau-père, il demandait au
régent de traverser les manèges que les ministres anglais et hanovriens
faisaient pour une paix particulière, négociations dont le succès serait
d'autant plus désagréable et plus embarrassant pour la France, que tout
le poids de la garantie de Sicile retomberait alors sur elle.

Le régent avait prévu les représentations et les instances du roi de
Prusse, et avait déjà agi auprès du czar pour l'engager d'entretenir une
étroite union avec ce prince comme le moyen d'établir pareillement cette
union entre la France et la Russie, les États du roi de Prusse étant
nécessaires pour cette communication. Kniphausen, envoyé de Prusse à
Paris, se réjouissait de voir que ceux qui étaient à la tête des
affaires pensaient que les alliances les plus naturelles et les plus
solides pour la France étaient celles qu'elle formerait avec le roi de
Suède et celui de Prusse. Il se flattait même que, s'il était possible
de conduire les affaires du nord à une bonne fin, les liaisons que la
France prenait avec l'Angleterre ne subsisteraient pas longtemps, parce
que l'esprit ni le goût de la nation n'était porté à se lier ni avec
l'Angleterre ni avec l'empereur. On croyait d'ailleurs que le régent
lui-même était ébranlé sur les affaires d'Espagne, et qu'il pourrait
changer de plan si on pouvait gagner du temps. Kniphausen assura son
maître qu'il n'y avait rien de visionnaire dans les avis qu'il lui
donnait sur ce sujet, qu'ils étaient conformes aux discours que tenaient
les principaux et les plus accrédités seigneurs de la cour de France\,;
que même le maréchal d'Huxelles l'avait assuré que le roi n'oublierait
rien pour procurer au roi de Prusse les moyens de finir la guerre du
nord à l'avantage et à la satisfaction de ce prince\,; cette base étant
nécessaire pour établir ensuite une amitié solide et permanente, qu'elle
serait cultivée à l'avenir par l'attention que la France donnerait aux
intérêts du roi de Prusse, qu'elle voulait désormais regarder comme les
siens propres\,; qu'elle ferait telle alliance qu'elle souhaiterait,
qu'elle y ferait entrer telles puissances qu'elle jugerait à propos\,;
enfin qu'il ne fallait pas qu'il fût étonné ni rebuté par les
ménagements que la France avait eus depuis quelque temps, et qu'elle
pourrait encore avoir pour l'Angleterre, parce qu'il fallait continuer à
tenir la même conduite jusqu'à ce qu'on pût parvenir au but qu'on se
proposait. Kniphausen fit d'autant plus de réflexions à ce discours du
maréchal d'Huxelles que, lorsqu'il fut fini, il lui demanda un grand
secret de tout ce qu'il lui avait confié. L'envoyé entendait d'ailleurs
les discours généraux qu'on tenait au sujet de la guerre d'Espagne.

Ce n'était pas seulement en France qu'elle recevait des
contradictions\,; les ministres d'Angleterre trouvaient aussi de fortes
oppositions en Hollande. Ils se plaignaient d'y voir un parti favorable
aux Espagnols par la seule raison de contredire l'Angleterre en toutes
choses. Si ce parti n'était pas assez considérable ni assez puissant
pour apporter aucun changement aux maximes suivies depuis longtemps, il
l'était cependant assez pour causer beaucoup d'embarras, même
d'obstacles aux affaires les plus importantes\,; il profitait de la
disposition de l'État généralement porté à vivre en bonne amitié avec
l'Espagne, car alors le seul désir des Hollandais, et le seul point
qu'ils croyaient conforme à leurs intérêts, était de conserver la paix,
et par ce moyen le commerce de la nation. Malgré cette disposition, les
Hollandais, craignant excessivement de déplaire à l'empereur et à
l'Angleterre, n'osèrent accorder à l'ambassadeur d'Espagne la permission
d'acheter des vaisseaux de guerre\,; dont le roi d'Espagne voulait faire
l'emplette en Hollande\,; quoique Beretti se vantât toujours que son
habileté l'emporterait sur les manèges de tous ceux qui s'y
opposaient\,; que les amirautés d'Amsterdam et de Rotterdam demandaient
aux États généraux la permission d'en vendre à l'Espagne, et que le
Pensionnaire, loin de s'y opposer, avait répondu\,: «\,Si nous en avons
trop, pourquoi n'en pas vendre à nos amis\,?» Ainsi Beretti, se comptant
sûr de son fait, n'était plus en peine que du payement\,; et Schreiner,
capitaine de vaisseau en Hollande, lui offrit des matelots et des
officiers, et de les conduire en Espagne, tous capables de bien servir.
Beretti ne fut pas si content du greffier Fagel, qui lui représenta lés
difficultés de cette affaire, et qui ne lui promit que faiblement ses
services là-dessus. Il ne fut pas plus gracieux aux plaintes que lui fit
Beretti des conditions du traité qui donnaient des États, disait-il, à
l'empereur, et du papier au roi d'Espagne. Fagel combattit toujours ses
raisons, et lui dit qu'on donnerait de telles sûretés à l'Espagne que
les papiers ne seraient pas sujets à la moindre altération. Tout était
encore en suspens en attendant le succès de l'envoi de Schaub à Vienne
et de Nancré à Madrid. Le projet de traité n'avait pas encore été
communiqué en forme aux États généraux\,; le public en pénétrait les
principales conditions, mais en ignorait le détail\,; on ne savait même
jusqu'à quel point la France concourrait aux desseins de l'Angleterre.

Beretti, avec sa prétendue sécurité, ne laissait pas de craindre de ne
pouvoir empêcher la Hollande de se soumettre aux idées de l'Angleterre
si elle était véritablement d'accord avec la France\,; cette république
se trouvait environnée par terre des États de l'empereur, et son
commerce par mer serait ruiné par l'Angleterre, si elle osait contredire
ses vues, jointes à celles de la France. On voulait encore douter à
Madrid des intentions de cette dernière couronne\,; ainsi Beretti eut
ordre d'agir de concert avec Châteauneuf pour y traverser les
négociations du marquis de Prié. Beretti comptait que jamais l'empereur
n'obligerait la république de prendre aucun engagement contre l'Espagne,
et que les principaux moteurs de la ligue auraient tant d'affaires chez
eux qu'il ne leur serait pas libre de se mêler du dehors. Il prévoyait
avec les politiques l'union prochaine du czar très mécontent de
l'Angleterre avec le roi de Suède et celui de Prusse, qui serait fatale
à l'Angleterre et à l'empereur, duquel l'électeur de Bavière devenait
l'ennemi, lequel dissimulait son dépit de ne pouvoir obtenir pour le
prince électoral son fils une des archiduchesses, porté d'ailleurs pour
les intérêts du roi d'Espagne. Ce fut un grand sujet de joie pour
Beretti de recevoir dans ces circonstances un projet dressé par la
compagnie des Indes occidentales de Hollande pour convenir avec le roi
d'Espagne d'un nouveau règlement à faire sur le commerce que les
directeurs de cette compagnie croyaient également avantageux de part et
d'autre. Ils demandaient le secret, et Beretti regardait comme une
victoire d'accoutumer les Hollandais à s'approcher des Espagnols, soit
pour le commerce, soit pour le militaire, persuadé que quelque jour les
effets en seraient très utiles à l'Espagne.

Monteléon, qui connaissait à quel point Albéroni était éloigné du projet
et de la paix, et qui n'osait lui déplaire, craignait une rupture avec
l'Angleterre, et continuait sa même adresse de représenter au premier
ministre sous le nom de l'abbé Dubois, ce qu'il lui avait dit ou ce
qu'il supposait qu'il en avait appris, n'osant hasarder ses
représentations sous le sien. Il assura donc Albéroni qu'il savait
positivement de cet abbé que la cour de Vienne n'accepterait pas le
projet, qu'elle se tiendrait même offensée de la proposition que le roi
d'Angleterre lui en avait faite. L'abbé Dubois prétendit même qu'il
avait déjà fort pressé le roi d'Angleterre et les ministres anglais
particulièrement Stanhope, d'employer enfin la force pour arrêter
l'humeur ambitieuse de l'empereur, l'unique moyen d'empêcher qu'il ne
mît l'Europe en feu étant que la France, l'Espagne et l'Angleterre unies
ensemble, prissent des mesures pour s'y opposer. Monteléon ajouta qu'il
savait, mais sous le secret et par un effet de la confiance intime que
l'abbé Dubois avait en lui, qu'il gagnait du terrain peu à peu, mais
qu'enfin ce progrès serait inutile si l'Espagne, de sort côté, ne
s'aidait\,; qu'elle devait se conformer à la constitution délicate,
extravagante et presque inexplicable du gouvernement d'Angleterre, et
faciliter au ministère anglais le moyen de se déclarer à découvert
contre la cour de Vienne. Ce moyen était que le roi d'Espagne fit voir
qu'il ne prenait pas en mauvaise part, et qu'il ne méprisait pas les
conditions du projet communiqué par l'Angleterre. Que, si Sa Majesté
Catholique y trouvait des difficultés, elle pouvait les représenter,
mais sans rompre les liens d'amitié et de confiance avec le roi
d'Angleterre\,; qu'elle devait, au contraire, pour son intérêt laisser
une porte ouverte aux expédients sans déclarer une volonté déterminée de
vouloir la guerre à toute force\,; que cette conduite prudente serait
totalement contraire à la négative hautaine et absolue que les ministres
anglais attendaient de Vienne\,; qu'ainsi le roi d'Espagne mettrait
cette cour dans son tort, et qu'il engagerait la nation anglaise en
général à se déclarer pour lui\,; que le ministère anglais, animé déjà
contre les Impériaux, agirait contre eux plus librement lorsqu'il
croirait le pouvoir faire avec sûreté\,; qu'il était encore dans la
crainte, parce que, s'il paraissait porté pour l'Espagne sans avoir de
sujet évident de se plaindre de l'empereur, les whigs mécontents, qui
parlaient alors en faveur de cette couronne, changeraient aussitôt de
langage et de sentiment.

Ces discours vrais ou supposés que Monteléon mettait dans la bouche de
l'abbé Dubois, étaient tirés, disait-il, de ses conversations avec les
ministres anglais, et croyant ces considérations importantes, cet abbé
l'avait prié de ne pas perdre un moment à les faire savoir au roi son
maître. Toutefois cet ambassadeur, quoique prévenu de l'importance dont
il était de faire tomber sur la cour de Vienne la haine du refus, et
persuadé de la nécessité de conserver une bonne intelligence avec la
cour d'Angleterre, n'avait osé différer de présenter le mémoire
qu'Albéroni lui avait ordonné de remettre aux ministres d'Angleterre au
sujet de l'escadre anglaise destinée pour la Méditerranée. Le seul effet
de ce mémoire fut d'exercer à Londres les raisonnements des
politiques\,; d'ailleurs, il ne suscita pas le moindre obstacle aux
desseins du roi d'Angleterre. Ce prince, prévoyant qu'il serait obligé
d'augmenter les dépenses de la marine, demanda qu'il fût réglé par un
acte du parlement que le parlement suivant abonnerait ces dépenses. Il
l'obtint, en sorte que par cet acte il devint le maître d'envoyer des
escadres où il le jugerait à propos, les fonds pour la dépense étant
déjà assignés. Ainsi Penterrieder n'eut pas la moindre inquiétude ni du
mémoire présenté par Monteléon, ni des représentations que quelques
négociants, surtout des intéressés dans \emph{l'Asiento}\footnote{Le
  sens de ce mot a été expliqué plus haut.}**, firent sur le préjudice
que l'interruption de la bonne correspondance avec l'Espagne ferait à
leur commerce, car, encore que l'empereur n'eût pas accepté le traité au
commencement de mars, il n'en était pas moins sûr de la route que
l'escadre anglaise tiendrait vers les côtes d'Italie. Penterrieder en
parlait en ces termes à La Pérouse, et pour faire voir la modération et
la clémence de Sa Majesté Impériale, il assurait qu'elle n'enverrait pas
même de troupes en Italie, ne voulant inquiéter personne, mais faire du
bien à tout le monde. Pour le prouver elle avait intention d'accorder au
roi d'Angleterre l'investiture de Brême et de Verden, lorsque la
campagne serait finie.

Cette bénignité accoutumée de la maison d'Autriche devait engager le roi
de Sicile à rechercher les bonnes grâces de l'empereur\,: c'était au
moins le discours de Penterrieder. Il faisait agir auprès de Provane le
secrétaire de Modène qui était à Londres\,; il laissait entrevoir des
apparences nouvelles à un accommodement, et faisait espérer que
l'empereur pourrait enfin se radoucir, à mesure que le roi de Sicile
ferait des pas pour regagner ses bonnes grâces. Il disait qu'il fallait
chercher des équivalents pour l'échange de la Sicile\,; que, s'il était
impossible d'en convenir, il ne le serait pas de céder au roi de Sicile
le royaume de Naples pour les posséder tous deux ensemble, donnant en
échange les autres États qu'il possédait actuellement. La Pérouse,
flatté de se trouver chargé d'une négociation sécrète avec le ministre
de l'empereur à Londres, pendant que la négociation d'une paix générale
occupait toute l'attention publique, n'oubliait rien pour faire croire à
son maître que la voie qu'il avait ouverte pour négocier était la plus
sûre et la meilleure qu'il pût trouver, et qu'il n'aurait pas même à
craindre d'être traversé par les Anglais, quoique promoteurs du projet
dont on attendait les réponses de Vienne et de Madrid. Il s'appuyait sur
les assurances que Penterrieder lui avait données, que tout le ministère
anglais, sans en excepter ni Stanhope, ni Craggs, était entièrement
dévoué à l'empereur\,; que toutes les caresses faites à l'abbé Dubois
étaient pures grimaces\,; que l'escadre destinée pour la Méditerranée
partirait au plus tôt\,; que déjà le consul anglais de Naples avait
ordre de faire préparer les provisions pour elle\,; qu'il n'y avait
point à se mettre en peine des murmures de la nation anglaise\,; qu'au
fond, elle craignait peu de rompre avec l'Espagne, parce que cette
interruption ne pouvait durer plus d'un an\,; que, pendant cet espace de
temps, il se formerait des compagnies anglaises qui se dédommageraient
dans les Indes espagnoles de la saisie que l'Espagne pourrait faire en
Europe. Quelques armateurs même offraient à Penterrieder d'arborer le
pavillon de l'empereur, et de faire des courses sur les Espagnols dans
la Méditerranée, si ce prince voulait leur donner des commissions.

Pendant que le ministre de l'empereur à Londres se croyait si sûr non
seulement des ministres de Georges, mais, encore des dispositions
générales de la nation anglaise sur la guerre d'Espagne, l'ambassadeur
d'Espagne à la Haye se tenait également assuré de la disposition
générale des Hollandais en faveur de son maître. Il crut en avoir une
preuve dans la permission qu'il obtint à la fin de mars d'acheter les
navires, de guerre que Castañeda devait ramener de Hollande en Espagne.
Le projet était d'en avoir sept à soixante-dix pièces de canon chacun.
Ces navires devaient être achetés sous le nom de marchands espagnols.
Beretti en était demeuré d'accord avec le Pensionnaire et d'autres
membres du gouvernement. Les États de Hollande avaient autorisé les
amirautés de la province à vendre les vaisseaux qu'elles pourraient
avoir au delà des trente que la république faisait armer pour la mer
Baltique. C'était donc au delà de ce nombre que Beretti se flattait d'en
trouver sept à choisir dans les amirautés d'Amsterdam, de Rotterdam et
de Zeelande. Il se vantait d'avoir surmonté par son habileté
l'opposition des provinces, parce qu'il s'agissait d'armer trente
vaisseaux pour le nord. Secondement l'empereur menaçait la république si
elle accordait cette permission\,; enfin les Anglais et les Portugais
traversaient secrètement la négociation, et mettaient en usage tant
d'intrigues et d'artifices pour en empêcher le succès, que Beretti ne
l'attribuait qu'à son savoir-faire, et puis à la bonne volonté que la
plus saine partie de la république avait pour le roi d'Espagne. Mais
Beretti n'était pas encore au bout de cette affaire, quelque assuré
qu'il s'en crût.

On disait publiquement alors que le roi de Sicile entrait dans la ligue,
et qu'il traitait avec l'empereur. Le régent avait communiqué en Espagne
les avis qu'il avait reçus de cette négociation secrète à Vienne.
Cellamare en avait officieusement averti Provane. Ce dernier, quoique
peu content, rendait cependant justice au régent. Il était persuadé que
ce prince voulait sincèrement procurer la paix, et qu'il la croyait
aussi conforme aux intérêts du roi et du royaume qu'aux siens
personnels. Albéroni ne douta pas un moment du double manège du roi de
Sicile. Persuadé que jamais il n'agissait de bonne foi, il conclut que
ce prince s'était proposé de voir enfin la guerre allumée de tous côtés
et les Impériaux chassés d'Italie. Mais il remarquait en même temps
autant de mauvaise foi de leur part que de faiblesse, accompagnée
d'autant d'artifice, pour détourner le mal qu'ils avaient à craindre\,;
et pour éviter le coup qu'il était aisé de leur porter\,; car ils
faisaient voir des pensées de paix, ils sollicitaient la France et
l'Angleterre de s'entremettre pour un accommodement\,; et la seule vue
de la cour de Vienne était, disaient-ils, de lier les mains au roi
d'Espagne par cet artifice, et d'empêcher les entreprises que
vraisemblablement il méditait, et qu'il pouvait aisément exécuter en
Italie par les troupes qu'il avait en Sardaigne. L'empereur n'avait pas
fait encore la paix avec les Turcs, par conséquent il était trop faible
pour défendre les États qu'il possédait en Italie, ses forces
principales étant occupées en Hongrie. Il voulait donc par de feintes
négociations gagner le temps de la paix, et se déployer après en force
sur l'Italie. Il reprochait à l'empereur que l'avidité de conserver et
d'étendre ses injustes usurpations sur l'Italie l'engageait à offrir aux
Turcs de leur céder Belgrade, et d'aimer mieux en obtenir une paix
honteuse dans le cours de ses victoires, qu'à tenir plus longtemps ses
troupes éloignées du lieu où il aimait mieux les employer.

Albéroni faisait de temps en temps des réflexions sur l'aveuglement
général et l'indolence fatale de tant de princes. Il en exceptait le roi
d'Espagne. Il prétendait qu'avec une bonne armée et de bonnes flottes il
demeurerait tranquillement chez lui, simple spectateur des maux que la
guerre causerait aux autres nations\,; que, s'il arrivait contre toute
apparence, qu'on vît de telles révolutions que ce prince fût contraint
de céder à la force, il aurait toujours sa ressource, et que, au pis
aller, il se retirerait sur son fumier (en France), résolution qui
pourrait un jour faire connaître à certaines gens (M. le duc d'Orléans)
que c'était s'égarer sur leurs propres intérêts que d'empêcher Sa
Majesté Catholique de porter hors de son continent des troupes et de
l'argent pour employer l'un et l'autre sur les frontières de France.
Enfin, il disait plus clairement que le régent se repentirait peut-être
un jour d'avoir négligé d'établir avec le roi d'Espagne, comme il le
pouvait aisément, l'union et la bonne intelligence dont dépendaient et
son honneur et son intérêt personnel. Albéroni, prévenu que la France et
l'Angleterre demanderaient, pour avancer la paix, que la Sardaigne fût
remise en dépôt pendant la négociation, déclara par avance que le roi
d'Espagne n'admettrait jamais une pareille proposition. Cette île était
l'entrepôt des troupes qu'il voulait envoyer en Italie. Ainsi, loin de
la remettre comme en séquestre, il prenait toutes les mesures
nécessaires pour la bien garder. Albéroni protestait en même temps que
le roi d'Espagne voulait venger ses outrages et soutenir ses droits,
quand même il serait seul et dépourvu de tout secours. Les ambassadeurs
d'Espagne en France et en Angleterre eurent ordre de parler en même
sens. Il fut enjoint particulièrement à Monteléon de renouveler ses
protestations, et de ne rien omettre pour faire bien connaître à la
nation anglaise le préjudice qu'elle souffrirait de l'engagement qu'on
voulait la forcer de prendre avec l'empereur, sans raison et contre
l'intérêt de cette nation, enfin dans un temps où les grâces qu'elle
avait obtenues du roi d'Espagne étaient trop récentes pour en avoir
perdu le souvenir. D'un autre côté, il s'épuisait en vives et fortes
représentations à la France\,; mais, les jugeant fort inutiles, il
continuait à prendre les mesures que l'état de l'Espagne pouvait
permettre pour se préparer à faire vigoureusement la guerre. Il
travaillait principalement à ramasser un nombre de vaisseaux suffisant
pour faire croire que l'Espagne avait suffisamment des forcés maritimes.
Plus il y travaillait, plus il trouvait que l'entreprise de mettre sur
pied une marine était, disait-il, un abîme. Il avait espéré d'acheter
des navires en Hollande, de les y trouver tout équipés et en état de
servir\,; cette espérance s'évanouissait, et malgré les belles paroles
de Beretti, Albéroni pénétrait qu'il ne devait en attendre rien de réel.
Il se plaignait de la négligence de Castañeda, et en général de ne
trouver en Espagne personne qui pût le soulager et qu'il pût regarder
comme un homme de confiance.

Il se figura que le roi de Suède serait peut-être de quelque secours aux
affaires du roi d'Espagne\,; qu'en aidant aux Suédois à rentrer en
Allemagne, on remplacerait avantageusement par cette diversion celle que
les Turcs avaient faite jusqu'alors en Hongrie, et qu'une prochaine paix
était prête à terminer. Beretti eut ordre d'examiner si le roi de Suède
avait en Hollande quelque sujet, homme de mérite, et en ce cas de lui
parler et de lui confier que, le roi d'Espagne étant sur le point
d'attaquer vivement l'empereur, il serait de l'intérêt de la Suède de
profiter de cette conjoncture. Si celui à qui Beretti parlerait
représentait que son maître, manquant d'argent, n'était pas en état
d'entrer dans de pareils projets, Beretti avait pouvoir de lui offrir,
mais seulement comme de lui-même, d'écrire au cardinal, et de le
disposer à fournir de l'argent à la Suède, lui proposant de prendre en
échange du cuivre ou des bois pour la marine. La paix aurait mis fin à
ces agitations, la négociation en était entre les mains d'Albéroni.
Nancré, étant arrivé à Madrid vers la fin de mars, lui avait exposé le
plan du traité concerté entre la France et l'Angleterre, et communiqué
depuis à Vienne. Il n'était pas encore alors aussi avantageux pour le
roi d'Espagne qu'il le fut depuis, car les Anglais avaient toujours en
tête de démembrer l'État de Toscane, de faire revivre l'ancienne
république de Pise, et de comprendre Livourne dans cet État ainsi
renouvelé.

Un tel projet fut mal reçu. Albéroni en ayant entendu toutes les
conditions le traita de fou et de chimérique\,; dit qu'en ayant rendu
compte à Leurs Majestés Catholiques, elles avaient répondu que jamais
elles n'avaient entendu rien de plus indigeste et de plus visionnaire\,;
que la reine surtout était offensée de l'opinion que le régent avait
d'elle, et de voir qu'il la crût capable d'une perfidie telle que le
serait de penser seulement, non de consentir à dépouiller un prince qui
lui tenait lieu de père. Albéroni plaignit Nancré, et dit qu'il était
malheureux qu'un homme d'honneur et d'esprit comme lui fût chargé d'une
si mauvaise commission\,; que, si le régent eût jeté plus tôt les yeux
sur lui, et que dès l'année précédente il l'eût envoyé en Espagne au
lieu de Louville, Son Altesse Royale ne se trouverait pas en des
engagements dont les suites et le dénouement ne tourneraient peut-être à
l'avantage ni de la France ni de l'Espagne. Albéroni prétendit que
Nancré avait représenté l'état de la France si malheureux qu'à peine
elle pourrait mettre en cas de guerre deux mille hommes en mouvement. Il
avait répondu qu'il trouvait une contradiction manifeste entre cet état
de faiblesse et les engagements que le régent avait pris avec
l'Angleterre, puisque certainement il se trouverait obligé à mettre plus
de deux mille hommes en mouvement s'il voulait tenir sa promesse. Le roi
d'Espagne, dans l'audience qu'il donna à Nancré, lui répondit qu'il
examinerait les propositions qu'il avait faites. L'intention d'Albéroni
était de prendre du temps pour être instruit des réponses de l'empereur,
avant que d'en rendre une positive de la part du roi d'Espagne.

Le colonel Stanhope était encore à Madrid, chargé des affaires et des
ordres du roi d'Angleterre. Nancré et lui agissant pour la même cause
agirent aussi d'un parfait concert, et Albéroni leur répondit également
à tous deux. Stanhope lui demanda si le roi d'Espagne enverrait des
troupes en Italie, et s'il exercerait des actes d'hostilité pendant
qu'on traitait actuellement la paix. Le colonel voulait obtenir une
promesse de cessation d'armes de l'Espagne pendant la négociation. Le
cardinal parut choqué du discours que le colonel lui tenait entre ses
dents. Il répondit que Sa Majesté Catholique ferait passer huit mille
hommes en Sardaigne, tant pour se défendre contre les entreprises des
Allemands, que, parce que l'empereur envoyait lui-même continuellement
des troupes dans l'État de Milan et dans le royaume de Naples\,; qu'au
reste elle n'était pas en état d'exercer présentement aucun acte
d'hostilité, et que vraisemblablement les réponses de Vienne
arriveraient avant que l'Espagne pût rien entreprendre. En même temps
qu'Albéroni faisait voir par ses réponses si peu de dispositions à la
paix, il pressait avec plus de diligence que jamais les préparatifs de
guerre. Tous les officiers sans exception eurent ordre de se rendre à
leurs corps. On disposa toutes les choses nécessaires pour
l'embarquement de quatre régiments de dragons qui de Barcelone devaient
être transportés en Sardaigne avec leurs chevaux. L'intendant de marine
eut ordre de préparer à Barcelone les vivres nécessaires pour
l'embarquement de vingt bataillons. On fit venir à Madrid le marquis de
Lede et don Joseph Patiño, l'homme de confiance d'Albéroni, pour leur
donner les ordres du roi d'Espagne. Tout était en mouvement pour la
guerre, jusqu'à Riperda, encore ambassadeur de Hollande, qui promit
d'engager au service d'Espagne quelques Hollandais, officiers généraux
de mer dans le service de ses maîtres.

\hypertarget{chapitre-xix.}{%
\chapter{CHAPITRE XIX.}\label{chapitre-xix.}}

1718

~

{\textsc{Menaces d'Albéroni sur le refus de ses bulles de Séville.}}
{\textsc{- Il s'emporte contre le cardinal Albane.}} {\textsc{- Manèges
d'Aldovrandi pour le servir et soi-même.}} {\textsc{- L'empereur
s'oppose aux bulles de Séville\,; accuse Albéroni de traiter avec les
Turcs.}} {\textsc{- Acquaviva embarrasse le pape par une forte demande
et très plausible.}} {\textsc{- Prétendues preuves de l'accusation
contre Albéroni.}} {\textsc{- Secret et scélérat motif d'Albéroni pour
la guerre.}} {\textsc{- Conduite de Cellamare en conséquence.}}
{\textsc{- L'empereur consent à tous les points du traité de Londres.}}
{\textsc{- Cellamare déclare que l'Espagne n'acceptera point le
traité.}} {\textsc{- Le régent dépêche à Londres.}} {\textsc{- Manèges,
inquiétudes, fougues, menaces d'Albéroni.}} {\textsc{- Ses
déclamations.}} {\textsc{- Son emportement contre le traité de la paix
d'Utrecht.}} {\textsc{- Fureur d'Albéroni sur les propositions de
Nancré, surtout contre la cession de la Sicile à l'empereur.}}
{\textsc{- Il proteste que le roi d'Espagne n'acceptera jamais le
traité, quoi qu'il en puisse arriver.}} {\textsc{- Ses vanteries\,; ses
imprécations.}} {\textsc{- Ne laisse pas de traiter Nancré avec beaucoup
de distinction et d'apparente confiance.}} {\textsc{- Fureur, menaces et
manèges d'Albéroni sur le refus de ses bulles de Séville.}} {\textsc{-
Albéroni dépité sur l'achat des vaisseaux en Hollande, ou Beretti se
trompe de plus en plus, déclare qu'il n'en a plus que faire\,; menace.}}
{\textsc{- Manège sur l'escadre anglaise.}} {\textsc{- Sage conduite de
Monteléon.}} {\textsc{- Négociation secrète du roi de Sicile à Vienne.}}
{\textsc{- Propos de l'abbé Dubois à Monteléon.}} {\textsc{- Doubles
manèges des Anglais sur la paix, avec l'Espagne et avec l'empereur.}}
{\textsc{- Sentiment de Monteléon.}} {\textsc{- Dangereux manèges du roi
de Sicile.}} {\textsc{- Le roi d'Angleterre s'oppose ouvertement à son
désir d'obtenir une archiduchesse pour le prince de Piémont.}}

~

Pendant qu'Albéroni se disposait à faire la guerre aux puissances
temporelles de l'Europe, il ne ménageait pas beaucoup la spirituelle du
pape, et déclarait hautement que Leurs Majestés Catholiques avaient
autant de ressentiment qu'ils avaient de mépris de la conduite misérable
que la cour de Rome avait à leur égard dans la vue de ménager les
Allemands. Albéroni, sous prétexte d'excuser le pape, disait que le peu
d'attention de Sa Sainteté pour Leurs Majestés Catholiques, et la
complaisance qu'elle avait pour leurs ennemis, procédaient des
impertinences du cardinal Albane\,; qu'il apprenait même, par les
lettres de Vienne, que c'était par les conseils de ce cardinal que le
comte de Gallas avait en dernier lieu bravé Sa Sainteté. Il ajouta que
le roi d'Espagne avait dessein d'envoyer enfin à Rome quelque esprit
turbulent, quelque homme de caractère à parler fortement, soit qu'il
fallût dire au cardinal Albane quatre mots à l'oreille, soit, qu'il
convînt de découvrir au pape le manège que son neveu, conduit par un
intérêt vil et sordide, pratiquait avec les Allemands, manège indigne
qui déconcertait absolument les serviteurs de Sa Sainteté par les
fausses démarches qu'on lui faisait faire, en sorte qu'Albéroni, se
mettant à la tête de ceux qui soutenaient avec plus de zèle les intérêts
du saint-siège, se plaignait de se voir hors d'état de rien faire
d'utile auprès du roi d'Espagne. Le nonce Aldovrandi, toujours attentif
à ménager le premier ministre, dont la protection lui paraissait
absolument nécessaire pour l'avancement de sa fortune, ne cessait
d'exalter ses bonnes intentions, et de conseiller au pape de profiter
d'une conjoncture où les dispositions du roi d'Espagne pour l'Église
étaient excellentes aussi bien que celles d'Albéroni. Le nonce
représenta qu'on irritait l'un et l'autre en refusant si longtemps les
bulles de Séville\,; qu'il était cependant essentiel pour la religion
d'entretenir le roi d'Espagne dans les sentiments qu'il avait eus
jusqu'alors, et de ne le pas irriter quand il y avait lieu de craindre
des divisions déplorables en Espagne\,; que plusieurs évêques de ce
royaume étaient attachés à la doctrine de saint Thomas\,; que plusieurs
de l'université d'Alcala suivaient la même doctrine\,; qu'ils
commençaient à trouver dans la constitution plusieurs articles
contraires aux leçons de cette école\,; que déjà quelques évêques
s'excusaient de parler et d'écrire au sujet de la constitution, sous
prétexte de leur crainte de se commettre avec le tribunal du
saint-office, à qui seul la publication des décrets apostoliques était
réservée. Ce nonce, loin d'imiter celui de France, concluait que, si
Rome voulait conserver l'Espagne, il fallait ménager non seulement le
roi d'Espagne et son ministre, mais de plus qu'il était nécessaire de
s'accommoder à la manière de penser des évêques. Ceux dont les
intentions étaient les meilleures souhaitaient d'être invités pour avoir
lieu de parler, ou de la part du pape, ou du moins de celle de son
nonce. Il croyait qu'il ne pouvait leur refuser cette satisfaction, et
que, de plus, il serait nécessaire de leur insinuer d'éviter de poser
l'infaillibilité du pape pour principe de leurs arguments. Mais parmi
ces souplesses pour obtenir ces bulles si désirées, l'empereur vint à la
traverse et s'y opposa ouvertement. Il fit dire au pape, par Gallas son
ambassadeur, qu'on avait découvert à Vienne, par des lettres
interceptées en Transylvanie, qu'Albéroni avait entamé un traité avec
Ragotzi par le prince de Cellamare, et qu'il s'agissait de former une
ligue entre le roi d'Espagne et la Porte. Gallas déclara qu'il en avait
les preuves, et qu'il en instruirait les cardinaux lorsque le pape
voudrait proposer Albéroni pour l'archevêché de Séville. La moindre
instance faite au pape, de la part de l'empereur, était menace. Il
tremblait à la voix des Allemands, le coeur lui manquait. Le point
principal de sa politique était de gagner du temps. Acquaviva,
connaissant parfaitement son caractère, crut à propos de profiter des
apprêts de l'Espagne pour l'Italie, et de parler ferme dans un temps où
tout se préparait dans les ports d'Espagne pour faire passer des
vaisseaux dans la Méditerranée. Il dit donc, après avoir insisté
fortement sur les bulles de Séville, que Sa Majesté Catholique ne
doutait pas que Sa Sainteté ne voulût bien accorder aux vaisseaux
espagnols les ports d'Ancône et de Civitta-Vecchia, et regarder en cette
occasion ce prince comme du même pays. Il ajouta que la proposition
était d'autant plus juste que, lorsque les Allemands marchèrent à la
conquête du royaume de Naples, Sa Sainteté leur accorda bon passage par
toutes les terres de l'Église\,; qu'elle devait regarder la démarche du
roi d'Espagne plutôt comme un avertissement de bienséance que comme une
demande, parce qu'il n'était pas à croire que le pape voulût forcer Sa
Majesté Catholique à recourir aux armes pour obtenir ce qui lui était dû
avec autant de justice. Acquaviva n'eut pas réponse sur-le-champ.
Quelques jours après, ayant envoyé l'auditeur de rote, Herrera, la
demander à Paulucci, ce cardinal lui dit que le pape n'était pas encore
déterminé sur cet article. L'auditeur insistant, Paulucci répliqua que
Sa Sainteté n'accordait ni ne refusait encore, qu'elle répondrait dans
le cours de la semaine, qu'il paraissait cependant que la chose pouvait
recevoir encore quelque difficulté.

Les preuves que Gallas prétendait avoir de la négociation entamée par le
cardinal Albéroni avec la Porte ottomane consistaient en deux lettres,
qu'on disait que l'ambassadeur turc, aux conférences de la paix, avait
remises à Belgrade à l'ambassadeur d'Angleterre. Les Impériaux
soutenaient que, pendant qu'Albéroni traitait directement à la Porte
pour y exciter à la continuation de la guerre, l'ambassadeur d'Espagne
en France avait traité secrètement à Paris pour la même fin avec le
prince Ragotzi. Ils soupçonnaient même le régent au sujet de cette
négociation secrète, et croyaient que, si Son Altesse Royale ne l'avait
pas approuvée, au moins elle ne l'ignorait pas. Cellamare démentit
hautement les bruits répandus sur ce sujet par les ministres de
l'empereur, faisant toutefois connaître que, quand même le fait dont ils
l'accusaient serait vrai, il n'aurait point à s'en justifier.

La cour d'Espagne espérait encore au commencement d'avril que la paix
avec les Turcs était encore éloignée. D'autres motifs confirmaient
encore cette cour a rejeter les propositions du traité qui se négociait
à Londres. Comme la paix ne convenait pas aux vues d'Albéroni, et qu'il
croyait que le trouble général de l'Europe était nécessaire pour appuyer
ceux qu'il voulait exciter en France, rien n'ébranlait ses résolutions.
Il savait que l'empereur envoyait de nouvelles troupes en Italie. On
disait que ce prince était sûr du roi de Sicile, qu'il ne dépendait que
de la cour de Vienne de conclure, quand elle voudrait, aux conditions
qu'il lui plairait d'imposer, le traité que deux Piémontais négociaient
secrètement avec cette cour. Ces dispositions, le nombre d'ennemis qui
s'unissaient contre l'Espagne, le peu d'espérance d'avoir des alliés
utiles, l'apparence morale de succomber étant dénué de tout secours,
enfin aucune de toutes les considérations les plus pressantes, ne
pouvait faire changer l'opposition que Sa Majesté Catholique, entraînée
par son ministre, témoignait pour le projet que la France et
l'Angleterre lui proposaient. Cellamare, suivant les ordres du roi son
maître, ne perdait aucune occasion de parler contre ce traité. Il disait
qu'il ne comprenait pas que les ministres de France eussent pu seulement
l'examiner. Il attaquait la disposition faite de la Sicile comme une
clause qui détruisait absolument le fondement de la paix d'Utrecht.
Stairs pour l'adoucir voulut lui faire sentir l'intérêt que les
Napolitains, dont les biens étaient confisqués par l'empereur,
trouveraient à la conclusion d'un traité où la restitution réciproque
des confiscations serait stipulée comme un des principaux articles\,;
mais Cellamare était trop délié pour témoigner inutilement, avant que la
paix fût faite, la satisfaction qu'il aurait de rentrer par cette voie
dans la jouissance de ses biens. Il se plaignit au contraire plus
fortement et de la négociation et du mystère que l'on faisait au roi
d'Espagne de ce qui se passait dans le cours d'une affaire où ce prince
avait tant d'intérêt. On commençait à parler d'une rupture prochaine
entre la France et l'Espagne. Cellamare dit qu'il n'était pas inquiet de
ces bruits, mais qu'il voyait avec déplaisir que le fondement de ces
discours, si éloignés des sentiments du roi et de la nation française,
et si éloignés des intérêts de Sa Majesté, était la crainte excessive
que le gouvernement avait de se trouver engagé dans une guerre
nouvelle\,; que cette crainte était cause que le régent se rendait sourd
à toutes les représentations tendantes à l'engager à prendre les armes.
Il ajoutait qu'il était à craindre que Son Altesse Royale, agissant sur
ce principe, n'offrît aux Anglais des choses aussi peu convenables à son
propre honneur qu'elles seraient contraires aux intérêts de l'Espagne\,;
que celui de M. le duc d'Orléans était de ne pas s'opposer aux desseins
que Sa Majesté Catholique pouvait former contre les ennemis communs si
naturels de sa maison, et de laisser à ce prince le moindre lieu de
soupçonner que les sentiments de Son Altesse Royale à son égard ne
fussent pas sincères.

Suivant les instructions d'Albéroni, Cellamare traitait de pot-pourri le
traité fait à Londres. Il se flattait même d'avoir obligé le maréchal
d'Huxelles à convenir de l'importance dont il était de ne pas altérer
par quelque résolution imprudente, et par le désir singulier de
soutenir, au préjudice du roi d'Espagne, des projets avantageux à
l'empereur, l'union qu'il était si nécessaire à maintenir entre les
François et les Espagnols. Après cet aveu du maréchal d'Huxelles,
Cellamare lui dit qu'on prétendait que l'abbé Dubois et Chavigny,
engoués tous deux de leurs négociations, travaillaient à les soutenir
par la violence\,; que leur vue était d'unir le régent au roi
d'Angleterre, dont le procédé devenait de jour en jour plus suspect au
roi d'Espagne\,; que cette union n'empêcherait pas cependant que la
réception favorable que Nancré avait eue à Madrid ne fût suivie de
toutes sortes de bons traitements, quoique d'ailleurs le roi d'Espagne
eût lieu de juger que cet envoyé était chargé de propositions peu
agréables à Sa Majesté Catholique. Pendant que l'ambassadeur d'Espagne
s'expliquait ainsi à celui qui devait en rendre compte au régent, il
parlait avec moins de modération aux différents ministres que les
princes d'Italie entretenaient à Paris. Il leur disait que le roi son
maître détestait la chaîne qu'on prétendait imposer à leurs
souverains\,; que les propositions de la France seraient mal reçues à
Madrid\,; que l'espérance de la succession de Parme était méprisée du
roi et de la reine d'Espagne\,; que l'un et l'autre avaient en horreur
le projet de remettre la Sicile entre les mains des Autrichiens, et que
Leurs Majestés Catholiques regardaient la proposition de laisser le
reste de l'Italie en l'état où elle se trouvait lors comme pernicieuse.
Il gémissait ensuite, soit avec ces ministres, soit avec d'autres, sur
ce que la France voulait la paix à quelque prix que ce fût, parce que le
régent la croyait nécessaire pour la validité des renonciations. C'était
une partie des manèges que Cellamare faisait pour acquérir des amis au
roi son maître, et pour empêcher l'exécution du traité. La cour de
Vienne, qui en devait recueillir les principaux avantages, ne se
pressait pas cependant d'y souscrire, et dans la fin de mars les
principaux ministres de l'empereur étaient encore partagés sur le parti
que ce prince devait prendre. Enfin la conclusion de la paix avec les
Turcs devenant plus que jamais probable au commencement d'avril,
l'empereur consentit à tous les points du traité. On dit même alors que
l'accommodement du roi de Sicile était fait, et que le mariage d'une
archiduchesse avec le prince de Piémont était une des principales
conditions.

Le prince de Cellamare, suivant ses ordres, déclara que le roi son
maître n'accepterait jamais un tel traité\,; que, tout l'avantage étant
pour la maison d'Autriche, l'acceptation de l'empereur ne serait pas un
exemple pour Sa Majesté Catholique. Malgré ces protestations, on ne
désespéra pas encore de le persuader. Comme le roi d'Espagne n'avait pas
refusé positivement, le régent dépêcha un courrier exprès pour porter à
Madrid la nouvelle du consentement de l'empereur, espérant que, le roi
d'Espagne voyant les principales puissances de l'Europe concourir
également à l'exécution de ce projet, Sa Majesté Catholique surmonterait
aussi sa répugnance à l'accepter. En effet, elle n'avait point rendu de
réponse précise\,; le cardinal avait seulement amusé Nancré et le
colonel Stanhope, en leur disant qu'il fallait attendre la réponse de
Vienne avant que le roi d'Espagne prît sa dernière résolution. Ce
premier ministre se contentait de combattre le projet de toutes ses
forces, en toutes ses parties, et de se retrancher sur la juste horreur
que la reine d'Espagne avait conçue sur ce qui se proposait à l'égard de
Parme. S'il se contenait un peu en parlant aux ministres de France et
d'Angleterre, il se déchaînait avec les autres, et furieusement contre
la paix d'Utrecht, et s'emporta même un jour jusqu'à dire à
l'ambassadeur de Portugal, que ce ne serait pas le premier traité rompu
aussitôt que conclu. Toutefois il affectait de ménager Nancré\,; il
avait avec lui de longues conférences tête à tête\,; l'accueil que
Nancré recevait de la cour était très distingué. Enfin, à juger par les
démarches extérieures, on pouvait penser que cette négociation
particulière était agréable au roi d'Espagne et à son ministre. Bien des
gens même soupçonnèrent qu'il y avait peut-être quelque intelligence
secrète entre les deux cours, que celle d'Angleterre ignorait et dont
elle serait la dupe. On s'épuisait en raisonnements\,; on jugeait bien,
par l'empressement de tant de préparatifs de guerre, que l'Espagne
rejetterait le traité\,; mais on ne pouvait se figurer qu'elle voulût
faire la guerre sans alliés, et on se persuadait qu'elle était assurée
de la France ou du roi de Sicile, parce que nulle autre alliance ne lui
paraissait si naturelle. Le roi de Sicile venait encore d'envoyer depuis
peu le président Lascaris à Madrid, quoiqu'il y eût l'abbé del Maro pour
ambassadeur ordinaire. On ne doutait donc point de quelque liaison
secrète, ou déjà prise, ou prête à prendre avec lui. Mais ces
raisonnements étaient vains, l'Espagne était véritablement sans pas un
allié. Son tout-puissant ministre déplorait inutilement l'aveuglement de
toute l'Europe, de la France surtout, qui manquait selon lui la plus
belle occasion du monde, et la plus facile, de mettre des bornes à la
puissance de l'empereur, et de chasser pour toujours les Allemands
d'Italie. À l'égard du roi de Sicile, quoiqu'il comptât peu sur l'envoi
de Lascaris, et qu'il ne doutât point que ce prince ne traitât avec le
ministre arrivé de Vienne à Turin, il avait une telle opinion de
l'infidélité de la Savoie, qu'il ne doutait pas que l'empereur n'en fût
trompé si la France voulait s'unir contre lui à l'Espagne. Malgré toute
l'affectation de fermeté et de tout espérer de la guerre, Albéroni
éprouvait de grandes agitations intérieures sur l'incertitude des succès
où il allait se livrer. Il avouait que, le roi d'Espagne étant seul,
l'entreprise était fort difficile\,; il disait qu'il satisfaisait au
moins à son honneur et montrait le chemin aux autres princes\,; il
laissait échapper des menaces contre ceux qui, après coup, se voudraient
joindre à Sa Majesté Catholique\,; il ajoutait que la guerre n'était
point de son goût, et qu'il en avait de bons témoins, et se faisait un
mérite de toutes les iniquités qu'il attirait sur soi par le seul zèle
de bien servir son maître. C'était par ce zèle qu'il traitait le traité
de chimérique, les conditions d'impossibles, et qu'il s'étonnait que
l'abbé Dubois eût pu penser que l'Espagne donnât dans des absurdités
pareilles, et pût compter sur le frivole de garanties dont on la
leurrait. Il dit au colonel Stanhope qu'il croyait de la prudence de
faire quelquefois des réflexions sur les variations du gouvernement
d'Angleterre, fondées sur ses discussions domestiques et sur le
changement de tout le ministère et de tous ses principes, comme il était
arrivé à l'avènement et à la mort de la reine Anne, d'où il concluait
qu'on ne pouvait jamais compter de sa part sur rien de solide ni de
durable. Il déclamait contre la mauvaise foi de la France et de
l'Angleterre, convenues de tout, selon lui, avec l'empereur depuis
longtemps, dont les offices à Vienne n'étaient que grimaces
concertées\,; que ce projet, communiqué si tard à l'Espagne, et encore
par parties, était si peu secret, que toute la teneur en avait été
écrite depuis longtemps de Venise et de Rome, jusque-là qu'une gazette
de Florence s'en était moquée et s'en était expliquée fort nettement\,;
de là Albéroni s'exhalait en invectives sans mesures, en menaces
figurées et en d'autres plus ouvertes, pleines de vanteries, sur la
bonté du gouvernement qu'il avait établi et le grand pied où il était
venu à bout de remettre l'Espagne\,; il finissait par des avertissements
très malins et menaçants pour M. le duc d'Orléans.

Nancré s'était alors expliqué sur tous les points de sa commission\,;
Albéroni appela cela avoir enfin vomi tout ce qu'il avait apporté,
digéré et non digéré après un long secret. Il s'emporta avec fureur
contre l'échange de la Sicile pour la Sardaigne, le traita de
scandaleux, demanda si la France, non contente d'avoir arraché cette île
à l'Espagne, voulait encore la priver du droit de réversion stipulé par
le traité d'Utrecht, et mettre le comble à la puissance de l'empereur en
lui donnant les moyens de former une marine, la seule chose qui lui
manquait, de devenir le maître absolu de la Méditerranée, de
l'Adriatique, de l'Archipel, et d'y porter quand il lui plairait toutes
les forces du septentrion. Dans sa fureur, il traita ces projets de
bestialité, de fous ceux qui les approuvaient, d'abandonnés de Dieu\,;
l'abbé Dubois d'aveugle, de dupe des Anglais, de dépourvu de tout esprit
de conseil, et qui entraînait la France et le régent dans le précipice.
Il distinguait le maréchal d'Huxelles seul des auteurs et approbateurs
d'un si pernicieux conseil. Il protesta que, quoi qu'il pût arriver, le
roi d'Espagne ne changerait point de sentiment\,; qu'avec la fermeté
qu'il avait marquée dans les temps les plus malheureux, il ne recevrait
pas des lois honteuses avec quatre-vingt mille hommes bien lestes et
bien complets, des forces de nier au delà de ce que l'Espagne en avait
jamais eu, des finances réglées comme une horloge et le commerce des
Indes bien disposé\,; qu'il mourrait l'épée à la main s'il le fallait
plutôt que de laisser les Anglais distribuer et changer les États à leur
gré, en maîtres du monde, et que, si le roi d'Espagne y périssait, on
dirait que ceux qui avaient un intérêt commun avec lui auraient
contribué à sa perte. Il chargea Monteléon de parler à l'abbé Dubois
comme il parlait lui-même à Nancré, et de lui faire faire les mêmes
réflexions s'il en était capable. Furieux contre la France, il ne
l'était pas moins du refus de ses bulles de Séville. Il s'en plaignit en
termes très forts à Paulucci, traita l'accusation de Gallas contre lui
d'impostures infâmes, sacrilèges, d'invention diabolique\,; il assura
que, quelque mépris que le roi d'Espagne eût pour une si noire calomnie,
il s'en vengerait, non par une autre, mais par les armes, cette voie
étant la seule dont les rois doivent se servir, et laisser l'imposture
aux âmes viles. Il triompha ensuite de désintéressement et de désir de
tout sacrifice personnel, mais en déclarant que, l'outrage étant fait
aux justes droits de la couronne d'Espagne, le roi catholique les
soutiendrait avec la dernière vigueur. Parmi tant de divers
emportements, Albéroni traitait Nancré avec tant de distinction et
d'apparente confiance, que ceux qui ne voyaient que ces dehors croyaient
que la négociation faisait de grands progrès. On voyait néanmoins les
préparatifs de guerre pressés avec plus de diligence que jamais, et que
les discours des gens qui pouvaient être instruits ne tendaient
nullement à la paix.

Castañeda, chef d'escadre, envoyé depuis quelque temps en Hollande, pour
y acheter des vaisseaux pour l'Espagne, reçut de nouveaux ordres d'en
revenir. Albéroni avait besoin de lui pour l'exécution de ses desseins,
et fatigué des difficultés qui, malgré la confiance de Beretti,
retardaient toujours cette affaire, le cardinal dit qu'il n'en avait
plus besoin, et que l'Espagne avait assez de navires pour se faire
respecter dans la Méditerranée, résolue, à quelque prix que ce fût,
d'assurer l'équilibre de l'Europe ou de la mettre toute en combustion.
Outre les ministres impériaux, ceux d'Angleterre et de Portugal, quoique
sans guerre, avaient traversé tant qu'ils avaient pu l'achat des
vaisseaux. Beretti ne s'en était pas moins vanté comme on l'a vu\,; il
voulut même prendre à bon augure la nomination qui fut faite de députés
pour examiner cette affaire, et dit à Castañeda, qui en jugeait bien
plus sainement, que c'était par le peu d'usage qu'il avait de la forme
du gouvernement de Hollande. L'armement de cette république pour la
Baltique était encore incertain\,; mais celui de l'escadre anglaise pour
la Méditerranée était public avec sa destination pour cette mer, surtout
depuis les menaces de Monteléon là-dessus. Les ministres d'Espagne ne
savaient quel parti le régent prendrait en cette occasion pour ou contre
leur maître, ou s'il demeurerait neutre, et Beretti se plaignait
amèrement du silence de Madrid, et de se trouver en des conjonctures si
difficiles sans ordres et sans instructions. Monteléon dans Londres n'en
recevait pas plus que lui à la Haye. Albéroni désirait peut-être qu'ils
fissent des fautes, et croyait utile de conserver la liberté de
désavouer les ministres d'Espagne, et les engagements qu'ils auraient
pris quand il lui plairait de le faire\,; il ne s'était encore expliqué
précisément que sur l'envoi de l'escadre anglaise, parle mémoire qu'il
avait fait présenter par Monteléon. La cour et ses partisans affectaient
de souhaiter la paix, et répandaient dans le public que l'envoi de cette
escadre n'avait d'autre objet que de faire valoir la médiation de
l'Angleterre, et de procurer plus aisément par là une tranquillité
générale. Ceux qui étaient opposés à la cour de tout parti favorisaient
l'Espagne, pour contredire Georges et ses ministres. Les négociants
étaient alarmés dans la vue de l'interdiction prochaine de leur
commerce. Monteléon, parmi ces différentes dispositions, continuait de
conseiller de faire tomber sur la cour de Vienne le blâme du refus des
conditions du traité, en différant une réponse absolument négative, et
se contentant, en attendant la réponse de Vienne, de représenter
doucement les inconvénients de ces conditions. Lui-même agissait dans
cet esprit auprès de l'abbé Dubois, et il interprétait en mal tout ce
que l'empereur faisait dire par le roi d'Angleterre, tendant au refus ou
à l'acceptation. On savait qu'il y avait à Vienne des émissaires du roi
de Sicile, qui traitaient avec le prince Eugène fort secrètement, et la
négociation passait pour avancée. Schaub voulut demander quelque
éclaircissement là-dessus, mais il n'en put tirer d'autre sinon que la
négociation existait. Monteléon n'oublia rien pour rendre les Impériaux
suspects à Londres et à l'abbé Dubois quelque parti qu'ils prissent de
refuser ou d'accepter. Il voyait souvent l'abbé Dubois même avec une
sorte de confidence. Cet abbé l'assura que Georges tiendrait ferme sans
se laisser amuser ni tromper par les Impériaux\,; que, si l'Espagne
acceptait, l'escadre anglaise serait à la disposition du roi
catholique\,; si Vienne refusait, l'Angleterre laisserait agir
l'Espagne, et prendrait d'autres mesures de concert avec la France, si
le roi de Sicile traitait avec l'empereur\,; en ce cas l'Angleterre
pourrait se joindre avec la France et l'Espagne, et les aider à ramener
la Sicile sous la domination d'Espagne. Il dit que, si cette couronne
avait quelque complaisance, et qu'elle parût disposée à accepter le
projet, elle retirerait de grands avantages de cette démonstration\,;
que la conjoncture était d'autant plus favorable que le ministère
anglais était mécontent de l'empereur, et qu'il y avait eu de fortes
paroles entre Stanhope et Penterrieder. Monteléon était persuadé qu'au
point où en étaient les choses, il n'y avait de parti à prendre pour
l'Espagne que de céder aux conseils absolus de la France et de
l'Angleterre\,; mais il n'osait avouer ce qu'il pensait. Il savait que
ce serait déplaire à Albéroni avec qui il n'était pas assez bien pour
lui écrire d'une manière directement opposée aux sentiments d'un homme
si porté à la vengeance, si fougueux et si totalement puissant.

Cependant les ministres d'Angleterre, connaissant l'intérêt particulier
qu'ils avaient d'empêcher une guerre dont la nation commençait à leur
reprocher l'inutilité et les fâcheuses conséquences\,; essayaient
également d'amener l'empereur et le roi d'Espagne à la paix\,; mais ils
négociaient différemment à l'égard de l'un et de l'autre. Ils louèrent
Albéroni de la bonne foi dont il avait parlé au colonel Stanhope, et
dirent qu'elle avait suspendu la réponse aux instances de Monteléon sur
l'escadre, parce qu'il aurait été impossible de n'y pas user de termes
qui ne convenaient pas entre deux puissances amies, également
intéressées à entretenir entre elles la plus étroite union. Stanhope fit
valoir comme une marque d'attention qu'au lieu de répondre au mémoire de
Monteléon, il écrivait directement à Albéroni que l'escadre destinée
pour la Méditerranée servirait le roi d'Espagne, quelque parti que prît
l'empereur de refuser ou d'accepter le projet du traité. Il en exalta de
nouveau les avantages et de quelle importance il serait pour le roi
d'Espagne d'avoir un pied en Italie, et de mettre actuellement garnison
espagnole dans Livourne, assuré de la garantie des principales
puissances de l'Europe. Monteléon, flatté par ces discours, était
persuadé que le roi son maître réussirait s'il voulait contracter une
alliance solide avec la France, l'Angleterre et la Hollande\,; qu'il ne
tiendrait qu'à lui de stipuler de la part de ces puissances un
engagement formel d'empêcher à jamais les Impériaux d'exercer des
vexations en Italie, et sous des prétextes mendiés d'attaquer ces
princes dans leur liberté, leurs biens et leur souveraineté. Mais,
pendant que Stanhope lui donnait de si bonnes paroles et de si belles
espérances, ce ministre et Sunderland assuraient tous deux Penterrieder
que, si l'empereur voulait signer le traité, le roi d'Angleterre en
remplirait fidèlement les engagements, et qu'il se porterait aux
dernières extrémités contre l'Espagne.

Les ministres d'Angleterre crurent apparemment devoir s'expliquer si
clairement pour déterminer la cour de Vienne, parce qu'ils surent que la
négociation du roi de Sicile avançait, qu'elle était fortement appuyée
par quelques Espagnols impériaux que ce prince avait gagnés, et qu'ils
conseillaient à l'empereur de s'emparer de Parme et de Plaisance, pour
échanger cet État contre la Sicile. Les ministres piémontais
travaillaient également de tous côtés pour traverser le traité de
Londres, et pendant qu'ils faisaient leurs efforts à Vienne pour unir
leur maître avec l'empereur, ils se liaient eux-mêmes avec les ministres
des princes d'Italie, en France et en Angleterre, pour empêcher le
succès du projet concerté entre le régent et le roi d'Angleterre. Ce
prince connaissait combien les vues du roi de Sicile étaient
dangereuses, et par conséquent de quelle importance il était d'empêcher
qu'il ne réussît à Vienne, et que par ses manèges il ne parvînt au but
qu'il se proposait d'obtenir, une archiduchesse pour le prince de
Piémont. Ainsi, pour l'empêcher, le roi d'Angleterre fit connaître aux
ministres impériaux que, si les bruits qui couraient de ce mariage se
vérifiaient, il lui serait désormais impossible d'entretenir avec
l'empereur les mêmes liaisons et la même confiance qu'il avait eues par
le passé. Il ajouta même aux ordres qu'il donna là-dessus à
Saint-Saphorin des lettres pour l'empereur et pour l'impératrice Amélie,
mère des archiduchesses.

\hypertarget{note-i.-le-garde-des-sceaux-dargenson.}{%
\chapter{NOTE I. LE GARDE DES SCEAUX,
D'ARGENSON.}\label{note-i.-le-garde-des-sceaux-dargenson.}}

Le marquis d'Argenson donne dans ses Mémoires\footnote{Édition de 1825,
  p.~183 et suiv.} des détails assez étendus sur son père. Il ne sera
pas sans intérêt de les comparer avec ce que Saint-Simon dit de ce même
personnage. C'est un complément indispensable de ses Mémoires. Voici
quelques extraits des notes du marquis d'Argenson sur son père.

«\,Mon père naquit à Venise\,: il eut la république pour marraine, et
pour parrain le prince de Soubise, qui voyageait alors en Italie. J'ai
une lettre originale de Balzac\footnote{Jean-Louis Guez, seigneur de
  Balzac, gentilhomme du pays d'Angoumois, était en relation d'amitié
  avec la famille d'Argenson.} sur sa naissance\,: il prophétise une
grande illustration au \emph{petit Venise}. Mon père, ayant achevé ses
études à Paris, revint en Touraine. Il voulait servir\,; la tendresse
paternelle s'y opposa. L'âge gagnait\,; il était un peu tard pour
aborder une autre carrière. Mon père trouva des ressources du côté
maternel. M. Houlier, son aïeul maternel, vivait encore\,; il était
lieutenant général au bailliage d'Angoulême\,: il proposa de lui
résigner sa charge\,; c'était un des beaux ressorts du royaume. Mon père
accepta non sans répugnance, mais ne pouvant se faire au désoeuvrement.
Mon père eut de tout temps l'amour du travail\,; j'en possède des
preuves multipliées remarques sur ses lectures, dissertations sur la
politique, extraits historiques, études du droit public et particulier,
j'en ai des volumes. De quoi cela pouvait-il servir à un pauvre
gentilhomme campagnard, ou même à un juge de province\,? Mais cette
charge subalterne était déjà une magistrature.

«\,Cependant mon père était recherché par ce qu'il y avait de meilleure
compagnie dans la province\,; il était de toutes les fêtes, convive
aimable et plein d'enjouement, avec cela un esprit nerveux, une âme
forte, le coeur aussi courageux que l'esprit, de la finesse dans les
aperçus, de la justesse dans le discernement\,; peut-être ne se
connaissait-il pas lui-même\,; il ignorait la portée de son génie.

«\,Parfois il éprouvait bien des tracasseries de la part de ceux de sa
compagnie\,: on trouvait qu'il passait vite sur les formes pour en venir
plus tôt au fond et, à l'essentiel, c'est-à-dire à la justice. Il
accommodait les procès, épargnait les épices aux plaideurs\,; il faisait
beaucoup de bien\,; c'en était assez pour causer le récri de ces êtres
entichés des droits, c'est-à-dire des profits de leurs charges.

«\,Mais voici le commencement de la fortune de mon père, élévation qu'il
ne dut assurément qu'à lui-même et à ses talents, auxquels il ne
manquait qu'un plus grand théâtre pour être généralement reconnus. En
1691 ou 1692, on envoya dans les provinces une commission des
\emph{Grands Jours\footnote{Il s'agit probablement ici de la commission
  des \emph{Grands Jours}, qui se rendit à Poitiers en 1688. Voy. la
  Notice sur les Grands Jours, à la suite des \emph{Mémoires de
  Fléchier} (édit. Hachette, p.~315).}}. L'un des commissaires fut M. de
Caumartin, qui est devenu mon oncle. Quand la commission vint à
Angoulême, elle fut frappée au premier abord du mérite du lieutenant
général\,; il leur partit bien au-dessus de tout ce qu'ils avaient
rencontré dans leur tournée. M. de Caumartin, qui se piquait de
connaissances généalogiques, connaissait d'avance notre famille et le
rang qu'elle avait tenu en Touraine\,; il s'engoua particulièrement pour
mon père. M. de Caumartin était allié de M. de Pontchartrain, et
jouissait d'un grand crédit près de ce ministre. Il pressa mon père de
l'accompagner à Paris. Tous les commissaires se joignirent à lui\,; il
n'y eut qu'une voix, offres sincères de service. Mon père refusa quelque
temps\,; il n'aimait point les chimères. Pourtant, au bout de peu de
mois, une affaire majeure l'appela à Paris et l'y fit séjourner.

«\,M. de Caumartin en profita pour le faire connaître de M. de
Pontchartrain, pour lors contrôleur général, et depuis chancelier de
France. M. de Pontchartrain reconnut la vérité de ce qui lui avait été
dit, et retint mon père près de lui. Il le chargea d'abord, pour
l'éprouver, de quelques commissions fort épineuses, dont il se tira avec
succès. Telle fut celle de réformer les amirautés, de revoir les
règlements de marine, de recomposer le tribunal des prises\,; et dans
ces affaires de marine, mon père se rendit si capable en peu de temps,
que, M. de Pontchartrain le borgne\footnote{Fils du chancelier.}. Q
ayant été reçu en survivance, on lui donna mon père pour instructeur.

«\,Ensuite il eut la commission de procureur général pour la recherche
des francs fiefs et des amortissements. Il y fit des travaux incroyables
et fit rentrer au roi plusieurs millions, ne s'attirant que respect et
éloge de sa justice et de son intégrité de la part des parties mêmes que
l'on recherchait. Mon père se délit alors de sa charge d'Angoulême. M.
de Caumartin lui fit épouser sa soeur, et M. de Pontchartrain approuva
ce mariage. Mon père avait quarante et un ans\,; il était bien fait, une
physionomie plus expressive qu'agréable. Ma mère eût pu faire un
meilleur mariage pour la fortune, mais elle refusa tout autre parti dès
qu'elle l'eut connu.

«\,Ce mariage et l'obligeance de quelques amis mirent mon père en état
d'acheter une charge de maître des requêtes, sans laquelle, de son
temps, on ne pouvait parvenir à rien\,; car il régnait alors des
principes d'ordre qu'on néglige beaucoup trop sous le règne
actuel\footnote{Le marquis d'Argenson écrivait sous le règne de Louis
  XV.}. Son heureuse étoile voulut qu'elles fussent à très bas prix. Mon
père recueillit aussi quelques héritages en ligne collatérale. Le
vicomte d'Argenson, son oncle, qui fut, pendant plusieurs
années\footnote{1657-1660.}, gouverneur de la Nouvelle-France (ou
Canada), lui donna ou assura, en faveur du mariage, la plus grande
partie de sa fortune, entre autres son hôtel, vieille rue du Temple, où
mon père alla demeurer en 1696.

«\,Ainsi mon père put s'établir, prendre femme et charge. Peu de temps
après, il fut question pour lui de l'intendance de Metz. On préféra lui
confier la police de Paris, M. de La Reynie s'étant retiré. On sait
comment il s'est acquitté de cette charge, et quels talents il y a
déployés. Dans cette charge, mon père était véritablement ministre\,: il
travaillait directement avec le feu roi, et était avec ce monarque en
correspondance continuelle. Il a été dis fois question de l'appeler au
ministère\,: la brigue de cour, la ligue des ministres s'y sont toujours
opposées, toujours sous le prétexte qu'on ne saurait trouver personne
pour le remplacer à la police de Paris en des temps aussi difficiles que
ceux de la dernière guerre. On l'a cru l'ami des jésuites beaucoup plus
qu'il ne l'était en effet. Il les connaissait mieux que personne, et n'a
jamais fait grand'chose pour eux. Or ces gens n'aiment point qu'on ne
travaille qu'à demi dans leurs intérêts. Mon père était aussi
médiocrement bien avec M\textsuperscript{me} de Maintenon\,: elle savait
l'apprécier\,; mais il était peu lié avec cette dame. Il était attaché
au maître en droiture. Les ministres le craignaient\,; les courtisans
l'évitaient autant qu'il savait se passer d'eux. M. de Bâville a été
précisément clans la même situation en Languedoc, où ses succès l'ont
confiné, mais lui ont valu un pouvoir souverain.

«\,Mon père possédait à la fois la sagesse de volonté et le courage
d'exécution. Au milieu du travail immense dont il était surchargé, mon
père a toujours été le plus imponctuel de tous les hommes\,: il ne
savait jamais quelle heure il était, et faisait de la nuit le jour et du
jour la nuit, selon qu'il lui convenait. Forcé de s'occuper d'une
multitude de détails, la plupart très importants, mais de différents
genres, il les faisait quand il pouvait ou quand il voulait, à bâtons
rompus, et coupait ou interrompait sans cesse l'un pour l'autre. Mais
son génie, également sûr et actif, suffisait à tout\,; il retrouvait
toujours le bout de ses fils, quoiqu'il les rompît à tous moments, et
saisissait successivement cent objets différents sans les confondre.

«\,J'ai la conviction\footnote{\emph{Mémoires du marquis d'Argenson}
  (édit. 1825), p.~176 et suiv.} que, de tous les hommes qui ont été en
place de nos jours, aucun n'a mieux ressemblé au cardinal de Richelieu
que mon père. Assurément ce grand ministre n'eût point désavoué le lit
de justice des Tuileries (26 août 1718). Il suffit de rappeler les
événements qui y donnèrent lieu. Une révolution affreuse était
imminente\,; jamais on n'en fut plus près\,; il n'y avait plus qu'à
mettre \emph{le feu aux poudres}, suivant l'expression du cardinal
Albéroni dans sa lettre interceptée. Le régent, trahi par son propre
ministère, l'opiniâtreté des parlements, l'inquiétude des protestants de
Poitou, les troubles de Bretagne, la conspiration de Cellamare, dans
laquelle étaient impliquées nombre de personnes de Paris, et dont les
fils étaient ourdis à l'hôtel du Maine\,; les querelles entre les
princes du sang et les légitimés, entre la noblesse et les ducs et
pairs, entre les jansénistes et les molinistes\,; toutes ces causes de
discorde fomentées et soldées par l'argent de l'Espagne\,; n'est-ce rien
que d'avoir sauvé le royaume de cet affreux tumulte, et des guerres
civiles qu'eût certainement entraînées la résistance d'un prince aussi
courageux que l'était M. le duc d'Orléans\,?

«\,Depuis la mort de Louis XIV, mon père avait été en butte à tous ces
petits seigneurs qui obsédaient l'esprit du régent. On lui donnait des
dégoûts dans sa charge\,; et pourtant on sait que le régent lui avait
des obligations essentielles qu'il n'eût pu oublier sans se rendre
coupable de la plus haute ingratitude\footnote{Voy. \emph{Mémoires de
  Saint-Simon}, t. VII, p.~321, t. XV, p.~256, et les \emph{Mémoires du
  marquis d'Argenson} (édit. 1825), p.~190, 191.}. Mon père était
informé de tout ce qui se tramait\,; il en avertissait M. le duc
d'Orléans. Celui-ci ne voulut reconnaître la vérité que lorsque les
choses furent parvenues à une évidence extrême. Mon père avait attendu
M. le duc d'Orléans au Palais-Royal jusqu'à deux heures après minuit.
Enfin ce prince, de retour d'une partie de plaisir, lui donna audience,
et reconnut, à des preuves irrécusables, les dangers de sa position. Il
fallait prendre un grand parti\,: mon père fut fait garde des sceaux et
président du conseil des finances. Jamais il n'y eut un coup d'État plus
hardi que celui par lequel il sauva son prince et sa patrie. Ce fut,
suivant l'expression d'un contemporain, une vraie \emph{Catilinade dont
mon père fut le Cicéron}.

«\,Personne ne parlait mieux en public que mon père\,; moins brillant
par une érudition de légiste que par une éloquence forte de choses, de
grandes maximes et de pensées élevées.

«\,Il fallut ensuite réparer les brèches ouvertes par les ennemis de
l'État. Nul ne savait mon père propre à l'administration des finances
comme il se l'est montré\,; mais la qualité d'homme sage, aimant le bien
public, ferme, travailleur et bon économe, est de beaucoup préférable à
cette maudite science financière qui a perdu la France. Mon père n'a
jamais été la dupe de Law, et je pense même que, s'il n'eût dépendu que
de lui, il eût donné la préférence aux projets de MM. Pâris, qui,
voulant opposer système à système, avaient un plan d'actions sur les
fermes qui devait nécessairement pâlir devant le funeste clinquant des
actions mississippiennes. Law et mon père ne s'accordèrent jamais
pleinement ensemble. Pourtant mon père fit la faute de remettre au
lendemain lorsqu'il reçut l'ordre d'arrêter Law et de l'enfermer à la
Bastille, et c'est ce qui décida sa disgrâce. Mon père en fut peu
affecté\,; mais il le fut beaucoup plus lorsqu'il vit que cette défaveur
entraînait aussi celle de mon frère\footnote{Le comte d'Argenson, qui
  fut ministre de la guerre sous le règne de Louis XV.}, malgré la
promesse contraire qu'il avait reçue du régent.

«\,Mon père conduisait les choses de son ministère avec un secret
admirable. En voici la preuve. J'avais soupé en ville\,; je rentrais
chez moi à une heure après minuit\,; le suisse me dit que M. le garde
des sceaux me demandait. Il s'agissait d'écrire quinze lettres
circulaires, sur sa minute, à autant d'intendants, et de ne me pas
coucher que tout ne fût terminé. Mon frère avait fini sa tâche qui était
d'autant, et s'était couché par ordre de mon père. Je pris du café et ne
me mis au lit qu'à quatre heures du matin. Il s'agissait d'une
augmentation de monnaies qui surprit tout le monde\,; car on avait fait
courir le bruit d'une diminution. Le lendemain cet édit fut publié, et
l'on fit porter nos lettres par des courriers. Ainsi mon père ne s'était
point fié à la discrétion de ses commis\,; il avait poussé la prévoyance
jusqu'à venir s'assurer par lui-même si nous nous étions couchés tous
les deux après avoir terminé nos écritures, l'appât d'un bénéfice sûr
pouvant être pour tous autres une violente tentation de divulguer ce
secret.\,»

\hypertarget{note-ii.-difficultuxe9-des-ruxe9formes-au-xviiie-siuxe8cle.}{%
\chapter{NOTE II. DIFFICULTÉ DES RÉFORMES AU XVIIIe
SIÈCLE.}\label{note-ii.-difficultuxe9-des-ruxe9formes-au-xviiie-siuxe8cle.}}

Saint-Simon, dans un des plus curieux passages de ses Mémoires, dit que
tout bien est impossible en France, et il allègue comme preuve ses vains
efforts, lorsqu'il était du conseil de régence, pour détruire certains
abus financiers. On trouve à peu près la même opinion exprimée dans les
Mémoires du marquis d'Argenson. Il venait de passer par le ministère, et
son frère était encore ministre, de la guerre, lorsqu'il écrivit la
partie de ses Mémoires inédits que je vais citer. Elle est datée de 1751
(29 juin)\,:

«\,Tout le monde dit ici {[}en France{]}, que le roi devroit retrancher
la dépense. Le parlement vient de le lui dire assez hardiment. On fait
même l'honneur à M. de Machaut de dire que c'est lui qui le suggère au
parlement, et qu'au moins il est bien aise que cela soit dit, pour faire
rentrer le roi en lui-même. Mais a-t-on bien réfléchi et connu
\emph{combien la moindre réforme est difficile} en France, sur le pied
où sont les choses\,? Chacun se tient l'un à l'autre. Il faudroit qu'un
ministre offensât ce qu'il y a de plus grand à la cour pour toucher aux
écuries, aux bâtiments, à la bouche\footnote{Ce mot désignait tous les
  officiers employés pour le service de la table et des cuisines du roi.},
aux extraordinaires de la maison du roi\footnote{C'est-à-dire aux
  dépenses extraordinaires de la maison du roi.}, aux dépenses des
voyages, aux pensions, aux gouvernements donnés à des gens qui ne
méritent rien et qui sont riches, et à toutes ces dépenses qui
consomment les finances. On choqueroit, on offenseroit par là grièvement
la maîtresse\footnote{M\textsuperscript{me} de Pompadour, qui avait
  contribué à faire renvoyer du ministère le marquis d'Argenson.}, le
grand-maître de la maison du roi, le premier maître d'hôtel, le grand
écuyer, le premier écuyer, les dames du palais, etc. Leurs cabales,
leurs agréments, la cour, les grands, les valets, tout cela se tient
l'un à l'autre\,: ainsi toutes ressources ne sont que des gouttes d'eau
dans la mer. C'est ce qui vient d'arriver aux nouveaux emprunts\,: à
peine y a-t-il eu deux millions de portés pour rentes viagères qu'ils
ont été mis à payer la maison du roi, à qui l'on doit encore beaucoup
par delà.

«\,Pour ce retranchement des dépenses du roi, il faudroit donc que le
caractère de facilité du roi se réformât, ou bien qu'il se donnât un
premier ministre bien autorisé, qui fût maître de tout, et que le roi
soutînt dans toutes ses opérations avec grande fermeté\,; ce qui lui est
très difficile. Il faudroit que ce vizir ne vît seulement pas la
marquise\footnote{La marquise de Pompadour.}, bien éloigné de recevoir
d'elle des ordres à chaque opération, comme on fait aujourd'hui. Ce
vizir devroit d'abord former une \emph{commission de réformation},
composée d'une douzaine de magistrats des plus sévères, qui réduisît
toutes les dépenses de la cour au pied le plus juste, et {[}jugeât{]} le
sujet de renvoi dans les provinces de tous ceux qui n'ont que faire à la
cour ni à la ville. Il faudroit que la cour vînt résider à Paris, avec
l'usage de quelques maisons de campagne pour le roi, pour la reine et
pour la maison royale.\,»

L'énoncé seul de ces idées prouve combien les réformes étaient alors
difficiles, pour ne pas dire impossibles. Le marquis d'Argenson imputait
surtout à la cour l'opposition à toutes les améliorations, et la
proclamait la cause principale des malheurs de la France à cette époque.

«\,\emph{La cour\,! la cour\,! la cour\,!} Dans ce mot est tout le mal
de la nation. La cour est devenue le seul sénat de la nation\,: le
moindre valet de Versailles est sénateur\,; les femmes de chambre ont
part au gouvernement\,; si ce n'est pour ordonner, c'est du moins pour
empêcher les lois et les règles\,; et, à force d'empêcher, il n'y a plus
ni lois, ni ordre, ni ordonnateurs\,; à plus forte raison quand il
s'agiroit de réformation dans l'État. Quand la réforme seroit si
nécessaire, tout ministre tremble devant un valet\,; et combien cela
est-il plus vrai, quand une favorite a grand crédit, quand le monarque
est facile et trop bon pour ce qui l'entoure\,?

«\,Cet ascendant de la cour est venu ainsi, depuis qu'il y a une
capitale exprès pour la cour (Versailles). Sous le feu roi, on s'en
ressentit, mais moins\,; car il était haut, ferme, et autorisoitt
beaucoup ses ministres, quelque chose qu'on en pût dire. Mais sous lui
et sous Louis XV, les ministres, en revanche, ont beaucoup perfectionné
l'autorité monarchique, arbitraire, la cour augmentant par là de pouvoir
sur la nation. Le goût du luxe s'est accru, de sorte qu'à mesure que la
noblesse est devenue plus pauvre, l'honneur de dépenser avec goût, le
déshonneur de l'économie, se sont accrus, et nous plongent chaque jour
davantage dans la nécessité de dépenser, soit en nous ruinant, soit en
rapinant.

«\,La cour empêche toute réforme dans la finance et en augmente le
désordre.

«\,La cour corrompt l'état militaire de terre et de mer par promotions
de faveur, et empêche que les officiers ne s'élèvent au généralat par le
mérite et l'émulation.

«\,La cour empêche le mérite, l'autorité et la permanence aux ministres,
et à ceux qui travaillent sous eux aux affaires d'État.

«\,La cour corrompt les moeurs\,; elle prêche aux jeunes gens, qui
entrent dans leur carrière, l'intrigue et la vénalité, au lieu de
l'émulation par la vertu, le mérite et le travail\,; elle casse le col à
la vertu, dès qu'elle se présente.

«\,Elle nous appauvrit, de sorte que bientôt les financiers mêmes
n'auront plus d'argent.

«\,Elle empêche enfin le roi de régner et de retrouver en lui la vertu
qu'il a.

«\,Elle appauvrit les provinces, attirant à Paris toute la graisse des
provinces.\,»

\hypertarget{note-iii.-journal-inuxe9dit-de-nicolas-joseph-foucault.}{%
\chapter{NOTE III. JOURNAL INÉDIT DE NICOLAS-JOSEPH
FOUCAULT.}\label{note-iii.-journal-inuxe9dit-de-nicolas-joseph-foucault.}}

On a déjà parlé plus haut (t. XII, p.~502) du journal inédit de
Nicolas-Joseph Foucault. Un des passages contient lé récit de l'incendie
des vaisseaux français par les Anglais après la bataille navale de la
Hougue\footnote{La Hougue-Saint-Waast (département de la Manche). On
  écrit quelquefois la Hogue.}. Si, comme l'avait demandé Seignelay, la
côte de Normandie avait eu son port militaire, la flotte dispersée y
aurait trouvé un asile. Mais on a vu que Louvois s'y était opposé.
Nicolas-Joseph Foucault, qui était alors intendant de Caen, fut témoin
oculaire de cet événement et en adressa la relation au ministre de la
marine\footnote{Journal manuscrit, Bib. imp., n° 229 des 500 de Colbert,
  folio 81 et suiv.}.

«\,M. de Tourville arriva à la Hougue avec douze vaisseaux le dernier
mai 1692, au matin\,; il mouilla le soir à la rade, à la portée du canon
de terre, le fond du bassin de la Hougue étant très bon pour l'ancrage.
Mais M. de Sepville, neveu de M. le maréchal de Bellefonds, qui montait
\emph{le Terrible}, pour avoir voulu ranger de trop près l'île de
Tatiou, s'échoua sur une pointe de roche qui paraît de basse mer\,; et
comme nos vaisseaux pouvaient approcher plus près de terre, le sieur de
Combes, qui a dressé des plans pour faire un port à la Hougue, fut leur
marquer le mouillage, et sur les neuf heures au matin du 1er juin, les
douze vaisseaux\footnote{Il s'agit toujours ici des vaisseaux qui
  avaient échappé au désastre de la Hougue.} vinrent chacun prendre leur
place, les ennemis demeurant toujours mouillés à deux portées de canon
du plus avancé en mer de nos vaisseaux.

«\,M. de Tourville, accompagné de MM. d'Anfreville et de
Villette\footnote{La \emph{Société de l'Histoire de France} a publié des
  \emph{Mémoires du marquis de Villette}, où l'on trouve un récit de la
  bataille navale de la Hougue\,; p.~113-143.}, vint trouver le roi
d'Angleterre \footnote{Jacques II.} à la Hougue pour prendre l'ordre de
ce qu'ils avaient à faire. Ils proposèrent tous trois d'attendre
l'ennemi et de se défendre. M. de Villette ayant dit, dans son avis,
que, si le vaisseau qu'il commandait était marchand ou corsaire, il le
ferait échouer, mais que, s'agissant des vaisseaux du roi, il croyait la
gloire de Sa Majesté intéressée à les défendre jusques à l'extrémité, le
roi d'Angleterre et le maréchal de Bellefonds furent sans balancer de ce
sentiment, et il fut résolu que nos vaisseaux demeureraient mouillés et
attendraient les ennemis. MM. de Tessé, lieutenant général, Gassion et
Sepville, maréchaux de camp, mylord Melford, MM\hspace{0pt}. de Bonrepos
et Foucault, furent présents à cette délibération\,; et MM. de
Tourville, Anfreville et Villette retournèrent chacun à son bord pour
donner ordre à tout. M. de Foucault y fut avec eux, et entra dans le
vaisseau de M. de Villette pour savoir si lui ou les autres capitaines
avaient besoin de quelque chose. On lui demanda de la poudre, la plupart
des vaisseaux n'en ayant pas suffisamment, et même celle qu'ils avaient
eue à Brest étant trop faible, ne poussant pas le boulet de moitié si
loin que celle des ennemis. Au surplus, le vaisseau de M. de Villette
était en fort bon état, et on assura ledit sieur Foucault qu'aux ancres
près, les autres étaient de même.

«\,On envoya en diligence chercher toute la poudre qui était dans les
magasins de Valogne et de Carentan\,; mais elle ne servit de rien\,; car
la résolution qui avait été prise le matin de se défendre à l'ancre, fut
changée le soir par M. le maréchal de Bellefonds en celle de faire
échouer les vaisseaux\footnote{Voy. le motif de ce changement de
  résolution dans les \emph{Mémoires du marquis de Villette},
  p.~134-135.}\,; et {[}celle-ci{]} ne fut néanmoins exécutée que le
lendemain, 2 juin, à la pointe du jour, avec beaucoup de précipitation,
de désordre et d'épouvante, les matelots ne songeant plus qu'à quitter
les vaisseaux et à en tirer tout ce qu'ils purent, depuis la nuit du
dimanche 1er juin jusques au lendemain sept heures du soir, que les
ennemis, qui n'avaient fait que rôder autour de nos vaisseaux sans en
approcher à la portée du canon, pendant qu'ils les avaient vus à flot,
envoyèrent des chaloupes sonder et reconnaître l'état où ils étaient.

«\,Voyant qu'il n'avait été pris aucune précaution pour en défendre
l'approche, ils firent avancer avec la marée une chaloupe qui vint
mettre le feu au vaisseau de M. de Sepville, qui était le plus avancé en
mer et entièrement sur le côté. D'autres chaloupes suivirent cette
première avec un brûlot, et vinrent brûler les cinq autres vaisseaux qui
étaient échoués sous l'île de Tatiou. On tira, à la vérité, plusieurs
coups de canon du fort sur ces chaloupes, mais ce fut sans effet, de
même que les coups de mousquet que nos soldats tirèrent du rivage, et
les ennemis ramenèrent leur brûlot n'ayant pas été obligés de s'en
servir. Tout cela se passa à la vue du roi d'Angleterre et de M. le
maréchal de Bellefonds, qui étaient au lieu de Saint-Waast, près la
Hougue, où ils demeurèrent fort longtemps à considérer ce triste
spectacle.

«\,Le lendemain, à huit heures du matin, les ennemis revinrent avec la
marée du côté de la Hougue, où étaient les six autres vaisseaux échoués
sous le canon du fort\,; ils y envoyèrent plusieurs chaloupes qui les
abordèrent et les brûlèrent avec la même facilité qu'ils avaient trouvée
la veille pour les six premiers, nonobstant le feu du canon du fort, et
celui d'une batterie que M. le chevalier de Gassion avoir fait dresser à
barbette\footnote{Espèce de plate-forme sans épaulement, d'où le canon
  tire à couvert.}, qui seule produisit de l'effet, ayant écarté
quelques chaloupes dont elle tua plusieurs hommes.

«\,Lorsque les ennemis eurent mis le feu à ces six vaisseaux, ils eurent
l'audace d'avancer dans une espèce de havre où il y avoir vingt
bâtiments marchands, deux frégates légères, un yacht et un grand nombre
de chaloupes, tous échoués près de terre, et brûlèrent huit vaisseaux
marchands, entrèrent dans une gribane et un autre bâtiment, qu'ils
eurent la liberté et le loisir d'appareiller et d'emmener avec eux en
criant\,: \emph{Vive le roi\,!} et, sans la mer qui se retirait, ils
auraient brûlé ou enlevé le reste. La première expédition ne leur avait
pas coûté un homme\,; il y en a eu peu de tués ou blessés en celle-ci,
quoique les ennemis se soient approchés si près du rivage, qui était
bordé de mousquetaires, que le cheval du bailli de Montebourg, qui était
près du roi d'Angleterre, eut la jambe cassée d'un coup de mousquet tiré
des chaloupes anglaises. Elles s'étaient fait suivre par deux brûlots
qui, pour s'être trop avancés, échouèrent sur des pêcheries, et les
ennemis y mirent le feu en se retirant.

«\,Il n'y a pas lieu de s'étonner que cette seconde entreprise ait si
bien réussi pour eux\,; il était trop tard, après les premiers vaisseaux
brûlés, de prendre des précautions pour sauver les autres, la mer ayant
été basse pendant la nuit qui fut l'intervalle des deux actions, et par
conséquent il n'aurait pas été possible de se servir de nos frégates et
de nos chaloupes qui étaient échouées.

«\,Mais voici la grande faute que l'on a faite et qui a causé tout le
mal\,: c'est de n'avoir pas pris, dès le 31 mai au soir, que nos
vaisseaux arrivèrent, la résolution de les faire échouer\footnote{Cf.
  les \emph{Mémoires du marquis de Villette}, qui exprime la même
  opinion.}.\,»

On adopta trop tard, comme le prouve le même Journal, les mesures
nécessaires pour fortifier la côte de Normandie. Louvois n'était plus là
pour s'opposer aux projets de Vauban, et l'on songea à les mettre à
exécution en 1694. «\,Au mois de mai 1694, dit Foucault\footnote{Journal
  manuscrit, fol.~87 recto.}, M. de Vauban est venu à la Hougue, dont il
a visité les fortifications. Il a cru qu'il fallait faire plusieurs
redoutes le long de la côte et un camp retranché à la tête de
Carentan.\,» Foucault ajoute\,: «\,Il a été imposé cinquante mille
livres sur les trois généralités de Normandie pour les ouvrages de la
Hougue\footnote{Journal manuscrit, fol.~87 recto.}.\,» Ces
fortifications élevées sur les côtes de Normandie n'empêchèrent pas les
ennemis de bombarder Granville en 1695. «\,Le 18 juillet, écrit
Foucault\footnote{\emph{Ibid}., fol.~90.}, à neuf heures du matin, les
ennemis ont paru devant Granville au nombre de neuf vaisseaux de guerre
et neuf galiotes à bombes, qui ont mouillé un peu hors de la portée du
canon. Ils ont bombardé la ville jusqu'à six heures du soir, et ont jeté
cinq cents bombes. La première galiote a été obligée de se retirer par
notre canon. Il y a eu six maisons endommagées dans la ville, et sept à
huit couvertes de chaume dans le faubourg.\,»

\hypertarget{note-iv.-les-chanceliers-pendant-le-ruxe8gne-de-louis-xiv.}{%
\chapter{NOTE IV. LES CHANCELIERS PENDANT LE RÈGNE DE LOUIS
XIV.}\label{note-iv.-les-chanceliers-pendant-le-ruxe8gne-de-louis-xiv.}}

Dans une note du tome X, page 477, des \emph{Mémoires de Saint-Simon},
nous avons indiqué les chanceliers et gardes des sceaux de France
pendant la première moitié du XVIIe siècle. À l'occasion de la mort du
chancelier Le Tellier (30 octobre 1685), Saint-Simon caractérise les
chanceliers de la fin du siècle \footnote{Notes sur le Journal de
  Dangeau. Voy. le \emph{Journal du marquis de Dangeau} avec les
  additions de Saint-Simon (édit. Didot, t. Ier, p.~242-243).}. Nous
compléterons ce tableau par quelques extraits des Mémoires du marquis
d'Argenson. Voici d'abord la note de Saint-Simon\,:

«\,Boucherai, qui fut chancelier {[}à la mort de Le Tellier{]}, n'en
avait que la figure, mais telle qu'à peindre un chancelier exprès on
n'aurait pu mieux réussir\footnote{Voy., sur Boucherat, les
  \emph{Mémoires} de Saint-Simon, t. II, p.~296 et suiv.}. Il avait été
le conseil de M. de Turenne et son ami intime, et cela l'avait fort
avancé\,; du reste, pesant et de fort peu d'esprit et de lumières. Cette
alternative semblait fatale aux chanceliers. Séguier, un des grands
hommes de la robe en tout genre, l'avait été entre les deux
Aligre\footnote{La pensée de Saint-Simon est claire\,: il veut dire que
  Séguier fut chancelier entre le premier d'Aligre (chancelier de 1624 à
  1635), et le second d'Aligre (chancelier de 1674 à 1677). Mais la
  ponctuation adoptée dans le \emph{Journal de Dangeau} rend la phrase
  inintelligible\,; on l'a écrite ainsi\,: «\,Séguier, un des grands
  hommes de la robe en tout genre, l'avait été entre les deux\,; Aligre,
  père et fils, choisis pour être nuls, etc.\,» Il y a là une faute
  typographique qu'il est important de corriger.}, père et fils, choisis
pour être nuls, et dont la postérité n'a pas été plus espritée. Le
Tellier\footnote{Michel Le Tellier fut chancelier de 1677 à 1685.} fut
délié, adroit, souple, rusé, modeste, toujours entre deux eaux, toujours
à son but, plein d'esprit, de force, et en même temps d'agrément, de
douceur, de prévoyance\,; moins savant que lumineux, pénétrant et
connaisseur, {[}il{]} avait fait et fondé la plus haute fortune.
Boucherat\footnote{Chancelier de 1685 à 1699.} délassa de tant de
talents, et s'il en avait montré quelqu'un dans le degré de conseiller
d'État, ils demeurèrent étouffés dans les replis de sa robe de
chancelier. Il ne fut point ministre.\,»

Saint-Simon parle, dans la suite de cette note, des candidats à la
charge de chancelier qui furent opposés à Boucherat, et sur lesquels il
l'emporta. Le marquis d'Argenson n'est pas plus favorable que
Saint-Simon à Boucherat\footnote{\emph{Mémoires du marquis d'Argenson}
  (édit. de 1825), p.~141-142.}\,: «\, Lorsque je vins au monde (en
1694), il y avait déjà quelques années que le chancelier Le Tellier,
père de M. de Louvois, était mort. M. Boucherat était revêtu de cette
éminente dignité, qui eût été bien au-dessus de sa capacité, si les
temps eussent été plus difficiles, mais le pouvoir de Louis XIV était si
bien établi, les parlements si soumis, le droit de remontrances avait
été si restreint, ou, pour mieux dire, si bien ôté aux cours
supérieures, que l'on avait pu hardiment accorder cette place à un vieux
magistrat âgé de soixante et dix ans, et devenu presque le doyen du
conseil. Aussi M. Boucherat l'occupa-t-il très pacifiquement jusqu'à
l'âge de quatre-vingt-quatre ans qu'il mourut\footnote{Boucherat mourut
  le 2 septembre 1699.}, ne laissant que des filles. Il eut pour
successeur M. de Pontchartrain\footnote{Voy., sur Pontchartrain, les
  \emph{Mémoires} de Saint-Simon, t. II, p.~301-305.}, qui était, depuis
1689, contrôleur général des finances, et, depuis 1690, secrétaire
d'État de la marine et du département de Paris.

«\,M. de Pontchartrain prit la charge de chancelier comme une retraite.
Effectivement elle pouvait être regardée comme telle en ces temps de
soumission. Il se trouva bien heureux que le roi voulût lui accorder
pour successeur, dans le contrôle des finances, M. de Chamillart, et
dans ses départements (de la marine et de Paris), M. de Pontchartrain,
son fils. L'un et l'autre n'étaient, assurément point capables de le
remplacer dignement\,; mais ils le débarrassaient des soins les plus
fatigants. Il fallut pourtant bien qu'il continuât à conseiller son
fils, qui ne lui donnait pas toute la satisfaction qu'il en pouvait
espérer\footnote{Voy., sur le fils du chancelier, les \emph{Mémoires} de
  Saint-Simon, t. IV, p.~377 et suiv.}\,; ce qui l'engagea à une
retraite totale. Louis XIV était vieux et menaçait ruine\,; M. de
Pontchartrain était précisément du même âge\,; d'ailleurs il voulait
sagement éviter d'être obligé de porter au parlement l'édit qui
déclarait les princes légitimés habiles à succéder à la
couronne\footnote{Cet édit fut porté au parlement le 2 août 1714, et le
  chancelier de Pontchartrain s'était retiré en juillet. Il mourut en
  1727, âgé de quatre-vingt-neuf ans. Sa correspondance est conservée à
  la Biblioth. Imp. ms. f.~Mortemart, n.~60-61.}.

«\,M. Voysin fut chargé de cette opération, qui s'exécuta pourtant avec
la soumission que l'on montra pour les ordres de Louis XIV jusqu'au
moment de la mort de ce monarque, arrivée, comme chacun sait, le 1er
septembre 1715. M. Voysin, chancelier à peu près de la même force que M.
Boucherat, mourut fort à propos au mois de février 1717\footnote{Voy.,
  sur le chancelier Voysin, les \emph{Mémoires} de Saint-Simon, t. VII,
  p.~253 et suiv.}. Il fut remplacé par M. d'Aguesseau\footnote{Voy.,
  \emph{Ibid.}. t. XIV, p.~176 et suiv., le caractère du chancelier
  d'Aguesseau.}.

«\,Des trois derniers chanceliers du règne de Louis XIV, M. de
Pontchartrain était sans contredit le plus capable. Il avait été
conseiller au parlement de Paris. M. de Pontchartrain fut ensuite
pendant vingt ans premier président au parlement de Bretagne, et y donna
des preuves de fermeté, d'habileté et d'adresse, en ménageant ces têtes
bretonnes de tout temps si difficiles à conduire.\,»

\end{document}
